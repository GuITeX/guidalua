

\chapter{Costrutti di base}

\section{Il ciclo \key{for} e il condizionale \key{if}}
\label{secFondCicloIf}

Cominciamo con il contare i numeri pari contenuti in una tabella che funziona
come un array, ricordandoci che gli indici partono da 1 e non da 0. Rileggete
il capitolo precedente come utile riferimento.

Creiamo la tabella con il costruttore in linea e iteriamo con un ciclo
\key{for}:
\sourcecode{from_file [[code/e1-001.lua]]:add_output()}

Il corpo del ciclo \key{for} di Lua è il blocco compreso tra le parole chiave
obbligatorie \key{do} ed \key{end}. La variabile \key{i} interna al ciclo
assumerà i valori da 1 fino al numero di elementi della tabella, ottenuto con
l'operatore lunghezza \key{\#} valido anche per le stringhe.

Per ciascuna iterazione con il costrutto condizionale \key{if} incrementeremo un
contatore solo nel caso in cui l'elemento della tabella è pari. L'\key{if} ha
anch'esso bisogno di definire il blocco di codice e lo fa con le parole chiavi
obbligatorie \key{then} ed \key{end}, mentre \key{else} o \key{elseif} sono rami
di codice facoltativi.

Il controllo di parità degli interi si basa sull'operatore modulo, resto della
divisione intera \key{\%}. Infatti un numero pari è tale quando il resto della
divisione per 2 è zero. 

L'operatore di \emph{uguaglianza} è il doppio carattere di uguale \texttt{==} e
quello di \emph{disuguaglianza} è la coppia dei segni tilde e uguale
\verb|~=|. Naturalmente funzionano anche gli operatori di confronto \key{>},
\key{>=} e \key{<}, \key{<=}.


\subsection{Operatore di lunghezza}

Ma come si comporta l'operatore di lunghezza \texttt{\#} per le tabelle array
con indici non lineari? Per esempio, qual è il risultato del seguente codice:
\lines
local t = {}
t[1] = 1
t[2] = 2
t[1000] = 3
print(#t)
\endlines
\sourcecode{from_lines()}
e in questo caso cosa verrà stampato?
\lines
local t = {}
t[1000] = 123
print(#t)
\endlines
\sourcecode{from_lines()}

Avrete certamente capito che l'operatore \texttt{\#} tiene conto solamente
degli elementi con indici consecutivi a cominciare da 1 e s'interrompe quando
incontra \texttt{nil}. Infatti, l'operatore di lunghezza \texttt{\#} considera
per le tabelle il valore \texttt{nil} di un indice come termine dell'array.
L'operatore è usato molto spesso per inserire progressivamente elementi:
\lines
local t = {}
for i = 1, 100 do
    t[#t+1] = i*i
end
print(t[100])
\endlines
\sourcecode{from_lines()}


\section{Il ciclo \key{while}}
\label{secCicloWhile}

Passiamo a scrivere il codice per inserire in una tabella i fattori primi di un
numero. Fatelo per esercizio e poi confrontate il codice seguente che utilizza
l'operatore modulo \key{\%}:
\sourcecode{
    from_file [[code/e1-002.lua]]:add_output()
}

Così abbiamo introdotto anche il ciclo \key{while} perfettamente coerente con
la sintassi dei costrutti visti fino a ora: il blocco di codice ripetuto fino a
che la condizione è vera, è obbligatoriamente definito da due parole chiave,
quella di inizio è \key{do} e quella di fine è \key{end}.

Le variabili definite come locali nei blocchi del ciclo \key{for}, nei rami del
condizionale \key{if} e nel ciclo \key{while}, non sono visibili all'esterno.


\section{Intermezzo}

In Lua non è obbligatorio inserire un carattere delimitatore sintattico ma è
facoltativo il \key{;}. I caratteri spazio, tabulazione e ritorno a capo vengono
considerati dalla grammatica come separatori, perciò si è liberi di formattare
il codice come si desidera inserendo per esempio più istruzioni sulla stessa
linea. Solitamente non si utilizzano i punti e virgola finali, ma se ci sono due
assegnazioni sulla stessa linea --- stile sconsigliabile perché poco leggibile
--- li si può separare almeno con un \key{;}. Come sempre una forma stilistica
chiara e semplice vi aiuterà a scrivere codice più comprensibile anche a
distanza di tempo.

Generalmente è buona norma definire le nuove variabili il più vicino possibile
al punto in cui verranno utilizzate per la prima volta, un beneficio per la
comprensione ma anche per la correttezza del codice perché può evitare di
confondere i nomi e magari di introdurre errori.


\section{Il ciclo \key{for} con il passo}

Provate a scrivere il codice Lua che verifica se un numero è \emph{palindromo},
ovvero che gode della proprietà che le cifre decimali sono simmetriche come per
esempio avviene per il numero 123321. Confrontate poi questa soluzione:
\sourcecode{
    from_file [[code/e1-003.lua]]
    :select_lines [[prima_sol]]
    :add_output{delim_run=true}
}

La soluzione utilizza una tabella per memorizzare le cifre in ordine inverso
del numero da verificare, che vengono poi utilizzate successivamente nel ciclo
\key{for} dall'ultima --- la cifra più significativa --- fino alla prima per
ricalcolare il valore. Se il numero iniziale è palindromo allora il
corrispondente numero a cifre invertite è uguale al numero di partenza.

Nel ciclo \key{for} il terzo parametro opzionale -1 imposta il passo per la
variabile \key{i} che quindi passa dal numero di cifre del numero da
controllare (6 nel nostro caso) a 1.

In effetti non è necessaria la tabella:
\sourcecode{
    from_file [[code/e1-003.lua]]
    :select_lines [[seconda_sol]]
    :add_output{delim_run=true}
}


\section{\key{if} a rami multipli}

Il prossimo problema è il seguente: determinare il numero di cifre di un
intero. Ancora una volta, confrontate il codice proposto solo dopo aver cercato
una vostra soluzione.
\sourcecode{
    from_file [[code/e1-004.lua]]
}

Questo esempio mostra in azione l'\key{if} a più rami che in Lua svolge la
funzione del costrutto \key{switch} presente in altri linguaggi, con una nuova
parola chiave: \key{elseif}.

L'esempio è interessante anche per come viene introdotta la variabile
\key{digits}, cioè senza inizializzarla per poi assegnarla nel ramo opportuno
dell'\key{if}. Infatti una variabile interna a un blocco non sopravvive oltre,
per questo motivo dichiararla all'interno dell'\key{if} non è sufficiente.

Come è necessario \emph{non} premettere \key{local} nelle assegnazioni nei rami
del condizionale: in questo caso verrebbe creata una nuova variabile locale al
blocco che \emph{oscurerebbe} quella esterna con lo stesso nome. In altre
parole, al termine del condizionale \key{digits} varrebbe ancora \key{nil}, il
valore che assume nel momento della dichiarazione.


\section{Esercizi}

\begin{Exercise}[label=cos-01]
Contare quanti interi sono divisibili sia per 2 che per 3 nell'intervallo \( [1,
10\,000]\). Suggerimento: utilizzare l'operatore modulo \key{\%}, resto della
divisione intera tra due operandi.
\end{Exercise}

\begin{Exercise}[label=cos-02]
Determinare i fattori del numero intero \(5\,461\,683\) modificando il codice
riportato alla sezione~\ref{secCicloWhile} per includerne la molteplicità.
\end{Exercise}

\begin{Exercise}[label=cos-03]
Instanziare la tabella seguente con tre tabelle/array di tre numeri e calcolarne
il determinante della matrice corrispondente.
\lines
local t = {
    { 0,  5, -1},
    { 2, -2,  0},
    {-1,  0,  1},
}
\endlines
\sourcecode{from_lines()}
\end{Exercise}

\begin{Exercise}[label=cos-04]
Data la tabella seguente stampare in console il conteggio dei numeri pari e dei
numeri dispari contenuti in essa. Verificare che la somma di questi due
conteggi sia uguale alla dimensione della tabella.
\lines
local t = {
    45, 23, 56, 88, 96, 11,
    80, 32, 22, 85, 50, 10,
    32, 75, 10, 66, 55, 30,
    10, 13, 23, 91, 54, 19,
    50, 17, 91, 44, 92, 66,
    71, 25, 19, 80, 17, 21,
    81, 60, 39, 15, 18, 28,
    23, 10, 18, 30, 50, 11,
    50, 88, 28, 66, 13, 54,
    91, 25, 23, 17, 88, 90,
    85, 99, 22, 91, 40, 80,
    56, 62, 81, 71, 33, 30,
    90, 22, 80, 58, 42, 10,
}
\endlines
\sourcecode{from_lines()}
\end{Exercise}

\begin{Exercise}[label=cos-05]
Data la tabella precedente, scrivere il codice per costruire una seconda tabella
uguale alla prima ma priva di duplicati e senza alterare l'ordine degli interi.
\end{Exercise}

\begin{Exercise}[label=cos-06]
Data la tabella precedente costruire una tabella le cui chiavi siano i numeri
contenuti in essa e i valori siano il corrispondente numero di volte che la
chiave stessa compare nella tabella di partenza. Stampare poi in console il
numero che si presenta il maggior numero di volte.
\end{Exercise}

% end of file
