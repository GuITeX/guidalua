
\chapter{Iteratori}
\label{chFondIteratori}

Gli iteratori offrono un approccio semplice e unificato per scorrere uno alla
volta gli elementi di una collezione di dati. Vi dedicheremo un capitolo proprio
perché sono molto utili per scrivere codice efficiente ed elegante.

Il linguaggio Lua prevede il ciclo d'iterazione \emph{generic for} che
introduce la nuova parola chiave \key{'in'}\luak{in} secondo questa sintassi:
\begin{lines}
for <lista variabili> in iterator_function() do
-- codice
end
\end{lines}

Le tabelle di Lua sono oggetti che possono essere impiegati per rappresentare
degli array oppure dei dizionari. In entrambe i casi Lua mette a disposizione
due iteratori predefiniti rispettivamente tramite le funzioni
\fn{ipairs}\luastd{ipairs} e \fn{pairs}\luastd{pairs}.

Queste funzioni restituiscono un iteratore conforme alle specifiche del generic
for. Mentre impareremo più tardi a scrivere iteratori personalizzati,
dedicheremo le prossime due sezioni a questi importanti iteratori predefiniti
per le tabelle.


\section{Funzione \fn{ipairs}}

La funzione \fn{ipairs} restituisce un iteratore che a ogni ciclo genera due
valori: l'indice dell'array e il valore corrispondente. L'iterazione comincia
dalla posizione 1 e termina fino a quando il valore diventa \key{nil}:
\begin{lines}
-- una tabella array
local t = {45, 56, 89, 12, 0, 2, -98}

-- iterazione tabella come array
for i, v in ipairs(t) do
    print(i, v)
end
\end{lines}

Il ciclo con \fn{ipairs} è equivalente a questo codice:
\begin{lines}
-- una tabella array
local t = {45, 56, 89, 12, 0, 2, -98}
do
    local i, v = 1, t[1]
    while v do
        print(i, v)
        i = i + 1
        v = t[i]
    end
end
\end{lines}

Se non interessa il valore dell'indice è buona norma dare al nome di variabile
corridspondente un segno di underscore \key{\_} che in Lua è un identificatore
valido. Per esempio:
\begin{lines}
-- una tabella array
local t = {45, 56, 89, 12, 0, 2, -98}
local sum = 0
for _, elem in ipairs(t) do
    sum = sum + elem
end
print(sum)
\end{lines}

Se non vogliamo incorrere in errori è molto importante ricordarsi che con
\fn{ipairs} verranno restituiti i valori in ordine di posizione da 1 in poi e
fino a che non verrà trovato un valore \key{nil}. Se desiderassimo raggiungere
tutte le coppie chiave/valore dovremo far ricorso all'iteratore \fn{pairs}
che tratteremo nella prossima sezione.


\section{Funzione \fn{pairs}}
\label{secFondPairsIterator}

Questa funzione primitiva di Lua considera la tabella come un dizionario
pertanto l'iteratore restituirà in un ordine casuale tutte le coppie chiave
valore contenute nella tabella stessa.

Una tabella con indici a salti verrà iterata parzialmente da \fn{ipairs} ma
completamente da \fn{pairs} al prezzo di perdere l'ordinamento:
\sourcecode{
    file = [[code/e9-iter.lua]],
    select = [[uno]],
    run = true,
}

Il comportamento di questi due iteratori potrebbe lasciare perplessi ma è
coerente con le caratteristiche di Lua.


\section{Generic \key{for}}

Come può essere implementato un iteratore in Lua? Per iterare è necessario
mantenere alcune informazioni essenziali chiamate \emph{stato} dell'iteratore.
Per esempio l'indice a cui siamo arrivati nell'iterazione di una tabella/array
e la tabella stessa.

Perchè non utilizzare la closure per memorizzare lo stato dell'iteratore?

Abbiamo incontrato le closure nella sezione \ref{secClosure}. Proviamo a
scrivere il codice per iterare una tabella:
\sourcecode{
    file = [[code/e9-iter.lua]],
    select = [[due]],
}

Funziona, molto semplicemente. Non è stato necessario introdurre nessun nuovo
elemento al linguaggio. L'iteratore è solamente una questione d'implementazione
che tra l'altro in questo caso ricrea l'iteratore \fn{ipairs} visto poco fa.

Infatti, la funzione \fn{iter\_fn} mantiene nella closure lo stato
dell'iteratore --- l'indice \key{i} e la tabella \key{t} --- e restituisce uno
dopo l'altro gli elementi della tabella. Il ciclo \key{while}\luak{while}
infinito, s'interrompe quando il valore è \key{nil}.

Tuttavia, data l'importanza degli iteratori, Lua introduce il nuovo costrutto
chiamato \emph{generic for} che si aspetta una funzione proprio come la
\fn{iter} del codice precedente. E in effetti funziona:
\sourcecode{
    file = [[code/e9-iter.lua]],
    select = [[tre]],
}

Riassumendo, la costruzione di un iteratore in Lua si basa sulla creazione di
una funzione che restituisce uno alla volta gli elementi dell'insieme nella
sequenza desiderata. Una volta costruito l'iteratore, questo potrà essere
impiegato in un ciclo generic for.

Se per esempio si volesse iterare la collezione dei numeri pari compresi
nell'intervallo da 1 a 10, avendo a disposizione l'apposito iteratore
\fn{evenNum} che definiremo in seguito, potrei scrivere semplicemente:
\begin{lines}
for n in evenNum(1,10) do
    print(n)
end
\end{lines}


\section{L'esempio dei numeri pari}

Per definire un iteratore sui numeri pari di un intervallo dobbiamo creare una
funzione che restituisce a sua volta una funzione in grado di generare la
sequenza. L'iterazione termina quando giunti all'ultimo elemento, la funzione
restituirà il valore nullo \key{nil}, cosa che succede in automatico senza dover
esplicitare un'istruzione di \key{return}\luak{return}.

Potremo fare così: dato il numero iniziale per prima cosa potremo calcolare il
primo numero pari dell'intervallo usando l'operatore modulo \key{\%}\luas{\%} e
poi creare la funzione di iterazione in sintassi anonima che prima incrementa di
2 la variabile del ciclo --- ed ecco perché dovremo inizialmente sottrarle la
stessa quantità --- e poi, se questa è inferiore all'estremo superiore
dell'intervallo, restituisce indice e numero pari corrente. Ecco il codice
completo:
\sourcecode{
    file = [[code/e9-iter.lua]],
    select = [[iter_even]],
    run = true,
}

In questo esempio, oltre ad approfondire il concetto di iterazione basata sulla
closure di Lua, possiamo notare che il generic for effettua correttamente anche
l'assegnazione a più variabili di ciclo con le regole viste nella
sezione~\ref{secFondAssegnazione}.

Naturalmente, l'implementazione data di \fn{evenNum} è solo una fra quelle
possibili, e non è detto che non debbano essere considerate situazioni
particolari come quella in cui si passa all'iteratore un solo numero o
addirittura nessun argomento.


\section{Stateless iterator}
\label{secFondStatelessIter}

Una seconda versione del generatore di numeri pari può essere un buon esempio
di un iteratore in Lua che non necessita di una closure, per un risultato ancora
più efficiente, implementando uno \emph{stateless iterator}.

Per capire come ciò sia possibile dobbiamo conoscere nel dettaglio come funziona
il generic for in Lua; dopo la parola chiave \key{'in'}\luak{in} esso si aspetta
altri due parametri oltre alla funzione da chiamare a ogni ciclo: una variabile
che rappresenta lo stato invariante e la variabile di controllo.
\begin{lines}
for <var list> in <iter_fn>, <state>, <ctl_var> do
    ...
end
\end{lines}

La funzione d'iterazione verrà chiamata a ogni ciclo con due argomenti: lo stato
invariante e la variabile di controllo e ci si aspetta che restituisca uno o più
dati di ciclo. Quando questi valori saranno \key{nil} il ciclo termina.

Nel seguente codice la funzione \fn{evenNum} provvede a restituire i tre
parametri necessari: la funzione \fn{next\_even} come iteratore, lo stato
invariante, ovvero il numero a cui la sequenza dovrà fermarsi e la variabile di
controllo che è proprio il valore nella sequenza dei numeri pari.
\sourcecode{
    file = [[code/e9-iter.lua]],
    select = [[generic_for]],
}

Con gli iteratori abbiamo terminato l'esplorazione di base del linguaggio Lua.
Gli otto capitoli dal \ref{chFondAssignment} al \ref{chFondIteratori} sono
sufficienti per scrivere programmi utili perché trattano tutti gli argomenti
essenziali, il prossimo invece, tratterà il paradigma della programmazione a
oggetti in Lua.


\section{Esercizi}

\begin{Exercise}[label=iter-01]
Dopo aver definito una tabella con chiavi e valori stampare le singole coppie
tramite l'iteratore predefinito \fn{pairs}.
\end{Exercise}

\begin{Exercise}[label=iter-02]
Scrivere una funzione che accetta un array (una tabella con indici interi in
sequenza) di stringhe e utilizzando la funzione di libreria \fn{string.upper}
restituisca un nuovo array con il testo trasformato in maiuscolo. Per esempio da
\verb|{"abc", "def", "ghi"}| a \verb|{"ABC", "DEF", "GHI"}|.
\end{Exercise}

\begin{Exercise}[label=iter-03]
Scrivere la funzione/closure per l'iteratore che restituisce la sequenza dei
quadrati dei numeri naturali a partire da 1 fino a un valore dato.
\end{Exercise}

\begin{Exercise}[label=iter-04]
Scrivere la versione \emph{stateless} dell'iteratore dell'esercizio precedente.
\end{Exercise}

\begin{Exercise}[label=iter-05]
Scrivere la versione stateless dell'iteratore \fn{ipairs}. È possibile
implementarlo in modo che la funzione d'iterazione restituisca per il ciclo
generic for solamente l'elemento della tabella e non anche l'indice?
\end{Exercise}

\begin{Exercise}[label=iter-06]
Definire un iteratore stateless attraverso la funzione \fn{Range} che,
funzionando con il generic for, simuli invece il ciclo for numerico a variabile
intera, iterando su un intervallo con un dato passo.

Gli argomenti della funzione possono variare da uno a tre secondo questo schema:
\begin{lines}
Range(7)       --> 1, 2, 3, 4, 5, 6, 7
Range(2, 7)    --> 2, 3, 4, 5, 6, 7
Range(2, 7, 2) --> 2, 4, 6
\end{lines}

Il passo può essere negativo? Quali vantaggi potrebbe avere un iteratore di
questo tipo rispetto al ciclo for numerico? Suggerimento: considerare che le
funzioni in Lua sono valori di prima classe.
\end{Exercise}

% end of file
