
\chapter{Funzioni}

Le funzioni sono il principale mezzo di astrazione e lo strumento base per
strutturare il codice.

Coerentemente con il resto del linguaggio la sintassi di una funzione comprende
due parole chiave che servono per delimitare il blocco di codice di
definizione: \key{function} ed \key{end}. Una funzione può restituire dati
tramite la parola chiave \key{return}.

Come primo esempio, vi presento una funzione per calcolare l'ennesimo numero
della \href{http://it.wikipedia.org/wiki/Successione_di_Fibonacci}{serie di
Fibonacci}. Un elemento si ottiene sommando i due precedenti elementi avendo
posto uguale a 1 i primi due:
\sourcecode{
    from_file [[code/e2-001.lua]]
    :select_lines [[uno]]
}

Con le regole dell'assegnazione multipla una funzione può accettare più
argomenti. Se gli argomenti passati sono in eccesso rispetto a quelli che essa
prevede, quelli in più verranno ignorati. Se viceversa, gli argomenti sono
inferiori a quelli previsti allora a quelli mancanti verrà assegnato il valore
\key{nil}.

Ma questo vale anche per i dati di ritorno quando la funzione è usata come
espressione in un'istruzione di assegnamento. Basta inserire dopo l'istruzione
\key{return} la lista delle espressioni separate da virgole che saranno valutate
e assegnate alle corrispondenti variabili.

Per esempio, potremo modificare la funzione precedente per restituire la somma
dei primi \( n \) numeri di Fibonacci oltre che l'ennesimo elemento della serie
stessa e considerare un valore di default se l'argomento è \key{nil}:
\sourcecode{
    from_file [[code/e2-001.lua]]
    :select_lines [[due]]
    :add_output{delim_run = true}
}


\section{Funzioni: valori di prima classe, I}

In Lua le funzioni sono un tipo. Possono essere assegnate a una variabile e
possono essere passate come argomento a un'altra funzione, una proprietà che
non si trova spesso nei linguaggi di scripting e che offre una nuova
flessibilità al codice.

In realtà in Lua tutte le funzioni sono memorizzate in variabili. Per assegnare
direttamente una funzione a una variabile esiste la sintassi anonima:
\lines
add = function (a, b)
    return a + b
end
print(add(45.4564, 161.486))
\endlines
\sourcecode{from_lines()}

Essendo le funzioni valori di prima classe ne consegue che in Lua le funzioni
sono oggetti senza nome esattamente come lo sono gli altri tipi come i numeri e
le stringhe. Inoltre, la sintassi classica di definizione:
\lines
function variable_name (args)
    -- function body
end
\endlines
\sourcecode{from_lines()}

\noindent è solo \emph{zucchero sintattico} perché l'interprete Lua la tradurrà
automaticamente nel codice equivalente in sintassi anonima:
\lines
variable_name = function (args)
    -- function body
end
\endlines
\sourcecode{from_lines()}


\section{Funzioni: valori di prima classe, II}

Un esempio di funzione con un argomento funzione è il seguente, dove viene
eseguito un numero di volte dato, la stessa funzione priva di argomenti:
\sourcecode{
    from_file [[code/e3-001.lua]]
    :select_lines [[uno]]
}

Molto interessante. Nell'ultima riga di codice l'argomento è una funzione
definita in sintassi anonima che verrà eseguita 12 volte.

Per prendere confidenza con il concetto di \emph{funzioni come valori di prima
classe}, cambiamo il significato della funzione \fn{print}:
\lines
local orig_print = print
print = function (n)
    orig_print("Argomento funzione -> "..n)
end

print(12)
\endlines
\sourcecode{from_lines()}


\section{Tabelle e funzioni}

Se una tabella può contenere chiavi con qualsiasi valore allora può contenere
anche funzioni! Le sintassi previste sono queste, esplicitate con il codice
riportato di seguito:
\begin{compactitemize}
\item assegnare la variabile di funzione a una chiave di tabella;
\item assegnare direttamente la chiave di tabella con la definizione di funzione
in sintassi anonima;
\item usare il costruttore di tabelle per assegnare funzioni in sintassi
anonima.
\end{compactitemize}
\sourcecode{
    from_file [[code/e3-001.lua]]
    :select_lines [[due]]
}

Con questo meccanismo una tabella può svolgere il ruolo di \emph{modulo}
memorizzando funzioni utili a un certo scopo. In effetti la libreria standard
di Lua si presenta all'utente proprio in questo modo.



\section{Numero di argomenti variabile}

Una funzione può ricevere un numero variabile di argomenti rappresentati da tre
dot consecutivi \key{...}. Nel corpo della funzione i tre punti
rappresenteranno la lista degli argomenti, dunque possiamo o costruire con essi
una tabella oppure effettuare un'assegnazione multipla.

Un esempio è una funzione che restituisce la somma di tutti gli argomenti
numerici:
\sourcecode{
    from_file [[code/e3-001.lua]]
    :select_lines [[tre]]
}

Per inciso, anche la funzione base \fn{print} accetta un numero variabile di
argomenti. Il meccanismo è ancora più flessibile perché tra i primi argomenti
vi possono essere variabili ``fisse''. Per esempio il primo parametro potrebbe
essere un moltiplicatore:
\sourcecode{
    from_file [[code/e3-001.lua]]
    :select_lines [[quattro]]
}

Un'altra funzione predefinita \fn{select} consente di accedere alla lista degli
argomenti in dettaglio. Infatti nel codice precedente, se tra gli argomenti
compare un valore \key{nil} avremo problemi ad accedere ai valori successivi
perché --- come sappiamo già --- l'operatore di lunghezza \key{\#} considera il
\key{nil} come valore sentinella di fine array/tabella.

Il selettore prevede un primo parametro fisso seguito da una lista variabile di
valori rappresentata dai tre punti \key{...}. Se questo parametro è un intero
allora verrà considerato come indice per restituire l'argomento corrispondente.
Se invece il parametro è la stringa \key\# allora la funzione restituisce il
numero totale di argomenti, inclusi i \key{nil}.

Il codice seguente preso pari pari dal \href{http://www.lua.org/pil/}{PIL} ne è
un'applicazione:
\lines
for i = 1, select("#", ...) do
    local arg = select(i, ...)
    -- loop body
end
\endlines
\sourcecode{from_lines()}


\section{Omettere le parentesi se}

In Lua esiste la sintassi di chiamata a funzione semplificata che consiste nella
possibilità di ommettere le parentesi tonde \key{()}, ammessa solo se:
\begin{compactitemize}
\item alla funzione si passa un unico argomento di tipo stringa;
\item alla funzione si passa un unico argomento di tipo tabella.
\end{compactitemize}

Per esempio:
\sourcecode{
    from_file[[code/e7-funzioni.lua]]
    :select_lines [[anchequesto]]
}


\section{Closure}
\label{secClosure}

Chiudiamo il capitolo parlando con uno strano termine forse meglio noto agli
sviluppatori dei linguaggi funzionali: la \emph{closure}.

Questa proprietà di Lua amplia il concetto di funzione rendendo possibile
l'accesso dall'interno di essa ai dati presenti nel contesto esterno. Ciò è
possibile perché alla chiamata di una funzione viene creato uno spazio di
memoria del contesto esterno unico e indipendente.

\begin{quote}
\emph{%
Tutte le chiamate a una stessa funzione condivideranno una stessa closure.%
}
\end{quote}

Se questo è vero una funzione potrebbe incrementare un contatore creato al suo
interno, e anche qui prendo l'esempio di codice dal PIL:
\sourcecode{
    from_file [[code/e7-funzioni.lua]]
    :select_lines [[uno]]
}

Il codice definisce una funzione \fn{new\_counter} che restituisce una
funzione che ha accesso indipendente al contesto (la variabile \key{i}).

Tecnicamente la closure \emph{è} la funzione effettiva mentre invece la
funzione non è altro che il prototipo della closure.

Le closure consentono di implementare diverse tecniche utili in modo naturale e
concettualmente semplice. Una funzione di ordinamento potrebbe per esempio
accettare come parametro una funzione di confronto per stabilire l'ordine tra
due elementi tramite l'accesso a una seconda tabella esterna contenente
informazioni utili per l'ordinamento stesso.

Nel prossimo esempio mettiamo in pratica l'idea. Il codice utilizza una
funzione della libreria di Lua, che introdurremo nel prossimo capitolo, in
particolare \fn{table.sort}, per applicare l'algoritmo di ordinamento alla
tabella passata come argomento in base al criterio di ordine stabilito con la
funzione passata come secondo argomento in sintassi anonima.
\sourcecode{
    from_file [[code/e7-funzioni.lua]]
    :select_lines [[due]]
    :add_output{delim_run=true}
}


\section{Esercizi}

\begin{Exercise}[label=fn-01]
Scrivere una funzione che sulla base della stringa in ingresso \key{"+"},
\key{"-"}, \key{"*"}, \key{"/"} restituisca la funzione corrispondente per due
operandi.
\end{Exercise}

\begin{Exercise}[label=fn-02]
Scrivere la funzione che accetti due argomenti numerici e ne restituisca i
risultati delle quattro operazioni aritmetiche.
\end{Exercise}

\begin{Exercise}[label=fn-03]
Scrivere una funzione che restituisca il fattoriale di un numero memorizzandone
in una tabella di closure i risultati per evitare di ripetere il calcolo in
chiamate successive con pari argomento.
\end{Exercise}

\begin{Exercise}[label=fn-04]
Scrivere una funzione con un argomento opzionale rispetto al primo parametro
numerico che ne restituisca il seno interpretandolo in radianti se l'argomento
opzionale è \key{nil} oppure \key{rad}, in gradi sessadecimali se \key{deg} o
in gradi centesimali se \key{grd}.
\end{Exercise}

\begin{Exercise}[label=fn-05]
Scrivere una funzione che accetti come primo argomento una funzione \( f:
ℝ \to ℝ \) (prende un numero e restituisce un numero), come
secondo e terzo argomento i due valori dell'intervallo di calcolo e come quarto
argomento il numero di punti in cui suddividere l'intervallo. La funzione dovrà
stampare i valori che la funzione argomento assume nei punti d'ascissa così
definiti.
\end{Exercise}


% end of file
