
\chapter{Primi passi in \LuaTeX}

In questo capitolo affronteremo il tema dello scambio dati tra \TeX{} e Lua, per
poi approfondire le applicazioni nelle pagine successive, un territorio di gran
lunga ancora inesplorato.


\section{La primitiva \cs{directlua}}

Come già accennato nell'introduzione, il principale modo di eseguire codice Lua
in \LuaTeX{} è assegnarlo come argomento alla macro \cs{directlua}. Quello che
avviene è stabilito da queste regole:
\begin{compactenumerate}
\item
l'argomento di \cs{directlua} viene espanso --- può quindi contenere macro con
un testo di sostituzione --- ed eseguito come blocco;

\item
le variabili locali hanno validità solo all'interno del blocco mentre quelle
globali saranno valide anche in quelli di successive \cs{directlua};

\item
l'espansione di \cs{directlua} è vuota;
\end{compactenumerate}

Come esempio minimo consideriamo il seguente sorgente \LuaTeX{} che stampa in
console assieme agli altri messaggi emessi dal processo di compilazione l'ora di
inizio della compilazione:
\lines
% !TeX program = LuaTeX
\directlua{
local time = \the\time
local h = math.floor(time/60)
local m = time - h*60
print(h..":"..m)}% stampa '18:16'
\bye
\endlines
\sourcecode{from_lines()}

Al termine dell'espansione l'istruzione di assegnazione della variabile numerica
\key{time} contiene il minuto trascorso dall'inizio del giorno.

Per il formato \LuaLaTeX{} lo stesso file potrebbe essere:
\lines
% !TeX program = LuaLaTeX
\documentclass{article}
\directlua{
    local time = \the\time
    local h = math.floor(time/60)
    local m = time - h*60
    print(h..":"..m)
}
\begin{document}
\end{document}
\endlines
\sourcecode{from_lines()}


\subsection{Registro delle compilazioni}

Ammettiamo di voler mantenere un registro 





\endinput


\chapter{Lua in \LuaTeX}

Agli usuali moduli della libreria standard di Lua, sono stati aggiunti in
\LuaTeX{} così come negli altri motori di composizione estesi, speciali
funzionalità dedicate al controllo dello stato interno e alla creazione di
elementi tipografici.

\TeX{} detiene il controllo principale dell'esecuzione.





Ho inserito la riga magica usata da alcuni shell editor per selezionare il compositore, per esplicitare il programma per il quale il codice è scritto.


Per il formato ConTeXt...



\section{Le funzioni \fn{tex.print}}















% end of file

