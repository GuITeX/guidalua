
\chapter{Assegnazione e tipi predefiniti}
\label{chFondAssignment}


\section{Lua, proprio un bel nome}

Lua è un linguaggio semplice ma non banale. Il suo ambito di applicazione è
quello dei linguaggi di scripting: text processing, manutenzione del sistema,
elaborazioni su file dati, eccetera e lo si può anche trovare come linguaggio
embedded di programmi complessi come i videogiochi o altri applicativi
che danno la possibilità di essere programmati con esso dall'utente.

Lua è stato ideato da un gruppo di programmatori esperti
dell'\href{http://www.puc-rio.br/index.html}{Università Cattolica di Rio de
Janeiro} in Brasile. ``Lua'' si pronuncia LOO-ah e significa ``Luna'' in
portoghese!


\section{L'assegnamento}
\label{secFondAssegnamento}

Ci occupiamo ora di uno degli elementi di base dei linguaggi informatici:
l'istruzione di \emph{assegnamento}. Con questa operazione viene introdotto un
\emph{simbolo} nel programma associandolo a un valore che apparterrà a uno
dei possibili \emph{tipi} di dato.

La sintassi di Lua non sorprende: a sinistra compare il nome della variabile e
a destra l'espressione che fornirà il valore da assegnare al simbolo. Il
carattere di `\texttt{=}' funge da separatore:
\begin{lines}
a = 123
\end{lines}

Durante l'esecuzione di questo codice, Lua determina dinamicamente il tipo del
valore letterale `123' --- un numero --- creandolo in memoria col nome di
`\texttt{a}'.

L'istruzione di assegnamento omette il tipo di dato non essendone prevista una
dichiarazione esplicita. In altre parole, i dati hanno un tipo, ma ciò entra in
gioco solamente a tempo di esecuzione.

Altro concetto importante di Lua è che le variabili sono tutte globali a meno
che non si dichiari il contrario.


\subsection{Locale o globale?}
\label{secFondLocaleGlobale}

Una proprietà dell'assegnamento è che se non diversamente specificato Lua
istanzia i simboli nell'ambiente globale del codice in esecuzione. Se si
desidera creare una variabile locale rispetto al blocco di codice in cui è
definita, occorre premettere alla definizione la parola chiave \key{local}.

Le variabili locali evitano alcuni errori di programmazione e in Lua rendono il
codice più veloce. Le useremo \emph{sempre} quando un simbolo appartiene in modo
semantico a un blocco, per esempio al corpo di una funzione\footnote{Da notare
che in sessione interattiva, ovvero nel modo REPL dell'interprete Lua, ogni riga
è un blocco quindi le variabili locali non sopravvivono alla riga successiva.
Perciò in questa modalità si usano solo variabili globali.}.

Se si crea una variabile locale con lo stesso nome di una variabile globale
quest'ultima viene \emph{oscurata} e il suo valore sarà protetto da modifiche
fino a che il blocco in cui è definita la variabile locale non termina.


\subsection{Assegnazioni multiple}

In Lua possono essere assegnate più variabili alla volta nella stessa
istruzione. Questo significa che l'assegnamento è in realtà più complesso di
quello presentato fino a ora perché è possibile scrivere una lista di variabili
separate da virgole che assumeranno i valori corrispondenti della lista di
espressioni, sempre separate da virgole che compare dopo il segno di uguale:
\begin{lines}
local a, b = 0.45 + 0.23, "text" -- a = 0.68; b = "text"
\end{lines}

Quando il numero delle variabili non corrispondono a quello delle espressioni,
Lua assegnerà automaticamente valori \texttt{nil} o ignorerà le espressioni in
eccesso. Per esempio:
\begin{lines}
local a, b, c = 0.45, "text"  -- c vale nil
local x, y = "op", "qw", "lo" -- "lo" è un dato ignorato
\end{lines}

Nell'assegnazione Lua prima valuta le espressioni a destra e solo
successivamente crea le rispettive variabili secondo l'ordine della lista.
Perciò per scambiare il valore di due variabili, operazione chiamata
\emph{switch}, è possibile scrivere semplicemente:
\begin{lines}
x, y = y, x
\end{lines}

Un ulteriore esempio di assegnazione multipla è il seguente, a dimostrazione
che le espressioni della lista a destra vengono prima valutate e solo dopo
assegnate alle corrispondenti variabili nella lista di sinistra:
\begin{lines}
#[run]
local pi = 3.14159
local r = 10.8 -- raggio del cerchio
-- grandezze cerchio
local diam, circ, area = 2*r, 2*pi*r, pi*r^2
-- stampa grandezze
print("Diametro:", diam)
print("Circonferenza:", circ)
print("Area:", area)
\end{lines}

Le assegnazioni multiple sono interessanti ma sembra non siano così importanti,
possiamo infatti ricorrere ad assegnazioni singole. Diverranno invece molto
utili con le funzioni e con gli iteratori di cui ci occuperemo in seguito.


\section{Una manciata di tipi}
\label{secFondManciataTipi}

In Lua esistono una manciata di tipi. Essenzialmente, omettendone due di uso
avanzato, sono solo questi:
\begin{compactitemize}
\item \key{number} il tipo numerico\footnote{Solamente dalla versione 5.3 di Lua
vengono internamente distinti gli interi e i numeri in virgola mobile};
\item \key{string} il tipo stringa \( \to \) capitolo \ref{chFondStringhe};
\item \key{boolean} il tipo booleano \( \to \) capitolo \ref{chFondOpLogic};
\item \key{table} il tipo tabella \( \to \) capitolo \ref{chFondTabella};
\item \key{nil} il tipo nullo \( to \) sezione \ref{secFondTipoNil};
\item \key{function} il tipo funzione \( to \) capitolo \ref{chFondFunzioni}.
\end{compactitemize}

Il breve elenco suscita due osservazioni: tranne la tabella non esistono
tipi strutturati mentre le funzioni hanno il rango di tipo.

Questo fa capire molto bene il carattere di Lua: da un lato l'essenzialità ha
ridotto all'indispensabile i tipi predefiniti nel linguaggio, ma dall'altro ha
spinto all'inclusione di concetti intelligenti e potenti.


\subsection{Il tipo \key{nil}}
\label{secFondTipoNil}

Uno dei concetti più importanti che caratterizzano un linguaggio di
programmazione è la presenza o meno del tipo nullo. In Lua esiste e viene
chiamato \key{nil}\luak{nil}. Il tipo nullo ha un solo valore possibile,
anch'esso chiamato \key{nil}. Il nome è così sia l'unico valore possibile che il
tipo.

Leggere una variabile non istanziata non è un errore perché Lua restituisce
semplicemente \key{nil}, mentre assegnare il valore nullo a una variabile la
distrugge:
\begin{lines}
print(z)      --> stampa nil, la variabile 'z' non esiste
local z = 123 --> assegnamento di un tipo numerico
print(z)      --> stampa 123
z = nil       --> distruzione della variabile
\end{lines}


\section{Gli identificatori}

I nomi che possiamo dare a variabili e funzioni, sono stringhe di lettere
maiuscole o minuscole, numeri e trattini bassi \key{\_}\luas{\_} purché non
comincino con una cifra e non corrispondano alle \emph{keyword}, le parole
riservate del linguaggio.

In Lua gli identificatori sono \emph{case sensitive} cioè le stesse lettere ma
diverse nel maiuscolo o minuscolo, ne formano di distinti come \key{vaR} e
\key{Var}.

Si è quindi liberi di utilizzare nomi qualsiasi, tuttavia è conveniente aderire
a qualche \emph{convenzione} che aggiunga al nome un significato di categoria.
Per esempio, gli identificatori che iniziano con una lettera maiuscola si
possono riservare agli oggetti come nel capitolo \ref{chFondOop}, oppure quelli
che iniziano con un solo trattino basso, l'underscore \key{\_}, pur essendo
identificatori come tutti gli altri, si possono riservare per i nomi che debbano
intedersi come privati, spesso campi di tabella o funzioni ausiliarie.

Per esempio, nel capitolo degli iteratori \ref{chFondIteratori} si fa uso
dell'identificatore \key{\_} per le variabili di ciclo non utilizzate.
L'importante è che se si utilizza una convenzione dei nomi la si rispetti sempre
per non rendere il listato più a rischio di errori.

Sono invece da evitare i nomi che iniziano con un doppio trattino basso, che
potrebbero collidere con quelli dei metametodi (vedi il capitolo
\ref{chFondOop}) e quelli che pur iniziando con un singolo trattino basso hanno
poi lettere tutte in maiuscolo come per esempio \key{\_VERSION}\luak{\_VERSION},
perché potrebbero collidere con i nomi predefiniti di Lua.


\section{Il Garbage Collector}

I dati non più utili come quelli di cui non esiste più un \emph{riferimento} a
essi durante l'esecuzione, per esempio perché la variabile è stata riassegnata a
\key{nil}, oppure quelli locali nel momento in cui escono di scopo, vengono
automaticamente eliminati dal \emph{garbage collector} di Lua. Questo componente
solleva l'utente dalla gestione diretta della memoria e sopratutto dai deleteri
errori di programmazione che si possono facilmente compiere nel farlo, al prezzo
di una piccola diminuzione delle prestazioni in fase di esecuzione.

Al termine del programma tutte le risorse in memoria vengono automaticamente
liberate.

\section{Esercizi}

\begin{Exercise}[label=fond-01]
Scrivere il codice Lua che instanzi due variabili \key{x} e \key{y} al valore
12.34. Si assegni alle altre due variabili \key{sum} e \key{prod} la somma e il
loro prodotto delle prime. Si stampi in console i risultati.
\end{Exercise}

\begin{Exercise}[label=fond-02]
Scrivere il codice Lua che dimostri che modificare una variabile locale non
modifica il valore della variabile globale con lo stesso nome. Suggerimento:
utilizzare la coppia \key{do}/\key{end} per creare un blocco di codice con le
proprie variabili locali.
\end{Exercise}

% end of file
