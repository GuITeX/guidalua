

\chapter{Un registro delle compilazioni}
\label{iichRegistro}

Siamo giunti al primo capitolo di taglio applicativo. Ci occuperemo di creare
in Lua un sistema di registrazione delle compilazioni che si concretizza in un
file posizionato nella directory principale del progetto di documento.

Non è possibile reperire dati generali sulla compilazione eseguendo codice
durante la compilazione stessa. Dovremo farlo con un tool esterno. Solo in
questo modo è possibile ottenere il tempo di compilazione totale o accertarsi
del nome del file sorgente. Tuttavia la registrazione in fase di compilazione è
utile per imparare con Lua a lavorare con i file e sperimentare un processo di
sviluppo iterativo molto più rapido che con \TeX.

Chiameremo questo registro \code{history.log} verso cui invieremo una linea di
testo con alcune informazioni per ciscuna compilazione che il compositore
completi correttamente.


\section{Scrivere sul registro}

Iniziamo a implementare la gestione automatica del registro dalla funzionalità
chiave: la scrittura su file di testo.

Ci rivolgeremo alla libreria standard di Lua \code{io}\footnote{La completa
documentazione della libreria \code{io} si trova alla pagina
\url{https://www.lua.org/manual/5.3/manual.html\#6.8}.}:
\lines
-- write a line of text in a file
local function append(filename, line)
    local f = io.open(filename, "a+")
    f:write(line.."\n")
    f:close()
end
\endlines
\sourcecode{from_lines()}

La funzione \fn{io.open} restituisce un riferimento ad un oggetto che
rappresenta un canale di input/output del sistema operativo verso un file, che
useremo in colon notation (vedi alla sezione~\ref{secRectOop}). Lo specificatore
\code{a+} indica di aprire il file in \emph{append} senza distruggerne il
contenuto preesistente e se il file non esiste ne verrà creato uno vuoto. In
ogni caso otterremo il file aperto in scrittura.

La funzione \fn{append} può essere inserita in un sorgente \LuaTeX{}
compilabile come in questo sorgente:
\VerbatimInput{app-registro/01.tex}


\section{Dati di compilazione}

Costruite le fondamenta stabiliamo quali dati inserire nel registro: l'utente 
mano a mano che il lavoro sul documento procede eseguirà senza dubbio delle
compilazioni intermedie perciò registreremo le seguenti informazioni base:
\begin{compactenumerate}
\item nome del file sorgente,
\item dimensione del file sorgente,
\item data e ora della compilazione,
\item tempo di esecuzione della composizione,
\item nome del motore di composizione.
\end{compactenumerate}

Per formare la linea di registro, al posto della concatenazione di stringhe
useremo quella di una tabella di stringhe, più efficiente. Riempiremo la tabella
un dato alla volta in un blocco \cs{directlua} posizionato immediatamente prima
del termine del sorgente. Come prima istruzione invece, inseriremo il blocco di
codice Lua che definirà alcuni parametri come il nome del file del registro e il
separatore di campo testuale, e la funzione \fn{append}:
\VerbatimInput{app-registro/02.tex}


\subsection{La variabile \code{jobname}}

Il nome del file sorgente non è detto che sia contenuto nella variabile
\code{jobname} che possiamo trovare nella macro omonima oppure nel campo omonimo
della tabella \code{tex}, tabella Lua contiene alcune funzionalità importanti
implementate all'interno di \LuaTeX.

Il campo o la macro conterrà il nome del sorgente soltanto se l'utente non ha
valorizzato l'opzione \code{--jobname} nel comando di compilazione con un altro
nome.

Infatti, se provassimo a compilare il listato \code{02.tex} con il seguente
comando
\begin{Verbatim}
$ luatex --jobname=abc 02
\end{Verbatim}
otterremo un errore che blocca la compilazione. Nel secondo \cs{directlua} la
variabile locale \key{jobname} conterrebbe infatti la stringa \code{abc.tex}
che è un file che non esiste. Di conseguenza la funzione \fn{lfs.attributes}
restituisce \key{nil} che ovviamente non può essere indicizzato con la chiave
\key{size}.

La tabella \code{lfs} contiene la libreria Lua File System disponibile per i
sistemi operativi più diffusi, con alcune funzionalità sui file che non troviamo
di base in Lua perché l'interprete ha regole stringenti di portabilità. La
troviamo in \LuaTeX{} già compilata staticamente, mentre la documentazione
si trova all'indirizzo web
\url{https://keplerproject.github.io/luafilesystem/manual.html}.

Se la funzione \fn{lfs.attributes} non restituisce la tabella con i dati del
file significa dunque che nel comando di compilazione è stata attivata l'opzione
\code{--jobname} per assegnare un diverso nome al file PDF di uscita.
Modifichiamo dunque il codice in questo senso:
\lines
local jobname = tex.jobname
local attr = lfs.attributes(jobname..".tex")
if attr then
    tline[1] = jobname
    tline[2] = attr.size
else
    tline[1] = jobname..".pdf"
    tline[2] = "unknow"    
end
tline[3] = os.date()
tline[4] = os.clock() - register.start
tline[5] = status.luatex_engine
\endlines
\sourcecode{from_lines()}

Anche l'estensione \code{.tex} è un problema perché se fosse diversa ancora una
volta il codice non funzionerebbe, per esempio se il file sorgente avesse
estensione \code{.TEX}.


\subsection{Gli altri dati}

Come sappiamo dal capitolo~\ref{iChLibstd} la tabella \key{os} appartiene alla
libreria standard di Lua. La documentazione delle funzioni \fn{os.date} e
\fn{os.clock} si trova quindi nel reference ufficiale del linguaggio alla pagina
\url{https://www.lua.org/manual/5.3/manual.html#6.9}.

Infine, maggiori informazioni sulla tabella \key{status} da cui leggiamo il nome
del motore di composizione, possono essere reperite nel manuale di \LuaTeX{}.

Più precisamente la misura del tempo di compilazione che registriamo è il tempo
che trascorre tra i due momenti in cui eseguiamo la funzione \fn{os.clock}
perciò non comprende il tempo iniziale di avvio e caricamento del formato, né i
tempi finali di chiusura come la composizione del materiale non ancora emesso
sulla pagina.


\section{Creare il modulo e il pacchetto}

Sposteremo il codice Lua in un file esterno che rappresenterà un \emph{modulo},
così che potremo con un secondo file fornire l'interfaccia utente all'interno
di un formato, per esempio per \LaTeX.

Il modulo implementa funzioni Lua di alto livello scritte sulle funzioni del
compositore e di eventuali librerie esterne. Un \emph{pacchetto} fornisce
l'interfaccia utente con le regole sintattiche e l'integrazione con un dato
formato.

Questa separazione non è solo utile per un buon sviluppo, ma facilita l'accesso
alle funzionalità da parte degli utenti indipendentemente dal formato.

Allo stato attuale dello sviluppo del codice, il modulo Lua è il seguente:
\sourcecode{from_file [[app-registro/mod-history.lua]]}

Tutte le funzionalità sono racchiuse in un unica tabella Lua chiamata
\code{register}. Dall'esterno, caricando il modulo con la funzione \fn{require}
potremo assegnarne il riferimento a una variabile locale o globale a seconda
delle necessità.

L'uso della colon notation per chiamare le funzioni non fa parte di
un'implementazione a oggetti ma è solo un semplice modo per poter disporre di un
riferimento alla tabella contenitore con la chiave \key{self}. Ciò rende pulito
il codice ed elimina la necessità di creare una closure, (piccolo incremento
delle prestazioni) per includere il riferimento.

Il sorgente in plain \LuaTeX{} si semplifica in questo:
\VerbatimInput{app-registro/03.tex}

Per compilarlo correttamente il file Lua del modulo chiamato
\code{mod-history.lua} deve essere visibile. La soluzione più semplice è
copiarlo localmente nella directory del sorgente.

A questo punto è semplice creare un pacchetto \LaTeX{} d'interfaccia. Non ci
sarà alcuna macro utente ma un aggiunta al codice di chiusura del documento
con la macro \cs{AtEndDocument}. Basta caricare il pacchetto e tutte le
compilazioni del sorgente verranno registrate. Ecco il codice:
\VerbatimInput{app-registro/historylog.sty}

e un conseguente file sorgente compilabile con Lua\LaTeX{}:
\VerbatimInput{app-registro/doc.tex}

Compilatelo più volte e controllate poi il file locale \code{history.log}.


\section{Sviluppo}

Abbiamo fino a ora seguito in dettaglio la costruzione di un modulo per inserire
in un registro alcune informazioni sulle compilazioni del documento,
implementando la funzionalità minima.

In questa sezione, proveremo a dare al modulo una nuova struttura. Ci sono molti
modi per consentire all'utente di scegliere quali dati registrare e in quale
formato. L'obiettivo è quello di aggiungere funzionalità in modo che sia
semplice. Per esempio, se si vuole che la data sia trascritta nel formato ISO
\code{yyyy-mm-dd} oppure che la dimensione del file riporti unità di misura
che si adattano a seconda della magnitudo del valore.

Elaboreremo la struttura di un \emph{framework}.






% end of file

