
\chapter{La libreria standard di Lua}
\label{iiChLibstd}

In Lua sono immediatamente disponibili un folto gruppo di funzioni che ne
formano la \emph{libreria standard}. Si tratta di una collezione di funzioni
utili a svolgere compiti ricorrenti su stringhe, file, tabelle, eccetera, e si
trovano precaricate in una serie di tabelle.

L'elenco completo ma in ordine sparso con il nome della tabella/modulo
contenitore e la descrizione applicativa è il seguente:
\begin{center}
\begin{tabular}{ll}
\key{math} & matematica;\\
\key{table} & utilità sulle tabelle;\\
\key{string} & ricerca, sostituzione e pattern matching;\\
\key{io} & input/output facility, operazioni sui file;\\
\key{bit32} & operazioni bitwise (solo in Lua 5.2);\\
\key{os} & date e chiamate di sistema;\\
\key{coroutine} & creazione e controllo delle coroutine;\\
\key{utf8} & utilità per la codifica Unicode UTF-8 (da Lua 5.3);\\
\key{package} & caricamento di librerie esterne;\\
\key{debug} & accesso alle variabili e performance assessment.\\
\end{tabular}
\end{center}
La pagina web a \href{www.lua.org/manual/5.3/contents.html}{questo link}
fornisce tutte le informazioni di dettaglio sulla libreria standard di Lua~5.3.


\section{Libreria matematica}

Nella libreria memorizzata nella tabella \key{math} ci sono le funzioni
trigonometriche \fn{sin}, \fn{cos}, \fn{tan}, \fn{asin} eccetera --- che come
di consueto lavorano in radianti --- le funzioni esponenziali \fn{exp},
\fn{log}, \fn{log10}, quelle di arrotondamento \fn{ceil}, \fn{floor}, e quelle
per la generazione pseudocasuale di numeri come \fn{random}, e \fn{randomseed}.
Oltre a funzioni, la tabella include campi numerici come la costante \( \pi \).

Un esempio introduttivo è questo dove nella funzione \fn{one} viene definita una
funzione locale:
\sourcecode{file = [[code/e8-libstd.lua]], select = [[one]]}


\section{Libreria stringhe}

La libreria per le stringhe è memorizzata nella tabella \key{string} ed è una
delle più utili. Con essa si possono formattare campi e compiere operazioni di
ricerca e sostituzione. In effetti, in Lua non è infrequente elaborare grandi
porzioni di testo.


\subsection{Funzione \fn{string.format}}
\label{secFondStringFormat}

La funzione più semplice è quella di formattazione \fn{string.format}. Essa
restituisce una stringa prodotta con il formato definito dal primo argomento
dei dati forniti dal secondo argomento in poi.

Il formato è esso stesso specificato come una stringa contenente dei segnaposto
creati con il simbolo percentuale e uno specificatore di tipo. Per esempio
\verb|"%d"| indica il formato relativo a un numero intero, dove \key{d} sta per
digit mentre \verb|"%f"| indica il segnaposto per un numero decimale con \key{f}
che sta per float.

I campi formato derivano da quelli della funzione classica di libreria
\fn{printf} del C. Di seguito un esempio di codice:
\sourcecode{
    file = [[code/e8-libstd.lua]],
    select = [[fmt]],
    run = true,
}

Come avete potuto notare nel codice, è anche possibile fornire un ulteriore
specifica di dettaglio tra il carattere \key{\%} e lo specificatore di tipo, per
esempio per indicare il numero delle cifre decimali di arrotondamento.

Per elaborare il testo si utilizza di solito una libreria per le espressioni
regolari. Lua mette a disposizione alcune funzioni di sostituzione e
\emph{pattern matching} meno complete dell'implementazione dello standard
POSIX per le espressioni regolari ma molto spesso più semplici da utilizzare.

Esistono due strumenti di base, il primo è il \emph{pattern} e il secondo è la
\emph{capture}.


\subsection{Pattern}
\label{secFondPattern}

Il pattern è una stringa che può contenere campi chiamati \emph{classi} simili
a quelli per la funzione di formato visti in precedenza, che stavolta però si
riferiscono al singolo carattere, e questa differenza è essenziale.

La funzione di base che accetta pattern è \fn{string.match} che restituisce
la prima corrispondenza trovata in una stringa primo argomento corrispondente
al pattern dato come secondo argomento.

Per esempio, possiamo ricercare in un numero di tre cifre intere all'interno di
un testo con il pattern \verb|"%d%d%d"|:
\sourcecode{
    file = [[code/e8-libstd.lua]],
    select = [[pattern_one]],
    run = true,
}

Le classi carattere possibili sono le seguenti:
\begin{compactdescription}
  \item[\key{.}] un carattere qualsiasi;
  \item[\key{\%a}] una lettera;
  \item[\key{\%c}] un carattere di controllo;
  \item[\key{\%d}] una cifra;
  \item[\key{\%l}] una lettera minuscola;
  \item[\key{\%u}] una lettera maiuscola;
  \item[\key{\%p}] un carattere di interpunzione;
  \item[\key{\%s}] un carattere spazio;
  \item[\key{\%w}] un carattere alfanumerico;
  \item[\key{\%x}] un carattere esadecimale;
  \item[\key{\%z}] il carattere rappresentato con il codice 0.
\end{compactdescription}

Le classi ammettono quattro modificatori per esprimere le ripetizioni dei
caratteri:
\begin{compactdescription}
  \item[\key{+}] indica 1 o più ripetizioni;
  \item[\key{*}] indica 0 o più ripetizioni;
  \item[\key{-}] come \key{*} ma nella sequenza più breve;
  \item[\key{?}] indica 0 o 1 ripetizione;
\end{compactdescription}

Esempio:
\sourcecode{
    file = [[code/e8-libstd.lua]],
    select = [[pattern_two]],
    run = true,
}

\section{Capture}
\label{secFondCapture}

Il pattern può essere arricchito non solo per trovare corrispondenze ma anche
per restituirne parti componenti. Questa funzionalità viene chiamata
\emph{capture} e consiste semplicemente nel racchiudere tra parentesi tonde le
classi.

Per esempio per estrarre l'anno di una data nel formato \key{dd/mm/yyyy}
possiamo usare il pattern con la capture seguente \verb|"%d%d/%d%d/(%d%d%d%d)"|:
\sourcecode{
    file = [[code/e8-libstd.lua]],
    select = [[capture_one]],
    run = true,
}

Più capture ci sono nel pattern e altrettanti argomenti multipli di uscita
saranno restituiti:
\sourcecode{file = [[code/e8-libstd.lua]],
   select = [[capture_two]],
   run = true,
}


\subsection{La funzione \fn{string.gsub}}
\label{secFondGsub}

Abbiamo appena cominciato a scoprire le funzionalità dedicate al testo
disponibili nella libreria standard di Lua. Diamo solo un altro sguardo
presentando la funzione \fn{string.gsub}. Il suo nome sta per \emph{global
substitution}, ovvero la sostituzione di tutte le parti di un testo che
corrispondono a uno schema.

Per individuare le parti da sostituire è naturale pensare di utilizzare un
pattern e che sia possibile utilizzare le capture nel testo di sostituzione, per
esempio:
\sourcecode{
    file = [[code/e8-libstd.lua]],
    select = [[gsub]],
    run = true,
}

Il primo argomento è la stringa da ricercare, il secondo è il pattern, il terzo
è il testo di sostituzione per ciscuna corrispondenza, ma può anche essere una
tabella dove le chiavi corrispondenti al pattern saranno sostituite con i
rispettivi valori, oppure una funzione che dal singolo frammento di testo
estratto elabora la stringa di sostituzione.

Una funzione quindi assai flessibile. Mi viene in mente questo esercizio:
moltiplicare per 12 tutti gli interi in una stringa, ed ecco il codice:
\sourcecode{
    file = [[code/e8-libstd.lua]],
    select = [[gsubfn]],
    run = true,
}

A questo punto degli esempi avrete certamente capito che \fn{gsub} restituisce
anche il numero delle sostituzioni effettuate.

Tutte queste funzioni restituiscono una stringa costruita ex-novo e non
modificano la stringa originale di ricerca. In Lua le stringhe sono dati
immutabili.



\section{Esercizi}

\begin{Exercise}[label=libstd-01]
Qual è la differenza tra i campi di formato della funzione \fn{string.format} e
le classi dei pattern? Quali le somiglianze?
\end{Exercise}

\begin{Exercise}[label=libstd-02]
Stampare una data nel formato \key{dd/mm/yyyy} a partire dagli interi contenuti
nelle variabili \key{d}, \key{m} e \key{y}.
\end{Exercise}

\begin{Exercise}[label=libstd-03]
Cosa restituisce l'esecuzione della seguente funzione?
\sourcecode{
    file = [[code/e8-libstd.lua]],
    select = [[esercizio3]],
}
Quale pattern corrisponde a un numero decimale la cui parte intera può essere
omessa?
\end{Exercise}

\begin{Exercise}[label=libstd-04]
Come estrarre dal nome di un file l'estensione?
\end{Exercise}

\begin{Exercise}[label=libstd-05]
Come eliminare da un testo eventuali caratteri spazio iniziali e/o finali?
\end{Exercise}

\begin{Exercise}[label=libstd-06]
Il pattern \verb|"(%d+)/(%d+)/(%d+)"| è adatto per estrarre giorno, mese e
anno di una data presente in una stringa nel formato \verb|"dd/mm/yyyy"|?
\end{Exercise}

\begin{Exercise}[label=libstd-07]
Creare un esempio che utilizzi \fn{string.gsub} con una funzione in sintassi
anonima a due argomenti corrispondenti a due capture nel pattern di ricerca.
\end{Exercise}


% end of file
