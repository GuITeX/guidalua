

\chapter{Operatori logici}
\label{chFondOpLogic}

Il tipo \key{boolean}\luak{boolean} può assumere i valori \key{true}\luak{true}
oppure \key{false}\luak{false}.

A qualsiasi espressione Lua assegna un valore booleano se richiesto, per esempio
in un condizionale \key{if}\luak{if} o in un ciclo \key{while}\luak{while}. In
Lua un'espressione è vera se essa corrisponde al valore booleano
\key{true}\luak{true} oppure a un valore che non è \key{nil}\luak{nil},
altrimenti è falsa.

Gli operatori logici \key{and}\luak{and}, \key{or}\luak{or} e
\key{not}\luak{not} danno luogo ad alcune espressioni idiomatiche di Lua.
Cominciamo con \key{or}: è un operatore logico binario. Se il primo operando è
vero lo restituisce altrimenti restituisce il secondo. Per esempio nel seguente
codice \key{a} vale 123.
\begin{lines}
local a = 123 or "mai assegnato"
\end{lines}

L'operatore \key{and} --- anche questo binario come \key{or} --- restituisce il
primo operando se esso è falso altrimenti restituisce il secondo operando.

\section{Operatore ternario}
\label{secFondOperatoreTernario}

Con \key{and} e \key{or} combinati otteniamo l'operatore ternario del C++ in
Lua: ecco l'espressione in un esempio: se \key{a} è vera il risultato è \key{b}
altrimenti \key{c}:
\begin{lines}
local val = (a and b) or c
\end{lines}

Poiché \key{and} ha priorità maggiore rispetto a \key{or} nell'espressione
precedente possiamo omettere le parentesi per un codice ancor più idiomatico:
\begin{lines}
local val = a and b or c -- a ? b : c del C++
\end{lines}

Il massimo tra due numeri è un'espressione condizionale:
\begin{lines}
local x, y = 45.69, 564.3
local max
if x > y then
    max = x
else
    max = y
end
\end{lines}
ma con gli operatori logici è tutto più Lua:
\begin{lines}
local x, y = 45.69, 564.3
local max = (x > y) and x or y
\end{lines}

L'operatore logico \key{not} restituisce \key{true} se l'operando è \key{nil}
oppure se è \key{false} e, viceversa, restituisce \key{false} se l'operando non
è \key{nil} oppure è \key{true}. Alcuni esempi:
\begin{lines}
print(not 5)       --> 'false'
print(not not 5)   --> 'true'
print(not true)    --> 'false'
print(not false)   --> 'true'
print(not nil)     --> 'true'
\end{lines}

L'operatore di negazione può essere usato per controllare se una variabile è
valida oppure no. Per esempio possiamo controllare se in una tabella esiste il
campo \key{prezzo}:
\begin{lines}
local t = {} -- una tabella vuota
if not t.prezzo then -- t.prezzo è nil
    print("assente")
else
    print("presente")
end

t.prezzo = 12.00
if not t.prezzo then
    print("assente")
else
    print("presente")
end
\end{lines}


\section{Esercizi}

\begin{Exercise}[label=oplogic-01]
Prevedere il risultato delle seguenti espressioni Lua:
\begin{lines}
local a = 1 or 2
local b = 1 and 2
local c = "text" or 45

local d = not 12 or "ok"
local e = not nil or "ok"
\end{lines}
\end{Exercise}

\begin{Exercise}[label=oplogic-02]
Nel seguente codice, se il valore del primo condizionale è \key{true} cosa
stamperà invece il secondo condizionale?
\begin{lines}
if "stringa" then print "it's not 'nil'" end
if "stringa" == true then
    print("it's 'true'")
else
    print("it's not 'true'")
end
\end{lines}
\end{Exercise}

\begin{Exercise}[label=oplogic-03]
Come distinguere se una variabile contiene il valore \key{false} o il valore
\key{nil}?  
\end{Exercise}

\begin{Exercise}[label=oplogic-04]
Usando gli operatori logici di Lua codificare l'espressione che restituisce la
stringa \verb|"più grande di 100"|, \verb|"uguale"| o
\verb|"più piccolo di 100"| a seconda del valore numerico fornito.
\end{Exercise}

% end file
