
% presentazione della guida e informazioni di base per eseguire gli esempi

\chapter{Introduzione}

\section{Presentazione della guida}

Questa guida tematica è dedicata alla programmazione in Lua all'interno dei
motori di composizione del sistema \TeX. Con Lua è possibile compiere sia
elaborazioni generiche come interrogare basi di dati o elaborazioni tipografiche
interagendo con il compositore interno.

Con Lua, accanto a \TeX, è più semplice ed efficiente effettuare calcoli
numerici o avvalersi di avanzati sistemi esterni. Sono funzionalità possibili
con la nuova generazione di compositori, che ampliano notevolmente gli scenari
applicativi.

Se da un lato è auspicabile che queste potenzialità diventino disponibili per
gli utenti finali per mezzo di moduli e pacchetti, dall'altro è utile fornire
dettagli ed esempi per implementare proprie soluzioni o poter scrivere moduli
condividendo idee di sviluppo.

Sono certamente molte le cose da conoscere: un nuovo linguaggio molto diverso da
\TeX{}, numerosi dettagli sul funzionamento interno dei compositori Lua-powered,
nuovi problemi di organizzazione del codice, il bilanciamento tra Lua e \TeX.
Per questo, ho pensato di contribuire con questa guida cercando di presentare il
quadro della crescente complessità del sistema.


\section{Piano della guida}

La guida è divisa in tre parti: la prima offre una panoramica rapida per
iniziare subito con Lua (parte \ref{partApp}), la seconda tratta delle basi
del linguaggio Lua (parte \ref{partFoundation}) e la terza tratta di esempi
applicativi con l'uso delle librerie interne di composizione (parte
\ref{partApp}).

Tra gli argomenti della guida ci sono:
\begin{compactitemize}
\item basi del linguaggio Lua \( \to \) dal capitolo \ref{chAssignment},
\item differenza tra motore e formato di composizione \( \to \) capitolo
\ref{iichExplain},
\item tecniche di programmazione e di rappresentazione dei dati,
\item interazione tra Lua e lo stato interno del motore di composizione.
\end{compactitemize}


\section{Origine della guida}

Per illustrare i concetti del linguaggio ho preso spunto da un breve corso su
Lua che scrissi qualche tempo fa per il blog
\href{http://parliamodi-ubuntu.blogspot.it}{Lubit Linux} di Luigi Iannoccaro
che mi propose di realizzare un progetto di divulgazione su Lua. Luigi ha
acconsentito all'utilizzo di quegli appunti per produrre questa guida tematica.


\section{Contribuire alla guida}

Spero che i lettori vorranno contribuire al testo inviando le propri soluzioni o
nuovi contributi piccoli o grandi. Lo si può fare attraverso lo strumento che
preferite, scrivendomi un messaggio email all'indirizzo
\href{mailto:giaconet.mailbox@gmail.com}{giaconet.mailbox@gmail.com}, oppure
utilizzando il repository
\href{https://github.com/GuITeX/guidalua}{\texttt{git}} dei sorgenti della
guida, eseguendo un Pull Request o aprendo una discussione.


\section{Altre risorse}

La risorsa principale per imparare Lua, a cui si rimanda per tutti gli
approfondimenti, è certamente il PIL acronimo del titolo del libro
\emph{Programming In Lua} di Roberto Ierusalimschy, principale Autore
di Lua. Questo testo non solo è completo e autorevole ma è anche ben scritto e
composto\footnote{Tra l'altro il libro ufficiale su Lua viene composto in
\LaTeX{} e commercializzato per contribuire allo sviluppo del linguaggio
stesso.}.

Quanto a \LuaTeX{} il riferimento è il suo manuale che, come quasi tutta la
documentazione nel sistema \TeX{}, può essere visualizzato a video con il
comando da terminale:
\begin{Verbatim}
$ texdoc luatex
\end{Verbatim}

% end of file
