


Come tutti i linguaggi di scripting non occorre compilare il codice Lua. Basta
scrivere il programma in un file di testo --- preferibilmente assegnandogli
l'estensione \key{.lua} --- e lanciare il comando d'esecuzione in un terminale
\tcmd{\$ lua nomefile}.

Per scrivere il codice vi consiglio di utilizzare l'editor
\href{http://www.scintilla.org/SciTE.html}{SciTE} perché è già predisposto per
eseguire programmi Lua ed è programmabile in Lua. Tra l'altro l'output del
programma viene visualizzato nella \emph{output tab}. Il tasto funzione F5
esegue il programma visualizzato nella scheda attiva in quel momento e il tasto
F8 attiva/disattiva la finestra di output, che è in pratica un terminale
incorporato nell'editor.

Prepariamoci a scrivere il primo programma in Lua, ovviamente il fatidico
\href{http://en.wikipedia.org/wiki/%22Hello,_world!%22_program"}{Hello world!},
salvando il seguente codice in un file chiamato \key{primo.lua} ed eseguiamolo
direttamente in SciTE premendo il tasto funzione F5 o al terminale con il
comando \tcmd{\$ lua primo.lua}:
\lines
print("Hello world!")
\endlines
\sourcecode{from_lines():add_output()}

In ambiente Linux o Mac OS~X, come per tutti gli altri linguaggi di scripting,
possiamo inserire come prima riga del file a cui sono stati assegnati i permessi
di esecuzione, la
\href{http://en.wikipedia.org/wiki/Shebang_%28Unix%29}{Sha-Bang}. In questo modo
basterà semplicemente scriverne il nome nel terminale, e in questo caso è
conveniente omettere l'estensione. Ecco il programma con la Sha-Bang:
\lines
#!/usr/bin/env lua
print("Hello world!")
\endlines
\sourcecode{from_lines()}

Lua è in grado di passare al programma gli argomenti che l'utente digita nel
terminale separandoli da spazi, tramite la tabella array chiamata \key{arg} che
conterrà all'indice 0 il nome dello script, e ai successivi indici i vari
argomenti utente come tipo stringa. Questa proprietà assieme alla tecnica della
sha-bang, dà all'utente la possibilità di scrivere programmi in Lua come se
fossero comandi da terminale, almeno per gli ambienti di classe *nix.

\chapter{Preparazione di un interprete Lua}

Poiché l'interprete Lua non interferisce con \LuaTeX{} e occupa una quantità di
memoria davvero piccola, è consigliabile installarlo per poter usare il modo
interattivo REPL e per sperimentare più comodamente.

Per seguire queste chiaccherate andrà benissimo una versione di Lua uguale o
superiore alla 5.1. Tuttavia è preferibile installare la versione che
corrisponde a quella inclusa in \LuaTeX{}, che viene riportata anche nel suo
manuale.

L'interprete Lua è scritto in Ansi~C e per questo è disponibile praticamente
per tutte le piattaforme, eventualmente compilandone i sorgenti.

Per verificare l'installazione è possibile dare il seguente comando da
terminale per stampare la versione dell'interprete:
\begin{Verbatim}
    $ lua -v
    Lua 5.2.3  Copyright (C) 1994-2013 Lua.org, PUC-Rio
\end{Verbatim}



\subsection{Linux}

Per i sistemi Linux fate ricorso al package manager della vostra distribuzione,
per esempio per le distribuzioni Debian e derivate come Ubuntu, è sufficiente
dare il comando:
\begin{Verbatim}
    $ sudo apt-get install lua
\end{Verbatim}



\subsection{Windows}

Per Windows fate riferimento alla pagina
\href{http://luabinaries.sourceforge.net/index.html}{LuaBinaries}. In caso di
problemi potete scaricare un file autoinstallante dal progetto
\href{https://github.com/rjpcomputing/luaforwindows}{Lua for Windows} fermo
però alla versione 5.1.5 di Lua ma completo di alcune librerie utili.



\subsection{Mac OSX}

Per Mac OSX andate alla pagine \href{http://rudix.org/packages/lua.html}{Rudix}
e individuate il package adatto per la versione del vostro Mac.



\section{Come eseguire il codice}

Operativamente, ci sono molti modi per eseguire codice Lua. La modalità usuale
è installarne l'interprete e questo capitolo riporta le procedure per i tre
principali sistemi operativi.

Se si dispone di una recente distribuzione del sistema \TeX{}, per esempio
TeX~Live, uno \emph{script} in Lua può anche essere eseguito da \LuaTeX{}, per
esempio con il comando da terminale:
\begin{Verbatim}
$ luatex --luaonly nomefile
\end{Verbatim}

Nella parte~\ref{partLuaTeX} vedremo come sia possibile inserire il codice Lua
all'interno di un sorgente \texttt{.tex}. Per esempio, poiché sto componendo
questa guida in \LuaLaTeX{} --- che altro non è che \LuaTeX{} con il formato
\LaTeX{} precaricato --- scrivendo il codice:
\begin{Verbatim}
\directlua{tex.print(lua.version)}
\end{Verbatim}
posso dirvi che la versione di Lua inclusa è \directlua{tex.print(lua.version)}.


\subsection{Modalità interattiva}

Aprite un terminale, o console in Windows, e digitate semplicemente il comando
\texttt{lua}. Entreremo nella modalità interattiva dove potremo
digitare istruzioni una alla volta. Per uscire tornando al prompt, digitate la
funzione \texttt{os.exit()} seguita da un bel tasto invio.

In modalità interattiva possiamo comodamente controllare quale tipo ha un
valore, utilizzando la funzione \texttt{type()}. Essa restituisce una stringa
che corrisponde al nome del tipo di un'espressione, stringa che possiamo a sua
volta stampare con la funzione \texttt{print()} in console. Digitiamo
\footnote{Il doppio trattino inserisce un commento di riga.}:
\lines
a = 123
print(type(123))    --> stampa "number"
print(type(a))      --> stampa "number"
print(type("123"))  --> stampa "string"
\endlines
\sourcecode{from_lines()}


% end of file

