% secondo tutorial

\section{La tabella dei pesi}

Dopo la calcolatrice si presenta un altro problema compositivo: una tabella che
riporta per vari diametri, area e peso della barra d'acciaio di lunghezza
unitaria. I diametri variano da 6 a 32 millimetri con passo 2.

L'idea è definire una sorta di iteratore a due componenti. Per esempio, se
volessimo una tabella con due colonne, la prima con gli interi da 1 a 10 e la
seconda con i rispettivi quadrati, dovremo poter definire solo la regola di
costruzione delle righe. Un secondo componente si occuperà di applicarla le
volte necessarie.

Il primo componente è una funzione generatrice, input del secondo componente,
che può essere qualsiasi purché sia definita sempre con due argomenti: il primo
è il contatore di riga e il secondo è l'array di riga. Nel caso d'esempio si
dovrà memorizzare nell'array in posizione 1 il contatore stesso e in posizione 2
il suo quadrato:
\begin{lines}
local function row_func(counter, row)
    row[1] = counter
    row[2] = counter^2
end
\end{lines}

A ben vedere potremo fare a meno del secondo parametro \code{row} se
restituissimo direttamente un nuovo array di riga, tuttavia in questo modo il
codice risulta più efficiente.

L'idea iniziale si concretizza nell'attribuire alla funzione che calcola la
generica riga il ruolo di \emph{regola di definizione} dell'intera tabella.
Il ruolo del secondo componente è assunto dalla classe \key{Row}. Essa ha il
compito di applicare la regola qualsiasi essa sia, riga dopo riga, senza
conoscerla. Tra il primo e il secondo componente vi è quindi una netta
separazione e questo è indice di buona progettazione.

\directlua{
Row = {}
Row.__index = Row
function Row:new(fn_next, start, stop, step)
    if not stop then
        start, stop = 1, start
    end
    local o = {
        fn_next = fn_next,
        start = start,
        stop = stop,
        step = step or 1
    }
    setmetatable(o, self)
    return o
end

function Row:next()
    local var = self.var
    if not var then
        var = self.start
    else
        var = var + self.step
    end
    if var <= self.stop then
        self.var = var
        local fn = self.fn_next
        fn(var, self)
        return true
    end
end
}

\begin{tcbfloat}[sidebyside,righthand width=21mm,label=lsttwocols]{%
    Listato parziale per la tabella a due colonne}
\begin{lines}
#[tex]
#[tcolorbox]
#[indexfile=app-tut/E0-003-tab.tex]
% !TeX program = LuaLaTeX
% filename: app-tut/E0-003-tab.tex
\documentclass{article}
% preambolo non riportato
\begin{document}
\begin{tabular}{rr}\directlua{
local row = Row:new(
    function (c, r) r[1]=c; r[2]=c^2 end, 10
)
local par = string.char(92)..string.char(92)
while row:next() do
    tex.print(row[1].."&"..row[2]..par)
end
}\end{tabular}
\end{document}
\end{lines}
\tcblower
\begin{tabular}{rr}
\directlua{
local row = Row:new(function (c, r) r[1]=c; r[2] = c^2 end, 10)
local par = string.char(92)..string.char(92)
while row:next() do
    tex.print(row[1]..[[&]]..row[2]..par)
end
}
\end{tabular}
\end{tcbfloat}

Esaminiamo la costruzione della tabella d'esempio in \LuaLaTeX{} nel
listato~\ref{lsttwocols}. Il metodo \fn{new} della classe \code{Row} accetta
proprio una funzione come primo argomento e il valore totale delle righe come
secondo argomento. In Lua le funzioni sono valori come tutti gli altri.

All'interno dell'ambiente \amb{tabular} c'è solo codice Lua: costruito l'oggetto
\key{row} un ciclo \key{while} esegue l'iterazione con il metodo \fn{next}. Come
si può verificare dall'implementazione con il paradigma a oggetti che segue, è
proprio \fn{next} a chiamare a ogni passo la funzione di generazione:

\CLRmarginpar{%
Metametodo \key{\_\_index}\\
\gotosec{} \ref{secFondMetaIndex}}[true]
%
\CLRmarginpar{%
Costruttore\\
\gotosec{} \ref{secFondCostruttore}}
%
\begin{lines}
Row = {}
Row.__index = Row
-- costruttore
function Row:new(fn_next, start, stop, step)
    if not stop then
        start, stop = 1, start
    end
    local o = {
        fn_next = fn_next,
        start = start,
        stop = stop,
        step = step or 1
    }
    setmetatable(o, self)
    return o
end
-- iteratore
function Row:next()
    local var = self.var
    if not var then
        var = self.start
    else
        var = var + self.step
    end
    if var <= self.stop then
        self.var = var
        local fn = self.fn_next
        fn(var, self)
        return true
    end
end
\end{lines}

Siamo in fase iniziale e questo giustifica l'assenza di controlli sui dati di
input e la gestione degli errori, parte essenziale di ogni programma. Ci si
potrà preoccupare in seguito in una fase di consolidamento, di errori e altri
dettagli.

Per esempio, stiamo trascurando le conseguenze possibili del fatto che il codice
Lua è all'interno del sorgente \TeX{} e che per questo ci potrebbero essere
effetti non desiderati dovuti all'espansione dell'argomento della primitiva
\cs{directlua}. Un esempio? Riceveremmo un errore con il blocco della
compilazione se nel preambolo caricassimo il pacchetto \pack{polyglossia} con
l'opzione \opz{babelshorthands} per la lingua italiana.

Nel codice abbiamo infatti usato il doppio apice come delimitatore dei valori
letterali delle stringhe, carattere che diviene attivo con quelle impostazioni.
Per fortuna ci sono altri modi più sicuri di delimitare i valori letterali delle
stringhe in Lua in questi casi, proprie del linguaggio.


\subsection{Dati tabellari}

La funzione generatrice è generale potendo costruire qualsiasi tipo di vista
tabellare sui dati, sia essi ricavati con il calcolo come nel caso precedente
oppure già definiti in memoria, per esempio come una lista di nomi di file con
relativa dimensione in byte:
\begin{lines}
local data = {
    {"files.txt",  4710},
    {"lib.lua"  ,   330},
    {"parse.lua", 50995},
    {"path.txt" ,  2150},
}
\end{lines}

Come dovremo definire il primo componente, la funzione \fn{row\_func}, per
costruire la tabella a due colonne nome e dimensione file? La risposta è nel
listato~\ref{lstTabData}. Nel codice incontriamo una ridondanza perché dobbiamo
specificare il numero di righe nel costruttore \fn{new} quando questo dato è
invece derivabile dai dati. 

\begin{tcbfloat}[sidebyside,righthand width=32mm,label=lstTabData]{%
    Generazione della tabella da un set di dati}
\begin{lines}
#[tex]
#[tcolorbox]
#[indexfile=app-tut/E0-004-tab.tex]
% !TeX program = LuaLaTeX
% filename: app-tut/E0-004-tab.tex
\documentclass{article}
% preambolo non riportato
\begin{document}
\begin{tabular}{lr}\directlua{
local data = {
    {"files.txt",  4710},
    {"lib.lua"  ,   330},
    {"parse.lua", 50995},
    {"path.txt" ,  2150},
}
local function row_func(counter, row)
    row[1] = data[counter][1]
    row[2] = data[counter][2]
end
local row = Row:new(row_func, 4)
local p = string.char(92); p = p..p
while row:next() do
    tex.print(row[1].."&"..row[2]..p)
end
}\end{tabular}
\end{document}
\end{lines}
\tcblower
\begin{tabular}{lr}
\directlua{
local data = {
  {'files.txt'    ,  4710},
  {'lib.lua'      ,   330},
  {'parse.lua'    , 50995},
  {'path.txt'     ,  2150},
}
local function row_func(c, row)
    row[1] = data[c][1]
    row[2] = data[c][2]
end
local row = Row:new(row_func, 4)
local p = string.char(92); p = p..p
while row:next() do
tex.print(row[1]..'&'..row[2]..p)
end
}
\end{tabular}
\end{tcbfloat}

In Lua è normale poter chiamare una funzione variando il tipo dei suoi argomenti
Se passassimo al metodo \fn{new} direttamente la tabella dati anziché la
funzione generatrice, il codice sarebbe in grado di rilevare la differenza e
costruire sia la funzione \fn{row\_func} che il numero totale di righe.

In effetti nel codice del file \file{E0-004-tab.tex} si trova un'implementazione
che fa proprio questo. Questa forma di polimorfismo interno del costruttore può
essere convertita in una forma esterna a beneficio dell'interfaccia che esprime
in modo diretto all'utente l'adattibilità della classe, nonché della chiarezza
del codice, implementando più di un costruttore, uno per ciscun tipo di dato in
ingresso con cui è possibile costruire la classe.

Per esempio già i nomi dei costruttori \fn{from\_func} o \fn{from\_file}
definiscono esplicitamente il tipo di sorgente dati da cui costruire l'oggetto.
Aggiungere un nuovo tipo di sorgente per i dati in forma tabellare significa
aggiungere un nuovo costruttore. Il codice dei costruttori già implementati non
ne risente mentre invece se il polimorfismo fosse interno, non potremo evitare
di modificare l'unico costruttore.


\subsection{Template di riga}

Proseguiamo adesso con il migliorare il modo in cui generare il codice di riga
nell'ambiente \key{tabular}. Stiamo infatti usando la concatenazione di
stringhe, un modo non molto efficiente e nemmeno comodo. Potrebbe essere più
conveniente specificare una sorta di template con segnaposto come la stringa
seguente per un'ipotetica tabella a due colonne:
\begin{lines}
template = [[\textbf{<1>} & <2>\\]]
\end{lines}

Il numero tra parentesi acute \code{<>} indica l'indice di riga corrispondente a
quello definito dalla funzione \fn{row\_func}.

Per implementare il template di riga, basta aggiungere alla classe \code{Row} un
iteratore che a ogni passo generi la stringa di codice della riga di tabella:

\CLRmarginpar{%
\fn{string.gsub}\\
\gotosec{} \ref{secFondGsub}}[true]
%
\CLRmarginpar{%
Pattern\\
\gotosec{} \ref{secFondPattern}}
%
\CLRmarginpar{%
Capture\\
\gotosec{} \ref{secFondCapture}}
%
\CLRmarginpar{%
Stateless iterator\\
\gotosec{} \ref{secFondStatelessIter}}
%
\begin{lines}
function Row:iter_template(tmpl)
    local iter_fn = function(row, i)
        if not i then
            i = row.start
        else
            i = i + row.step
        end
        if i <= self.stop then
            self.fn_next(i, self)
            local s = tmpl:gsub("<(%d+)>", function (s)
                local n = tonumber(s)
                return row[n]
            end)
            return i, s
        end
    end
    return iter_fn, self, nil
end
\end{lines}

Al di la di considerazioni di efficienza legate all'uso della funzione di
libreria \fn{gsub}, l'iteratore in effetti funziona come dimostra il seguente
codice per \LuaLaTeX{} estratto dal file \file{app-tut/E0-005-tab.tex} parte
della guida. Abbiamo inserito la macro \cs{noexpand} per bloccare l'espansione
delle control sequence\footnote{Certo mi ostino ancora a non utilizzare il
pacchetto \pack{luacode}.}:
\begin{lines}
#[tex]
\begin{tabular}{lr}
\directlua{
local tmpl = [[\noexpand\textbf{<1>} & <2>\noexpand\\]]
for _, s in row:iter_template(tmpl) do
   tex.print(s)
end
}
\end{tabular}
\end{lines}

Torniamo alla nostra tabella dei pesi. La funzione generatrice e il template di
riga saranno le seguenti:
\begin{lines}
local function row_func(diam, row)
    row[1] = diam
    local area = math.pi * (diam/20)^2
    local fmt = string.char(37)..'0.3f'
    row[2] = fmt:format(area)
    row[3] = fmt:format(0.785*area)
end
row = Row:new(row_func, 6, 32, 2)
tmpl = [[\noexpand\textbf{<1>} & <2> & <3>\noexpand\\]]
\end{lines}

\directlua{
function Row:iter_template(tmpl)
    local iter_fn = function(row, i)
        if not i then
            i = row.start
            row.counter = 0
        else
            i = i + row.step
        end
        if i <= self.stop then
            row.counter = row.counter + 1
            self.fn_next(i, self)
            local perc = string.char(37)
            local s = tmpl:gsub('<('..perc..'d+)>', function (s)
                local n = tonumber(s)
                return assert(row[n])
            end)
            return i, s
        end
    end
    return iter_fn, self, nil
end

local function row_func(diam, row)
    row[1] = diam
    local area = math.pi * (diam/20)^2
    local fmt = string.char(37)..'0.3f'
    row[2] = fmt:format(area)
    row[3] = fmt:format(0.785*area)
end
row = Row:new(row_func, 6, 32, 2)
tmpl = [[\noexpand\textbf{<1>} & <2> & <3>\noexpand\\]]
}
e il risultato è:
\begin{center}
\small
\begin{tabular}{lrr}
\directlua{
for _, s in row:iter_template(tmpl) do
   tex.print(s)
end
}
\end{tabular}
\end{center}

Miglioriamo ora il codice della funzione generatrice aggiungendo il metodo
\fn{insert} alla classe \code{Row}, a tre argomenti: il numero di colonna
\key{col}, il valore da inserire nella cella \key{val} e infine il valore
opzionale di arrotondamento numerico \key{prec}. Eccone una sua implementazione
molto semplice:
\begin{lines}
function Row:insert(col, val, prec)
    if prec then
        local p = string.char(37)
        local fmt = string.format(p..p.."0."..p.."df", prec)
        val = string.format(fmt, val)
    end
    self[col] = val
    return self
end
\end{lines}

Il nuovo metodo restituisce l'oggetto stesso così che possiamo concatenare più
inserimenti di cella. Ecco come la funzione di generazione può semplificarsi:
\begin{lines}
local function row_func(diam, row)
    local area = math.pi * (diam/20)^2
    row:insert(1, diam)
       :insert(2, area, 3)
       :insert(3, 0.785*area, 3)
end
\end{lines}

Sono scomparse le acrobazie per il formato numerico a favore della compattezza.
Un ulteriore miglioramento ci consente di evitare di dover controllare
l'espansione quando inseriamo il testo del template di riga grazie al comando
\cs{detokenize}.

Introduciamo in proposito una nuova macro \cs{printrow} che ha come argomento il
template che rappresenta il modello della generica riga della tabella:
\begin{lines}
#[tex]
\newcommand{\printrow}[1]{\directlua{
local tmpl = [=[\detokenize{#1}]=]
for _, s in row:iter_template(tmpl) do
   tex.print(s)
end
}}
\end{lines}

Per non introdurre un secondo argomento, nell'istanziare l'oggetto della classe
\code{Row} dovremo solo ricordarci di chiamare la variabile \key{row}, lo stesso
nome usato nella definizione di \cs{printrow}. Mettiamo subito al lavoro la
nuova macro:
\begin{lines}
#[tex]
\begin{tabular}{lrr}
\printrow{\textbf{<1>} & <2> & <3>\\}
\end{tabular}
\end{lines}

Molto semplice: si definisce prima la funzione generatrice e con essa si
costruisce l'oggetto \key{Row}, poi si scrive il codice \LaTeX{} dando alla
macro \cs{printrow} il template con i segnaposto.

Molto importante è far corrispondere i numeri di cella dei segnaposto del
template con i valori che la funzione di riga inserisce nella varie posizioni.


\subsection{Composizione finale}

L'ultimo passo è migliorare l'aspetto della tabella. Con il pacchetto
\pack{booktabs} aggiungiamo un'intestazione e un filetto ogni tre righe per
facilitare la lettura dei dati. Dobbiamo così modificare la funzione di riga per
derminare se il numero di riga è multiplo di tre --- senza usare l'operatore
modulo \key{\%} di Lua perché non ci troviamo in un file esterno:
\begin{lines}
local function fn(diam, row)
    local area = math.pi * (diam/20)^2
    local peso = 0.785*area
    local c = row.counter
    local midrule = ""
    if c - 3*math.floor(c/3) == 0
    and not (diam == row.stop) then
        midrule = string.char(92).."midrule"
    end
    row:insert(1, diam)
       :insert(2, area, 3)
       :insert(3, peso, 6)
       :insert(0, midrule)
end
\end{lines}

Introduciamo anche il pacchetto \pack{siunitx} \cite{pkg:siunitx} utilissimo per
comporre numeri, unità di misura e tabelle, con questo ambiente \amb{tabular}
ridisegnato:
\begin{lines}
#[tex]
\begin{tabular}{
    c
    S[table-format=4.3]
    S[table-format=1.3]
    S[table-format=1.6]
}
\toprule
\diameter & {Sviluppo} & {Sezione} & {Peso}\\
\small\si{mm} & {\small\si{cm^2/m}} & {\small\si{cm^2}} &
    {\small\si{daN/m}}\\
\midrule
\printrow{\(\mathbf{<1>}\) & <4> & <2> & <3>\\<0>}
\bottomrule
\end{tabular}
\end{lines}

Il progetto completato si trova nel file \file{app-tut/E0-006-tab.tex}, dove ho
aggiunto alla tabella la colonna con il calcolo della superficie laterale delle
barre.

\directlua{
function Row:insert(col, val, prec)
    if prec then
        local p = string.char(37)
        local fmt = string.format(p..p..'0.'..p..'df', prec)
        val = string.format(fmt, val)
    end
    self[col] = val
    return self
end

local function fn(diam, row)
    local area = math.pi * (diam/20)^2
    local peso = 0.785*area
    local sup_lat = 10 * math.pi * diam
    local c = row.counter
    local midrule = ''
    if c - 3*math.floor(c/3) == 0 then
        midrule = string.char(92)..'midrule'
    end
    row:insert(1, diam)
       :insert(2, area, 3)
       :insert(3, peso, 6)
       :insert(4, sup_lat, 3)
       :insert(0, midrule)
end
row = Row:new(fn, 6, 32, 2)
}

\newcommand{\printrow}[1]{\directlua{
local tmpl = [=[\detokenize{#1}]=]
for _, s in row:iter_template(tmpl) do
   tex.print(s)
end
}}

\begin{table}[bp]
\caption{La tabella dei pesi generata automaticamente definendone la regola di
generazione e il template di riga.}
\label{tabTabPesi}
\centering
\small
\begin{tabular}{
    c
    S[table-format=4.3]
    S[table-format=1.3]
    S[table-format=1.6]
}
\toprule
\diameter & {Sviluppo} & {Sezione} & {Peso}\\
\small\si{mm} & {\small\si{cm^2/m}} & {\small\si{cm^2}} & {\small\si{daN/m}}\\
\midrule
\printrow{\(\mathbf{<1>}\) & <4> & <2> & <3>\\<0>}
\bottomrule
\end{tabular}
\end{table}

\subsection{Conclusioni}

La nostra classe \code{Row} ci permette di costruire tabelle iterative in Lua in
modo del tutto generale, compiendo calcoli numerici e ogni sorta di possibili
elaborazioni. Molti altri affinamenti sono possibili come il caricamento di dati
esterni oppure l'uso di pipeline di operatori all'interno dei segnaposto dei
template. Anche questo tutorial si chiude perciò con una lista di nuove idee da
implementare con Lua.

La tabella~\ref{tabTabPesi} mostra il risultato finale.

% end of file
