

\chapter{La tabella}
\label{iChTabella}

In questo capitolo parleremo della \emph{tabella}, l'unico tipo strutturato
predefinito di Lua. Diamone subito la definizione: la tabella è un
\emph{dizionario} cioè l'insieme non ordinato di coppie chiavi/valore e, allo
stesso tempo, anche un \emph{array} cioè una sequenza ordinata di valori.

In Lua ne è previsto quindi un uso molteplice: se le chiavi sono numeri interi
la tabella sarà un array, se le chiavi sono di altro tipo, per esempio stringhe,
avremo un dizionario.

Le chiavi possono essere di tutti i tipi previsti da Lua tranne che \key{nil},
mentre i valori possono appartenere a qualsiasi tipo. Nulla vieta che in una
stessa tabella coesistano chiavi di tipo diverso.

Dal punto di vista sintattico, una tabella di Lua è un oggetto racchiuso tra
parentesi graffe e, la più semplice quella vuota, si crea così:
\lines
local t = {} -- una tabella vuota
\endlines
\sourcecode{from_lines()}

Per assegnare e ottenere il valore associato a una chiave si utilizzano le
parentesi quadre, l'operatore di indicizzazione, ecco un esempio:
\lines
local t = {}
t["key"] = "val"
print(t["key"]) --> stampa "val"
\endlines
\sourcecode{from_lines()}

Stando alla definizione che abbiamo dato, una tabella può avere chiavi anche di
tipo differente, e infatti è proprio così e ciò vale anche per i valori. In
questo esempio una tabella ha chiavi di tipo numerico e di tipo stringa con
valori a sua volta di tipo diverso:
\lines
local t = {}
t["key"] = 123
t[123] = "key"
print(t["key"]) --> stampa il tipo numerico 123
print(t[123])   --> stampa il tipo stringa "key"
\endlines
\sourcecode{from_lines()}


\section{La tabella è un oggetto}

Cosa significa che la tabella di Lua è un oggetto? Vuol dire che la tabella è
un dato in memoria gestito con un riferimento. In conseguenza, se si copia una
variabile a una tabella in una seconda variabile questa farà riferimento ancora
alla stessa tabella e non una sua copia:
\lines
local t = {}
t[1], t[2] = 10, 20
-- copia la tabella o il riferimento?
local other = t
t[1] = t[1] + t[2] -- modifichiamo t
-- l'altra variabile riflette la modifica?
assert(t[1] == other[1])
\endlines
\sourcecode{from_lines()}

Con la funzione \fn{assert} si può esprimere l'equivalenza logica tra due
espressioni. Essa ritorna l'argomento se questo è \key{true} oppure se non è
\key{nil}, altrimenti termina l'esecuzione del programma riportando l'errore
descritto eventualmente da un secondo argomento testuale.

Il fatto che la tabella è un oggetto è la premessa fondamentale per la
programmazione a oggetti in Lua e per scrivere codice più compatto nelle
elaborazioni su tabelle dalla struttura molto complessa.

In effetti possiamo annidare in una tabella ulteriori tabelle assegnandole come
valore a corrispondenti chiavi, con complessità arbitraria. In altri termini
una tabella può rappresentare una struttura ad albero senza limiti teorici.
Poiché essa è gestita attraverso un riferimento nell'albero vi saranno solamente
i corrispondenti riferimenti mentre il dato effettivo sarà presente in qualche
altra parte della memoria.


\section{Il costruttore e la dot notation}

Dunque la tabella è un tipo di dato molto flessibile, è un oggetto, ed è
sufficientemente efficiente. Può essere usata in moltissime diverse situazioni
ed è ancora più utile grazie all'efficacia del suo \emph{costruttore}.

Ispirato al formato di dati bibliografici di BibTeX, uno dei programmi storici
del sistema \TeX{} usato per la gestione delle bibliografie nei documenti
\LaTeX, il costruttore di Lua può creare tabelle da una sequenza di
chiavi/valori inserite tra parentesi graffe:
\lines
local t = { a = 123, b = 456, c = "valore"}
\endlines
\sourcecode{from_lines()}

La chiave appare come il nome di una variabile ma in realtà nel costruttore
essa viene interpretata come una chiave di tipo stringa. Così l'esempio
precedente è equivalente al seguente codice:
\lines
-- codice equivalente
local t = {}
t["a"] = 123
t["b"] = 456
t["c"] = "valore"
\endlines
\sourcecode{from_lines()}

La notazione del costruttore non ammette l'utilizzo diretto di chiavi
numeriche. Se occorrono è necessario utilizzare le parentesi quadre per
racchiudere il numero che fa da indice:
\lines
-- chiavi numeriche nel costruttore?
local t_error = { 20 = 123 }
local t_ok = { [20] = 123 }
\endlines
\sourcecode{from_lines()}

Invece, se nel costruttore omettiamo le chiavi, otteniamo una tabella array con
indici interi impliciti in sequenza a partire da 1, contrariamente alla maggior
parte dei linguaggi dove l'indice comincia da 0. Ecco un esempio:
\lines
local t = { 30, 8, 500 }
print(t[1] + t[2] + t[3]) --> stampa 538
\endlines
\sourcecode{from_lines()}

Non è tutto. L'efficacia sintattica del costruttore è completata dalla
\emph{dot notation}, valida solamente per le chiavi di tipo stringa: il campo di
una chiave di tipo stringa si indicizza scrivendone la chiave dopo il nome del
riferimento della tabella, separato dal carattere \key{.}:
\lines
local t = { chiave = "123" }
assert(t.chiave == t["chiave"])
\endlines
\sourcecode{from_lines()}

Prestate attenzione perché all'inizio si può male interpretare il risultato del
costruttore della tabella se unito alla dot notation:
\lines
local chiave = "ok"
local t = { ok = "123"} -- t.ok == t[chiave]

-- attenzione!
local k = "ok"
print( t.k ) --> stampa nil: "k" non è definita in t
print( t[k]) --> stampa "123"
-- t[k] == t["ok"] == t.ok
-- t.k diverso da t[k] !!!
\endlines
\sourcecode{from_lines()}

Non confondete il nome di variabile con il nome del campo in dot notation!

Riassumendo, indicizzare una tabella con una variabile restituisce il valore
associato alla chiave uguale al valore della variabile, mentre indicizzare in
dot notation con il nome uguale a quello della variabile restituisce il valore
associato alla chiave corrispondente alla stringa del nome.


\section{Esercizi}

\begin{Exercise}[label={tab-01}]
Scrivere il codice Lua che memorizzi in una tabella i primi 10 numeri primi
usandone il costruttore.
\end{Exercise}

\begin{Exercise}[label={tab-02}]
Utilizzando la dot notation è possibile utilizzare caratteri spazio nel nome
della chiave delle tabelle?
\end{Exercise}

\begin{Exercise}[label={tab-03}]
Scrivere il codice Lua che stampi il valore associato alle chiavi \key{paese} e
\key{codice}, e il numero medio di comuni per regione, per la seguente tabella.
Stampare inoltre il numero di abitanti della capitale.
\lines
local t = {
    paese = "Italia",
    lingua = "italiano",
    codice = "IT",
    regioni = 20,
    province = 110,
    comuni = 8047,
    capitale = {"Roma", "RM", abitanti = 2753000},
}
\endlines
\sourcecode{from_lines()}
\end{Exercise}


% end of file
