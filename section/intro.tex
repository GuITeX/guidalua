
% presentazione della guida e informazioni di base per eseguire gli esempi

\chapter{Introduzione}

\section{Presentazione della guida}

Questa guida tematica è dedicata alla programmazione in Lua all'interno dei
motori di composizione del sistema \TeX. Quel che occorre conoscere è l'ambito
in cui ci troviamo per poter avvalerci di funzionalità ben più semplici da
codificare che non con gli strumenti tradizionali messi a disposizione dal solo
\TeX{} o dalle macro del formato.

Elaborare dati, effettuare calcoli numerici, eseguire compiti in pratica
impossibili con il compositore tradizionale come connettersi a database
relazionali o avvalersi di avanzati sistemi esterni, sono funzionalità
possibili con la nuova generazione di compositori, in scenari applicativi
notevolmente ampliati.

Se da un lato è auspicabile che queste potenzialità diventino disponibili per
gli utenti finali per mezzo di moduli e pacchetti tali da minimizzare la
necessità di programmare, dall'altro è utile fornire dettagli ed esempi per
implementare proprie soluzioni e collaborare condividendo idee di sviluppo.

Sono certamente molte le cose da conoscere: un nuovo linguaggio molto diverso da
\TeX{}, numerosi dettagli sul funzionamento interno dei compositori Lua-powered,
nuovi problemi di organizzazione del codice. Per questo, ho pensato di
contribuire con questa guida cercando di presentare il quadro della crescente
complessità del sistema.

Tra gli argomenti della guida ci sono:
\begin{compactitemize}
\item basi del linguaggio Lua, dal capitolo \ref{chAssignment},
\item la differenza tra motore e formato di composizione, capitolo \ref{iichExplain},
\item tecniche di programmazione e di rappresentazione dei dati,
\item l'interazione con lo stato interno del motore di composizione.
\end{compactitemize}

A mio parere se si vuole puntare sullo sviluppo razionale di progetti
documentali non bisogna mai dimenticare \TeX{} a favore di Lua, anzi occorre
conoscerlo più a fondo per riuscire ad equilibrarne le componenti in azione per
costruire documenti di qualità.

Auguro perciò ai lettori Happy LuaTeXing!


\section{Lua, proprio un bel nome}

Lua è un linguaggio semplice ma non banale. Il suo ambito di applicazione è
quello dei linguaggi di scripting: text processing, manutenzione del sistema,
elaborazioni su file dati, eccetera e lo si può anche trovare come linguaggio
embedded di programmi complessi come i videogiochi o altri applicativi
che danno la possibilità di essere programmati con esso dall'utente.

Lua è stato ideato da un gruppo di programmatori esperti
dell'\href{http://www.puc-rio.br/index.html}{Università Cattolica di Rio de
Janeiro} in Brasile. ``Lua'' si pronuncia LOO-ah e significa ``Luna'' in
portoghese!


\section{Piano della guida}

La guida è divisa in due parti: la prima tratta delle basi del linguaggio Lua e
la seconda le applica in esempi concreti con l'uso delle librerie interne del
compositore.

La risorsa principale per imparare Lua a cui si rimanda per tutti gli
approfondimenti è certamente il PIL acronimo del titolo del libro
\emph{Programming In Lua} di Roberto Ierusalimschy, principale Autore
di Lua. Questo testo non solo è completo e autorevole ma è anche ben scritto e
composto\footnote{Tra l'altro il libro ufficiale su Lua viene composto in
\LaTeX{} e commercializzato per contribuire allo sviluppo del linguaggio
stesso.}.

Quanto a \LuaTeX{} il riferimento è il suo manuale che, come quasi tutta la
documentazione nel sistema \TeX{}, può essere visualizzato a video con il
comando da terminale:
\begin{Verbatim}
$ texdoc luatex
\end{Verbatim}


\section{Contribuire alla guida}

Spero che i lettori vorranno contribuire al testo inviando le propri soluzioni o
nuovi contributi anche piccoli. Lo si può fare attraverso lo strumento che
preferite, scrivendomi un messaggio email, oppure utilizzando il repository
\texttt{git} dei sorgenti.


\section{Origine della guida}

Per illustrare i concetti del linguaggio ho preso spunto da un breve corso su
Lua che scrissi qualche tempo fa per il blog
\href{http://parliamodi-ubuntu.blogspot.it}{Lubit Linux} di Luigi Iannoccaro
che mi propose di realizzare un progetto di divulgazione su Lua. Luigi ha
acconsentito all'utilizzo di quegli appunti per produrre questa guida tematica.


\section{Come eseguire gli esercizi}

Per iniziare è certamente molto utile eseguire noi stessi esempi ed esercizi di
programmazione. Nella guida ne trovate alcuni alla fine di ciascun capitolo
della parte prima. 

Questa sezione vi introduce brevemente al programma \prog{texlua} che già
trovate compreso in ogni recente distribuzione \TeX{}. Si tratta dell'interprete
Lua controparte di \texttt{luatex} nell'esecuzione del codice Lua come
suggerisce il nome.

Rispetto all'interprete \prog{lua} standard esso non comprende la modalità
interativa detta REPL\footnote{Read–eval–print loop.} con cui si digita una
linea di codice alla volta senza dover creare un file per fare semplici prove. 

Il codice andrà memorizzato in un file con estensione \texttt{.lua}, in questo
modo: in un file \texttt{primo.lua} digitiamo questa unica riga di codice:
\begin{Verbatim}
print("Hello World!")
\end{Verbatim}
apriamo una finestra di terminale\footnote{Maggiori dettagli per diversi sistemi
operativi sulla linea di comando possono essere trovati nella guida tematica
dedicata scaricabile dal sito GuIT.} e lanciamo il comando:
\begin{Verbatim}
$ texlua primo.lua
\end{Verbatim}

Ora che sappiamo come eseguire codice Lua, concentriamoci con i prossimi
capitoli sulle basi del linguaggio. Torneremo nella seconda parte della guida su
ulteriori modalità di esecuzione e a conoscere importanti dettagli
sull'esecuzione di Lua all'interno dei motori di composizione.


% end of file
