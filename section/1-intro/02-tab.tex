% secondo tutorial

\section{Tabella dei pesi}

Dopo la calcolatrice si presenta un altro problema compositivo: una tabella che
riporta per vari diametri, area e peso della barra d'acciaio di lunghezza
unitaria. I diametri variano da 6 a 32 millimetri con passo 2.

L'idea è definire una sorta di iteratore a due componenti. Per esempio, se
volessimo una tabella con due colonne, la prima con gli interi da 1 a 10 e la
seconda con i rispettivi quadrati, dovremo definire solo la funzione di calcolo
e il numero delle righe totali, perché un secondo componente si occuperà di
applicarla le volte necessarie.

Il primo componente variabile della funzione generatrice, può essere qualsiasi
purché sia definita per due argomenti: il primo il contatore di riga e il
secondo l'array di riga. Nel caso d'esempio si dovrà memorizzare nell'array il
contatore in posizione 1 e il quadrato in posizione 2:
\begin{Verbatim}
local function row_func(counter, row)
    row[1] = counter
    row[2] = counter^2
end
\end{Verbatim}

L'idea iniziale è quindi realizzata se attribuiamo alla funzione che calcola la
generica riga il concreto ruolo di \emph{regola di definizione} dell'intera
tabella. A ben vedere potremo fare a meno del secondo parametro \code{row} se
restituissimo direttamente un nuovo array di riga, tuttavia in questo modo il
codice risulta più efficiente.

\directlua{
Row = {}
Row.__index = Row
function Row:new(fn_next, start, stop, step)
    if not stop then
        start, stop = 1, start
    end
    local o = {
        fn_next = fn_next,
        start = start,
        stop = stop,
        step = step or 1
    }
    setmetatable(o, self)
    return o
end

function Row:next()
    local var = self.var
    if not var then
        var = self.start
    else
        var = var + self.step
    end
    if var <= self.stop then
        self.var = var
        local fn = self.fn_next
        fn(var, self)
        return true
    end
end
}

Il secondo componente costante, la classe di libreria \key{Row}, ha il compito
di applicare la regola ad ogni riga della tabella, qualsiasi essa sia. Prima di
passare alla sua implementazione esaminiamo la costruzione della tabella
d'esempio in \LuaLaTeX{}. Il metodo \fn{new} accetta proprio una funzione come
primo argomento e il valore totale di righe come secondo argomento.

In Lua le funzioni sono valori come tutti gli altri. L'esempio minimo
compilabile è il seguente:
\begin{tcolorbox}[sidebyside,righthand width=21mm]
\begin{Verbatim}[numbers=none,xleftmargin=0pt]
% !TeX program = LuaLaTeX
% filename: app-start/E0-003-tab.tex
\documentclass{article}
% preambolo non riportato
\begin{document}
\begin{tabular}{rr}\directlua{
local row = Row:new(
    function (c, r) r[1]=c; r[2]=c^2 end, 10
)
local par = string.char(92)..string.char(92)
while row:next() do
    tex.print(row[1].."&"..row[2]..par)
end
}\end{tabular}
\end{document}
\end{Verbatim}
\tcblower
\begin{tabular}{rr}
\directlua{
local row = Row:new(function (c, r) r[1]=c; r[2] = c^2 end, 10)
local par = string.char(92)..string.char(92)
while row:next() do
    tex.print(row[1]..[[&]]..row[2]..par)
end
}
\end{tabular}
\end{tcolorbox}

All'interno dell'ambiente \amb{tabular} c'è solo codice Lua: costruito l'oggetto
\key{row} un ciclo \key{while} esegue l'iterazione con il metodo \fn{next}. Come
si può verificare dall'implementazione con il paradigma a oggetti che segue, è
\fn{next} a chiamare a ogni passo la funzione di generazione di riga:
%
\tcbdocmarginnote[enlarge top initially by=\margindown]{%
Metametodo \key{\_\_index}\\
\gotosec{} \ref{secFondMetaIndex}}
%
\tcbdocmarginnote[enlarge top initially by=\dimexpr\margindown+12mm\relax]{%
Costruttore\\
\gotosec{} \ref{secFondCostruttore}}
%
\begin{Verbatim}
Row = {}
Row.__index = Row
-- costruttore
function Row:new(fn_next, start, stop, step)
    if not stop then
        start, stop = 1, start
    end
    local o = {
        fn_next = fn_next,
        start = start,
        stop = stop,
        step = step or 1
    }
    setmetatable(o, self)
    return o
end
-- iteratore
function Row:next()
    local var = self.var
    if not var then
        var = self.start
    else
        var = var + self.step
    end
    if var <= self.stop then
        self.var = var
        local fn = self.fn_next
        fn(var, self)
        return true
    end
end
\end{Verbatim}

Siamo in fase iniziale e questo giustifica l'assenza di controlli sui dati di
input e la conseguente gestione degli errori, parte essenziale di ogni
programma. Ci si potrà preoccupare in seguito in una fase di consolidamento di
errori e altri dettagli.

Per esempio, stiamo trascurando le conseguenze possibili del fatto che il codice
Lua è all'interno del sorgente \TeX{} e che per questo ci potrebbero essere
problemi dovuti all'espansione dell'argomento della primitiva \cs{directlua}. Un
esempio? Riceveremmo un errore con il blocco della compilazione se usassimo i
delimitatori delle stringhe doppi apici se nel preambolo si caricasse il
pacchetto \pack{polyglossia} con l'opzione \opz{babelshorthands} per la lingua
italiana, che rende attivo proprio il doppio apice. Per fortuna ci sono altri
modi più sicuri di delimitare i valori letterali delle stringhe in Lua in questi
casi, proprie del linguaggio.

Ogni tipo di dati potrà essere rappresentato in forma tabellare, anche dati non
calcolati come nomi di file con relativa dimensione in byte, dati come i
seguenti, istanziati dal costruttore di tabelle di Lua che elaborando una
tabella che a sua volta ne contiene altre quattro:
\begin{Verbatim}
local data = {
    {"files.txt",  4710},
    {"lib.lua"  ,   330},
    {"parse.lua", 50995},
    {"path.txt" ,  2150},
}
\end{Verbatim}

Come dovremo definire il componente variabile, ovvero la funzione
\fn{row\_func}, per costruire la tabella a due colonne nome file, e dimensione?
\begin{tcolorbox}[sidebyside,righthand width=32mm]
\begin{Verbatim}[numbers=none,xleftmargin=0pt]
% !TeX program = LuaLaTeX
% filename: app-start/E0-004-tab.tex
\documentclass{article}
% preambolo non riportato
\begin{document}
\begin{tabular}{lr}\directlua{
local data = {
    {"files.txt"    ,  4710},
    {"lib.lua"      ,   330},
    {"parse.lua"    , 50995},
    {"path.txt"     ,  2150},
}
local function row_func(counter, row)
    row[1] = data[counter][1]
    row[2] = data[counter][2]
end
local row = Row:new(row_func, 4)
local p = string.char(92); p = p..p
while row:next() do
    tex.print(row[1].."&"..row[2]..p)
end
}\end{tabular}
\end{document}
\end{Verbatim}
\tcblower
\begin{tabular}{lr}
\directlua{
local data = {
  {'files.txt'    ,  4710},
  {'lib.lua'      ,   330},
  {'parse.lua'    , 50995},
  {'path.txt'     ,  2150},
}
local function row_func(c, row)
    row[1] = data[c][1]
    row[2] = data[c][2]
end
local row = Row:new(row_func, 4)
local p = string.char(92); p = p..p
while row:next() do
tex.print(row[1]..'&'..row[2]..p)
end
}
\end{tabular}
\end{tcolorbox}

In questo esempio incontriamo una ridondanza perché dobbiamo specificare il
numero di righe nel costruttore \fn{new} quando questo dato è invece
derivabile dai dati. 

In Lua è normale chiamare una funzione passando per lo stesso argomento una
variabile di un tipo di dato oppure un'altra variabile di un tipo diverso. Se
passassimo al metodo \fn{new} direttamente la tabella dati anziché la funzione
generatrice, si potrebbe riconoscerne il tipo e costruire di conseguenza sia la
funzione \fn{row\_func} che il numero totale di righe al posto dell'utente. In
effetti nel codice del file \file{E0-004-tab.tex} si trova un'implementazione
che fa proprio questo.

Tuttavia potremo optare per implementare altri costruttori della classe, per
esempio con il nome di \fn{from\_data} o \fn{from\_file}, specifici per un tipo
di sorgente dati, e indubbiamente è la soluzione da preferire. Potremo anche
sperimentare il punto di vista per colonne anziché quello per righe mantenendo
la generalità della classe.

Proseguiamo adesso migliorando il modo in cui generare il codice di riga
nell'ambiente \key{tabular}. Stiamo infatti usando la concatenazione di
stringhe, un modo non molto efficiente e nemmeno comodo. Potrebbe essere più
conveniente specificare una sorta di template con segnaposto come la stringa
seguente per un'ipotetica tabella a due colonne:
\begin{Verbatim}[numbers=none]
template = [[\textbf{<1>} & <2>\\]]
\end{Verbatim}

Il numero tra parentesi acute, i segni di minore e maggiore, indica l'indice di
riga così come definito nelle funzioni \fn{row\_func}.

Per far questo, basta aggiungere alla classe \code{Row} un metodo d'iterazione
che a ogni passo ritorni la stringa risultato, e che segua le specifiche perché
possa essere usato in un ciclo \key{for}:
%
\tcbdocmarginnote[enlarge top initially by=\margindown]{%
\fn{string.gsub}\\
\gotosec{} \ref{secFondGsub}}
%
\tcbdocmarginnote[enlarge top initially by=\dimexpr\margindown+9mm\relax]{%
Pattern\\
\gotosec{} \ref{secFondPattern}}
%
\tcbdocmarginnote[enlarge top initially by=\dimexpr\margindown+18mm\relax]{%
Capture\\
\gotosec{} \ref{secFondCapture}}
%
\begin{Verbatim}
function Row:iter_template(tmpl)
    local iter_fn = function(row, i)
        if not i then
            i = row.start
        else
            i = i + row.step
        end
        if i <= self.stop then
            self.fn_next(i, self)
            local s = tmpl:gsub("<(%d+)>", function (s)
                local n = tonumber(s)
                return row[n]
            end)
            return i, s
        end
    end
    return iter_fn, self, nil
end
\end{Verbatim}

Al di la di considerazioni di efficienza legate all'uso della funzione di
libreria \fn{gsub}, l'iteratore in effetti funziona come dimostra il seguente
codice per \LuaLaTeX{} estratto dal file \file{app-start/E0-005-tab.tex}
parte della guida, dove abbiamo inserito la macro \cs{noexpand} per bloccare
l'espansione delle control sequence\footnote{Certo mi ostino ancora a non
utilizzare il pacchetto \pack{luacode}.}:
\begin{Verbatim}
\begin{tabular}{lr}
\directlua{
local tmpl = [[\noexpand\textbf{<1>} & <2>\noexpand\\]]
for _, s in row:iter_template(tmpl) do
   tex.print(s)
end
}
\end{tabular}
\end{Verbatim}

Torniamo alla nostra tabella dei pesi. La funzione generatrice e il template di
riga saranno le seguenti:
\begin{Verbatim}
local function row_func(diam, row)
    row[1] = diam
    local area = math.pi * (diam/20)^2
    local fmt = string.char(37)..'0.3f'
    row[2] = fmt:format(area)
    row[3] = fmt:format(0.785*area)
end
row = Row:new(row_func, 6, 32, 2)
tmpl = [[\noexpand\textbf{<1>} & <2> & <3>\noexpand\\]]
\end{Verbatim}

\directlua{
function Row:iter_template(tmpl)
    local iter_fn = function(row, i)
        if not i then
            i = row.start
            row.counter = 0
        else
            i = i + row.step
        end
        if i <= self.stop then
            row.counter = row.counter + 1
            self.fn_next(i, self)
            local perc = string.char(37)
            local s = tmpl:gsub('<('..perc..'d+)>', function (s)
                local n = tonumber(s)
                return assert(row[n])
            end)
            return i, s
        end
    end
    return iter_fn, self, nil
end

local function row_func(diam, row)
    row[1] = diam
    local area = math.pi * (diam/20)^2
    local fmt = string.char(37)..'0.3f'
    row[2] = fmt:format(area)
    row[3] = fmt:format(0.785*area)
end
row = Row:new(row_func, 6, 32, 2)
tmpl = [[\noexpand\textbf{<1>} & <2> & <3>\noexpand\\]]
}
e il risultato è:
\begin{center}
\begin{tabular}{lrr}
\directlua{
for _, s in row:iter_template(tmpl) do
   tex.print(s)
end
}
\end{tabular}
\end{center}

Miglioriamo ora il codice della funzione generatrice aggiungendo il metodo
\fn{insert} alla classe \code{Row}, a tre argomenti: il numero di colonna
\key{col}, il valore da inserire nella cella \key{val} e infine il valore
opzionale di arrotondamento numerico \key{prec}. Eccone una sua implementazione
molto semplice:
\begin{Verbatim}
function Row:insert(col, val, prec)
    if prec then
        local p = string.char(37)
        local fmt = string.format(p..p.."0."..p.."df", prec)
        val = string.format(fmt, val)
    end
    self[col] = val
    return self
end
\end{Verbatim}

Il nuovo metodo restituisce l'oggetto stesso così che possiamo concatenare più
inserimenti di cella. Ecco come la funzione di generazione può semplificarsi:
\begin{Verbatim}
local function row_func(diam, row)
    local area = math.pi * (diam/20)^2
    row:insert(1, diam)
       :insert(2, area, 3)
       :insert(3, 0.785*area, 3)
end
\end{Verbatim}

Sono scomparse le acrobazie per il formato numerico a favore della compattezza.
Un ulteriore miglioramento ci consente di evitare di dover controllare
l'espansione quando inseriamo il testo del template di riga grazie al comando
\cs{detokenize}.

Introduciamo in proposito una nuova macro \cs{printrow} che ha come argomento il
template che rappresenta il modello della genrica riga della tabella:
\begin{Verbatim}
\newcommand{\printrow}[1]{\directlua{
local tmpl = [=[\detokenize{#1}]=]
for _, s in row:iter_template(tmpl) do
   tex.print(s)
end
}}
\end{Verbatim}

Per non introdurre un secondo argomento, nell'istanziare l'oggetto della classe
\code{Row} dovremo solo ricordarci di chiamare la variabile come \key{row}, lo
stesso nome usato nella definizione di \cs{printrow}. Mettiamo subito al lavoro
la nuova macro:
\begin{Verbatim}
\begin{tabular}{lrr}
\printrow{\textbf{<1>} & <2> & <3>\\}
\end{tabular}
\end{Verbatim}

Molto semplice: si definisce prima la funzione generatrice e con essa si
costruisce l'oggetto \key{Row}, poi si scrive il codice \LaTeX{} dando alla
macro \cs{printrow} il template con i segnaposto.

Molto importante è far corrispondere i numeri di cella dei segnaposto del
template con i valori che la funzione di riga inserisce nella varie posizioni.

L'ultimo passo è migliorare l'aspetto della tabella. Con il pacchetto
\pack{booktabs} aggiungiamo un'intestazione e un filetto ogni tre righe per
facilitare la lettura dei dati. Dobbiamo così modificare la funzione di riga per
derminare se il numero di riga è multiplo di tre --- senza usare l'operatore
modulo \key{\%} di Lua perché non ci troviamo in un file esterno:
\begin{Verbatim}
\directlua{
local function fn(diam, row)
    local area = math.pi * (diam/20)^2
    local peso = 0.785*area
    local c = row.counter
    local midrule = ""
    if c - 3*math.floor(c/3) == 0
    and not (diam == row.stop) then
        midrule = string.char(92).."midrule"
    end
    row:insert(1, diam)
       :insert(2, area, 3)
       :insert(3, peso, 6)
       :insert(0, midrule)
end
}
\end{Verbatim}

Introduciamo anche il pacchetto \pack{siunitx} utilissimo per comporre numeri,
unità di misura e tabelle, con questo ambiente \amb{tabular} ridisegnato:
\begin{Verbatim}
\begin{tabular}{
    c
    S[table-format=4.3]
    S[table-format=1.3]
    S[table-format=1.6]
}
\toprule
\diameter & {Sviluppo} & {Sezione} & {Peso}\\
\small\si{mm} & {\small\si{cm^2/m}} & {\small\si{cm^2}} & {\small\si{daN/m}}\\
\midrule
\printrow{\(\mathbf{<1>}\) & <4> & <2> & <3>\\<0>}
\bottomrule
\end{tabular}
\end{Verbatim}

Il progetto completo si trova nel file \file{app-star/E0-006-tab.tex}, dove ho
aggiunto alla tabella la colonna con il calcolo della superficie laterale delle
barre.


\subsection{Conclusioni}

La nostra classe \code{Row} ci permette di costruire tabelle iterative in Lua in
modo del tutto generale, compiendo calcoli numerici e ogni sorta di possibili
elaborazioni. Molti altri affinamenti sono possibili come il caricamento di dati
esterni oppure l'uso di pipeline di operatori all'interno dei segnaposto dei
template. Anche questo tutorial si chiude perciò con una lista di nuove idee da
implementare con Lua.

La tabella sottostante mostra il risultato finale.

\directlua{
function Row:insert(col, val, prec)
    if prec then
        local p = string.char(37)
        local fmt = string.format(p..p..'0.'..p..'df', prec)
        val = string.format(fmt, val)
    end
    self[col] = val
    return self
end

local function fn(diam, row)
    local area = math.pi * (diam/20)^2
    local peso = 0.785*area
    local sup_lat = 10 * math.pi * diam
    local c = row.counter
    local midrule = ''
    if c - 3*math.floor(c/3) == 0 then
        midrule = string.char(92)..'midrule'
    end
    row:insert(1, diam)
       :insert(2, area, 3)
       :insert(3, peso, 6)
       :insert(4, sup_lat, 3)
       :insert(0, midrule)
end
row = Row:new(fn, 6, 32, 2)
}

\newcommand{\printrow}[1]{\directlua{
local tmpl = [=[\detokenize{#1}]=]
for _, s in row:iter_template(tmpl) do
   tex.print(s)
end
}}

\begin{center}
\begin{tabular}{
    c
    S[table-format=4.3]
    S[table-format=1.3]
    S[table-format=1.6]
}
\toprule
\diameter & {Sviluppo} & {Sezione} & {Peso}\\
\small\si{mm} & {\small\si{cm^2/m}} & {\small\si{cm^2}} & {\small\si{daN/m}}\\
\midrule
\printrow{\(\mathbf{<1>}\) & <4> & <2> & <3>\\<0>}
\bottomrule
\end{tabular}
\end{center}

% end of file
