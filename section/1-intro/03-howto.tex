

\chapter{Sul sistema \TeX{} e Lua}
\label{iichExplain}

Questo capitolo fornisce informazioni sulla differenza tra motore di
composizione e formato, e sull'esecuzione di codice Lua all'interno di un
sorgente \TeX.

I listati che riportano nella seconda riga di commento il nome del file
corrispondente, sono scaricabili dal
\href{https://github.com/GuITeX/guidalua}{repository} del progetto della guida.


\section{Motori di composizione e formati}

Un sorgente \TeX{} contiene testo e macro. Il testo formerà i capoversi, i
titoli e il resto del documento, mentre le macro ne stabiliranno l'aspetto e
la struttura. I \emph{motori di composizione} dispongono di particolari macro
dette \emph{primitive} implementate direttamente in essi, e la possibilità di
definire nuove macro per svolgere compiti specifici e ricorrenti.

Queste nuove macro il cui codice è generalmente scritto da veri esperti,
possono essere aggregate in una sorta di libreria di alto livello che prende
il nome di \emph{formato}, come \LaTeX{} o Con\TeX t, perché per esse sono
state stabilite nuove e coerenti regole di sintassi.

Per esempio, nel formato \LaTeX{} è presente il concetto di \emph{ambiente}
con la coppia di macro di delimitazione \cs{begin} e \cs{end}. I motori di
composizione hanno l'abilità nella fase iniziale della compilazione di
caricare il formato in forma precompilata.

Se si avvia un qualsiasi motore di composizione della famiglia \TeX{} verrà
caricato di default il formato più semplice chiamato \emph{plain}. Se si vuole
invece utilizzare un diverso formato, per esempio il più diffuso \LaTeX{},
occorre passarne il nome all'opzione \texttt{-{}-fmt} nel comando di
compilazione al terminale.

Tuttavia, data l'importanza per gli utenti dei formati di alto livello, sono
stati predisposti appositi comandi scorciatoia. Per esempio il programma
\prog{pdflatex} rimanda all'effettivo motore di composizione, il tradizionale
\prog{pdftex}, con l'istruzione di caricare il formato \LaTeX{}, e così anche
per \prog{lualatex} con \prog{luatex}.

Riassumendo, i motori di composizione sono programmi tipografici mentre i
formati sono insiemi coerenti di macro basate sulle primitive di sistema. I
nomi dei programmi disponibili nel sistema \TeX{} possono confondere se non si
conosce questa importante distinzione: alcuni di essi sono comandi scorciatoia
per identificare sia il motore sia il formato e non un motore di composizione
vero e proprio.


\subsection{Compositori Lua-powered}

\LuaTeX{} è un programma che elabora un file di testo contenente codice \TeX{}
per comporne il corrispondente file PDF, quindi è un motore di composizione.
Nella famiglia \TeX{} ci sono almeno altri due compositori dotati
dell'interprete Lua, LuaHB\TeX{} e Luajit\TeX{}.

Tutti e tre questi compositori possono eseguire il formato \LaTeX. Come detto in
apertura di sezione, esiste il programma \prog{lualatex} scorciatoia a un
compositore che carica il formato \LaTeX.

Dalla TeX Live 2020 questo compositore è \prog{luahbtex}. Per rendercene conto
basta scrivere in un terminale il nome del programma, premere invio e leggere
l'output poi premere CTRL+C per chiuderne l'esecuzione di prova:
\begin{Verbatim}
> lualatex
This is LuaHBTeX, Version 1.12.0 (TeX Live 2020/W32TeX)
 restricted system commands enabled.
**
\end{Verbatim}

Per ulteriore informazione, \prog{luahbtex} è il motore di composizione
\prog{luatex} in cui è stato sostituito il componente per il calcolo della
forma dei font con il modulo HarfBuzz, mentre \prog{luajittex} è un'altra
variante di \prog{luatex} in cui l'interprete Lua è stato sostituito con
LuaJIT un'implementazione indipendente più veloce dell'interprete ufficiale che
sfrutta le tecniche di compilazione denominate \emph{Just In Time}.

In generale un sorgente \TeX{} che contiene codice Lua viene correttamente
compilato da qualsiasi dei tre compositori grazie al mantenimento della
compatibilità.


\subsection{Lua in \LuaTeX}

Per illustrare l'esecuzione di codice Lua all'interno di un sorgente \LuaTeX,
consideriamo la stampa di un semplice testo nell'output di console con il
seguente sorgente completo, dove il codice Lua va inserito come argomento della
primitiva \cs{directlua}:
\begin{Verbatim}
% !TeX program = LuaTeX
\directlua{
    print("Hello World!")
}
\bye
\end{Verbatim}
il testo uscirà tra gli altri messaggi di output senza che sia prodotto un file
PDF. Ciò significa che \cs{directlua} è una macro espandibile con risultato
vuoto.

La prima linea di commento è una \emph{riga magica}, comodissima nel dare
istruzione allo shell editor sul programma da usare per compilare il documento
ma ignorata durante la compilazione stessa\footnote{La sintassi delle righe
magiche dipende dall'editor, in questa guida essa è scritta secondo le regole
di TeX Works.}. Qui le useremo se pertinenti per aiutare il lettore a
stabilire il contesto di esecuzione del codice.

Se il sorgente è memorizzato nel file \texttt{primo.tex}, possiamo verificare
quanto previsto in un terminale lanciando il comando:
\begin{Verbatim}
$ luatex primo
\end{Verbatim}
e per il sistema operativo Windows e la distribuzione TeX Live 2020, l'output
in console è:
\begin{Verbatim}
This is LuaTeX, Version 1.12.0 (TeX Live 2020/W32TeX) 
    restricted system commands enabled.
(./primo.texHello World!
)
warning  (pdf backend): no pages of output.
Transcript written on primo.log.
\end{Verbatim}


\subsection{Lua in Lua\LaTeX}
\label{secLuaInLuaLaTeX}

Con Lua\LaTeX{} si ottiene lo stesso risultato ma con il sorgente scritto nella
sintassi \LaTeX, ovvero:
\begin{Verbatim}
% !TeX program = LuaLaTeX
\documentclass{article}
\directlua{
    print("Hello World!")
}
\begin{document}
\end{document}
\end{Verbatim}
e questa volta il comando di compilazione è:
\begin{Verbatim}
$ lualatex primo
\end{Verbatim}

Avremo potuto inserire la macro all'interno dell'ambiente \amb{document} anziché
nel preambolo. Quando \TeX{} incontra \cs{directlua} ne \emph{espande}
l'argomento e passa a Lua il controllo che esegue immediatamente il codice per
poi restituirlo di nuovo a \TeX{} al termine dell'esecuzione.


\section{Passaggio di dati}

La comunicazione dati bidirezionale tra \TeX{} e Lua può avvenire con la
tecnologia dei nodi, come vedremo, certamente quella più avanzata e complessa
ma non è l'unica: i dati possono arrivare a Lua tramite l'espansione, mentre
nella direzione opposta è possibile scrivere del testo nella lista di input
del compositore con le funzioni della famiglia \fn{tex.print}.

Interrompete la lettura della guida per provare a scrivere la funzione Lua
\fn{fact} che calcola il fattoriale di un intero. La useremo per dimostrare come
avviene questo scambio di dati. L'idea è definire un nuovo comando che accetti
un numero e ne stampi il fattoriale:
\begin{Verbatim}
\newcommand{\fattoriale}[1]{\directlua{
    local n = #1
    tex.print(tostring(fact(n)))
}}
\end{Verbatim}

La definizione si basa sulla macro base \cs{newcommand} dove si indica, se
esistono, il numero degli argomenti tra parentesi quadre e la definizione. Ogni
argomento ha un segnaposto che inizia con il carattere \code{\#} seguito dal
numero del gruppo\footnote{Fate riferimento ha una buona guida per \LaTeX{} per
saperne di più sulla definizione di macro utente.}. Quando \TeX{} incontra un
segnaposto lo sostituisce con il corrispondente testo introdotto dall'utente.
Per questo quando scriveremo \cs{fattoriale}\code{\{5\}} il codice Lua
effettivamente eseguito sarà:
\begin{Verbatim}
local n = 5
tex.print(tostring(fact(n)))
\end{Verbatim}

La funzione \fn{tex.print} inserisce l'argomento nella lista d'ingresso del
testo. Quando termina l'esecuzione del blocco di codice Lua i prossimi caratteri
che si troverà il compositore saranno le cifre che compongono il fattoriale.

Il listato del sorgente completo da confrontare con la vostra personale
soluzione è:
\VerbatimInput{app-start/E1-001-fattoriale.tex}


\section{Globale o locale}

Prendendo spunto ancora dall'esempio precedente, soffermiamoci sul comportamento
nei diversi blocchi \cs{directlua} delle definizioni locali e globali: come
descritto dalle specifiche di Lua, tutto quello che è locale a un blocco non è
più disponibile al di fuori di esso. Nel contesto di \LuaTeX{} il codice
contenuto in una macro \cs{directlua} forma un blocco.

Per questo motivo se separassimo il codice che calcola il fattoriale dalla
definizione della macro utente \cs{fattoriale}, dovremo definire la funzione
come globale. Altrimenti riceveremmo un errore di chiamata di un valore
\code{nil} dal secondo blocco:
\VerbatimInput{app-start/E1-002-fattoriale.tex}

Le definizioni globali tuttavia possono determinare errori dovuti alla
collisione dei nomi essendo l'ambiente globale unico per tutto il sorgente.
Spesso, come per gli esempi della guida, non è un problema ma è buona norma
definire una tabella di \emph{namespace} dove memorizzare le funzioni limitando
la collisione dei nomi solamente al nome del riferimento alla tabella.

Questa buona prassi diviene \emph{obbligatoria} quando stiamo scrivendo codice
applicativo in contesti di utilizzo reale.

Come ulteriore esempio, nel listato\footnote{Questo listato è interessante
perché è migliorabile sia per eliminare le ripetizioni di codice a cui è
costretto l'utente sia nell'efficienza di esecuzione.} che segue è mostrato come
si possa formattare il numero del fattoriale con il pacchetto
\pack{siunitx}: \VerbatimInput{app-start/E1-003-fattoriale.tex}


\section{Espansione di macro}

La comunicazione di un input tra \TeX{} e Lua può avvenire anche per espansione
di macro. Come esempio minimo consideriamo il seguente sorgente \LuaTeX{} che
stampa l'ora di inizio della compilazione avvalendosi del contatore \cs{time}
che indica i minuti trascorsi dall'inizio del giorno:
\VerbatimInput{app-start/E1-004-time.tex}

Al termine dell'espansione l'istruzione di assegnazione della variabile numerica
\key{time} sarà l'effettiva e corretta sintassi Lua. Per il formato \LuaLaTeX{}
lo stesso file potrebbe essere:
\VerbatimInput{app-start/E1-005-time.tex}

Al capitolo~\ref{iichRegistro} vedremo che anche con la libreria standard di Lua
è possibile ottenere la data e l'ora corrente.


\section{Caratteri speciali}

Alcuni simboli hanno un diverso significato per \TeX{} e per Lua, per esempio il
carattere cancelletto, la stessa backslash o il simbolo di percento. \LuaTeX{}
non si occupa direttamente di modificare il significato dei simboli che si
sovrappongono.

Le descrizioni dei conseguenti errori possono essere criptici ma ci sono almeno
quattro diverse soluzioni:
\begin{compactitemize}
\item scrivere il codice Lua in file esterni,
\item utilizzare codice \TeX{} per gestire l'espansione dei simboli o i codici
di categoria,
\item usare i codici ASCII per creare stringhe che contengono i simboli, con la
funzione \fn{string.char},
\item utilizzare il pacchetto \pack{luacode}.
\end{compactitemize}

Come preferenza personale non utilizzo l'ottimo pacchetto \pack{luacode},
anche per evitare una dipendenza in più. Tuttavia non appena il codice Lua
cresce in numero di linee o diventa un componente di un progetto reale, quasi
sempre si devono utilizzare file esterni per una più agevole gestione del
progetto.

Rimando alla documentazione del pacchetto \pack{luacode} per i dettagli sulla
collisione dei simboli. Vedrete che esso offre due comandi e due ambienti per
poter far convivere il codice Lua in un sorgente \LaTeX.

Nel codice Lua si possono usare i commenti in stile \TeX{} con il simbolo del
percento perché il processo d'espansione li elimina \emph{prima} di passare il
codice all'interprete, ma non si possono usare i commenti in stile Lua con il
doppio trattino a meno che non siano all'ultima linea. Infatti, per la stessa
espansione tutto il codice Lua nella macro \cs{directlua} è inviato
all'interprete come un unica linea di codice perciò il commento dopo un doppio
trattino si estenderebbe non solo a fine riga ma a tutto il codice che segue.

Per fortuna la grammatica Lua a differenza di Python, consente la libera
scrittura e identazione del codice.


\section{Le librerie disponibili in \LuaTeX}

Agli usuali moduli della libreria standard di Lua, sono stati aggiunti in
\LuaTeX{}, così come per gli altri motori di composizione estesi, speciali
funzionalità dedicate al controllo dello stato interno e alla creazione di
elementi tipografici.





\section{La primitiva directlua}

Abbiamo ora tutte le informazioni per svolgere alcuni esercizi e applicazioni
fin dai prossimi capitoli. Non rimane che ricordare che il principale modo di
eseguire codice Lua in \LuaTeX{} è assegnarlo come argomento alla macro
\cs{directlua}. Quello che avviene è stabilito da queste regole:
\begin{compactenumerate}
\item l'argomento di \cs{directlua} viene espanso ed eseguito come blocco, può
quindi contenere macro o argomenti macro con un testo di sostituzione;

\item le variabili locali hanno validità solo all'interno del gruppo/blocco
mentre quelle globali saranno valide anche in quelli di successive
\cs{directlua};

\item l'espansione di \cs{directlua} è vuota;
\end{compactenumerate}


% end of file
