
% presentazione della guida e informazioni di base per eseguire gli esempi

\chapter{Presentazione}

\section{Motivazione}

Questa guida tematica è dedicata alla programmazione in Lua all'interno dei
motori di composizione del sistema \TeX.

Con Lua è possibile compiere sia elaborazioni generiche come l'interrogazione di
basi di dati che elaborazioni tipografiche interagendo con il compositore
interno. Con Lua rispetto a \TeX, è più semplice ed efficiente effettuare
calcoli numerici o avvalersi di avanzate librerie esterne.

La nuova generazione di compositori amplia così notevolmente gli scenari
applicativi. Se da un lato è auspicabile che queste potenzialità diventino
disponibili per gli utenti finali per mezzo di moduli e pacchetti, dall'altro è
utile fornire dettagli ed esempi per implementare proprie soluzioni o per poter
scrivere nuovi moduli condividendone lo sviluppo con la community.

Sono certamente molte le cose da conoscere: un nuovo linguaggio molto diverso da
\TeX{}, numerosi dettagli sul funzionamento interno dei compositori Lua-powered,
nuovi problemi di organizzazione del codice, di bilanciamento tra Lua e \TeX,
eccetera. Per questo, ho pensato di contribuire con questa guida cercando di
presentare il quadro della crescente complessità del sistema.


\section{Piano della guida}

La guida è divisa in tre parti: la prima offre una panoramica rapida in forma di
\emph{tutorial} per iniziare subito con Lua seguendo i passi di un ipotetico
utente alle prese con problemi compositivi (parte \ref{partTutorial}), la
seconda tratta delle basi del linguaggio Lua (parte \ref{partFoundation}) e
contiene numerosi esercizi, e infine la terza illustra esempi applicativi con
l'uso delle librerie di composizione cn i nodi (parte \ref{partApp}).

Tra gli argomenti ci sono:
\begin{compactitemize}
\item differenza tra motore e formato di composizione \( \to \) capitolo
\ref{iichExplain},
\item basi del linguaggio Lua \( \to \) dal capitolo \ref{chAssignment},
\item tecniche di programmazione e di rappresentazione dei dati \( \to \) dal
capitolo \ref{iichRegistro},
\item la tecnologia dei nodi in Lua \( \to \) capitolo \ref{iichTartaglia}.
\end{compactitemize}


\section{Origine della guida}

Per illustrare i concetti del linguaggio ho preso spunto da un breve corso su
Lua che scrissi qualche tempo fa per il blog
\href{http://parliamodi-ubuntu.blogspot.it}{Lubit Linux} di Luigi Iannoccaro
che mi propose di realizzare un progetto di divulgazione su Lua. Luigi ha
acconsentito all'utilizzo di quegli appunti per produrre questa guida tematica.


\section{Contribuire e collaborare}

Spero che i lettori vorranno contribuire al testo inviando proprie soluzioni o
nuovi contributi piccoli o grandi. Lo si può fare attraverso lo strumento che
preferite, scrivendomi un messaggio di posta elettronica all'indirizzo
\href{mailto:giaconet.mailbox@gmail.com}{\texttt{giaconet.mailbox@gmail.com}},
oppure utilizzando il
\href{https://github.com/GuITeX/guidalua}{\texttt{repository git}} dei sorgenti
della guida, eseguendo un Pull Request o aprendo una discussione premendo il
pulsante Issues.


\section{Altre risorse}

La risorsa principale per imparare Lua, a cui si rimanda per tutti gli
approfondimenti, è certamente il PIL \cite{PIL} acronimo del titolo del libro
\emph{Programming In Lua} di Roberto Ierusalimschy, principale Autore di Lua.
Questo testo non solo è completo e autorevole ma è anche ben scritto e
composto\footnote{Tra l'altro il libro ufficiale su Lua viene composto in
\LaTeX{} e commercializzato per contribuire allo sviluppo del linguaggio
stesso.}.

La seconda importante risorsa su Lua si trova in rete all'indirizzo
\href{https://www.lua.org/manual/5.3/}{\texttt{www.lua.org/manual/5.3/}} ed è il
\emph{reference} del linguaggio \cite{web:luaref} con le specifiche di sintassi,
metametodi, funzioni di libreria, C API eccetera.

Quanto a \LuaTeX{} il riferimento è il suo manuale \cite{prg:luatex} che, come
quasi tutta la documentazione nel sistema \TeX{}, può essere visualizzato a
video con il comando da terminale:
\begin{Verbatim}[numbers=none]
$ texdoc luatex
\end{Verbatim}


\section{Note di lettura}

Nei listati compilabili riportati nella guida compare alla prima linea la
\emph{riga magica}, un commento utile per dare istruzioni all'editor sul
compilatore da usare, ma che qui informerà il lettore per aiutarlo a stabilire
il contesto del codice.

Se presente nel progetto, alla seconda riga dei listati si troverà invece il
nome del file che il lettore potrà scaricare ed eseguire per i propri
esperimenti dal \href{https://github.com/GuITeX/guidalua}{\texttt{repository
git}}.

Nella parte \ref{partFoundation} ho cercato di non dare per scontati i concetti
fondamentali della programmazione. Ovviamente il lettore già preparato procederà
più velocemente nel prendere dimestichezza con Lua. Ho invece escluso dalla
guida \TeX{}, per esempio non spiegando come si definisce una macro utente o
come si lavora con il formato \LaTeX3. Rimando senza indugio alla copiosa
documentazione disponibile a cominciare da quella scaricabile dal sito \GuIT.

Diamo quindi inizio a questa nuova avventura lunare.

% end of file
