
\chapter{Programmazione a oggetti in Lua}
\label{iiChOop}

In sintesi, il paradigma della \emph{Object Oriented Programming} \textsc{oop},
si basa sulla creazione di entità indipendenti chiamate \emph{oggetti}. Ciascun
oggetto incorpora sia dati che funzioni, che prendono il nome di \emph{metodi}.

Ogni oggetto è un'istanza che fa parte di una stessa famiglia chiamata
\emph{classe}, una sorta di prototipo che rappresenta un ``tipo di dati''. Le
classi possono essere ricavate da altre classi con il meccanismo
dell'\emph{ereditarietà} per specializzarne il comportamento.

Per instanziare un oggetto di una classe si utilizza un metodo speciale chiamato
\emph{costruttore}, che valida gli eventuali dati in ingresso e instanzia in
memoria l'oggetto.

In questo capitolo ritroveremo tutti questi concetti del paradigma della
programmazione a oggetti dal punto di vista di Lua. Con essi la struttura del
problema non è più pensata in termini di funzioni, ma attraverso la
rappresentazione dei suoi elementi concettuali in classi e le loro relazioni di
ereditarietà.

Negli ultimi anni, la programmazione a oggetti è stata ripensata tant'è che nei
linguaggi di nuova generazione come Go e Rust non è inclusa nel modo classico.
Ciò non toglie che essa possa rendere più intuitiva la programmazione in Lua, in
special modo per chi sviluppa applicazioni per \LuaTeX.


\section{Il minimalismo di Lua}

Il linguaggio Lua non è progettato con gli stessi obiettivi di Java o del C++, i
due linguaggi più noti per la programmazione a oggetti, non possiede un
controllo preventivo del tipo, non prevede il concetto sintattico di classe, non
offre alcun meccanismo per dichiarare come privati campi e metodi, e lascia al
programmatore più di un modo per implementare la \textsc{oop}.

Tuttavia Lua basandosi sulle tabelle offre il pieno supporto ai principi del
paradigma a oggetti senza perdere le caratteristiche minimali del linguaggio.


\section{Una classe Rettangolo}
\label{secRectOop}

Costruiremo una classe per rappresentare un rettangolo. Si tratta di un ente
geometrico definito da due soli parametri \emph{larghezza} e \emph{altezza}, e
dotato di proprietà come l'area e il perimetro, che implementeremo come metodi.

Un primo tentativo potrebbe essere questo:
\sourcecode{
    file = [[code/e10-oop.lua]],
    select = [[primo]],
}

Ci accorgiamo presto che questo tentativo è difettoso in quanto non rispetta
l'indipendenza degli oggetti rispetto al loro nome. Infatti il prossimo test
fallisce:
\sourcecode{
    file = [[code/e10-oop.lua]],
    select = [[due]],
}

Il problema sta nel fatto che nel metodo \fn{area} compare il particolare
riferimento alla tabella \key{Rettangolo}. La soluzione non può che essere
l'introduzione del riferimento dell'oggetto come parametro esplicito nel metodo
stesso, ed è la stessa utilizzata anche dagli altri linguaggi di programmazione
che supportano gli oggetti. Vedremo tra poco come, allo stesso modo dei
linguaggi \textsc{oop}, anche in Lua si possa nascondere il riferimento
con la colon notation.

Secondo quest'idea dovremo riscrivere il metodo \fn{area} in questo modo (in
Lua il riferimento implicito all'oggetto deve chiamarsi \key{self} pertanto
abituiamoci fin dall'inizio a questa convenzione):
\sourcecode{
    file = [[code/e10-oop.lua]],
    select = [[tre]],
}

Fino a ora abbiamo costruito l'oggetto sfruttando le caratteristiche della
tabella e la particolarità che consente di assegnare una funzione a una
variabile. Questo punto è importante: le chiavi nella tabella dell'oggetto
possono contenere sia funzioni/metodi sia valori/campi perciò sovrascrivere una
chiave nell'intenzione di introdurre un nuovo campo/metodo porta a errori.


\section{Colon notation}

Da questo momento entra in scena l'operatore \key{:} --- che chiameremo
\emph{colon notation}. L'operatore \key{:} fa in modo che le seguenti due
espressioni siano perfettamente equivalenti anche se le rende differenti dal
punto di vista concettuale agli occhi del programmatore:
\sourcecode{
    file = [[code/e10-oop.lua]],
    select = [[colon_notation]],
}

Questo operatore è il primo nuovo elemento che Lua introduce per facilitare la
programmazione orientata agli oggetti. Se si accede a un metodo memorizzato in
una tabella con l'operatore due punti \key{:} anziché con l'operatore \key{.}
allora l'interprete aggiungerà implicitamente un primo parametro chiamandolo
\key{self} con il riferimento alla tabella stessa, insomma puro zucchero
sintattico.


\section{Metatabelle}

Il linguaggio Lua si fonda sull'essenzialità tanto che supporta la
programmazione a oggetti utilizzando quasi esclusivamente le proprie risorse di
base senza introdurre nessun nuovo costrutto. In particolare in Lua si utilizza
la tabella, l'unica struttura dati predefinita nel linguaggio, assieme a
particolari funzionalità dette \emph{metatabelle} e \emph{metametodi}.

Il salto definitivo nella programmazione \textsc{oop} consiste nel poter
costruire una \emph{classe} senza ogni volta assemblare campi e metodi,
introducendo un qualcosa che faccia da stampo per gli oggetti.

In Lua l'unico meccanismo disponibile per compiere questo ultimo importante
passo consiste nelle \emph{metatabelle}. Esse sono normali tabelle contenenti
funzioni dai nomi prestabiliti che vengono chiamati quando si verificano
particolari eventi come l'esecuzione di un'espressione di somma tra due tabelle
con l'operatore \key{+}. Ogni tabella può essere associata a una metatabella e
questo consente di creare degli insiemi di oggetti che condividono una stessa
aritmetica.

Metatabelle e metametodi quindi, possono rendere il codice intuitivo e compatto,
sono quindi una funzionalità indipendente dalla programmazione a oggetti.

I nomi delle funzioni di una metatabella vengono detti \emph{metametodi} e
iniziano tutti con un doppio trattino basso. Per esempio nel caso della somma
sarà richiesta la funzione \fn{\_\_add} nella metatabella associata al primo
addendo o se non esiste a quella del secondo addendo.

Per assegnare una metatabella si utilizza la funzione \fn{setmetatable}. Essa ha
due argomenti tabella, la prima è l'oggetto e la seconda è la tabella con i
metametodi.


\section{Il metametodo \fn{\_\_tostring}}

Il metametodo più semplice di tutti è \fn{\_\_tostring}. Esso viene
invocato se una tabella è data come argomento alla funzione \fn{print} per
ottenere il valore stringa da stampare. Se non esiste una metatabella
associata con questo metametodo verrà stampato l'indirizzo di memoria della
tabella:
\sourcecode{
    file = [[code/e10-oop.lua]],
    select = [[tostring]],
}


\section{Il metametodo \key{\_\_index}}
\label{secFondMetaIndex}

Il metametodo che interessa la programmazione a oggetti in Lua è
\key{\_\_index}. Esso interviene quando viene chiamato un campo di una tabella
che non esiste e che normalmente restituirebbe il valore \key{nil}. Un esempio
di codice chiarirà il meccanismo:
\sourcecode{
    file = [[code/e10-oop.lua]],
    select = [[index]],
}

Tornando all'oggetto \key{Rettangolo} riscriviamo il codice creando adesso una
tabella che assume il ruolo concettuale di una vera e propria classe:
\sourcecode{
    file = [[code/e10-oop.lua]],
    select = [[classe_rect]],
}

Queste poche righe di codice racchiudono il meccanismo della creazione di una
nuova classe in Lua: abbiamo infatti assegnato a una nuova tabella \key{r} la
metatabella con funzione di classe \key{Rettangolo}. Quando viene richiesta la
stampa del campo \key{b}, poiché tale campo non esiste nella tabella vuota
\key{r} verrà ricercato il metametodo \key{\_\_index} nella metatabella
associata che è appunto la tabella \key{Rettangolo}.

A questo punto il metametodo restituisce semplicemente la tabella
\key{Rettangolo} stessa e questo fa sì che tutti i campi e i metodi siano
ereditati da essa per essere accessibili da \key{r}. Il campo \key{b} e il
metodo \fn{area} del nuovo oggetto \key{r} sono in realtà quelli definiti nella
tabella \key{Rettangolo}.

Se volessimo creare invece un rettangolo assegnando direttamente la dimensione
dei lati dovremo semplicemente crearli in \key{r} con i nomi previsti dalla
classe: \key{b} e \key{h}. Il metodo \fn{area} sarà ancora caricato dalla
tabella \key{Rettangolo} ma i campi numerici con le nuove misure dei lati
saranno quelli interni dell'oggetto \key{r} e non quelli della metatabella
poiché semplicemente esistono in \key{r}.

Questa costruzione funziona ma può essere migliorata con l'introduzione del
costrutture come vedremo meglio in seguito. L'oggetto \key{Rettangolo} apparirà
sempre più concettualmente simile a una classe.


\section{Il costruttore}
\label{secFondCostruttore}

Proponendoci ancora la rappresentazione del concetto di rettangolo, completiamo
il quadro introducendo il costruttore della classe. Il lavoro che dovrà
svolgere questa speciale funzione sarà quello di inizializzare i campi
argomento in una delle tante modalità possibili e una volta effettuato il
controllo di validità degli argomenti.

Il codice completo della classe \key{Rettangolo} è il seguente:
\sourcecode{
    file = [[code/e10-oop.lua]],
    select = [[newrect]],
}

Il costruttore \fn{new} accetta una tabella come argomento, altrimenti ne crea
una vuota, controlla gli eventuali parametri geometrici, assegna la metatabella
e restituisce l'oggetto. Alla funzione arriva il riferimento implicito a
\key{Rettangolo} grazie alla colon notation, per cui \key{self} è un riferimento
alla stessa tabella del di quello della variabile \key{Rettangolo}.

Quando viene passata una tabella con uno o due campi sulle misure dei lati al
costruttore, l'oggetto disporrà delle misure come valori interni effettivi,
cioè dei parametri indipendenti che costituiscono il suo stato interno. Lo
sviluppatore può fare anche una diversa scelta, quella per esempio di
considerare la tabella argomento del costruttore come semplice struttura di
chiavi/valori da sottoporre al controllo di validità e poi includere in una
nuova tabella con modalità e nomi che riguardano solo l'implementazione interna
della classe.

Essendo il costruttore una normale funzione Lua, i suoi argomenti possono essere
più di uno e di diverso tipo, mentre la classe può averne anche più di uno con
nomi differenti.


\section{Questa volta un cerchio}

Per capire ancor meglio i dettagli e renderci conto di come funziona il
meccanismo automatico delle metatabelle, costruiamo una classe \key{Cerchio} che
annoveri fra i suoi metodi una funzione che modifichi il valore del raggio
aggiungendovi una misura:
\sourcecode{
    file = [[code/e10-oop.lua]],
    select = [[cerchio]],
}

Nella sezione del codice utente viene dapprima creato un cerchio senza fornire
alcun valore per il raggio. Ciò significa che quando stampiamo il valore del
raggio con la successiva istruzione otteniamo \( 0 \) che è il valore di default
del raggio dell'oggetto \key{Cerchio}, per effetto della chiamata a
\key{\_\_index} della metatabella.

Fino a questo momento la tabella dell'oggetto \key{o} non contiene alcun campo
\key{radius}. Cosa succede allora quando è chiamato il comando
\key{o:addToRadius(12.342)}?

Il metodo \fn{addToRadius} contiene una sola espressione. Come da regola viene
prima valutata la parte a destra ovvero \key{self.radius + v}. Il primo termine
assume il valore previsto in \key{Cerchio} --- quindi zero --- grazie al
metametodo, e successivamente il risultato della somma uguale all'argomento
\key{v} è memorizzato nel campo \key{o.radius} che viene creato effettivamente
solo in quel momento.


\section{Ereditarietà}

Il concetto di ereditarietà nella programmazione a oggetti consiste nella
possibilità di derivare una classe da un'altra per specializzarne il
comportamento.

L'operazione di derivazione incorpora automaticamente nella sottoclasse tutti i
campi e i metodi della classe base. Dopodiché si implementano o si modificano i
metodi della classe derivata creando una gerarchia di oggetti.

In Lua l'operazione di derivazione consiste molto semplicemente nel creare un
oggetto con il costruttore della classe base e modificarne o aggiungerne metodi
o campi.

Vediamo un esempio semplice dove si rappresenta il concetto generale di una
persona che svolge attività sportiva e da questo, il concetto di una persona
che svolge uno specifico sport:
\sourcecode{
    file = [[code/e10-oop.lua]],
    select = [[sportivo]],
}

Continua tutto a funzionare per via della ricerca effettuata dal metametodo
\key{\_\_index} che funziona a ritroso fino alla classe base.


\section{Esercizi}

\begin{Exercise}[label=oop-01]
Aggiungere alla classe \key{Rettangolo} riportata nel testo il metametodo
\fn{\_\_tostring} che stampi in console il rettangolo dalle dimensioni
corrispondenti ad altezza e larghezza usando i caratteri \key{+} per gli
spigoli e i caratteri \key{-} e \key{|} per disegnare i lati. Utilizzare le
funzioni di libreria \fn{string.rep} e \fn{string.format}.
\end{Exercise}

\begin{Exercise}[label=oop-02]
Creare una classe corrispondente al concetto di numero complesso e implementare
le quattro operazioni aritmetiche tramite metametodi (riferimento matematico
\href{http://it.wikipedia.org/wiki/Numero_complesso#Operazioni_con_i_numeri_complessi}{qui}).
Aggiungere anche il metodo \fn{\_\_tostring} per stampare il numero complesso e
poter controllare i risultati di operazioni di test.
\end{Exercise}

\begin{Exercise}[label=oop-03]
Ideare una classe base e una classe derivata dandone un'implementazione.
\end{Exercise}

% end of file
