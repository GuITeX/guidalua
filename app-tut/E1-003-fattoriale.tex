% !TeX program = LuaLaTeX
% filename: app-tut/E1-003-fattoriale.tex
\documentclass{article}
\usepackage{siunitx}
% tabella di namespace
\directlua{
assert(not app)
app = {}
function app.fact(n)
    local res = 1
    for i = 1, n do
        res = res * i
    end
    return res
end
}
% user command
\newcommand{\fattoriale}[1]{\num{\directlua{
    local n = assert(tonumber(#1))
    tex.print(tostring(app.fact(n)))
}}}
\begin{document}
\noindent\( 10! = \fattoriale{10} \)\\
\( 11! = \fattoriale{11} \)\\
\( 12! = \fattoriale{12} \)\\
\( 13! = \fattoriale{13} \)\\
\( 14! = \fattoriale{14} \)\\
\( 15! = \fattoriale{15} \)\\
\( 16! = \fattoriale{16} \)\\
\( 17! = \fattoriale{17} \)\\
\( 18! = \fattoriale{18} \)
\end{document}
