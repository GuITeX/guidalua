

\chapter{Un registro delle compilazioni}
\label{iiiChRegistro}

Creeremo in Lua un sistema di registrazione delle compilazioni che si
concretizza in un file posizionato nella directory principale del progetto del
documento.

Non è possibile reperire dati generali sulla compilazione eseguendo codice
durante la compilazione stessa. Dovremo farlo con un tool esterno. Solo in
questo modo è possibile ottenere il tempo di compilazione totale o accertarsi
del nome del file sorgente. Tuttavia la registrazione in fase di compilazione è
utile per imparare con Lua a lavorare con i file e sperimentare un processo di
sviluppo iterativo molto più rapido che con \TeX.

Chiameremo questo registro \file{history.log} verso cui invieremo una linea di
testo informativa per ciscuna compilazione che il compositore completi
correttamente.


\section{Scrivere sul registro}

Iniziamo a implementare la gestione automatica del registro dalla funzionalità
chiave: la scrittura su file. Ci rivolgeremo alla libreria standard di Lua,
modulo \code{io}\footnote{La completa documentazione della libreria \code{io} si
trova alla pagina \url{https://www.lua.org/manual/5.3/manual.html\#6.8}.}:
\begin{lines}
-- write a line of text in a file
local function append(filename, line)
    local f = io.open(filename, "a+")
    f:write(line.."\n")
    f:close()
end
\end{lines}

La funzione \fn{io.open} restituisce un riferimento a un oggetto che rappresenta
un canale di input/output del sistema operativo verso un file i cui metodi vanno
chiamati in colon notation (vedi alla sezione~\ref{iiSecRectOop}). \fn{open}
accetta il nome e la modalità di apertura del file. Lo specificatore \verb|"a+"|
indica di aprire il file in \emph{append} senza distruggerne il contenuto
preesistente. In caso il file non esista ne verrà creato uno nuovo perciò in
ogni caso otterremo il file aperto in scrittura.

La funzione \fn{append} può essere provata in un sorgente plain \LuaTeX{}
compilabile come in questo:
\sourcecode{file = [[app-reg/01.tex]]}


\section{Dati di compilazione}

Costruite le fondamenta stabiliamo quali dati inserire nel registro: l'utente,
mano a mano che il lavoro sul documento procede, eseguirà senza dubbio delle
compilazioni intermedie perciò registreremo le seguenti informazioni di base:
\begin{compactenumerate}
\item il nome del file sorgente,
\item la dimensione del file sorgente,
\item data e ora della compilazione,
\item il tempo di esecuzione della composizione,
\item il nome del motore di composizione.
\end{compactenumerate}

Per formare la linea di registro, al posto della concatenazione di stringhe
useremo quella di una tabella di stringhe, più efficiente. Riempiremo la tabella
un dato alla volta in un blocco \cs{directlua} posizionato immediatamente prima
del termine del sorgente. Come prima istruzione invece, inseriremo il blocco di
codice Lua che definirà alcuni parametri come il nome del file del registro e il
separatore di campo testuale, e la funzione \fn{append}:
\sourcecode{file = [[app-reg/02.tex]]}


\subsection{La variabile \code{jobname}}

Il nome del file sorgente non è detto che sia contenuto nella variabile
\code{jobname} che possiamo trovare anche nella macro omonima oppure nel campo
omonimo della tabella \code{tex}.

Il campo o la macro conterrà il nome del sorgente soltanto se l'utente non ha
valorizzato l'opzione \code{--jobname} nel comando di compilazione con un altro
nome o se non è stato modificato il campo \code{tex.jobname}.

Infatti, se provassimo a compilare il listato \code{02.tex} con il seguente
comando
\begin{Verbatim}[numbers=none,xleftmargin=0pt]
$ luatex --jobname=abc 02
\end{Verbatim}
otterremo un errore che blocca la compilazione. Nel secondo \cs{directlua} la
variabile locale \key{jobname} conterrebbe infatti la stringa \code{abc.tex}
che è un file che non esiste. Di conseguenza la funzione \fn{lfs.attributes}
restituisce \key{nil} che ovviamente non può essere indicizzato con la chiave
\key{size}.

La tabella \code{lfs} contiene la libreria Lua File System disponibile per i
sistemi operativi più diffusi, con alcune funzionalità sui file che non troviamo
di base in Lua perché l'interprete ha regole stringenti di portabilità. La
troviamo in \LuaTeX{} già compilata staticamente. La documentazione si trova
all'indirizzo web
\url{https://keplerproject.github.io/luafilesystem/manual.html}.

Se la funzione \fn{lfs.attributes} non restituisce la tabella con i dati del
file significa dunque che nel comando di compilazione è stata attivata l'opzione
\code{--jobname} per assegnare un diverso nome al file PDF di uscita.
Modifichiamo dunque il codice eseguito in chiusura in questo senso:
\begin{lines}
local tline = {}
local jobname = tex.jobname
local attr = lfs.attributes(jobname..".tex")
if attr then
    tline[1] = jobname
    tline[2] = attr.size
else
    tline[1] = "unknow"
    tline[2] = "unknow"    
end
tline[3] = os.date()
tline[4] = os.clock() - register.start
tline[5] = status.luatex_engine
register:append(tline)
\end{lines}

Anche l'estensione \code{.tex} è un problema perché se fosse diversa ancora una
volta il codice non funzionerebbe, per esempio se il file sorgente avesse
estensione \code{.TEX}.


\subsection{Gli altri dati}

Come sappiamo dal capitolo~\ref{iiChLibstd} la tabella \key{os} appartiene alla
libreria standard di Lua. La documentazione delle funzioni \fn{os.date} e
\fn{os.clock} si trova quindi nel reference ufficiale del linguaggio alla pagina
\url{https://www.lua.org/manual/5.3/manual.html#6.9}.

Infine, maggiori informazioni sulla tabella \key{status} da cui leggiamo il nome
del motore di composizione, possono essere reperite nel manuale di \LuaTeX{}.

Quanto alla misura del tempo di compilazione che registriamo, più precisamente
si tratta del tempo che trascorre tra i due momenti in cui eseguiamo la funzione
\fn{os.clock} perciò non comprende il tempo iniziale di avvio e caricamento del
formato, né i tempi finali di chiusura come la composizione del materiale non
ancora emesso sulla pagina.


\section{Creare il modulo e il pacchetto}

Per consentire l'utilizzo del registro da qualsiasi sorgente occorre spostare il
codice Lua in un file esterno di \emph{modulo} e scrivere un secondo file di
\emph{pacchetto} per fornire un'interfaccia utente conforme al formato, per
esempio per \LaTeX{}.

Questa separazione non è solo utile al processo di sviluppo, facilitando
l'accesso alle funzionalità da parte degli utenti indipendentemente dal formato.

Allo stato attuale dello sviluppo del codice, il modulo Lua è il seguente:
\sourcecode{file = [[app-reg/mod-history.lua]]}

Tutte le funzionalità sono incluse in un unica tabella Lua chiamata
\code{register}. Dall'esterno, caricando il modulo con la funzione \fn{require}
potremo assegnarne il riferimento a una variabile locale o globale a seconda
delle necessità.

L'uso della colon notation per chiamare le funzioni non fa parte di
un'implementazione a oggetti ma è solo un semplice modo per poter disporre di un
riferimento alla tabella contenitore con la chiave \key{self}. Ciò rende pulito
il codice ed elimina la necessità di creare una closure con un piccolo
incremento delle prestazioni per includere il riferimento alla tabella stessa.

Il sorgente in plain \LuaTeX{} si semplifica in questo:
\sourcecode{file = [[app-reg/03.tex]]}

Per compilarlo correttamente il file Lua del modulo chiamato
\code{mod-history.lua} deve essere nel percorso di ricerca. La soluzione più
semplice è copiarlo in locale nella directory del sorgente.

A questo punto è facile creare un pacchetto per \LaTeX{}. Non ci sarà alcuna
macro utente ma un aggiunta al codice di chiusura del documento con la macro
\cs{AtEndDocument}. Ecco il codice:
\sourcecode{file = [[app-reg/historylog.sty]]}

Basta caricare il pacchetto e tutte le compilazioni del sorgente verranno
registrate come per questo file sorgente per Lua\LaTeX{}. Compilatelo più volte
e controllate poi il file locale \file{history.log}:
\sourcecode{file = [[app-reg/doc.tex]]}

Il file di registro conterrà informazioni simili a queste:
\begin{Verbatim}[numbers=none,xleftmargin=0pt]
03.tex, 141, Sat Apr 10 17:31:35 2021, 0.00073999999999996, luatex
03.tex, 141, Sat Apr 10 17:31:38 2021, 0.00083600000000006, luatex
03.tex, 141, Sat Apr 10 17:31:40 2021, 0.00073299999999998, luatex
\end{Verbatim}


\section{Sviluppo}

Abbiamo costruito da zero un modulo Lua per inserire in un registro alcune
informazioni sulle compilazioni dei sorgenti. Le funzionalità sono minime ma si
tratta di un buon punto di partenza per dare risposta a nuove domande: quale
struttura dovremo dare al modulo in modo che sia semplice aggiungere nuove
funzionalità o modificare la lista delle informazioni registrabili.

A poco a poco prende forma e si precisa la struttura di un \emph{framework}
grazie all'attività di progettazione in passi successivi. Ciò che era molto meno
evidente con lo sviluppo dei pacchetti \LaTeX{} è invece chiaro con i moduli
Lua: la necessità di ricercare soluzioni con l'organizzazione del codice.


\subsection{Dati per descrivere dati}

Nel registro ogni linea contiene informazioni ma nel codice non ne esiste una
rappresentazione di alto livello, un oggetto che si possa aggiungere o
eliminare, o stampare in un diverso formato.

I concetti espressi sono due: dare la possibilità di scegliere le informazioni
da registrare e il loro formato, e rendere indipendente il registro dalle
informazioni. Ciò si risolve con l'introduzione di oggetti \code{Info}, insieme
di:
\begin{compactitemize}
\item un identificativo,
\item l'informazione testuale da registrare,
\item una serie di possibili regole di formato.
\end{compactitemize}

Il costruttore di tabelle è una struttura in grado di rappresentare ogni tipo di
dato e la seguente è la diretta descrizione dei componenti di \code{Info}:
\begin{lines}
{
    id = "EngineName",
    info = status.luatex_engine,
    fmt = {
        upper = function (engine_name)
            return engine_name:upper()
        end,
    }
}
\end{lines}

In questo codice l'informazione registrabile è il nome del compositore. I nomi
dei campi dovranno essere sempre gli stessi così che il componente del registro
possa elaborarli. Intenderemo il valore associato al campo \key{info} di tipo
stringa o di tipo funzione. In quest'ultimo caso la funzione dovrà attendersi
l'oggetto stesso come argomento e dovrà restituire un valore stringa.

Per esempio, nell'oggetto \key{ElapsedTime} che misura gran parte del tempo di
compilazione, il campo \key{info} è una funzione definita secondo le regole
appena enunciate:
\begin{lines}
{
    id = "ElapsedTime",
    info = function (t)
        return tostring(os.clock() - t._start)
    end,
    _start = os.clock(),
    fmt = {
        second = function (t) ... end,
    }
}
\end{lines}


\subsection{Nuova implementazione del registro}

Una volta stabilito il tipo di dato non rimane che implementare il codice che
gestisce gli oggetti \code{Info}. Le operazioni fondamentali sono:
\begin{compactitemize}
\item installazione di un oggetto \key{Info} nel registro, metodo \fn{install};
\item aggiungere un oggetto a una lista di registrazione specificando una
eventuale funzione di formato da applicare, metodo \fn{add\_to\_log};
\item rimuovere un oggetto \key{Info} dalla lista di uscita verso il registro,
metodo \fn{remove\_from\_log};
\item generare una lista di uscita di registrazione o caricarne una predefinita,
metodo costruttore con sintassi di chiamata di funzione;
\item impostare un formato tra quelli disponibili di un oggetto \key{Info} nella
lista di uscita, metodo \fn{set\_format};
\item ottenere il testo formattato pronto per la registrazione, metodo
\key{log}.
\end{compactitemize}

Non spiegherò tutti i dettagli per non eccedere con la descrizione del codice,
tratterò solo i punti salienti lasciando libero il lettore di consultare il
codice completo nel file \file{app-reg/lib-logger.lua}.

Di base, la possibilità di selezionare l'insieme di informazioni da registrare
comporta il dover distinguere la lista di oggetti disponbili che chiameremo
\key{\_iArchive} da quella degli oggetti che saranno inseriti nell'output di
nome \key{\_logList}. La nuova implementazione del registro inizia con il
definire le due liste interne di oggetti \key{Info}:
\begin{lines}
local register = {
    _iArchive = {},
    _logList = {},
}
\end{lines}
Stiamo usando la convenzione che i nomi dei campi privati inizino tutti con un
carattere di trattino basso.

Esaminiamo il metodo \fn{install} chen come stabilito dall'interfaccia pubblica,
ha il compito di verificare la struttura attesa di un generico oggetto
\code{Info} e in caso di successo di memorizzarla nell'archivio interno:
\begin{lines}
-- check and install a new `Info` object
function register:install(itab)
    local id = assert(itab.id) -- mandatory string field `id`
    assert(type(id) == "string")
    -- have we an `id` already saved?
    local arch = self._iArchive
    assert(not arch[id], "id '"..id.."' is not new")
    -- check `info` field
    assert(itab.info, "`info` field not found for '"..id.."'")
    -- check `fmt` field
    local fmt = itab.fmt
    if fmt then
        assert(type(fmt) == "table")
        local flag = false
        for k, f in pairs(fmt) do
            flag = true
            assert(type(k) == "string")
            assert(type(f) == "function")
        end
        assert(flag, "Empty format function table")
    end
    arch[id] = itab
    return self
end
\end{lines}

L'archivio d'installazione è una tabella dizionario indicizzata con il valore
del campo \key{id} presente nella tabella dell'oggetto \key{Info} stesso. In
alternativa, avrei potuto rendere esterno il nome dell'oggetto passandolo come
parametro esplicito al metodo \fn{install}.

Come forma di verifica dei dati abbiamo usato la funzione \fn{assert}, e non una
più sofisticata gestione degli errori poiché contiamo sul fatto che chi utilizza
la libreria non abbia esigenze d'impostazione frequenti ma intervenga di tanto
in tanto nel modificare le registrazioni delle compilazioni.

Più complessa è la struttura che definisce la lista delle informazioni da
registrare. Il motivo è che il codice deve essere eseguito in due distinti
momenti: al caricamento del pacchetto e alla fine del documento, altrimenti non
potremo registrare l'intervallo di tempo trascorso che in difetto corrisponde al
tempo di compilazione.

Quando il codice è caricato, l'oggetto \key{ElapsedTime} è installato assieme a
tutti gli altri tra i predefiniti della libreria, e in conseguenza in quel
momento viene letto l'orologio di sistema. 

Alla fine del processo di compilazione, le informazioni verranno calcolate e,
se presente in output l'oggetto \key{ElapsedTime}, rileggerà l'orologio
misurando il tempo trascorso.

Nulla vieta di implementare altre strategie come quella di rendere la lettura
dell'orologio al caricamento della libreria, un dato del registro stesso e poi
dare accesso a questo tipo di parametri comuni a tutti gli oggetti \key{Info}.
Sono scelte di progettazione molto importanti che divengono chiare durante lo
sviluppo a cui si può dare risposta con criteri di semplicità ma che certamente
sono il campo dell'ingegneria del software.

Nell'ambito di questa seconda iterazione di sviluppo, tornando cioè alla
strategia di oggetti \key{Info} autonomi cioè in grado di fornire l'informazione
senza dover accedere a parametri del registro, descriviamo la forma
della lista di selezione.

Per ciascuna informazione, i dati utili sono il nome identificativo e i dati di
eventuali funzioni di formato composti dal nome della funzione seguito da
possibili parametri. Uno schema aiuta a disegnare e implementare la lista:
\begin{lines}
{
    { <id_info>, <id_fmtFunc>, <param>, <param>, ...},
    { <id_info>, <id_fmtFunc>, <param>, <param>, ...},
    ...
}
\end{lines}

Per esempio, se volessimo registrare in prima posizione il nome del compositore
utilizzato per la compilazione, oggetto \key{EngineName}, con un testo tutto in
maiuscolo, formato \key{upper}, all'indice 1 della lista \key{\_logList} ci
dovrebbe essere la tabella:
\begin{lines}
{
    {"EngineName", "upper"},
    ...
}
\end{lines}

Il limite di questa struttura dati è che viene considerata un'unica eventuale
funzione formato, anche se con eventuali parametri secondari. Non è possibile
applicare quindi una funzione formato su un dato già formattato, limitazione
eliminabile modificando la struttura dati.

Cominciamo con il metodo \fn{remove\_from\_log}. Serve a eliminare
un'informazione dalla lista:
\begin{lines}
function register:remove_from_log(id_info)
    assert(type(id_info) == "string")
    local t = self._logList
    local i = assert(
        self:_find_index(t, id_info), id_info.." not found"
    )
    table.remove(t, i)
    return self
end
\end{lines}

Il metodo si basa su una funzione di utilità \fn{\_find\_index} che scorre la
lista e restituisce, se esiste, la posizione di un oggetto \key{Info}:
\begin{lines}
function register:_find_index(t, id_name)
    local i = 1
    local res
    while t[i] do
        if t[i][1] == id_name then
            res = i
            break
        end
        i = i + 1
    end
    return res
end
\end{lines}

Per evitare la creazione di una closure la funzione ausiliaria è stata definita
come membro della libreria, tuttavia la si può sostituire con un dizionario
identificatore/posizione da memorizzare nella tabella del registro e aggiornare
ogni volta che si modifica la lista.

Esaminiamo ora il codice del metodo \fn{add\_to\_log} che aggiunge
un'informazione tra quelle da registrare in uscita inserendola nella posizione
giusta nella lista:
\begin{lines}
function register:add_to_log(info, index)
    local ll = self._logList
    local len = #ll
    local arch = self._iArchive
    local is_pos
    local log_entry = {} -- final data
    if type(info) == "string" then
        is_pos = self:_find_index(ll, info)
        log_entry[1] = info
    else
        assert(type(info) == "table")
        local id = info.id
        if id then -- info is a `Info` object
            self:install(info) -- check and install
            log_entry[1] = id
        else
            -- [index 1] expected a `Info` tab or a string
            local o = info[1]
            local id
            if type(o) == "string" then
                is_pos = self:_find_index(ll, o)
                log_entry[1] = o
                id = o
            else
                assert(type(o) == "table")
                self:install(o) -- check and install
                log_entry[1] = o.id
                id = o.id
            end
            -- format function?
            local id_fmt = info[2] -- name of the format function
            if id_fmt then
                assert(type(id_fmt) == "string")
                -- check if the fmt function exists
                local fmt_repo = arch[id].fmt
                if fmt_repo then
                    assert(fmt_repo[id_fmt])
                else
                    error(
                        "No format function available for '"..
                        id.."' Info"
                    )
                end
                log_entry[2] = id_fmt
                local i = 3
                while info[i] do
                    log_entry[i] = info[i]
                    i = i + 1
                end
            end
        end
    end
    local id = log_entry[1]
    assert(arch[id], "`Info` id '"..id.."' not found")
    if is_pos == nil then -- insert or append
        if index then
            assert(type(index) == "number")
            assert(index >= 1 and index <= len + 1)
            table.insert(ll, index, log_entry)
        else
            ll[#ll + 1] = log_entry
        end
    else -- update an existing log entry
        if index then
            assert(type(index) == "number")
            assert(index >= 1 and index <= len)
            if not (index == is_pos) then
                local o = table.remove(ll, is_pos)
                table.insert(ll, index, o)
                is_pos = index
            end
        end
        local id_fmt = log_entry[2]
        local fmt_repo = arch[id].fmt
        if fmt_repo then
            if id_fmt then
                assert(
                    fmt_repo[id_fmt],
                    "format funcion '"..id_fmt.."' not found"
                )
                local entry = ll[is_pos]
                entry[2] = id_fmt
                local i = 3
                while log_entry[i] do -- copy parameters
                   entry[i] = log_entry[i]
                   i = i + 1
                end
                entry[i] = nil -- safe breaking the param list
            else
                -- rise eventually a warning message
                -- "unuseful insertion call"
            end
        else
            error(
                "Info '"..id..
                "' doesn't have any format function"
            )
        end
    end
    return self
end
\end{lines}

Il metodo è piuttosto articolato perché prevede un elevato grado di polimorfismo
degli argomenti. Con esso possiamo non solo inserire in uscita un'informazione
specificandone o meno la posizione, ma anche installarne e inserirne una nuova
dandone la definizione, e indicare o meno una funzione di formato con o senza
parametri.

In fondo, si installa un nuovo oggetto \key{Info} solo per utilizzarlo e di
questo la libreria tiene conto permettendo all'utente di chiamare un solo
metodo. In altre parole, sarebbe assurdo che l'utente installi un tipo di
informazione senza poi farne uso.

Il codice di \fn{add\_to\_log} è diviso in due parti, nella prima viene
esaminato il dato di input e nella seconda viene modificata la lista di
informazioni di registro. Vediamone alcuni esempi d'uso.

Per spostare nella posizione 1 con il formato \key{upper} l'informazione
\key{EngineName} si chiama:
\begin{lines}
reg:add_to_log({"EngineName", "upper"}, 1)
\end{lines}
e per aggiungere un testo di commento in fondo alla lista creando un nuovo
oggetto \key{Info}:
\begin{lines}
reg:add_to_log{
    id = "CommentText",
    info = "Compilazione di fase due"
}
\end{lines}
e ancora per inserire in posizione 2 un nuovo tipo di informazione con la
larghezza di pagina in punti a tre decimali:
\begin{lines}
local pw = {
    id = "PageWidth",
    info = function () return tex.pagewidth end,
    fmt = {
        point = function (pw, n)
            n = n or 2
            if n == 0 then
                return tostring(p)
            else
                assert(type(n), "number")
                assert(n>0)
                local f1 = "%%0.%df pt"
                local f2 = f1:format(n)
                return f2:format(pw/65536) 
            end
        end
    },
}
reg:add_to_log({pw, "point", 3}, 2)
\end{lines}

Esaminiamo infine il metodo \fn{log}. Non fa altro che leggere la lista delle
informazioni e per ciascuna di esse richiamarne la definizione ottenendo il
valore stringa da registrare su cui eventualmente applicare la funzione di
formato. Il risultato non è la stringa di testo da memorizzare nel file di
registro, ma ancora un tipo strutturato, cioè la lista delle coppie
identificatore oggetto e informazione. Infatti, come vedremo nelle prossime
sezioni il componente che gestisce gli oggetti \key{Info} non si occupa di
interagire con file esterni.
\begin{lines}
function register:log()
    local tline = {}
    local arch = self._iArchive
    for pos, info in ipairs(self._logList) do
        local id = info[1]
        local iobj = assert(arch[id])
        local datalog = assert(iobj.info)
        if type(datalog) == "function" then
            datalog = assert(datalog(iobj))
        end
        datalog = assert(tostring(datalog))
        local id_fmt = info[2]
        if id_fmt then
            assert(type(id_fmt) == "string")
            local fn_fmt = assert(
                iobj.fmt[id_fmt], id_fmt.." field not found"
            )
            assert(type(fn_fmt)=="function")
            datalog = fn_fmt(datalog, table.unpack(info, 3))
        end
        tline[#tline + 1] = {id, datalog}
    end
    return tline
end
\end{lines}

\subsection{Inizializzare la libreria}

Il registro può essere inizializzato o con la lista utente di informazioni in
uscita, oppure se questa non è fornita con una lista di default, tuttavia
l'utente potrebbe dimenticare di inizializzare la lista. Per questo motivo, è
possibile imporre l'inizializzazione nascondendo dietro a una chiamata di
funzione il riferimento del modulo del registro.

Con il metametodo \fn{\_\_call} e le closure di Lua si fa in modo che l'oggetto
restituito al caricamento della libreria non sia il modulo atteso \key{register}
ma un oggetto vuoto con l'unica via della chiamata di inizializzazione:
\begin{lines}
#[tex]
\directlua{
local lib_init = require "lib-logger"
local register = lib_init("EngineName", "ElapsedTime", ...)
}
\end{lines}

Se l'utente non fornisce alcuna lista sarà restituito il riferimento al modulo
con la lista di default, altrimenti con la lista utente, come nell'esempio
sopra.

Questo comportamento si affida alla tecnica delle metatabelle ed è possibile
anche l'alternativa diretta che non esamineremo. Dovreste poter riconoscere il
punto nel codice che richiede la costruzione della closure:
\begin{lines}
local mt = {
    __call = function (_, ilist)
        return register:_define_or_default(ilist)
    end
}

local lib = {}
setmetatable(lib, mt)
return lib
\end{lines}

La funzione ausiliria, sempre dal nome che inizia con un trattino basso a
indicare un membro privato, è la seguente che sfrutta il metodo
\fn{add\_to\_log} perciò l'utente è libero di creare qualsiasi composizione di
informazioni, sia con oggetti \key{Info} predefiniti sia con quelli di propria
definizione, con o senza funzioni di formato:
\begin{lines}
function register:_define_or_default(i_list)
    i_list = i_list or self._default_log_entry_list
    assert(type(i_list) == "table")
    for _, info in ipairs(i_list) do
        self:add_to_log(info)
    end
    return self
end
\end{lines}


\subsection{Metodi di uscita}

Dal registro le informazioni possono essere inviate al file esterno ma nulla
vieta di creare un modulo indipendente che implementi diversi formati e
destinazioni. Per questo nel file \file{lib-logger.lua} si trova il modulo
\key{loglib} che mette a disposizione una o più modalità di output.

I metodi del modulo si attendono in ingresso la lista delle coppie
identificatore e informazione, sempre lo stesso tipo di dato, per poi creare una
nuova linea di testo in un file nel formato CSV\footnote{Comma Separated
Value.}, l'unico effettivamente implementato in questa seconda iterazione di
sviluppo, oppure inviare i dati a un database, o spedirli in un messaggio di
posta elettronica.


\subsection{Il punto di vista dell'utente}

Per usare la libreria l'utente carica il relativo file, inizializza il registro
con la lista di default o con una personalizzata, e seleziona la modalità di
output. Solamente in caso di nuovi oggetti \key{Info} l'utente dovrebbe scrivere
l'opportuno codice Lua.

Si possono fare esperimenti scaricando il pacchetto di seconda generazione
\file{app-reg/logger.sty} e il file di prova \file{app-reg/doc2.tex}, secondo le
istruzioni della sezione Note di lettura del capitolo di apertura della guida.

Una terza iterazione di sviluppo, potrebbe porsi lo scopo di minimizzare il
numero di righe di codice necessarie a implementare un nuovo oggetto \key{Info},
consentendo di evitare ripetizioni. Su questa idea, il punto chiave potrebbe
essere l'inserire gli oggetti \key{Info} in una gerarchia di classi,
implementando in una classe base funzioni di formato che i singoli oggetti
ereditano automaticamente.


\section{Conclusioni}

Dalla progettazione della libreria del registro abbiamo imparato due cose: la
prima è l'importanza di realizzare gli schemi delle strutture dati perché ciò
aiuta sia nel definirle sia nell'implementarne i metodi. In Lua infatti possiamo
contare solo sulla tabella.

La seconda, è che lo sviluppo genera non solo risposte ma anche nuove domande, e
che perciò la complessità del problema può essere affrontata in successive
iterazioni dove una nuova struttura del codice prende il posto di quella dello
stadio precedente, fino a raggiungere il framework che risponde ai migliori
criteri.


% end of file
