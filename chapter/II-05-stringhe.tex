
\chapter{Il tipo stringa}
\label{chFondStringhe}

In Lua le stringhe rappresentano uno dei tipi di base del linguaggio.
Per rappresentare valori stringa letterali ci sono tre diversi delimitatori:
\begin{compactitemize}
\item doppi apici: carattere \verb|"|;
\item apice semplice: carattere \key{'};
\item doppie parentesi quadre: delimitatori \key{[[} e \key{]]} con o senza un
numero corrispondente di caratteri \key{=}, per esempio \key{[==[} e \key{]==]}.
\end{compactitemize}

In una stringa delimitata da doppi apici possiamo inserire liberamente apici
semplici e viceversa, e caratteri non stampabili come il ritorno a capo
(\cs{n}) e la tabulazione (\cs{t}), tramite il carattere di escape backslash
che quindi va inserito esso stesso come doppio backslash (\cs{\textbackslash}):
\sourcecode{file = [[code/e1-005.lua]], run = true}

In Lua non esiste il tipo carattere quindi gli Autori del linguaggio hanno
pensato di utilizzare i delimitatori normalmente destinati a rappresentarne la
forma letterale, per consentire all'utente di creare stringhe contenenti i
delimitatori stessi, senza utilizzare l'escaping.

Sono comunque ammessi i simboli \key{\textbackslash}\verb|"| e
\key{\textbackslash '} che rappresentano i caratteri corrispondenti, come si
vede nella variabile \key{s5} del codice precedente.

Il terzo tipo di delimitatore per le stringhe è una coppia si parentesi quadre e
ha la proprietà di ammettere il ritorno a capo. Si possono così introdurre nel
sorgente interi brani di testo nel quale i caratteri di escaping non saranno
interpretati.
\sourcecode{file = [[code/e1-006.lua]], run = true}

Se per caso nel testo fossero presenti i delimitatori di chiusura è possibile
inserire un numero qualsiasi di caratteri \key{=} tra le parentesi quadre,
purché il numero sia lo stesso per i delimitatori di apertura e chiusura,
esempio:
\sourcecode{ file = [[code/e1-007.lua]]}


\section{Commenti multiriga}

Questi delimitatori variabili con numero qualsiasi di segni \key{=} li troviamo
anche nei commenti multiriga di Lua. Abbiamo incontrato fino a ora i commenti di
riga che si introducono nel codice con un doppio trattino \key{--}.

I commenti multiriga sono comodi quando si vuol escludere dall'esecuzione un
intero blocco di righe: iniziano con i doppi trattini seguiti da un delimitatore
di stringa multiriga e terminano con la corrispondente chiusura:
\begin{lines}
-- questo è un commento di riga

--[[
questo è un commento
multiriga
]]

--[=[
e anche questo è un commento
multiriga
]=]
\end{lines}

Normalmente in Lua i commenti multiriga vengono chiusi premettendo i doppi
trattini anche al gruppo delimitatore di chiusura. Questo è solo un trucco per
riattivare rapidamente il codice eventualmente contenuto nel commento, basta uno
spazio per far trasformare il commento multiriga in uno semplice:
\begin{lines}
--[[ righe di codice non attive
local tab = {}
--]]

-- [[ notare lo spazio dopo i doppi trattini
-- questo codice invece viene eseguito

local tab = {}
--]] -- e questo diventa una normale riga di commento
\end{lines}


\section{Concatenazione stringhe e immutabilità}

In Lua l'operatore \key{..} concatena due stringhe, in questo modo:
\begin{lines}
#[run]
local s1 = "Hello" .. " " .. "world"
local s2 = s1 .. " OK"
s2 = s2 .. "."

print(s1 .. "!")
print(s2)
\end{lines}

Il concetto importante riguardo alle stringhe è se queste siano o no immutabili.
Se non lo sono la concatenazione di stringhe non comporta la creazione di una
nuova stringa ma la modifica in memoria.

In Lua, come in molti altri linguaggi, le stringhe sono invece immutabili. Ciò
significa che una volta create, le stringhe non possono più essere modificate.
Nel codice precedente, l'operazione di concatenare il carattere punto in coda
alla stringa \key{s2}, genera una nuova stringa che è assegnata alla stessa
variabile al posto del valore originale.

Per poche operazioni di concatenazione ciò non è un problema, ma in alcuni casi
invece si. Consideriamo il seguente codice apparentemente innocuo:
\begin{lines}
local s = ""  -- stringa vuota
for i = 1, 100 do
    s = s .. "**"
end
print(#s) -- l'operatore # funziona anche per le stringhe!
--           contandone i byte
\end{lines}

Ma cosa succede in dettaglio? Perché questo codice non è efficiente? A ogni
concatenazione viene creata una nuova stringa. La prima volta vengono copiati
due byte per dare la stringa \verb|"**"|. La seconda iterazione la memoria
copiata sarà di 4 byte, e alla terza di 6 byte, eccetera.

A ogni iterazione la memoria copiata cresce di due byte con il risultato che per
produrre una stringa di 200 asterischi (200 byte) avremo copiato in totale la
memoria equivalente a 10100 byte!

In Java e negli altri linguaggi con stringhe immutabili normalmente si corre ai
ripari mettendo a disposizione una struttura dati o una funzione che risolve il
problema, per esempio un tipo \key{StringBuffer} di Java. In Lua la soluzione è
una funzione della libreria \key{table} che, anticipando rispetto alle nostre
chiaccherate è \fn{table.concat}\luastd{table.concat}:
\begin{lines}
local t = {}
for i = 1, 100 do
    t[#t + 1] = "**"
end
local s = table.concat(t)
print(#s) --> stampa 200
\end{lines}

Le singole stringhe \verb|"**"| sono state memorizzati come elementi di una
tabella in sequenza. La funzione \fn{table.concat} li concatena in una sola
stringa senza usare concatenazioni parziali quindi in modo efficiente. Nel caso
specifico avremo dovuto usare la funzione \fn{string.rep}\luastd{string.rep}
considerando \fn{table.concat} nel caso generale.


\section{Esercizi}

\begin{Exercise}[label=string-01]
Come fare in Lua per creare una stringa letterale contenente sia il
carattere apice semplice che doppio?
\end{Exercise}

\begin{Exercise}[label=string-02]
Quale sarà il risultato dell'esecuzione del seguente codice?
\begin{lines}
local s = "'"..'"'.."ok"..[["']]
print(s)
\end{lines}
\end{Exercise}

\begin{Exercise}[label=string-03]
Creare la stringa \verb|"\/"|.
\end{Exercise}

\begin{Exercise}[label=string-04]
Scrivere un programma che a partire dalla stringa \verb|"*"| crei e stampi la
stringa di 64 asterischi senza utilizzare l'operatore di concatenazione o la
funzione \fn{string.rep}.
\end{Exercise}

\begin{Exercise}[label=string-05]
Scrivere un programma che a partire dalla stringa \verb|"*"| crei e stampi
la stringa di 64 asterischi usando l'operatore di concatenazione il minimo
indispensabile di volte.
\end{Exercise}

% end of file
