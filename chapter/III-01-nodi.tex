

\chapter{Nodi}
\label{iiiChNodi}


\section{Introduzione pratica ai nodi}

In alcuni moduli si incontrano dati che devono essere scritti lettera per
lettera all'interno di una sequenza di caselle di uguale larghezza. Per
riprodurre in \TeX{} una simile struttura dovremo alternare i singoli caratteri
con uno spazio che dipende dalle larghezze dei caratteri adiacenti.

La distanza \( G \) tra i centri dei glifi è constante ed è uguale alla somma di
metà della larghezza del carattere di sinistra, della metà della larghezza del
carattere di destra più lo spazio tra di loro è costante. Se \(w_\mathrm{i} \) e
\(w_\mathrm{i+1} \) sono le larghezze di due glifi consecutivi allora la
distanza \( g \) interna dai loro bordi, quella che dovremo effettivamente
inserire, soddisfa l'equazione:
\[
    G = \frac{w_\mathrm{i}}{2} + g + \frac{w_\mathrm{i+1}}{2}
\]

In \LuaTeX, grazie alla libreria interna \code{node} è possibile costruire ogni
tipo di struttura tipografica pronta per il posizionamento sulla pagina. Queste
strutture sono le stesse che il compositore costruirebbe leggendo il codice di
un sorgente \TeX{}. Creandole con Lua ci sostituiamo a esso nell'eseguire le
prime fasi di lettura dei token, espansione ed esecuzione, arrivando
direttamente alla fase finale detta visuale.

Come si crea un nodo? Ciascun nodo si crea sempre con lo stesso schema semplice
e diretto: si chiama la funzione \fn{new} di \code{node} con l'identificatore
del tipo e sull'oggetto restituito si impostano i parametri.

Per un nodo glifo i parametri obbligatori sono il numero del font, che deve
essere già caricato, e il codice del glifo nel font. La funzione Lua potrebbe
essere:
\begin{lines}
function glyph(code)
    local n = node.new("glyph")
    n.font = font.current()
    n.char = code
    return n.width, n
end
\end{lines}

I nodi dimensione sono chiamati \key{glue} e rappresentano le lunghezze
elastiche grazie alla quali \TeX{} è così flessibile nel comporre la pagina. Nel
nostro esempio interessa solamente la parte rigida e la funzione Lua potrebbe
essere:
\begin{lines}
function glue(x)
    local g = node.new("glue")
    g.width = x
    return g
end
\end{lines}

I nodi si assemblano poi in liste che possono essere assemblate su più livelli.
Un capoverso per esempio, è un unico nodo di tipo scatola verticale che contiene
una lista di scatole orizzontali corrispondenti alle singole righe alternate con
nodi di spaziatura verticale. Ogni scatola orizzontale a sua volta è una lista
di nodi glifo, dimensioni e altri tipi di nodi come quelli per il kerning.

Tecnicamente questi oggetti sono tipi Lua detti \emph{userdata}, con
implementazione in C e fuori dalla gestione automatica della memoria, vanno
quindi usati con cautela facendo in modo che il ciclo di vita sia sempre
\emph{creazione}, \emph{impostazione parametri}, e \emph{invio alla
destinazione}.

La destinazione può essere un registro di tipo scatola oppure il processo di
uscita tramite la funzione \fn{node.write}. Riutilizzare un oggetto nodo già
finalizzato comporta possibili crash del programma.

Altri problemi derivano invece dal non eliminare esplicitamente nodi creati ma
poi non finalizzati. Per esempio, se si deve costruire un oggetto ripetitivo si
può creare la lista dell'elemento e poi finalizzare un suo clone. La copia in un
nuovo oggetto nodo è molto veloce ma alla fine ci si deve ricordare di eliminare
dalla memoria il nodo originale con l'apposita funzione
\fn{node.free}\luastd{node.free} o simili.

Ora che abbiamo definito le funzioni di creazione di glifi e dimensioni non
rimane che scrivere il codice per la creazione della lista. Carattere per
carattere concateneremo un nodo dimensione con un nodo glifo usando la funzione
\fn{node.insert\_after}\luastd{node.insert\_after}:
\begin{lines}
#[indexfile=code/f1-001.tex]
-- filename: code/f1-001.tex
local s = "TESTO"
local gap = tex.sp("20pt")
local w1 = gap
local head, last
for n in string.utfvalues(s) do
    local wn, gn = glyph(n)
    local g = glue(gap - (w1 + wn)/2)
    head, last = node.insert_after(head, last, g)
    head, last = node.insert_after(head, last, gn)
    w1 = wn
end
local hbox = node.hpack(head)
tex.box["sptxt"] = hbox
\end{lines}

La lista che fa capo al nodo iniziale contenuto nella variabile \key{head},
viene inserita in una scatola orizzontale con la funzione
\fn{node.hpack}\luastd{node.hpack} così da poterla memorizzare nel registro
\TeX{} di nome \key{sptxt} da creare in un punto precedente del codice del
sorgente.

A questo punto si può inserire nel sorgente ovunque serva il contenuto del
registro con i classici comandi \LaTeX.


\subsection{Disegno delle caselle}





% end of file
