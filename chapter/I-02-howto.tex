

\chapter{Sul sistema \TeX{} e Lua}
\label{iChExplain}

Questo capitolo fornisce informazioni sulla differenza tra motore di
composizione e formato, e sull'esecuzione di codice Lua all'interno di un
sorgente \TeX.


\section{Motori di composizione e formati}

Un sorgente \TeX{} contiene testo e macro. Il testo formerà i capoversi, i
titoli e il resto del documento, mentre le macro ne stabiliranno l'aspetto e la
struttura. I \emph{motori di composizione} costruiscono il documento eseguendo
macro dette \emph{primitive}, sono cioè implementate in modo nativo per svolgere
compiti elementari, e macro definite con esse per svolgere sequenze ricorrenti.

Le macro sequenze dette di alto livello, possono costituire una vera e propria
libreria il cui codice è generalmente scritto da veri esperti, che prende il
nome di \emph{formato}, come \LaTeX{} o Con\TeX t, perché per esse ne è stata
ricercata la coerenza interna e sono state stabilite nuove regole di sintassi
comune.

Per esempio, nel formato \LaTeX{} è presente il concetto di \emph{ambiente} con
la coppia di macro di delimitazione \cs{begin\{\}} e \cs{end\{\}}, oppure quello
di suddivisione del sorgente in preambolo e in corpo del documento.

Con le macro primitive o con quelle del formato, l'utente può implementare
funzionalità specifiche oppure caricare macro definite in pacchetti o moduli che
integrano o estendono quelle del formato.

I motori di composizione hanno l'abilità nella fase iniziale della compilazione
di caricare il formato in forma precompilata. Se si avvia un qualsiasi programma
di composizione della famiglia \TeX{} verrà caricato di default il formato più
semplice chiamato \emph{plain}. Se si vuole invece utilizzarne uno diverso, per
esempio il più diffuso \LaTeX{}, occorre specificarne il nome al compilatore con
l'opzione \texttt{-{}-fmt} nel comando al terminale.

Tuttavia, data l'importanza per gli utenti dei formati di alto livello, sono
stati predisposti appositi comandi scorciatoia. Per esempio il programma
\prog{pdflatex} rimanda all'effettivo motore di composizione \prog{pdftex} con
l'istruzione di caricare il formato \LaTeX{}, e così anche per \prog{lualatex}
con \prog{luatex}.

Per essere chiari, i due comandi seguenti sono del tutto equivalenti, nel primo
si invoca direttamente il motore di composizione mentre nel secondo lo si fa con
una scorciatoia:
\begin{Verbatim}[numbers=none]
$ pdftex --fmt=pdflatex mydoc
$ pdflatex mydoc
\end{Verbatim}

Quando il nome del formato corrisponde al nome del programma scorciatoia, non li
si deve confondere: il programma \prog{latex} utilizza ancora il compositore
\prog{pdftex} con il formato \LaTeX{} con uscita in DVI\footnote{DVI sta per
Device Indipendent e oggi è in disuso a favore del formato PDF. Tuttavia è utile
far notare che un programma compositore può fornire anche più formati di
uscita.}.

I file binari dei formati sono usualmente generati o rigenerati in automatico
dalle utility della distribuzione nel momento dell'installazione o quando
l'utente esegue un aggiornamento del sistema che contiene modifiche alle macro
del formato, perciò è molto raro che ci si trovi nella situazione di doverne
generare uno.

Riassumendo, i motori di composizione sono programmi di composizione tipografica
mentre i formati sono insiemi coerenti di macro basate sulle primitive del
compositore. I nomi dei programmi disponibili nel sistema \TeX{} possono
confondere se non si conosce questa importante distinzione: alcuni di essi sono
infatti comandi scorciatoia per eseguire un motore di composizione con un
particolare formato.


\subsection{Compositori Lua-powered}

\LuaTeX{} è un programma che elabora un file di testo contenente codice \TeX{}
per comporne il corrispondente file PDF, l'uscita di default, quindi è un motore
di composizione. Nella famiglia \TeX{} ci sono almeno altri due compositori
dotati dell'interprete Lua, LuaHB\TeX{} e LuaJIT\TeX{}.

Tutti e tre questi compositori possono eseguire il formato \LaTeX. Come detto in
apertura di capitolo, il programma \prog{lualatex} esegue un compositore che
carica il formato \LaTeX. Dalla TeX Live 2020 questo compositore è
\prog{luahbtex}. Per rendercene conto basta scrivere in un terminale il nome del
programma con l'opzione \code{-{-}version}:
\begin{Verbatim}[numbers=none]
> lualatex --version
This is LuaHBTeX, Version 1.13.0 (TeX Live 2021)

Execute  'luahbtex --credits'  for credits and version details.
...
\end{Verbatim}

Per compilare un sorgente \LaTeX{} con \LuaLaTeX{}, con la TeX Live 2021 i due
comandi equivalenti sono:
\begin{Verbatim}[numbers=none]
$ luahbtex --fmt=lualatex mydoc
$ lualatex mydoc
\end{Verbatim}

Per ulteriore informazione, \prog{luahbtex} è il motore di composizione
\prog{luatex} in cui è stato sostituito il componente per il calcolo della forma
dei font con il modulo HarfBuzz \cite{lib:harfbuzz}, mentre \prog{luajittex} è
un'altra variante di \prog{luatex} in cui l'interprete Lua è stato sostituito
con LuaJIT \cite{prg:luajit} un'implementazione indipendente più veloce
dell'interprete ufficiale che sfrutta le tecniche di compilazione denominate
\emph{Just In Time}.

In generale un sorgente \TeX{} che contiene codice Lua viene correttamente
compilato da qualsiasi dei tre compositori grazie al mantenimento della
compatibilità.


\subsection{Codice Lua in \LuaTeX}
\label{iChLuaInLuaTeX}

Per illustrare l'esecuzione di codice Lua all'interno di un sorgente \LuaTeX,
consideriamo la stampa nell'output di console di un testo semplice. Il codice
Lua va inserito come argomento della primitiva \cs{directlua}:
\begin{lines}
#[tex]
% !TeX program = LuaHBTeX
\directlua{
    print("Hello World!")
}
\bye
\end{lines}

Se il sorgente è memorizzato nel file \texttt{primo.tex}, possiamo verificare
quanto previsto in un terminale lanciando il comando:
\begin{Verbatim}[numbers=none]
$ luahbtex primo
\end{Verbatim}
e per il sistema operativo Linux e la distribuzione TeX Live 2021, l'output
in console è:
\begin{Verbatim}
This is LuaHBTeX, Version 1.13.0 (TeX Live 2021) 
restricted system commands enabled.
(./primo.texHello World!
)
warning  (pdf backend): no pages of output.
Transcript written on t.log.  
\end{Verbatim}

Il testo uscirà tra gli altri messaggi di output senza che sia stato prodotto un
file PDF. Ciò significa che \cs{directlua} è una macro espandibile con risultato
vuoto.


\subsection{Codice Lua in Lua\LaTeX}
\label{iSecLuaInLuaLaTeX}

Con Lua\LaTeX{} si ottiene lo stesso risultato ma con il sorgente scritto nella
sintassi \LaTeX, ovvero:
\begin{lines}
#[tex]
% !TeX program = LuaLaTeX
\documentclass{article}
\directlua{
    print("Hello World!")
}
\begin{document}
\end{document}
\end{lines}
e questa volta il comando di compilazione è:
\begin{Verbatim}[numbers=none]
$ lualatex primo
\end{Verbatim}

Avremo potuto inserire la macro all'interno dell'ambiente \amb{document} anziché
nel preambolo. Quando \TeX{} incontra \cs{directlua} ne \emph{espande}
l'argomento e passa a Lua il controllo che esegue immediatamente il codice per
poi restituirlo di nuovo a \TeX{} al termine dell'esecuzione.


\section{La primitiva \key{directlua}}
\label{iSecDirectLua}

Abbiamo ora tutte le informazioni per svolgere alcuni esercizi e applicazioni.
Non rimane che ricordare che il principale modo di eseguire codice Lua in
\LuaTeX{} è assegnarlo come argomento alla macro \cs{directlua}. Quello che
avviene è stabilito da queste regole:
\begin{compactenumerate}
\item l'argomento di \cs{directlua} viene espanso ed eseguito come blocco, può
quindi contenere macro o argomenti macro con un testo di sostituzione;

\item le variabili locali hanno validità solo all'interno del gruppo/blocco
mentre quelle globali saranno valide anche in quelli di successive
\cs{directlua};

\item l'espansione di \cs{directlua} è vuota;
\end{compactenumerate}

Per documentarsi nel dettaglio esiste un'ottima risorsa su Overleaf, la popolare
piattaforma di compilazione e gestione progetti \LaTeX{} online, a
\href{https://cs.overleaf.com/learn/latex/Articles/An_Introduction_to_LuaTeX_(Part_2):_Understanding_%5Cdirectlua}{questo indirizzo web}.


\section{Passaggio di dati}
\label{iSecPassaggioDati}

La comunicazione dati bidirezionale tra \TeX{} e Lua può avvenire con la
tecnologia dei nodi, come vedremo, certamente quella più avanzata e complessa
ma non è l'unica: i dati possono arrivare a Lua tramite l'espansione, mentre
nella direzione opposta è possibile scrivere del testo nella lista di input
del compositore con le funzioni della famiglia \fn{tex.print}.

Interrompete la lettura della guida per provare a scrivere la funzione Lua
\fn{fact} che calcola il fattoriale di un intero. La useremo per dimostrare come
avviene questo scambio di dati. L'idea è definire un nuovo comando che stampi il
fattoriale del numero argomento:
\begin{lines}
#[tex]
\newcommand{\fattoriale}[1]{\directlua{
    local n = #1
    tex.print(tostring(fact(n)))
}}
\end{lines}

Quando \TeX{} incontra una macro utente con un segnaposto\footnote{Fate
riferimento ha una buona guida per \LaTeX{} per saperne di più sulla definizione
di macro utente.} lo sostituisce con i dati corrispondenti a un singolo token
oppure a un gruppo di token tra parentesi graffe. Per questo quando nel sorgente
scriviamo \cs{fattoriale}\code{\{5\}}, dopo la sostituzione il codice Lua
effettivamente eseguito sarà:
\begin{lines}
local n = 5
tex.print(tostring(fact(n)))
\end{lines}

La funzione \fn{tex.print} inserisce l'argomento stringa nell'input d'ingresso
come se fossero stati letti dal sorgente. Quando a termine l'esecuzione del
blocco di codice Lua della macro \cs{fattoriale} i successivi caratteri che si
troverà a elaborare il compositore saranno le cifre 1, 2, e 0 e l'espansione
della macro sarà stata vuota.

Il listato del sorgente completo dell'esercizio proposto da confrontare con la
vostra personale soluzione è:
\sourcecode{file = [[app-tut/E1-001-fatt.tex]]}


\section{Globale o locale}
\label{iSecGlobaleLocale}

Prendendo spunto ancora dall'esempio precedente, soffermiamoci sul comportamento
nei diversi blocchi \cs{directlua} delle definizioni locali e globali: come
descritto dalle specifiche di Lua, tutto quello che è locale a un blocco non è
più disponibile al di fuori di esso. Nel contesto di \LuaTeX{} il codice
contenuto in una macro \cs{directlua} forma un blocco.

Per questo motivo se separassimo il codice che calcola il fattoriale dalla
definizione della macro utente \cs{fattoriale}, per non ricevere un errore di
chiamata di un valore \key{nil} dal secondo blocco dovremo definire la funzione
come globale come in:
\sourcecode{file = [[app-tut/E1-002-fatt.tex]]}

Le definizioni globali tuttavia possono determinare errori dovuti alla
collisione dei nomi definiti in altre parti del sorgente essendo l'ambiente
appunto globale e quindi unico. Spesso, come per gli esempi della guida, non è
un problema ma è buona norma definire una tabella di \emph{namespace} dove
memorizzare le funzioni limitando la collisione dei nomi solamente al nome del
riferimento alla tabella.

Questa buona prassi diviene \emph{obbligatoria} quando stiamo scrivendo codice
applicativo in contesti di utilizzo reale.

Come ulteriore esempio, nel listato\footnote{Questo listato è interessante
perché migliorabile sia per eliminare le ripetizioni di codice a cui è costretto
l'utente sia nell'efficienza di esecuzione.} che segue è mostrato come si possa
formattare il numero del fattoriale con il pacchetto \pack{siunitx}:
\sourcecode{file = [[app-tut/E1-003-fatt.tex]]}


\section{Espansione di macro}

La comunicazione di un dato tra \TeX{} e Lua può avvenire anche per espansione
di macro. Come esempio minimo consideriamo il seguente sorgente \LuaTeX{} che
stampa l'ora di inizio della compilazione avvalendosi del contatore \cs{time}
che indica i minuti trascorsi dall'inizio del giorno:
\sourcecode{file = [[app-tut/E1-004-time.tex]]}

Al termine dell'espansione l'istruzione di assegnazione della variabile numerica
\key{time} sarà l'effettiva e corretta sintassi Lua. Per il formato \LuaLaTeX{}
lo stesso file potrebbe essere:
\sourcecode{file = [[app-tut/E1-005-time.tex]]}

Nel capitolo~\ref{iiiChRegistro} vedremo che anche con la libreria standard di
Lua è possibile ottenere la data e l'ora corrente.


\section{Caratteri speciali}

Alcuni simboli hanno un diverso significato per \TeX{} e per Lua, per esempio il
carattere cancelletto, lo stesso backslash o il simbolo di percento. \LuaTeX{}
non si occupa direttamente di modificare il significato dei simboli che si
sovrappongono.

Le descrizioni dei conseguenti errori possono essere criptici ma ci sono almeno
quattro diverse soluzioni:
\begin{compactitemize}
\item scrivere il codice Lua in file esterni,
\item utilizzare codice \TeX{} per gestire l'espansione dei simboli o i codici
di categoria,
\item usare i codici ASCII per creare stringhe che contengono i simboli, con la
funzione \fn{string.char},
\item utilizzare il pacchetto \pack{luacode}.
\end{compactitemize}

Come preferenza personale non utilizzo l'ottimo pacchetto \pack{luacode},
anche per evitare una dipendenza in più. Tuttavia non appena il codice Lua
cresce in numero di linee o diventa un componente di un progetto reale, quasi
sempre si devono utilizzare file esterni per una più agevole gestione del
progetto.

Rimando alla documentazione del pacchetto \pack{luacode} per i dettagli sulla
collisione dei simboli. Vedrete che esso offre due comandi e due ambienti per
poter far inserire codice Lua in un sorgente \LaTeX{} con varie impostazioni di
caratteri.

Nel codice Lua si possono usare i commenti in stile \TeX{} con il simbolo del
percento perché il processo d'espansione li elimina \emph{prima} di passare il
codice all'interprete, ma non si possono usare i commenti in stile Lua con il
doppio trattino a meno che non siano all'ultima linea. Infatti, per la stessa
espansione tutto il codice Lua nella macro \cs{directlua} è inviato
all'interprete come un unica linea di codice perciò il commento dopo un doppio
trattino si estenderebbe non solo a fine riga ma a tutto il codice che segue.

Per fortuna la grammatica Lua a differenza di Python, consente la libera
scrittura e i rientri del codice altrimenti questo meccanismo non potrebbe
funzionare.


\section{Le librerie disponibili in \LuaTeX}

Agli usuali moduli della libreria standard di Lua, sono stati aggiunti in
\LuaTeX{}, così come negli altri motori di composizione estesi, speciali
funzionalità dedicate al controllo dello stato interno e alla creazione di
elementi tipografici che chiameremo librerie interne, e funzionalità generiche
che chiameremo librerie esterne.

Tutte queste librerie formano un nutrito elenco come riporta dettagliatamente il
manuale dell'utente. Tra quelle interne più note ricordiamo:
\begin{compactitemize}
\item \code{node}, gestione dei nodi tipografici;
\item \code{token}, scansione dei token per la raccolta di argomenti, creazione
di macro da codice, e molto altro;
\item \code{tex}, funzioni di ingresso dell'input del compositore, accesso a
registri e parametri interi;
\item \code{harfbuzz}, informazioni sui font e i singoli glifi (solo per il
compositore \prog{luahbtex});
\item \code{mplib}, \hologo{METAPOST} per la grafica vettoriale.
\end{compactitemize}

Tra le librerie esterne troviamo:
\begin{compactitemize}
\item \code{lfs}, Lua File System per visitare l'albero delle directory e
ottenere informazioni sui file;
\item \code{socket}, connessioni di rete con protocolli SMTP, HTTP e FTP;
\item \code{luazip}, compressione e decompressione di file e cartelle;
\item \code{lpeg}, Parsing Expression Grammar, dati da testo strutturato.
\end{compactitemize}

Alcune di queste librerie richiedono l'autorizzazione esplicita dell'utente
all'esecuzione per mezzo di opzioni di compilazione.


\section{\LaTeX{} con \code{pdflatex} o \code{lualatex}}
\label{iSecPassareALuaTeX}

I sorgenti \LaTeX{} possono essere compilati sia con il tradizionale motore
\code{pdftex} che con il compositore \code{luahbtex} con minime modifiche.
Vediamo quali sono affiancando i preamboli minimi nei due casi:\\
\begin{tcolorbox}[width=(\linewidth/2-1mm),equal height group=A,nobeforeafter]
\begin{lines}
#[tex]
#[tcolorbox]
% !TeX program = pdfLaTeX
\documentclass{article}
\usepackage[T1]{fontenc}
\usepackage[utf8]{inputenc}
\usepackage[italian]{babel}
\end{lines}
\end{tcolorbox}
%
\begin{tcolorbox}[width=(\linewidth/2-1mm),equal height group=A,nobeforeafter]
\begin{lines}
#[tex]
#[tcolorbox]
% !TeX program = LuaLaTeX
\documentclass{article}
\usepackage{fontspec}
\usepackage[italian]{babel}
\end{lines}
\end{tcolorbox}

Nel preambolo di un sorgente scritto per \code{pdflatex} si caricano tre
pacchetti fondamentali \pack{fontenc} per la codifica dei font, \pack{inputenc}
per quella del sorgente, e \pack{babel} per le lingue.

Oggi la stragrande maggioranza degli utenti, e se non siete tra loro fatelo
subito, codificano i file sorgenti con UTF-8. Il testo in Unicode infatti
elimina ogni incoveniente quando si condividono i file tra diversi sistemi
operativi\footnote{Per informazioni dettagliate sulle codifiche rimando alla
guida tematica \GuIT{}
\href{http://www.guitex.org/home/images/doc/GuideGuIT/introcodifiche.pdf}{%
Introduzione alle codifiche in entrata e uscita} di Claudio Beccari e Tommaso
Gordini \cite{gt:codifiche}.}.

Per \LuaTeX{} è obbligatoria la codifica UTF-8 perciò la prima modifica al
preambolo è semplicemente l'eliminazione del pacchetto \pack{inputenc}. Se i
vostri file non fossero codificati UTF-8, si può utilizzare un tool come
\code{iconv} come spiegato nella guida tematica menzionata in nota.

Per i font non è più necessario il pacchetto \pack{fontenc}. Per selezionarne
nel documento si usa il pacchetto \pack{fontspec} e le sue macro
\cs{setmainfont}, \cs{setmonofont} eccetera. I font matematici vanno invece
gestiti con il pacchetto \pack{unicode-math} e con la macro \cs{setmathfont}.

L'immediato vantaggio di passare da pdf\LaTeX{} a \LuaLaTeX{} è proprio nel
poter comporre il documento con i font Open Type. Per la mia esperienza la
compatibilità con i numerosi pacchetti \LaTeX{} è ottima mentre è possibile
utilizzare i pacchetti scritti in Lua ormai numerosi.

Ulteriori informazioni si possono trovare nel documento \texttt{lualatex-doc},
al solito disponibile a video con l'utility \code{texdoc}, dove viene
considerato anche il compositore \XeLaTeX{}.

% end of file
