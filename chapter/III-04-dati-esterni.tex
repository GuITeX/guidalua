

\chapter{Dati esterni}
\label{iiiChDati}

Il tema molto vasto di questo capitolo è la tecnologia che ci permette di
includere nei documenti dati esterni. Può capitare di voler comporre un buon
numero di documenti che condividono un insieme di dati, oppure di comporre un
documento che riporti tabelle con dati sperimentali residenti in file esterni.

Nel primo caso, vorremmo evitare di digitare sempre gli stessi dati in ciascun
documento con il rischio di sbagliare, mentre nel secondo cercheremo una
procedura che carichi in automatico i dati all'interno del documento
presentandoli con aspetto professionale.

Oppure ancora, potremo trovarci in ambito aziendale dove i dati che dovremo
includere nei documenti sono responsabilità di unità operative diverse dalla
nostra. Cercheremo un modo per condividere le informazioni in modo efficiente e
sicuro.

Gli scenari sono quindi molto vari e di conseguenza le tecnologie utili possono
essere le più diverse, per esempio nel fornire o meno la lettura sincrona dei
dati da un database. Comunque sia presenterò degli esempi concreti solamente per
contesti di gestione dati in cui è il motore di composizione a compiere le
elaborazioni affinché i dati siano disponibili all'interno del sorgente. A
questo ambito daremo il nome di \emph{progetti documentali}.

Per esempio, non è un progetto di documento quello in cui il programma elabora i
dati e crea il relativo file di testo che rappresenta il sorgente \TeX{}.
Infatti, così facendo non solo sarebbe possibile compilare il sorgente con i
compositori tradizionali come PDF\LaTeX{} o Xe\LaTeX{}, e quindi non saremo in
tema, ma non lavoreremo scrivendo il sorgente in modo diretto, ovvero nel
contesto ideale per comporre documenti complessi che contengono variazioni uno
rispetto all'altro sia nel contenuto sia nella struttura.


\section{Lua table}

Il modo più semplice per includere dati esterni è quello che non richiede la
scrittura di alcun modulo per caricare i dati. In altre parole, i dati devono
essere scritti secondo un formato nativo del linguaggio, nel nostro caso Lua.

Questo formato nativo per Lua non può che essere il costruttore di tabelle, che
abbiamo incontrato al capitolo~\ref{iiChTabella}. Applicheremo questa fantastica
tecnica per realizzare modi diversi di presentare gli stessi dati.








%\subsection{Lua type}


%\subsection{SQLite3 e i quadri economici}


%\subsection{Matematica simbolica}

% end of file
