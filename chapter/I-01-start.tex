
% codename "magpie"

\chapter{Let's start with Lua}
\label{iChStart}

Consideriamo un sorgente \LaTeX{} minimale:
\begin{lines}
#[tex]
% !TeX program = LuaLaTeX
\documentclass{article}
\begin{document}
\end{document}
\end{lines}

I suoi elementi sono l'utile commento che informa l'editor di quale compositore
usare per la compilazione detto riga magica, la dichiarazione di classe e infine
il corpo vuoto del documento.

Essendo il compositore \LuaLaTeX{} abbiamo a disposizione la nuova primitiva
\cs{directlua}. Potremo inserirla più volte in qualsiasi punto del sorgente. Se
la inserissimo nel preambolo del sorgente minimo avremo:

\CLRmarginpar{Lua in \LaTeX\\
\gotosec{} \ref{iSecLuaInLuaLaTeX}}[true]% true option means +16pt of vspace
%
\CLRmarginpar{\code{\backslash directlua}\\
\gotosec{} \ref{iSecDirectLua}}
%
\begin{lines}
#[tex]
% !TeX program = LuaLaTeX
\documentclass{article}
\directlua{<codice Lua>}
\begin{document}
\end{document}
\end{lines}

Qui sta il punto. La primitiva \cs{directlua} è come se commutasse il modo di
compilazione da \LaTeX{} a Lua, un linguaggio di programmazione per scopi
generali, efficiente e portabile. Qualsiasi cosa faccia il \code{<codice Lua>}
racchiuso tra graffe esso sarà eseguito dall'interprete Lua e non da \TeX.

Quali sono le conseguenze? Qual è l'utilità dell'interprete Lua?

\CLRmarginpar{Passare a \LuaLaTeX\\
\gotosec{} \ref{iSecPassareALuaTeX}}
%
La risposta a queste ottime domande è duplice. La prima è che il motore
\prog{luatex} rispetto a \prog{pdftex} è stato migliorato in molti aspetti, per
esempio è in grado di gestire i font di tipo Open Type nativamente. Solo questo
lo rende molto interessante per l'utente.

La seconda è che Lua può \emph{integrare} l'elaborazione \TeX{} eseguendo
compiti complessi come il calcolo numerico, l'interrogazione di database,
l'editing del testo, l'interazione con i servizi di rete, la costruzione di
oggetti tipografici di complessità arbitraria, la grafica vettoriale, eccetera
eccetera.


\section{Un primo esempio}

Desideriamo che nel nostro documento ci sia un testo allineato sulla diagonale
della pagina. Per ottenerlo usando il comando \cs{rotatebox} del pacchetto
\pack{graphicx} è necessario calcolare il valore dell'angolo di rotazione in
gradi conoscendo la misura di larghezza e altezza della pagina e questo implica
il calcolo dell'arcotangente del rapporto. Nel caso di questa guida:
\[
    \alpha = \frac{180}{\pi}\arctan{\frac{h}{w}} = \directlua{
        local w = tex.pagewidth
        local h = tex.pageheight
        local angle = (180/math.pi) * math.atan(h/w)
        tex.print(tostring(angle))
    }
\]

Applicando qualche metodo numerico o pacchetto, sono certo che \TeX{} sia in
grado di calcolare l'angolo con sufficiente precisione pur non disponendo di
alcuna libreria in virgola mobile. Con Lua invece è un calcolo diretto in
floating point:

\CLRmarginpar{Tabella di Lua\\
\gotosec{} \ref{iiChTabella}}[true]% true option means +16pt of vspace
%
\CLRmarginpar{Libreria \code{math}\\
\gotosec{} \ref{iiSecMathLibrary}}
%
\begin{lines}
#[tex]
#[indexfile=code/e0-001.tex]
% filename: code/e0-001.tex
\directlua{
    local w = tex.pagewidth
    local h = tex.pageheight
    local alpha = (180/math.pi) * math.atan(h/w)
    token.set_macro('angdeg', alpha)
}
\end{lines}

Qualche commento sulle quattro linee di codice Lua\footnote{Il listato completo
del sorgente è scaricabile dalla directory \code{code} con il nome di
\file{e0-001.tex} dall'archivio dei sorgenti della guida.}: le prime due
leggono le dimensioni della pagina in due variabili locali --- rappresentano
perciò i dati di ingresso dell'elaborazione --- ma attenzione i campi
\key{pagewidth} e \key{pageheight} della tabella \key{tex}, senza entrare ora
nel dettaglio di cosa sia un campo o una tabella in Lua, non sono valori in sola
lettura di valore pari a quello dei corrispondenti parametri dimensionali. Sono
bensì le dimensioni \cs{pagewidth} e \cs{pageheight} stesse.

La terza linea esegue il calcolo in virgola mobile in doppia precisione e
l'ultima crea una macro \cs{angdeg} con il valore dell'angolo --- rappresenta
percò l'output dell'elaborazione --- esattamente come se fosse stata creata con
\cs{def}.

Nel motore di composizione, Lua ha accesso diretto ai registri dimensionali e ai
contatori e può creare macro. Significa che l'implementazione del motore di
composizione stesso è nativamente interfacciato con Lua.


\newcommand{\ordinanomi}[1]{\directlua{
    local list = {#1}
    table.sort(list)
    tex.print(table.concat(list, ', '))
}}

\section{Un secondo esempio}

Abbiamo una lista dei nomi di partecipanti a un evento, ma vorremmo evitare di
ordinarli a mano. La lista si presenta come un elenco separato da virgole dei
nomi (già ordinati): \ordinanomi{'Giacomo', 'Tonio', 'Elena', 'Anna', 'Stefano',
'Milena', 'Federico', 'Giovanni', 'Massimo', 'Basilio', 'Beatrice'}.

In Lua bastano tre righe di codice per definire il comando \cs{ordina} che ha
come unico argomento la lista:

\CLRmarginpar{Libreria \code{table}\\
\gotosec{} \ref{iiSecTableLibrary}}[true]% true option means +16pt of vspace
%
\CLRmarginpar{\fn{tex.print}\\
\gotosec{} \ref{iSecPassaggioDati}}
%
\begin{lines}
#[tex]
#[indexfile=code/e0-002.tex]
% filename: code/e0-002.tex
\newcommand{\ordina}[1]{\directlua{
    local list = {#1}
    table.sort(list)
    tex.print(table.concat(list, ", "))
}}
\end{lines}

Il dato in ingresso, la lista dei nomi non ordinata, viene costruito grazie al
meccanismo della sostituzione degli argomenti di una macro. Il dato in uscita,
la lista dei nomi ordinata, perviene al compositore come testo in lettura come
se fosse stato presente nel file sorgente.

I tre passi del codice sono la costruzione della lista con una tabella,
l'ordinamento degli elementi, e l'invio al buffer di lettura dei caratteri gli
elementi concatenati e separati con una virgola.

Anche qui, se non fosse chiaro quanto avete appena letto è perché vorrei
introdurre prima come si definisce un comando \LaTeX{} che contiene solamente
codice Lua rispetto ai concetti del linguaggio che comunque potete raggiungere
con i riferimenti al margine ai capitoli della parte~\ref{partFoundation}.

Nel documento un tale comando si usa in questo modo, con i nomi necessariamente
da delimitare come stringhe:
\begin{lines}
#[tex]
#[indexfile=code/e0-002.tex]
% filename: code/e0-002.tex
\ordina{"Giacomo", "Tonio", "Elena", "Anna", "Stefano",
    "Milena", "Federico", "Giovanni", "Massimo", "Basilio",
    "Beatrice"}
\end{lines}

Con qualche riga di codice in più, sarebbe semplice eliminare dall'input la
delimitazione dei nomi come stringhe, oppure inserire un criterio di ordinamento
più complesso come quello per cognome e nome.


\section{Direzioni di lettura}

Il prossimo capitolo~\ref{iChExplain} si occupa di spiegare le particolarità dei
motori di composizione Lua-powered distinguendoli dal formato, e di come possano
eseguire sorgenti \LaTeX{}.

I capitoli dal~\ref{iiChAssignment} fino al~\ref{iiChOop} trattano delle basi
del linguaggio Lua.

I tutorial applicativi in Lua

comporre documenti \TeX. Implementeremo una \emph{calcolatrice}
(capitolo~\ref{iChCalcolatrice}) e comporremo una tabella numerica ripetitiva
(capitolo~\ref{iChTabellaPesi}).


% end of file
