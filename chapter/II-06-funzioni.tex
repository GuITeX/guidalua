
\chapter{Funzioni}
\label{chFondFunzioni}

Le funzioni sono il principale mezzo di astrazione e organizzazione del codice.

Coerentemente con il resto del linguaggio la sintassi dichiarativa di una
funzione comprende due parole chiave che servono per delimitare il blocco di
codice: \key{function} ed \key{end}. Una funzione può restituire dati tramite la
parola chiave \key{return}.

Come primo esempio, consideriamo una funzione per calcolare l'ennesimo numero
della \href{http://it.wikipedia.org/wiki/Successione_di_Fibonacci}{serie di
Fibonacci}. Un elemento si ottiene sommando i due precedenti elementi avendo
posto uguale a 1 i primi due:
\sourcecode{file = [[code/e2-001.lua]], select = [[uno]]}

Con le regole dell'assegnazione multipla una funzione può accettare più
argomenti. Se gli argomenti passati sono in eccesso rispetto a quelli che essa
prevede, quelli in più verranno ignorati. Viceversa, se gli argomenti sono
inferiori a quelli previsti allora a quelli mancanti verrà assegnato il valore
\key{nil}\luak{nil}.

Le stesse regole valgono anche per i dati di uscita quando la funzione è usata
come espressione in un'istruzione di assegnazione. Dopo l'istruzione
\key{return}\luak{return} si può scrivere una lista delle espressioni separate
da virgole che saranno assegnate alle corrispondenti variabili.

Per esempio, potremo modificare la funzione precedente per restituire anche la
somma dei primi \( n \) numeri di Fibonacci oltre che l'ennesimo elemento della
serie stessa e considerare un valore di default se l'argomento è \key{nil}:
\sourcecode{file = [[code/e2-001.lua]], select = [[due]], run = true,}


\section{Funzioni: valori di prima classe, I}

In Lua le funzioni sono un tipo. Possono essere assegnate a una variabile,
passate come argomento a un'altra funzione e restituite da una funzione come
valore, una proprietà che non si trova spesso nei linguaggi di scripting e che
offre più flessibilità al codice.

A ben vedere in Lua tutte le funzioni sono memorizzate in variabili e di per se
non hanno un nome. Il modo di definizione è quindi la \emph{sintassi anonima}:
\begin{lines}
add = function (a, b)
    return a + b
end
print(add(45.4564, 161.486))
\end{lines}

Questa importante proprietà si riassume dicendo che in Lua le funzioni sono
valori di \emph{prima classe}: sono assegnate a variabili e non hanno un nome
esattamente come non lo hanno gli altri tipi come numeri e stringhe.

Per comodità il linguaggio ammette anche la sintassi classica di definizione:
\begin{lines}
function variable_name(args)
    -- function body
end
\end{lines}
L'interprete Lua la tradurrà automaticamente nel codice equivalente in sintassi
anonima, una caratteristica nascosta detta \emph{zucchero sintattico}:
\begin{lines}
variable_name = function (args)
    -- function body
end
\end{lines}


\section{Funzioni: valori di prima classe, II}

Un esempio di funzione con un argomento funzione è il seguente, dove si vuole
eseguire per un certo numero di volte consecutivamente la funzione argomento:
\sourcecode{
    file = [[code/e3-001.lua]],
    select = [[uno]],
}

Molto interessante. Nell'ultima riga di codice l'argomento è una funzione
definita in sintassi anonima (che verrà eseguita 12 volte).

Essendo le funzioni valori di prima classe è possibile riassegnare variabili per
cambiarne il significato, come per esempio con la funzione predefinita
\fn{print}:
\begin{lines}
#[run]
local orig_print = print
print = function (n)
    orig_print("Argomento funzione -> "..n)
end

print(12)
\end{lines}


\section{Tabelle e funzioni}

Se una tabella può contenere chiavi con qualsiasi valore allora può contenere
anche funzioni! Le sintassi previste sono queste, esplicitate con il codice
riportato di seguito:
\begin{compactitemize}
\item assegnare la variabile di funzione a una chiave di tabella;
\item assegnare direttamente la chiave di tabella con la definizione di funzione
in sintassi anonima;
\item usare il costruttore di tabelle per assegnare funzioni in sintassi
anonima.
\end{compactitemize}
\sourcecode{
    file = [[code/e3-001.lua]],
    select = [[due]],
}

Con questo meccanismo una tabella può svolgere il ruolo di \emph{modulo}
memorizzando funzioni utili in un gruppo. In effetti la libreria standard
di Lua si presenta all'utente proprio in questo modo.


\section{Numero di argomenti variabile}

Una funzione può ricevere un numero variabile di argomenti rappresentati da
\key{...}\luas{...} tre caratteri punto consecutivi\footnote{Questa
caratteristica è chiamata \emph{variadic function}.}. Nel corpo della funzione i
tre punti rappresenteranno la lista degli argomenti, dunque possiamo o costruire
con essi una tabella oppure effettuare un'assegnazione multipla.

Un esempio è una funzione che restituisce la somma di tutti gli argomenti
numerici:
\sourcecode{
    file = [[code/e3-001.lua]],
    select = [[tre]],
}

Nell'ultima riga di codice si può notare che anche la funzione predefinita
\fn{print} accetta un numero variabile di argomenti.

Il meccanismo è ancora più flessibile perché tra i primi argomenti vi possono
essere variabili "fisse". Per esempio il primo parametro potrebbe essere un
moltiplicatore:

\sourcecode{
    file = [[code/e3-001.lua]],
    select = [[quattro]],
}

La funzione predefinita \fn{select}\luastd{select} consente di accedere alla
lista degli argomenti in dettaglio. Infatti nel codice precedente, se tra gli
argomenti compare un valore \key{nil} avremo problemi ad accedere ai valori
successivi perché --- come sappiamo già --- l'operatore di lunghezza \key{\#}
determina la lunghezza della sequenza in base ai bordi cioè alle posizioni dei
valori non nulli seguite da valori \key{nil} (maggiori dettagli alla sezione \S
\ref{secFondLenOperator}).

Il selettore prevede un primo parametro fisso seguito da una lista variabile di
valori rappresentata dai tre punti \key{...}. Se questo parametro è un intero
allora verrà considerato come indice per restituire l'argomento corrispondente.
Se invece il parametro è la stringa \key{\#} allora la funzione restituisce il
numero totale di argomenti \emph{inclusi} i \key{nil}.

Il codice seguente preso pari pari dal \href{http://www.lua.org/pil/}{PIL} ne è
un esempio:
\begin{lines}
for i = 1, select("#", ...) do
    local arg = select(i, ...)
    -- loop body
end
\end{lines}


\section{Omettere le parentesi}

In Lua esiste la sintassi di chiamata a funzione semplificata che consiste nella
possibilità di ommettere le parentesi tonde \key{()}\luas{()}. È ammessa solo se
alla funzione si passa un unico argomento di tipo stringa o di tipo tabella:
\sourcecode{
    file = [[code/e7-funzioni.lua]],
    select = [[anchequesto]],
}


\section{Closure}
\label{secClosure}

Chiudiamo il capitolo parlando di uno strano termine forse meglio noto agli
sviluppatori nei linguaggi funzionali: la \emph{closure}.

Questa proprietà di Lua amplia il concetto di funzione rendendo possibile
l'accesso dall'interno di essa a dati presenti nel contesto esterno. Ciò è
possibile perché alla chiamata di una funzione viene creato uno spazio di
memoria del contesto esterno unico e indipendente.

\begin{quote}
\emph{%
Tutte le chiamate a una stessa funzione condivideranno una stessa closure.%
}
\end{quote}

Se questo è vero una funzione potrebbe incrementare un contatore creato al suo
interno, e anche qui prendo l'esempio di codice dal PIL:
\sourcecode{
    file = [[code/e7-funzioni.lua]],
    select = [[uno]],
}

Il codice definisce una funzione \fn{new\_counter} che restituisce una
funzione che ha accesso indipendente al contesto (la variabile \key{i}).

Tecnicamente la closure \emph{è} la funzione effettiva mentre invece la
funzione non è altro che il prototipo della closure.

Le closure consentono di implementare diverse tecniche utili in modo naturale e
concettualmente semplice. Una funzione di ordinamento potrebbe per esempio
accettare come parametro una funzione di confronto per stabilire l'ordine tra
due elementi tramite l'accesso a una seconda tabella esterna contenente
informazioni utili per l'ordinamento stesso.

Nel prossimo esempio mettiamo in pratica l'idea. Il codice utilizza la funzione
\fn{table.sort} della libreria di Lua che introdurremo nel prossimo capitolo,
per applicare l'algoritmo di ordinamento alla tabella primo argomento in
base al criterio stabilito con la funzione passata come secondo
argomento in sintassi anonima.
\sourcecode{
    file = [[code/e7-funzioni.lua]],
    select = [[due]],
    run = true,
}


\section{Esercizi}

\begin{Exercise}[label=fn-01]
Scrivere una funzione che sulla base della stringa in ingresso \verb|"+"|,
\verb|"-"|, \verb|"*"|, \verb|"/"| restituisca la funzione corrispondente per
due operandi.
\end{Exercise}

\begin{Exercise}[label=fn-02]
Scrivere la funzione che accetti due argomenti numerici e ne restituisca i
risultati delle quattro operazioni aritmetiche.
\end{Exercise}

\begin{Exercise}[label=fn-03]
Scrivere una funzione che restituisca il fattoriale di un numero memorizzandone
in una tabella di closure i risultati per evitare di ripetere il calcolo in
chiamate successive con pari argomento.
\end{Exercise}

\begin{Exercise}[label=fn-04]
Scrivere una funzione con un argomento opzionale rispetto al primo parametro
numerico che ne restituisca il seno interpretandolo in radianti se l'argomento
opzionale è \key{nil} oppure \verb|"rad"|, in gradi sessadecimali se
\verb|"deg"| o in gradi centesimali se \verb|"cen"|.
\end{Exercise}

\begin{Exercise}[label=fn-05]
Scrivere una funzione \fn{ordinates} che accetti come primo argomento una
funzione \( f: ℝ \to ℝ \) (prende un numero e restituisce un numero), come
secondo e terzo argomento i due valori dell'intervallo di calcolo e come quarto
argomento il numero di punti \( n \geq 2 \) in cui suddividere equamente
l'intervallo, e restituisca i valori in sequenza che la funzione \( f \) assume
nei punti d'ascissa così definiti, in una tabella/array con indice iniziale 1.

Verificare poi che le due tabelle calcolate da \fn{ordinates} con le funzioni di
libreria \key{math.sin} e \key{math.cos}, nello stesso intervallo \( 0 \), \(
\pi/2 \) con \( n = 101 \) contengano lo stesso valore nella posizione di indice
51.
\end{Exercise}


% end of file
