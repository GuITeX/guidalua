

\chapter{Come eseguire gli esercizi}

È certamente fondamentale eseguire noi stessi esempi ed esercizi di
programmazione allo scopo di acquisire la padronanza di Lua. Nella guida ne
trovate alcuni alla fine di ciascun capitolo della parte seconda. 

Questa sezione vi introduce brevemente al programma \prog{texlua} che già
trovate compreso in ogni recente distribuzione \TeX{}. Si tratta dell'interprete
Lua controparte di \prog{luatex}.

Rispetto all'interprete \prog{lua} standard \prog{texlua} non ha la modalità
interattiva REPL\footnote{Read–eval–print loop.} con cui si digita una linea di
codice alla volta in un prompt interattivo, modalità molto utile per fare prove
velocemente.

Il codice dunque, andrà memorizzato in un file con estensione \texttt{.lua}.
Come esempio elementare, digitiamo questa unica riga di codice in un file
di testo chiamato \file{primo.lua}:
\begin{lines}
print("Hello World!")
\end{lines}
apriamo una finestra di terminale\footnote{Maggiori dettagli per diversi sistemi
operativi sulla linea di comando possono essere trovati nella guida tematica
dedicata \emph{Guida alla console} scaricabile dal sito \GuIT.} e lanciamo il
comando:
\begin{Verbatim}[numbers=none,xleftmargin=0pt]
$ texlua primo.lua
\end{Verbatim}

Ora che sappiamo come eseguire il codice Lua, concentriamoci con i prossimi
capitoli sulle basi del linguaggio. Torneremo nella terza parte della guida su
ulteriori modalità di esecuzione anche per il codice Lua interno a sorgenti
\TeX.


