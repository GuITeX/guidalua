

\chapter{La tabella}
\label{iiChTabella}

In questo capitolo parleremo della \emph{tabella}, l'unico tipo strutturato
predefinito di Lua e il suo costruttore. Diamone subito la definizione: la
tabella è un \emph{dizionario} cioè l'insieme non ordinato di coppie
chiavi/valore e, allo stesso tempo, anche un \emph{array} cioè una sequenza
ordinata di valori.

In Lua ne è previsto quindi un uso molteplice: se le chiavi sono numeri interi
la tabella sarà un array, se le chiavi sono di altro tipo, per esempio stringhe,
sarà un dizionario.

Le chiavi possono essere di tutti i tipi previsti da Lua tranne che
\key{nil}\luak{nil}, mentre i valori possono appartenere a qualsiasi tipo. Nulla
vieta che in una stessa tabella coesistano chiavi di tipo diverso.

Per creare una tabella si usa il \emph{costruttore}. Come vedremo tra poco, la
sua forma più semplice è quella che corrisponde a una tabella vuota e consiste
in una coppia di parentesi graffe:
\begin{lines}
local t = {}      -- una tabella vuota
\end{lines}

Per assegnare e ottenere il valore associato a una chiave si utilizzano le
parentesi quadre, l'operatore di indicizzazione, ecco un esempio:
\begin{lines}
local t = {}      -- tabella vuota
t["key"] = "val"  -- nuovo elemento chiave/valore
print(t["key"])   --> stampa "val"
\end{lines}

Stando alla definizione che abbiamo dato, una tabella può avere chiavi anche di
tipo differente, e infatti è proprio così e ciò vale anche per i valori. In
questo esempio una tabella ha chiavi di tipo numerico e di tipo stringa con
valori a sua volta di tipo diverso:
\begin{lines}
local t = {}
t["key"] = 123
t[123] = "key"
print(t["key"]) --> stampa il tipo numerico 123
print(t[123])   --> stampa il tipo stringa "key"
\end{lines}


\section{La tabella è un oggetto}

Cosa significa che la tabella di Lua è un oggetto? Vuol dire che la tabella è un
dato in memoria gestito con un riferimento. La variabile a cui viene assegnato
l'oggetto contiene l'informazione della posizione in memoria della tabella. Per
rendersene conto è sufficiente copiare una variabile a una tabella in una
seconda variabile e verificare che non si tratta di una sua copia:
\begin{lines}
local t = {}
t[1], t[2] = 10, 20
-- copia la tabella o il riferimento?
local other = t
t[1] = t[1] + t[2] -- modifichiamo t
-- l'altra variabile riflette la modifica?
assert(t[1] == other[1])
\end{lines}

Con la funzione \fn{assert}\luastd{assert} si può dichiarare l'equivalenza
logica tra due espressioni oppure imporre la verità di un'espressione prima di
assegnarla a una variabile. Se l'argomento è vero essa si comporta in modo
neutro restituendo l'argomento stesso, ma se è falso, essa termina l'esecuzione
del programma stampando la descrizione d'errore opzionalmente fornita dal
secondo argomento stringa.

Come vedremo nel capitolo \ref{iiChOpLogic}, un'espressione è vera se vale
\key{true} oppure non è \key{nil}, e falsa se vale \key{false} oppure è
\key{nil},

I tipi come i numeri invece non sono oggetti e assegnare una variabile che li
contiene a un'altra comporta la copia del dato, come si può verificare con un
codice simile al precedente.

Il fatto che la tabella sia un oggetto è la premessa fondamentale per la
programmazione a oggetti in Lua e per scrivere codice più compatto nelle
elaborazioni su tabelle dalla struttura complessa.

In effetti possiamo annidare in una tabella ulteriori tabelle assegnandole come
valore a corrispondenti chiavi, con complessità arbitraria. In altri termini una
tabella può rappresentare una struttura ad albero senza limiti teorici. Poiché
essa è gestita attraverso un riferimento nell'albero vi saranno solamente i
corrispondenti riferimenti mentre il dato effettivo sarà rappresentato in
un'altra parte della memoria.


\section{Il costruttore e la dot notation}

Dunque la tabella è un tipo di dato molto flessibile, è un oggetto, ed è
sufficientemente efficiente. Può essere usata in moltissime diverse situazioni
ed è ancora più utile grazie all'efficacia del suo \emph{costruttore}.

Ispirato al formato dei dati bibliografici di \BibTeX, uno dei programmi storici
del sistema \TeX{} usato per la gestione delle bibliografie nei documenti
\LaTeX, il costruttore di Lua può creare tabelle da una sequenza di
chiavi/valori inserite tra parentesi graffe:
\begin{lines}
local t = { a = 123, b = 456, c = "valore" }
\end{lines}

La chiave appare come il nome di una variabile ma in realtà nel costruttore
essa viene interpretata come una chiave di tipo stringa. Così l'esempio
precedente è equivalente al seguente codice:
\begin{lines}
-- codice equivalente
local t = {}
t["a"] = 123
t["b"] = 456
t["c"] = "valore"
\end{lines}

La notazione del costruttore non ammette l'utilizzo diretto di chiavi
numeriche. Se occorrono è necessario utilizzare le parentesi quadre per
racchiudere il numero che fa da indice:
\begin{lines}
-- chiavi numeriche nel costruttore?
local t_error = { 20 = 123 }
local t_ok = { [20] = 123 }
\end{lines}

Invece, se nel costruttore omettiamo le chiavi, otteniamo una tabella array con
indici interi impliciti in sequenza a partire da 1, contrariamente alla maggior
parte dei linguaggi dove l'indice comincia da 0. Ecco un esempio:
\begin{lines}
local t = { 30, 8, 500 }
print(t[1] + t[2] + t[3]) --> stampa 538
\end{lines}

Non è tutto. L'efficacia sintattica del costruttore è completata dalla \emph{dot
notation}, valida solamente per le chiavi di tipo stringa: il valore associato a
una chiave stringa si ottiene scrivendo la chiave dopo il nome del riferimento
della tabella, separato dal carattere \key{.}\luas{.}:
\begin{lines}
local t = { chiave = "123" }
assert(t.chiave == t["chiave"])
\end{lines}

Prestate attenzione perché all'inizio si può male interpretare il risultato del
costruttore della tabella se unito alla dot notation:
\begin{lines}
local chiave = "ok"
local t = { ok = "123"} -- t.ok == t[chiave]

-- attenzione!
local k = "ok"
print( t.k ) --> stampa nil: "k" non è definita in t
print( t[k]) --> stampa "123"
-- t[k] == t["ok"] == t.ok
-- t.k è diverso da t[k] !!!
\end{lines}

Non confondete il nome di variabile con il nome del campo in dot notation!

Riassumendo, indicizzare una tabella con una variabile restituisce il valore
associato alla chiave uguale al valore della variabile, mentre indicizzare in
dot notation con il nome uguale a quello della variabile restituisce il valore
associato alla chiave corrispondente alla stringa del nome.


\section{Esercizi}

\begin{Exercise}[label={tab-01}]
Scrivere il codice Lua che memorizzi in una tabella i primi 10 numeri primi
usandone il costruttore.
\end{Exercise}

\begin{Exercise}[label={tab-02}]
Utilizzando la dot notation è possibile utilizzare caratteri spazio nel nome
della chiave delle tabelle?
\end{Exercise}

\begin{Exercise}[label={tab-03}]
Scrivere il codice Lua che stampi il valore associato alle chiavi \key{paese} e
\key{codice}, e il numero medio di comuni per regione, per la seguente tabella.
Stampare inoltre il numero di abitanti della capitale.
\begin{lines}
local t = {
    paese = "Italia",
    lingua = "italiano",
    codice = "IT",
    regioni = 20,
    province = 110,
    comuni = 8047,
    capitale = {"Roma", "RM", abitanti = 2753000},
}
\end{lines}
\end{Exercise}

\begin{Exercise}[label={tab-04}]
Ricreare la stessa tabella \key{t} dell'esercizio precedente ma \emph{senza}
usare il costrutture.
\end{Exercise}

\begin{Exercise}[label={tab-05}]
Dato il triangolo di vertici di coordinate \( A = \left( -1, 8\right) \), \( B =
\left( 5, 3\right) \), e \( C = \left( 1, -3\right) \), creare con il
costruttore una tabella di tabelle array che rappresenti il triangolo e con essa
calcolare le coordinate del baricentro.
\end{Exercise}

\begin{Exercise}[label={tab-06}]
Rappresentare con una tabella il triangolo dell'esercizio precedente utilizzando
come chiavi i nomi assegnati dei vertici. Dalla tabella ricavare poi una seconda
tabella che rappresenti il triangolo aventi i vertici nei punti medi dei lati.
\end{Exercise}

\begin{Exercise}[label={tab-07}]
Istanziare una tabella dizionario con il costruttore a rappresentare i propri
dati anagrafici. Stampare poi su più linee il proprio indirizzo postale.
\end{Exercise}

% end of file
