
% esercitazioni semplici con richiami ai fondamenti

\chapter{La calcolatrice}
\label{iChCalcolatrice}

Siamo giunti al primo capitolo di taglio applicativo. Creeremo una
\emph{calcolatrice} sottoforma di una macro \cs{expr} che accetti un'espressione
numerica e ne stampi il risultato. È davvero utile non dover più calcolare a
parte il risultato e ricopiarlo nel sorgente del documento \LaTeX{}.

Tentiamo qualcosa di molto semplice con Lua, assegnare l'espressione a una
variabile per poi stamparla nel documento:

\CLRmarginpar{Lua in \TeX\\
\gotosec{} \ref{iSecLuaInLuaLaTeX}}[true]% true option means +16pt of vspace
%
\CLRmarginpar{Variabili locali\\
\gotosec{} \ref{iSecGlobaleLocale}, \gotosec{} \ref{iiSecLocaleGlobale}}
%
\CLRmarginpar{\fn{tex.print}\\
\gotosec{} \ref{iSecPassaggioDati}}
%
\begin{lines}
#[tex]
#[indexfile=app-tut/E0-001-expr.tex]
% !TeX program = LuaLaTeX
% filename: app-tut/E0-001-expr.tex
\documentclass{article}
\newcommand\expr[1]{\directlua{
    local result = #1
    tex.print(tostring(result))
}}
\begin{document}
Finalmente una calcolatrice:
\( 1.24 (7.45 + 11.21) = \expr{1.24*(7.45 + 11.21)}\)
\end{document}
\end{lines}

\newcommand\expr[1]{\directlua{
    local result = #1
    tex.print(tostring(result))
}}
compilando con \LuaLaTeX{} il risultato è:
\begin{tcolorbox}
Finalmente una calcolatrice: \( 1.24 (7.45 + 11.21) = \expr{1.24*(7.45 + 11.21)}\)
\end{tcolorbox}

Un buon inizio. Nel sorgente all'interno della macro \cs{directlua} il primo
argomento è stato sostituito con l'espressione che viene poi valutata da Lua.
Nessun pacchetto aggiuntivo caricato, qualsiasi espressione numerica è lecita, e
questo solo e soltanto usando Lua incluso in \LuaTeX.

Funziona anche con le stringhe, a patto di delimitarne il valore, e con le
espressioni booleane. Proviamo:
\begin{tcolorbox}
\begin{lines}
#[tex]
#[tcolorbox]
\( 56.9 > 78.42 \) è \texttt{\expr{ 56.9 > 78.42 }}
\end{lines}
\tcblower
\( 56.9 > 78.42 \) è \texttt{\expr{ 56.9 > 78.42 }}
\end{tcolorbox}


\section{Espressioni booleane personalizzate}

E se si volessero sostituire le rappresentazioni testuali dei valori vero e
falso? Ecco la modifica:

\CLRmarginpar{%
Ciclo \key{if}\\
\gotosec{} \ref{iiSecCicloIf}}[true]
%
\CLRmarginpar{%
Tipo \key{boolean}\\
\gotosec{} \ref{iiSecManciataTipi}}
%
\CLRmarginpar{%
Operatore ternario\\
\gotosec{} \ref{iiSecOperatoreTernario}}
%
\begin{lines}
#[tex]
\newcommand\expr[1]{\directlua{
    local result = #1
    if type(result) == "boolean" then
        result = result and "vero" or "falso"
    end
    tex.print(tostring(result))
}}
\end{lines}

\renewcommand\expr[1]{\directlua{
    local result = #1
    if type(result) == 'boolean' then
        result = result and 'vero' or 'falso'
    end
    tex.print(tostring(result))
}}

Un semplice test ci conforterà sulla correttezza del codice e si funziona:
\begin{tcolorbox}[sidebyside]
\begin{lines}
#[tcolorbox]
\expr{100 == 100 and 7 > 3}
\expr{-10 < -100}
\end{lines}
\tcblower
\expr{100 == 100 and 7 > 3}\\
\expr{-10 < -100}
\end{tcolorbox}


\section{Arrotondamento numerico}

Vorrei poter regolare l'arrotondamento del risultato numerico della calcolatrice
ricorrendo a un argomento opzionale separato dall'espressione da una virgola:

\begin{center}
\CLRmarginpar{%
\fn{string.format}\\
\gotosec{} \ref{iiSecStringFormat}}[true]%
\begin{lines}
#[tex]
\newcommand\expr[1]{\directlua{
    local result, dec = #1
    if type(result) == "boolean" then
        result = result and "vero" or "falso"
    elseif type(result) == "number" and dec then
        local perc = string.char(37)
        local fmt1 = perc..perc.."0."..perc.."df"
        local fmt2 = string.format(fmt1, dec)
        result = string.format(fmt2, result)
    end
    tex.print(tostring(result))
}}
\end{lines}
\end{center}

\renewcommand\expr[1]{\directlua{
    local result, dec = #1
    if type(result) == 'boolean' then
        result = result and 'vero' or 'falso'
    elseif type(result) == 'number' and dec then
        local perc = string.char(37)
        local fmt1 = perc..perc..'0.'..perc..'df'
        local fmt2 = string.format(fmt1, dec)
        result = string.format(fmt2, result)
    end
    tex.print(tostring(result))
}}

Stavolta il codice perde un po' di chiarezza perché non è possibile usare
direttamente il carattere percento \code{\%} che verrebbe interpretato come
inizio di un commento nel costruire le stringhe di formato. Ovviamente questo
non succederebbe se il codice fosse in un file separato o se fosse racchiuso in
un ambiente \pack{luacode} dell'omonimo pacchetto \LaTeX{} \cite{pkg:luacode}.

Mettiamo alla prova la modifica alla macro \cs{expr}:
\begin{tcolorbox}
\begin{lines}
#[tex]
#[tcolorbox]
\(\sqrt{2} + \sqrt{3} \approx \expr{ 2^0.5 + 3^0.5, 2}\)
\end{lines}
\tcblower
\(\sqrt{2} + \sqrt{3} \approx \expr{ 2^0.5 + 3^0.5, 2}\)
\end{tcolorbox}


\section{Funzioni matematiche}

Potremo trovare una sintassi un po' più chiara per indicare il numero di cifre a
cui arrontondare il risultato, tuttavia c'è di mezzo un problema più urgente:
poter usare funzioni matematiche come seno e coseno. Se scrivessimo
\verb=\expr{sin(1)^2 + cos(1)^2}= non otterremo il valore unitario ma un errore.
Dovremo infatti usare la scomoda notazione
\code{math.}\meta{funzione}/\meta{costante} come in
\verb=\expr{math.cos(math.pi)}= invece della più naturale \verb=\expr{cos(pi)}=.
Ma ci vuole poco a riassegnare le funzioni a nomi locali per far si che
l'identità trigonometrica precedente sia un'espressione valida:
\begin{lines}
#[tex]
\newcommand\expr[1]{\directlua{
    local cos = math.cos
    local sin = math.sin
    local result, dec = #1
    if type(result) == "boolean" then
        result = result and "vero" or "falso"
    elseif type(result) == "number" and dec then
        local perc = string.char(37)
        local fmt1 = perc..perc.."0."..perc.."df"
        local fmt2 = string.format(fmt1, dec)
        result = string.format(fmt2, result)
    end
    tex.print(tostring(result))
}}
\end{lines}

\renewcommand\expr[1]{\directlua{
    local cos = math.cos
    local sin = math.sin
    local result, dec = #1
    if type(result) == 'boolean' then
        result = result and 'vero' or 'falso'
    elseif type(result) == 'number' and dec then
        local perc = string.char(37)
        local fmt1 = perc..perc..'0.'..perc..'df'
        local fmt2 = string.format(fmt1, dec)
        result = string.format(fmt2, result)
    end
    tex.print(tostring(result))
}}

Una prova della calcolatrice potenziata con le funzioni trigonometriche ci dirà
se tutto funziona ancora bene:
\begin{tcolorbox}
\begin{lines}
#[tex]
#[tcolorbox]
\(\sin^2(1/2) + \cos^2(1/2) =\expr{sin(0.5)^2 + cos(0.5)^2, 8}\).
\end{lines}
\tcblower
\(\sin^2(1/2) + \cos^2(1/2) = \expr{sin(0.5)^2 + cos(0.5)^2, 8}\).
\end{tcolorbox}
e per un'espressione booleana:
\begin{tcolorbox}
\begin{lines}
#[tex]
#[tcolorbox]
A \(1/3\) l'identità è \emph{\expr{sin(1/3)^2 + cos(1/3)^2 == 1}}
\end{lines}
\tcblower
A \( 1/3 \) l'identità è \emph{\expr{sin(1/3)^2 + cos(1/3)^2 == 1}}
\end{tcolorbox}

Finora ogni nuova funzionalità aggiunta alla calcolatrice non ha presentato
difficoltà. Possiamo inserire o meno il risultato in ambiente matematico,
arrotondarlo al numero di decimali desiderato e usare funzioni matematiche
dell'efficiente libreria in virgola mobile di Lua, un linguaggio che si sta
dimostrando semplice da usare e molto efficace.


\section{Costanti numeriche}

Continuamo con un nuovo passo: aggiungere costanti numeriche definite
dall'utente, una sorta di \emph{funzione di memoria} della calcolatrice. Per
inserire variabili letterali in un'espressione abbiamo bisogno che il loro
valore numerico sia inizializzato ma non possiamo ricorrere alla stessa tecnica
con cui abbiamo risolto l'inserimento delle funzioni trigonometriche.

Non è possibile infatti codificare variabili locali senza conoscerne il nome,
perché è un dato fornito dall'utente. Servirebbe una sorta di metaprogrammazione
come con le macro dei linguaggi compilati. Leggendo più a fondo la
documentazione di Lua, si scopre che è possibile intervenire sull'ambiente delle
variabili globali \code{\_ENV}\footnote{Per ulteriori informazioni
sull'\emph{environment} di Lua rimando al PIL dove si trova un intero capitolo
ad esso dedicato.} di un \emph{chunk}, e anzi, a ben vedere il problema di
rendere visibili simboli di costanti è lo stesso che quello di rendere
disponibili nell'espressione le funzioni matematiche con nomi abbreviati.
Facciamo un tentativo ripartendo con il codice da zero:

\CLRmarginpar{%
Tipo \code{table}\\
\gotosec{} \ref{iiChTabella}}[true]
%
\CLRmarginpar{%
Iteratore \fn{pairs}\\
\gotosec{} \ref{iiSecPairsIterator}}
%
\begin{lines}
#[tex]
\directlua{
calclib = {}
for name, object in pairs(math) do
    calclib[name] = object
end
}
\newcommand\expr[1]{\directlua{
do
    local _ENV = calclib
    ans = #1
end
tex.print(tostring(calclib.ans))
}}
\end{lines}

Stiamo sfruttando una tecnica piuttosto interessante: all'interno di un blocco
viene riassegnata localmente la variabile \code{\_ENV} a \code{calclib}, una
tabella in cui vi abbiamo riversato tutte le funzioni e le costanti matematiche
della libreria \code{math} di Lua. Alla riga 10, tutti quei nomi saranno
visibili come variabili globali proprio quando l'espressione deve essere
valutata.

Non solo, come effetto collaterale, il risultato dell'ultimo calcolo sarà
disponibile nella successiva espressione memorizzato nella variabile \code{ans}
come succede con altri tool matematici! Proviamolo:
%
\directlua{
calclib = {}
for name, object in pairs(math) do
    calclib[name] = object
end
}
\renewcommand\expr[1]{\directlua{
do
    local _ENV = calclib
    ans = #1
end
tex.print(tostring(calclib.ans))
}}
%
\begin{tcolorbox}[sidebyside]
\begin{lines}
#[tex]
#[tcolorbox]
\expr{pi/4}\\
\expr{cos(ans)}\\
\expr{acos(ans)}
\end{lines}
\tcblower
\expr{pi/4}\\
\expr{cos(ans)}\\
\expr{acos(ans)}
\end{tcolorbox}

Molto bene. Non ci resta che aggiungere con la stessa tecnica la memorizzazione
delle costanti attivando l'argomento opzionale della macro \cs{expr}. Tra le
parentesi quadre potremo fornire all'espressione una lista di costanti nel
formato chiave/valore con separatore il carattere virgola.

Memorizzeremo le costanti indicate dall'utente solamente se il loro nome non è
già stato utilizzato o se non è un un nome di funzione. Inoltre, specificando
una stringa solo come chiave tra le opzioni, potremo implementare la
memorizzazione del risultato dell'espressione stessa, così che sia poi
riutilizzabile:
\begin{lines}
#[tex]
\directlua{
calclib = {}
for name, object in pairs(math) do
    calclib[name] = object
end
}
\newcommand\expr[2][]{\directlua{
do
    local error, pairs, assert, type = error, pairs, assert, type
    local _ENV = calclib
    local opt = {#1}
    local mem = opt[1]; opt[1] = nil
    for c, val in pairs(opt) do
        if _ENV[c] then
            error("Duplicated key '"..c.."' in constant name")
        else
            _ENV[c] = val
        end
    end
    ans = #2
    if mem then
        assert(type(mem) == "string")
        if _ENV[mem] then
            error("Duplicated key '"..mem.."' in memory index")
        else
            _ENV[mem] = ans
        end
    end
end
tex.print(tostring(calclib.ans))
}}
\end{lines}

\renewcommand\expr[2][]{\directlua{
do
    local error, pairs, assert, type = error, pairs, assert, type
    local _ENV = calclib
    local opt = {#1}
    local mem = opt[1]; opt[1] = nil
    for c, val in pairs(opt) do
        if _ENV[c] then
            error([[Duplicated key ']]..c..[[' in constant name]])
        else
            _ENV[c] = val
        end
    end
    ans = #2
    if mem then
        assert(type(mem) == [[string]])
        if _ENV[mem] then
            error([[Duplicated key ']]..mem..[[' in memory index]])
        else
            _ENV[mem] = ans
        end
    end
end
tex.print(tostring(calclib.ans))
}}

Eccone un esempio:
\begin{tcolorbox}
\begin{lines}
#[tex]
#[tcolorbox]
\( b = \expr[b=10, h=20]{b} \), % oppure \expr["b", h=20]{10}
\( h = \expr{h} \),
\( M = \expr[m = 1000]{m}\), % oppure \expr["m"]{1000}
\( \sigma = M/W_\mathrm{x} = \expr[w=(b*h^2)/6]{m/w}\).
\end{lines}
\tcblower
\( b = \expr[b = 10]{b} \),
\( h = \expr[h = 20]{h} \),
\( M = \expr[m = 1000]{m}\),
\( \sigma = M/W_\mathrm{x} = \expr[w=(b*h^2)/6]{m/w}\).
\end{tcolorbox}

Poiché anche i valori assegnati alle costanti sono valutati da Lua \emph{dopo}
la modifica dell'environment, anche per le costanti nelle opzioni della macro
\cs{expr} è possibile assegnare espressioni usando tutte le funzioni matematiche
e tutte le costanti precedentemente definite. Da questo punto in poi, possiamo
presentare il valore di \( W_\mathrm{x} \) scrivendo nel sorgente
\verb=\expr{w}= che da \expr{w}.

\section{Conclusioni}

Le principali funzionalità della calcolatrice sono state implementate in Lua,
certo non tutte. Per esempio potremo far eseguire calcoli coinvolgendo anche i
registri contatori o dimensionali di \TeX{}, per esempio per determinare delle
coordinate in un disegno, oppure considerare che constanti dai nomi speciali
come \code{M1}, \code{M2} eccetera si comportino come registri di memoria della
calcolatrice e quindi che possano essere sovrascritti e possano funzionare da
accumulatori.

% end of file
