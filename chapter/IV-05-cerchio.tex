

\chapter{Disegno del cerchio}
\label{ivChCerchio}

In questo capitolo esploreremo un'altro tipo di nodo: il \emph{pdfliteral}.
Con esso possono essere inserite nel file PDF figure vettoriali definite
attraverso istruzioni grafiche. Per questi nodi il compositore non fa altro che
inserirli direttamente e senza alcun controllo nel file di uscita perché siano
interpretati unicamente dal programma di visualizzazione e stampa PDF.

Dovremo quindi prestare attenzione alla correttezza delle istruzioni che
definiamo seguendo la sintassi prevista dal formato PDF e contenuta nel PDF
reference distribuito da Adobe. Un errore potrebbe invalidare l'intero
documento.

Se si è utenti del sistema \TeX{} ci sarà certamente capitato di dover
realizzare diagrammi o figure, perciò abbiamo già utilizzato le istruzioni
pdfliteral ma non direttamente. I pacchetti grafici come \pack{picture2e} o
\pack{TikZ} infatti, offrono un'interfaccia di alto livello più compatta e
sicura alle primitive grafiche PDF.

Per conoscere questo tipo di tecnologia, implementeremo in Lua il disegno di
cerchi.


\section{Le curve di Bezier}


\section{Cerchio a otto curve}


\section{SVG}





% end of file
