
\chapter{La libreria standard di Lua}
\label{iiChLibstd}

In Lua sono immediatamente disponibili un folto gruppo di funzioni che ne
formano la \emph{libreria standard}. Si tratta di una collezione di funzioni
utili a svolgere compiti ricorrenti su stringhe, file, tabelle, eccetera, e si
trovano precaricate in una serie di tabelle.

L'elenco completo ma in ordine sparso con il nome della tabella/modulo
contenitore e la descrizione applicativa è il seguente:
\begin{center}
\begin{tabular}{ll}
\key{math} & matematica;\\
\key{table} & utilità sulle tabelle;\\
\key{string} & ricerca, sostituzione e pattern matching;\\
\key{io} & input/output facility, operazioni sui file;\\
\key{bit32} & operazioni bitwise (solo in Lua 5.2);\\
\key{os} & date e chiamate di sistema;\\
\key{coroutine} & creazione e controllo delle coroutine;\\
\key{utf8} & utilità per la codifica Unicode UTF-8 (da Lua 5.3);\\
\key{package} & caricamento di librerie esterne;\\
\key{debug} & accesso alle variabili e performance assessment.\\
\end{tabular}
\end{center}
La pagina web a \href{www.lua.org/manual/5.3/contents.html}{questo link}
fornisce tutte le informazioni di dettaglio sulla libreria standard di Lua~5.3.


\section{Libreria matematica}
\label{iiSecMathLibrary}

Nella libreria memorizzata nella tabella \key{math} ci sono le funzioni
trigonometriche \fn{sin}\luastd{math.sin}, \fn{cos}\luastd{math.cos},
\fn{tan}\luastd{math.tan}, \fn{asin}\luastd{math.asin} eccetera --- che lavorano
in radianti --- le funzioni esponenziali \fn{exp}\luastd{math.exp},
\fn{log}\luastd{math.log}, \fn{log10}\luastd{math.log10}, quelle di
arrotondamento \fn{ceil}\luastd{math.ceil}, \fn{floor}\luastd{math.floor}, e
quelle per la generazione pseudocasuale di numeri come
\fn{random}\luastd{math.random}, e \fn{randomseed}\luastd{math.randomseed}.
Oltre a funzioni, la tabella include campi numerici come la costante \( \pi
\)\luak{math.pi}.

Un esempio introduttivo è questo dove nella funzione \fn{one} viene definita una
funzione locale:
\sourcecode{file = [[code/e8-libstd.lua]], select = [[one]]}

Accanto a funzioni matematiche vere e proprie troviamo la funzione
\fn{math.type}\luastd{{math.type}} che esamina l'argomento e restituisce
\verb|"integer"| o \verb|"float"| a seconda se esso è un intero o un numero in
virgola mobile, oppure \key{nil} se non si tratta di un numero. Questa funzione
è utile per eseguire il controllo di validità dei parametri in ingresso.

La funzione \fn{math.tointeger}\luastd{math.tointeger} è, come la precedente,
utile per verificare se l'argomento è un intero. Essa infatti tenta la
conversione e nel caso non sia possibile il risultato sarà \key{nil}. 

L'argomento può essere anche un numero o una stringa, e in questo è simile alla
funzione \fn{tonumber}\luastd{tonumber} ma solo per i valori interi:
\begin{lines}
#[run]
print(tonumber(123), math.tointeger(123))
print(tonumber(-89), math.tointeger(-89))
print(tonumber(40.5 * 2), math.tointeger(40.5 * 2))
print(tonumber(40.5 * 3), math.tointeger(40.5 * 3))
print(tonumber("89"), math.tointeger("89"))
print(tonumber("89.6"), math.tointeger("89.6"))
\end{lines}


\section{Libreria tabelle}
\label{iiSecTableLibrary}

Per le tabelle la libreria standard fornisce funzionalità attraverso la tabella
\key{table}. Tra le funzioni più importanti troviamo
\fn{table.concat}\luastd{table.concat} che concatena tutti i valori nelle
posizioni sequenza in una stringa. Per esempio:
\begin{lines}
-- table.concat(list [, sep [, i [, j]]])
assert(table.concat{1,2,3,4,5,6} == "123456")
\end{lines}

Ci sono anche utili argomenti opzionali: un secondo argomento stringa che sarà
usato come separatore dei vari elementi, un terzo argomento intero che è
l'indice dal quale gli elementi saranno raccolti e un quarto argomento come
indice finale.

Questa funzione è molto importante nelle applicazioni perché consente di
concatenare stringhe in un modo efficiente quando il loro numero è maggiore di
una decina\footnote{L'ordine di grandezza del numero delle stringhe da
concatenare per cui conviene l'uso di \fn{table.concat} dovrebbe essere una
decina ma non ho fatto benchmark in proposito.}.

Altre funzioni molto utili sono \fn{table.insert}\luastd{table.insert} con la
sua controparte \fn{table.remove}\luastd{table.remove}, e
\fn{table.sort}\luastd{table.sort} per riordinare gli elementi della sequenza di
una tabella secondo criteri definibili di cui un esempio si trova nella
sezione~\ref{secClosure}.


\section{Libreria stringhe}

La libreria per le stringhe è memorizzata nella tabella \key{string} ed è una
delle più utili per questo l'approfondiremo più in dettaglio rispetto alle
altre. Con essa si possono formattare campi e compiere operazioni di ricerca e
sostituzione. In effetti, in Lua non è infrequente elaborare grandi porzioni di
testo.


\subsection{Funzione \fn{string.format}}
\label{secFondStringFormat}

La funzione più semplice è quella di formattazione
\fn{string.format}\luastd{string.format}. Essa restituisce una stringa prodotta
con il formato definito dal primo argomento dei dati forniti dal secondo
argomento in poi.

Il formato è esso stesso specificato come una stringa contenente dei segnaposto
creati con il simbolo percentuale e uno specificatore di tipo. Per esempio
\verb|"%d"| indica il formato relativo a un numero intero, dove \key{d} sta per
digit mentre \verb|"%f"| indica il segnaposto per un numero decimale con \key{f}
che sta per float.

I campi formato derivano da quelli della funzione classica di libreria
\fn{printf} del C. Di seguito un esempio di codice:
\sourcecode{
    file = [[code/e8-libstd.lua]],
    select = [[fmt]],
    run = true,
}

Come avete potuto notare nel codice, è anche possibile fornire un ulteriore
specifica di dettaglio tra il carattere \key{\%} e lo specificatore di tipo, per
esempio per indicare il numero delle cifre decimali di arrotondamento.

Per elaborare il testo si utilizza di solito una libreria per le espressioni
regolari. Lua mette a disposizione alcune funzioni di sostituzione e
\emph{pattern matching} meno complete dell'implementazione dello standard
POSIX per le espressioni regolari ma molto spesso più semplici da utilizzare.

Esistono due strumenti di base, il primo è il \emph{pattern} e il secondo è la
\emph{capture}.


\subsection{Pattern I}
\label{secFondPattern}

Il pattern è una stringa che può contenere campi chiamati \emph{classi} simili
a quelli per la funzione di formato visti in precedenza, che stavolta però si
riferiscono al \emph{singolo carattere} e questa differenza è essenziale.

La funzione di base che accetta un pattern è
\fn{string.match}\luastd{string.match}. Essa restituisce la prima corrispondenza
trovata in una stringa data come primo argomento che corrispondente al pattern
dato come secondo argomento.

Per esempio, possiamo ricercare un gruppo di tre cifre intere all'interno di un
testo con il pattern \verb|"%d%d%d"|:
\sourcecode{
    file = [[code/e8-libstd.lua]],
    select = [[pattern_one]],
    run = true,
}

Le classi carattere possibili sono le seguenti:
\begin{compactdescription}
  \item[\key{.}] un carattere qualsiasi;
  \item[\key{\%a}] una lettera;
  \item[\key{\%c}] un carattere di controllo;
  \item[\key{\%d}] una cifra;
  \item[\key{\%l}] una lettera minuscola;
  \item[\key{\%u}] una lettera maiuscola;
  \item[\key{\%p}] un carattere di interpunzione;
  \item[\key{\%s}] un carattere spazio;
  \item[\key{\%w}] un carattere alfanumerico;
  \item[\key{\%x}] un carattere esadecimale;
  \item[\key{\%z}] il carattere rappresentato con il codice 0.
\end{compactdescription}

Le classi ammettono quattro modificatori per esprimere le ripetizioni dei
caratteri:
\begin{compactdescription}
  \item[\key{+}] indica 1 o più ripetizioni;
  \item[\key{*}] indica 0 o più ripetizioni;
  \item[\key{-}] come \key{*} ma nella sequenza più breve;
  \item[\key{?}] indica 0 o 1 ripetizione;
\end{compactdescription}

Esempio:
\sourcecode{
    file = [[code/e8-libstd.lua]],
    select = [[pattern_two]],
    run = true,
}

\subsection{Capture}
\label{secFondCapture}

Il pattern può essere arricchito non solo per trovare corrispondenze ma anche
per restituirne parti componenti. Questa funzionalità viene chiamata
\emph{capture} e consiste semplicemente nel racchiudere tra parentesi tonde le
classi.

Per esempio per estrarre l'anno di una data nel formato \key{dd/mm/yyyy}
possiamo usare il pattern con la capture seguente \verb|"%d%d/%d%d/(%d%d%d%d)"|:
\sourcecode{
    file = [[code/e8-libstd.lua]],
    select = [[capture_one]],
    run = true,
}

Più capture ci sono nel pattern e altrettanti argomenti multipli di uscita
saranno restituiti:
\sourcecode{file = [[code/e8-libstd.lua]],
   select = [[capture_two]],
   run = true,
}


\subsection{Pattern II}

Mentre il carattere percento unito a una lettera forma una delle classi
carattere elencate alla sezione~\ref{secFondPattern}, la stessa lettera scritta
in maiuscolo rappresenta la classe carattere negazione.

La classe \verb|"%D"| rappresenta un carattere che non sia una cifra decimale,
così come \verb|"%S"| corrisponde al carattere che non ha corrispondenza con
\verb|"%s"|. Per esempio, per catturare una prima sequenza di eventuali
caratteri spazio in una stringa si può usare il pattern di ricerca
\verb|"(%s*)%S?"|. 

Due ulteriori caratteri speciali \verb|^| per l'inizio riga e \key{\$} per la
fine riga, svolgono il ruolo di ancoraggio del pattern. Possono essere usati
entrambi per esempio per eliminare da una stringa i caratteri spazio iniziali o
finali:
\begin{lines}
local pattern = "^%s*(.-)%s*$"
print("|"..string.match("ok", pattern).."|" )      --> "|ok|"
print("|"..string.match(" ok", pattern).."|" )     --> "|ok|"
print("|"..string.match(" ok ", pattern).."|" )    --> "|ok|"
print("|"..string.match(" ok ok ", pattern).."|" ) --> "|ok ok|"
\end{lines}

Passiamo ora ai \emph{char-set}. Si tratta di classi definite dall'utente
racchiudendo tra parentesi quadre caratteri o classi di carattere già definiti.
Per esempio, per estrarre da una stringa una sequenza di \( 0 \) e \( 1 \)
possiamo definire una nuova classe carattere che corrisponde o alla prima o alla
seconda cifra con \verb|[01]|.

Per esempio, per trovare corrispondenze di \emph{identificatori}, cioè sequenze
di caratteri alfanumerici che comprendono il trattino basso e che non iniziano
con una cifra, potremo usare il pattern \verb|"^[_%a][_%w]*"|:
\begin{lines}
local pattern = "^[_%a][_%w]*"
print(string.match("1var", pattern))    --> nil
print(string.match("_", pattern))       --> "_"
print(string.match("var", pattern))     --> "var"
print(string.match("_var", pattern))    --> "_var"
print(string.match("1var", pattern))    --> nil
print(string.match("var1_=5", pattern)) --> "var1_"
print(string.match("0", pattern))       --> nil
\end{lines}

Ulteriore esempio è la ricerca di corrispondenze per numeri interi con segno:
\begin{lines}
local pattern = "[+-]?%d+"
print(string.match("56", pattern))   --> "56"
print(string.match("+56", pattern))  --> "+56"
print(string.match("-56", pattern))  --> "-56"
print(string.match("-+56", pattern)) --> "+56"
print(string.match("0000", pattern)) --> "0000"
\end{lines}

In un pattern, tra parentesi quadre è ammesso l'uso del carattere trattino per
definire intervalli. Il pattern \verb|"[012345]"| che corrisponde a una classe
carattere per le prime sei cifre decimali, può essere definito anche con
\verb|"[0-5]"|.


\subsection{La funzione \fn{string.gsub}}
\label{secFondGsub}

Abbiamo appena cominciato a scoprire le funzionalità dedicate al testo
disponibili nella libreria standard di Lua. Diamo solo un altro sguardo
presentando la funzione \fn{string.gsub}\luastd{string.gsub}. Il suo nome sta
per \emph{global substitution}, ovvero la sostituzione di tutte le parti di un
testo che corrispondono a uno schema.

Per individuare le parti da sostituire è naturale pensare di utilizzare un
pattern e che sia possibile utilizzare le capture nel testo di sostituzione, per
esempio:
\sourcecode{
    file = [[code/e8-libstd.lua]],
    select = [[gsub]],
    run = true,
}

Il primo argomento è la stringa da ricercare, il secondo è il pattern, il terzo
è il testo di sostituzione per ciscuna corrispondenza, ma può anche essere una
tabella dove le chiavi corrispondenti al pattern saranno sostituite con i
rispettivi valori, oppure una funzione che dal singolo frammento di testo
estratto elabora la stringa di sostituzione.

Una funzione quindi assai flessibile. Mi viene in mente questo esercizio:
moltiplicare per 12 tutti gli interi in una stringa, ed ecco il codice:
\sourcecode{
    file = [[code/e8-libstd.lua]],
    select = [[gsubfn]],
    run = true,
}

A questo punto degli esempi avrete certamente capito che \fn{gsub} restituisce
anche il numero delle sostituzioni effettuate.

Tutte queste funzioni restituiscono una stringa costruita ex-novo e non
modificano la stringa originale di ricerca. In Lua le stringhe sono dati
immutabili.

Infine, le funzioni della libreria \key{string} possono essere chiamate su
variabili stringa usando la \emph{colon notation} che introdurremo alla
sezione~\ref{secFondColonNotation}, grazie ai metametodi:
\begin{lines}
string.match(s, pattern) -- equivalente a:
s:match(pattern)
\end{lines}


\subsection{La funzione \fn{string.gmatch}}

Con la funzione \fn{string.gmatch}\luastd{string.gmatch} è possibile iterare
tutte le corrispondenze di un pattern in una stringa in un ciclo generic for.
Gli argomenti da fornire sono la stringa e il pattern che può contenere delle
capture. Vediamo un primo esempio:
\begin{lines}
#[run]
local s = "trova le vocali"
for vowel in string.gmatch(s, "[aeiou]") do
    print(vowel)
end
\end{lines}

Se invece il pattern contiene delle capture, ad ogni iterazione altrettante
variabili di ciclo assumeranno i valori stringa delle catture:
\begin{lines}
#[run]
local s = "da A=(1,9) a B=(-9,5) passando per C=(0,3)"
local pattern = "([A-Z])=%(([+-]?%d+),([+-]?%d+)%)"
for p, x, y in string.gmatch(s, pattern) do
    print(p, tonumber(x), tonumber(y))
end
\end{lines}


\subsection{Altre funzioni utili}

Nominiamo per prima la funzione \fn{string.find}\luastd{string.find}. Essa
restituisce la posizione della corrispondenza nella stringa di ricerca, nella
forma della coppia d'indici di inizio e di fine della sequenza. La ricerca può
cominciare da una posizione qualsiasi in virtù del terzo argomento opzionale, e
se il pattern contiene una capture la corrispondenza verrà restituita come
ulteriore terzo risultato:
\begin{lines}
#[run]
-- string.find(s, pattern [, init [, plain]])
local s = "Un intero 123 e un secondo 456!"
local i, j = string.find(s, "123") -- i = 11, j = 13
print(string.find(s, "(%d+)", j+1))
\end{lines}

Conviene citare altre due funzioni utili per esempio per poter usare il simbolo
di percentuale nel codice Lua interno a sorgenti \TeX{}. Sono
\fn{string.byte}\luastd{string.byte} e \fn{string.char}\luastd{string.char}: la
prima da una lista di interi positivi calcola la stringa codificata con essi e
la seconda calcola i codici dei caratteri della stringa argomento. Per esempio:
\begin{lines}
#[run]
-- string.byte(s [, i [, j]])
-- string.char(···)
local s = "LuaTeX"
local t = {string.byte(s, 1, -1)}
assert(table.concat(t, ", ") == "76, 117, 97, 84, 101, 88")
assert(string.char(76, 117, 97, 84, 101, 88) == s)
\end{lines}


\section{Esercizi}

\begin{Exercise}[label=libstd-01]
Qual è la differenza tra i campi di formato della funzione \fn{string.format} e
le classi dei pattern? Quali le somiglianze?
\end{Exercise}

\begin{Exercise}[label=libstd-02]
Stampare una data nel formato \key{dd/mm/yyyy} a partire dagli interi contenuti
nelle variabili \key{d}, \key{m} e \key{y}.
\end{Exercise}

\begin{Exercise}[label=libstd-03]
Cosa restituisce l'esecuzione della seguente funzione?
\sourcecode{
    file = [[code/e8-libstd.lua]],
    select = [[esercizio3]],
}
Quale pattern corrisponde a un numero decimale la cui parte intera può essere
omessa?
\end{Exercise}

\begin{Exercise}[label=libstd-04]
Come estrarre dal nome di un file l'estensione?
\end{Exercise}

\begin{Exercise}[label=libstd-05]
Come eliminare da un testo eventuali caratteri spazio iniziali e/o finali?
\end{Exercise}

\begin{Exercise}[label=libstd-06]
Il pattern \verb|"(%d+)/(%d+)/(%d+)"| è adatto per estrarre giorno, mese e
anno di una data presente in una stringa nel formato \verb|"dd/mm/yyyy"|?
\end{Exercise}

\begin{Exercise}[label=libstd-07]
Creare un esempio che utilizzi \fn{string.gsub} con una funzione in sintassi
anonima a due argomenti corrispondenti ad altrettante capture nel pattern di
ricerca.
\end{Exercise}


% end of file
