% !TeX program = LuaLaTeX--shell-escape
%
% "Guida Tematica alla programmazione Lua in LuaTeX"
% Copyright (C) 2020-2021 Roberto Giacomelli,
% Gruppo utilizzatori Italiani di TeX e LaTeX
% All rights reserved
%
% https://www.guitex.org
% 
% License information see LICENSE text file

\documentclass[b5paper,11pt,openany]{guidatematica}
\ProvidesFile{guidalua.tex}[2021/05/30 v0.5 Guida al linguaggio Lua per LuaTeX]
\GetFileInfo{guidalua.tex}
\setmonofont[Scale=0.82]{Fira Mono}
\usepackage{sourcecode}
\usepackage{hologo}
\usepackage[backend=biber,style=alphabetic]{biblatex}
\addbibresource{guidalua.bib}

\makeindex[options={-s guidatematica.ist},title=Indice Analitico,intoc]

\fvset{
    fontsize=\small,
    numbers=left,
    xleftmargin=7.5mm,
    numbersep=5.75mm
}

\lstset{% general setup
    numbers=left,
    numberstyle=\tiny,
    xleftmargin=7.5mm,
    numbersep=5.75mm
}

\usepackage[
    pdftitle={Guida al linguaggio Lua per LuaTeX},
    pdfauthor={Roberto Giacomelli},
    pdfsubject={Guida tematica del GuIT},
    pdfkeywords={Lua, LuaTeX, guida, programming}
]{hyperref}

\usepackage{exercise}
\renewcounter{Exercise}[chapter]
\renewcommand{\ExerciseHeader}{\noindent\textsc{esercizio \ExerciseHeaderNB}}

\usepackage{tcolorbox}
\tcbuselibrary{skins}
\tcbset{
    sharp corners=all,
    colback=verdeguit!12!white,
    colframe=verdeguit!70!white,
    bicolor,
    colbacklower=white,
    boxrule=0.4pt,
    leftrule=3.2pt,
    fontupper=\small,
    fontlower=\small
}

% colori note a margine
\definecolor{sfondo}{rgb}{0.95,0.97,0.95}% verdeguit!5!white
\definecolor{bordo}{rgb}{0, 0.40, 0}% verdeguit
\setmarginnotes{6pt}{\dimexpr\foremargin-18pt\relax}{2.5pt}

\usepackage{siunitx}
\sisetup{output-decimal-marker=\virgoladecimale}

\frenchspacing

\newcommand\tcmd[1]{\normalfont\texttt{#1}}
\newcommand\key[1]{\normalfont\texttt{#1}}
\newcommand\code[1]{\normalfont\texttt{#1}}
\newcommand\fn[1]{\normalfont\texttt{#1()}}
\newcommand\gotosec{\textcolor{verdeguit}{\tiny\S}}
\newcommand\BibTeX{\textsc{Bib}\TeX}

% index special macros
\DeclareRobustCommand*\luastd[1]{%
\index{funzioni!#1@\fn{#1}}}
\DeclareRobustCommand*\luak[1]{%
\index{keyword!#1@\key{#1}}}
\DeclareRobustCommand*\luas[1]{%
\index{#1@\key{#1}}}

\makeatletter
\newcommand\guidalualicensebox{
\noindent
\begin{tikzpicture}
\begin{scope}[scale=0.18]
\guidalualogocmd
\end{scope}
\node[anchor=base] at (40mm,1.25mm) {\@@title};
\node[anchor=base] at (40mm,-3.5mm) {Copyright \textcopyright{} \the\year, \@@author};
\end{tikzpicture}\\[\baselineskip]
Questa documentazione è soggetta alla licenza LPPL
\href{http://www.latex-project.org/lppl.txt}{\LaTeX{} Project Public
License}, versione 1.3 o successive ed è curata dall'autore.}
\makeatother

\directlua{
    local d = [[\filedate]]
    token.set_macro('guidaluadate', d:gsub('/', '-'))
}

\licenza{%
\guidalualicensebox

\medskip Guide's \pack{biblatex} entry:\\
\begingroup
\ttfamily\footnotesize\noindent
\begin{tabular}{@{}l@{}l@{}l@{}}
\toprule
\multicolumn{3}{l}{\hspace{-6pt}@manual\{guit:guidalua,}\\
\hspace{8pt}
 & title        & =\{Guida al linguaggio Lua per Lua\cs{TeX}\},\\
 & author       & =\{Giacomelli, Roberto\},\\
 & date         & =\{\guidaluadate\},\\
 & version      & =\{\fileversion\},\\
 & pagetotal    & =\{\thelastpage\},\\
 & langid       & =\{italian\},\\
 & url          & =\{https://www.guitex.org/home/images/doc/GuideGuIT/guidalua.pdf\},\\
 & urldate      & =\{2021-04-01\},\\
 & organization & =\{GuIT, Gruppo Utilizzatori Italiani di \cs{TeX}\},\\
 & series       & =\{Guide Tematiche del GuIT\},\\
\}\\
\bottomrule
\end{tabular}
\endgroup
}

\newcommand\guidalualogocmd{
\draw[fill,verdeguit] (0,0) circle (2.2);
\draw[fill,verdeguit] (2.31, 2.31) circle (0.72);
\draw[fill, white] (0.8,0.8) circle (0.72);
}

\begin{document}
\newtcolorbox[blend into=listings]{tcbfloat}[2][]{float=htb,title={#2},#1}
\hypersetup{
    colorlinks=true,
    linkcolor=verdeguit,
    citecolor=verdeguit,
    urlcolor=verdeguit!80!red
}
\author{Roberto Giacomelli}
\date{\filedate{} --- \fileversion}
\title{Guida al linguaggio Lua per \LuaTeX}
\subtitle{
\begin{tikzpicture}[scale=0.82]
\node at (-2.31-0.72, 0) {};
\guidalualogocmd
\end{tikzpicture}
\vspace*{8.5ex}
}
\maketitle


% presentazione della guida e informazioni di base per eseguire gli esempi

\chapter{Presentazione}

Questa guida tematica è dedicata alla programmazione in Lua all'interno dei
motori di composizione del sistema \TeX{} per utenti avanzati e sviluppatori.

Come dimostreremo, la nuova generazione di compositori amplia notevolmente gli
scenari applicativi. Se da un lato è auspicabile che queste potenzialità
diventino disponibili per gli utenti finali per mezzo di moduli e pacchetti,
dall'altro è utile fornire dettagli ed esempi per implementare proprie soluzioni
o per poter scrivere nuovi moduli condividendone lo sviluppo con la community.

Sono certamente molte le cose da conoscere: un nuovo linguaggio molto diverso da
\TeX{}, numerosi dettagli sul funzionamento interno dei compositori Lua-powered,
nuovi problemi di organizzazione del codice, di bilanciamento tra Lua e \TeX,
eccetera. Per questo, ho pensato di contribuire con questa guida cercando di
presentare il quadro della crescente complessità del sistema.


\section{Piano della guida}

La guida è divisa in tre parti: la prima offre una panoramica rapida in forma di
\emph{tutorial} passo passo per iniziare subito con Lua e comprenderne come
eseguirne il codice all'interno del sorgente (parte \ref{partTutorial}), la
seconda tratta delle basi del linguaggio Lua e contiene numerosi esercizi (parte
\ref{partFoundation}), e infine la terza presenta esempi applicativi con lo
sviluppo di librerie e l'uso della tecnologia dei nodi (parte \ref{partApp}).

Tra gli argomenti ci sono:
\begin{compactitemize}
\item differenza tra motore e formato di composizione \( \to \) capitolo
\ref{iChExplain},
\item basi del linguaggio Lua \( \to \) dal capitolo \ref{chFondAssignment},
\item tecniche di programmazione e di rappresentazione dei dati \( \to \) dal
capitolo \ref{iiichRegistro},
\item la tecnologia dei nodi in Lua \( \to \) capitolo \ref{iichTartaglia}.
\end{compactitemize}


\section{Origine della guida}

Per illustrare i concetti del linguaggio ho preso spunto da un breve corso su
Lua che scrissi qualche tempo fa per il blog
\href{http://parliamodi-ubuntu.blogspot.it}{Lubit Linux} di Luigi Iannoccaro
che mi propose di realizzare un progetto di divulgazione su Lua. Luigi ha
acconsentito all'utilizzo di quegli appunti per produrre questa guida tematica.


\section{Contribuire e collaborare}

Spero che i lettori vorranno contribuire al testo inviando proprie soluzioni o
nuovi contributi, piccoli o grandi. Lo si può fare attraverso lo strumento che
preferite, scrivendomi un messaggio di posta elettronica all'indirizzo
\href{mailto:giaconet.mailbox@gmail.com}{\texttt{giaconet.mailbox@gmail.com}},
oppure utilizzando il
\href{https://github.com/GuITeX/guidalua}{\texttt{repository git}} dei sorgenti
della guida, eseguendo un Pull Request o ancora aprendo una discussione premendo
il pulsante Issues.


\section{Altre risorse su Lua}

La risorsa principale per imparare Lua, a cui rimando per tutti gli
approfondimenti, è certamente il PIL \cite{PIL}, acronimo del titolo del libro
\emph{Programming In Lua} di Roberto Ierusalimschy il principale Autore di Lua.
Questo testo non solo è completo e autorevole ma è anche ben scritto e
composto\footnote{Tra l'altro il libro ufficiale su Lua viene composto in
\LaTeX{} e commercializzato per contribuire allo sviluppo del linguaggio
stesso.}.

La seconda importante risorsa su Lua si trova in rete all'indirizzo
\href{https://www.lua.org/manual/5.3/}{\texttt{www.lua.org/manual/5.3/}} ed è il
\emph{reference} del linguaggio \cite{web:luaref} con le specifiche di sintassi,
metametodi, funzioni di libreria, C API, eccetera.

Quanto a \LuaTeX{} il riferimento è il suo manuale \cite{prg:luatex} che, come
quasi tutta la documentazione nel sistema \TeX{}, può essere visualizzato a
video con il comando da terminale:
\begin{Verbatim}[numbers=none]
$ texdoc luatex
\end{Verbatim}


\section{Note di lettura}
\label{secZeroIntroNote}

Nei listati compilabili riportati nella guida compare alla prima linea la
\emph{riga magica}, un commento utile per dare istruzioni all'editor sul
compilatore da usare, ma che qui aiuterà il lettore a stabilire il contesto del
codice. La sintassi delle righe magiche dipende dall'editor, in questa guida
useremo le regole di TeXworks.

Se presente nel progetto, alla seconda riga dei listati si troverà invece il
nome del file che il lettore potrà scaricare ed eseguire per i propri
esperimenti dal \href{https://github.com/GuITeX/guidalua}{\texttt{repository
git}}.

Nella parte \ref{partFoundation} ho cercato di non dare per scontati i concetti
fondamentali della programmazione. Ovviamente il lettore già preparato procederà
più velocemente nel prendere dimestichezza con Lua. Ho invece escluso dalla
guida \TeX{}, per esempio non spiegando come si definisce una macro utente o
come si lavora con il formato \LaTeX\(3\). Rimando senza indugio alla copiosa
documentazione disponibile a cominciare da quella scaricabile dal sito \GuIT.

\LuaTeX{} è un programma molto complesso, uno strumento eccezionalmente vasto.
Le applicazioni illustrate nella guida non possono che derivare dalle mie
conoscenze. Non metto in dubbio che ci siano modi migliori, più sicuri, più
efficienti, più accurati, più completi di riscriverne il codice.

Auguro buona e proficua lettura nell'esplorare l'avampost%
\begin{tikzpicture}[scale=0.042]
\guidalualogocmd
\end{tikzpicture} lunare.

% end of file


\newpage

\tableofcontents

\mainmatter*

\part{Il sistema Lua+\TeX}
\label{partStart}


% codename "magpie"

\chapter{Let's start with Lua}
\label{iChStart}

Consideriamo un sorgente \LaTeX{} minimale:
\begin{lines}
#[tex]
% !TeX program = LuaLaTeX
\documentclass{article}
\begin{document}
\end{document}
\end{lines}

I suoi elementi sono l'utile commento che informa l'editor di quale compositore
usare per la compilazione detto riga magica, la dichiarazione di classe e infine
il corpo vuoto del documento.

Essendo il compositore \LuaLaTeX{} abbiamo a disposizione la nuova primitiva
\cs{directlua}. Potremo inserirla più volte in qualsiasi punto del sorgente. Se
la inserissimo nel preambolo del sorgente minimo avremo:

\CLRmarginpar{Lua in \LaTeX\\
\gotosec{} \ref{iSecLuaInLuaLaTeX}}[true]% true option means +16pt of vspace
%
\CLRmarginpar{\code{\backslash directlua}\\
\gotosec{} \ref{iSecDirectLua}}
%
\begin{lines}
#[tex]
% !TeX program = LuaLaTeX
\documentclass{article}
\directlua{<codice Lua>}
\begin{document}
\end{document}
\end{lines}

Qui sta il punto. La primitiva \cs{directlua} è come se commutasse il modo di
compilazione da \LaTeX{} a Lua, un linguaggio di programmazione per scopi
generali, efficiente e portabile. Qualsiasi cosa faccia il \code{<codice Lua>}
racchiuso tra graffe esso sarà eseguito dall'interprete Lua e non da \TeX.

Quali sono le conseguenze? Qual è l'utilità dell'interprete Lua?

\CLRmarginpar{Passare a \LuaLaTeX\\
\gotosec{} \ref{iSecPassareALuaTeX}}
%
La risposta a queste ottime domande è duplice. La prima è che il motore
\prog{luatex} rispetto a \prog{pdftex} è stato migliorato in molti aspetti, per
esempio è in grado di gestire i font di tipo Open Type nativamente. Solo questo
lo rende molto interessante per l'utente.

La seconda è che Lua può \emph{integrare} l'elaborazione \TeX{} eseguendo
compiti complessi come il calcolo numerico, l'interrogazione di database,
l'editing del testo, l'interazione con i servizi di rete, la costruzione di
oggetti tipografici di complessità arbitraria, la grafica vettoriale, eccetera
eccetera.


\section{Un primo esempio}

Desideriamo che nel nostro documento ci sia un testo allineato sulla diagonale
della pagina. Per ottenerlo usando il comando \cs{rotatebox} del pacchetto
\pack{graphicx} è necessario calcolare il valore dell'angolo di rotazione in
gradi conoscendo la misura di larghezza e altezza della pagina e questo implica
il calcolo dell'arcotangente del rapporto. Nel caso di questa guida:
\[
    \alpha = \frac{180}{\pi}\arctan{\frac{h}{w}} = \directlua{
        local w = tex.pagewidth
        local h = tex.pageheight
        local angle = (180/math.pi) * math.atan(h/w)
        tex.print(tostring(angle))
    }
\]

Applicando qualche metodo numerico o pacchetto, sono certo che \TeX{} sia in
grado di calcolare l'angolo con sufficiente precisione pur non disponendo di
alcuna libreria in virgola mobile. Con Lua invece è un calcolo diretto in
floating point:

\CLRmarginpar{Tabella di Lua\\
\gotosec{} \ref{iiChTabella}}[true]% true option means +16pt of vspace
%
\CLRmarginpar{Libreria \code{math}\\
\gotosec{} \ref{iiSecMathLibrary}}
%
\begin{lines}
#[tex]
#[indexfile=code/e0-001.tex]
% filename: code/e0-001.tex
\directlua{
    local w = tex.pagewidth
    local h = tex.pageheight
    local alpha = (180/math.pi) * math.atan(h/w)
    token.set_macro('angdeg', alpha)
}
\end{lines}

Qualche commento sulle quattro linee di codice Lua\footnote{Il listato completo
del sorgente è scaricabile dalla directory \code{code} con il nome di
\file{e0-001.tex} dall'archivio dei sorgenti della guida.}: le prime due
leggono le dimensioni della pagina in due variabili locali --- rappresentano
perciò i dati di ingresso dell'elaborazione --- ma attenzione i campi
\key{pagewidth} e \key{pageheight} della tabella \key{tex}, senza entrare ora
nel dettaglio di cosa sia un campo o una tabella in Lua, non sono valori in sola
lettura di valore pari a quello dei corrispondenti parametri dimensionali. Sono
bensì le dimensioni \cs{pagewidth} e \cs{pageheight} stesse.

La terza linea esegue il calcolo in virgola mobile in doppia precisione e
l'ultima crea una macro \cs{angdeg} con il valore dell'angolo --- rappresenta
percò l'output dell'elaborazione --- esattamente come se fosse stata creata con
\cs{def}.

Nel motore di composizione, Lua ha accesso diretto ai registri dimensionali e ai
contatori e può creare macro. Significa che l'implementazione del motore di
composizione stesso è nativamente interfacciato con Lua.


\newcommand{\ordinanomi}[1]{\directlua{
    local list = {#1}
    table.sort(list)
    tex.print(table.concat(list, ', '))
}}

\section{Un secondo esempio}

Abbiamo una lista dei nomi di partecipanti a un evento, ma vorremmo evitare di
ordinarli a mano. La lista si presenta come un elenco separato da virgole dei
nomi (già ordinati): \ordinanomi{'Giacomo', 'Tonio', 'Elena', 'Anna', 'Stefano',
'Milena', 'Federico', 'Giovanni', 'Massimo', 'Basilio', 'Beatrice'}.

In Lua bastano tre righe di codice per definire il comando \cs{ordina} che ha
come unico argomento la lista:

\CLRmarginpar{Libreria \code{table}\\
\gotosec{} \ref{iiSecTableLibrary}}[true]% true option means +16pt of vspace
%
\CLRmarginpar{\fn{tex.print}\\
\gotosec{} \ref{iSecPassaggioDati}}
%
\begin{lines}
#[tex]
#[indexfile=code/e0-002.tex]
% filename: code/e0-002.tex
\newcommand{\ordina}[1]{\directlua{
    local list = {#1}
    table.sort(list)
    tex.print(table.concat(list, ", "))
}}
\end{lines}

Il dato in ingresso, la lista dei nomi non ordinata, viene costruito grazie al
meccanismo della sostituzione degli argomenti di una macro. Il dato in uscita,
la lista dei nomi ordinata, perviene al compositore come testo in lettura come
se fosse stato presente nel file sorgente.

I tre passi del codice sono la costruzione della lista con una tabella,
l'ordinamento degli elementi, e l'invio al buffer di lettura dei caratteri gli
elementi concatenati e separati con una virgola.

Anche qui, se non fosse chiaro quanto avete appena letto è perché vorrei
introdurre prima come si definisce un comando \LaTeX{} che contiene solamente
codice Lua rispetto ai concetti del linguaggio che comunque potete raggiungere
con i riferimenti al margine ai capitoli della parte~\ref{partFoundation}.

Nel documento un tale comando si usa in questo modo, con i nomi necessariamente
da delimitare come stringhe:
\begin{lines}
#[tex]
#[indexfile=code/e0-002.tex]
% filename: code/e0-002.tex
\ordina{"Giacomo", "Tonio", "Elena", "Anna", "Stefano",
    "Milena", "Federico", "Giovanni", "Massimo", "Basilio",
    "Beatrice"}
\end{lines}

Con qualche riga di codice in più, sarebbe semplice eliminare dall'input la
delimitazione dei nomi come stringhe, oppure inserire un criterio di ordinamento
più complesso come quello per cognome e nome.


\section{Direzioni di lettura}

Il prossimo capitolo~\ref{iChExplain} si occupa di spiegare le particolarità dei
motori di composizione Lua-powered distinguendoli dal formato, e di come possano
eseguire sorgenti \LaTeX{}.

I capitoli dal~\ref{iiChAssignment} fino al~\ref{iiChOop} trattano delle basi
del linguaggio Lua.

I tutorial applicativi in Lua

comporre documenti \TeX. Implementeremo una \emph{calcolatrice}
(capitolo~\ref{iChCalcolatrice}) e comporremo una tabella numerica ripetitiva
(capitolo~\ref{iChTabellaPesi}).


% end of file


\input{chapter/I-02-howto}

\part{Fondamenti del linguaggio Lua}
\label{partFoundation}



\chapter{Come eseguire gli esercizi}
\label{iiChEseguireEsercizi}

È certamente fondamentale eseguire noi stessi esempi ed esercizi di
programmazione allo scopo di acquisire la padronanza di Lua. Nella guida ne
trovate alcuni alla fine di ciascun capitolo della parte seconda. 

Questa sezione vi introduce brevemente al programma \prog{texlua} che già
trovate compreso in ogni recente distribuzione \TeX{}. Si tratta dell'interprete
Lua controparte di \prog{luatex}.

Rispetto all'interprete \prog{lua} standard \prog{texlua} non ha la modalità
interattiva REPL\footnote{Read–eval–print loop.} con cui si digita una linea di
codice alla volta in un prompt interattivo, modalità molto utile per fare prove
velocemente.

Il codice dunque, andrà memorizzato in un file con estensione \texttt{.lua}.
Come esempio elementare, digitiamo questa unica riga di codice in un file
di testo chiamato \file{primo.lua}:
\begin{lines}
print("Hello World!")
\end{lines}
apriamo una finestra di terminale\footnote{Maggiori dettagli per diversi sistemi
operativi sulla linea di comando possono essere trovati nella guida tematica
dedicata \emph{Guida alla console} scaricabile dal sito \GuIT.} e lanciamo il
comando:
\begin{Verbatim}[numbers=none,xleftmargin=0pt]
$ texlua primo.lua
\end{Verbatim}

Ora che sappiamo come eseguire il codice Lua, concentriamoci con i prossimi
capitoli sulle basi del linguaggio. Torneremo nella terza parte della guida su
ulteriori modalità di esecuzione anche per il codice Lua interno a sorgenti
\TeX.





\chapter{Assegnazione e tipi predefiniti}
\label{chFondAssignment}


\section{Lua, proprio un bel nome}

Lua è un linguaggio semplice ma non banale. Il suo ambito di applicazione è
quello dei linguaggi di scripting: text processing, manutenzione del sistema,
elaborazioni su file dati, eccetera e lo si può anche trovare come linguaggio
embedded di programmi complessi come i videogiochi o altri applicativi
che danno la possibilità di essere programmati con esso dall'utente.

Lua è stato ideato da un gruppo di programmatori esperti
dell'\href{http://www.puc-rio.br/index.html}{Università Cattolica di Rio de
Janeiro} in Brasile. In portoghese ``Lua'' si pronuncia LOO-ah e significa
``Luna''!


\section{L'assegnazione}
\label{secFondAssegnazione}

Ci occupiamo ora di uno degli elementi di base dei linguaggi informatici:
l'istruzione di \emph{assegnazione}. Con questa operazione viene introdotto un
\emph{simbolo} nel programma associandolo a un valore che apparterrà a uno
dei possibili \emph{tipi} di dato.

La sintassi di Lua non sorprende: a sinistra compare il nome della variabile e
a destra l'espressione che fornirà il valore da assegnare al simbolo. Il
carattere di `\texttt{=}' funge da separatore:
\begin{lines}
a = 123
\end{lines}

Durante l'esecuzione di questo codice, Lua determina dinamicamente il tipo del
valore letterale `123' --- un numero --- creandolo in memoria col nome di
`\key{a}'.

L'istruzione di assegnazione omette il tipo di dato non essendone prevista una
dichiarazione esplicita. In altre parole, i dati hanno un tipo, determinabile
con la funzione \fn{type}\luastd{type}, ma ciò non ha rilevanza semantica ed è
solo in fase di esecuzione che l'interprete determina il tipo di dato e in
funzione di questo ne memorizza il valore in memoria.

Altro concetto importante di Lua è che le variabili sono tutte globali a meno
che non le si dichiari locali al blocco di codice.


\subsection{Locale o globale?}
\label{secFondLocaleGlobale}

Una proprietà dell'assegnazione è che se non diversamente specificato Lua
crea i simboli nell'ambiente globale del codice in esecuzione. Se si
desidera creare una variabile locale rispetto al blocco di codice in cui è
definita, occorre premettere alla definizione la parola chiave \key{local}.

Le variabili locali evitano alcuni errori di programmazione e in Lua rendono il
codice più veloce. Le useremo \emph{sempre} quando un simbolo appartiene in modo
semantico a un blocco, per esempio al corpo di una funzione\footnote{Da notare
che in sessione interattiva, ovvero nel modo REPL dell'interprete Lua, ogni riga
è un blocco quindi le variabili locali non sopravvivono alla riga successiva.
Perciò in questa modalità si usano solo variabili globali.}.

Se si crea una variabile locale con lo stesso nome di una variabile globale
quest'ultima viene \emph{oscurata} e il suo valore sarà protetto da modifiche
fino a che il blocco in cui è definita la variabile locale non termina.


\subsection{Assegnazioni multiple}

In Lua possono essere assegnate più variabili alla volta nella stessa istruzione
con una sintassi in realtà più complessa di quella presentata fino a ora. A
sinistra del simbolo di uguale è ammessa una lista di variabili separate da
virgole e a destra una lista di espressioni sempre separate da virgole che, una
volta valutate, saranno assegnate in ordine.

Se si premette la parola chiave \key{local} tutte le variabili saranno locali.
La sintassi generale dell'assegmento multiplo è la seguente:
\begin{lines}
[local] var_1, var_2, var_3, ... = expr_1, expr_2, expr_3, ...
\end{lines}
perfettamente equivalente a:
\begin{lines}
[local] var_1 = expr_1
[local] var_2 = expr_2
[local] var_3 = expr_3
\end{lines}

Così
\begin{lines}
local a, b = 0.45 + 0.23, "text" -- a = 0.68; b = "text"
\end{lines}
è equivalente a:
\begin{lines}
local a = 0.45 + 0.23
local b = "text"
\end{lines}

Quando il numero delle variabili non corrisponde a quello delle espressioni, Lua
assegnerà automaticamente valori \texttt{nil} o ignorerà le espressioni in
eccesso. Per esempio:
\begin{lines}
local a, b, c = 0.45, "text"  -- c vale nil
local x, y = "op", "qw", "lo" -- "lo" è un dato ignorato
\end{lines}

Nell'assegnazione Lua prima valuta le espressioni a destra e solo
successivamente crea le rispettive variabili secondo l'ordine della lista.
Perciò per scambiare il valore di due variabili, operazione chiamata
\emph{switch}, è possibile scrivere semplicemente:
\begin{lines}
x, y = y, x
\end{lines}

Un ulteriore esempio di assegnazione multipla è il seguente, a dimostrazione
che le espressioni della lista a destra vengono prima valutate e solo dopo
assegnate alle corrispondenti variabili nella lista di sinistra:
\begin{lines}
#[run]
local pi = 3.14159
local r = 10.8 -- raggio del cerchio
-- grandezze cerchio
local diam, circ, area = 2*r, 2*pi*r, pi*r^2
-- stampa grandezze
print("Diametro:", diam)
print("Circonferenza:", circ)
print("Area:", area)
\end{lines}

Le assegnazioni multiple sono interessanti ma sembra non siano così importanti,
possiamo infatti ricorrere ad assegnazioni singole. Diverranno invece molto
utili con le funzioni e con gli iteratori di cui ci occuperemo in seguito.


\section{Una manciata di tipi}
\label{secFondManciataTipi}

In Lua esistono una manciata di tipi. Essenzialmente, omettendone due di uso
avanzato, sono solo questi sei:
\begin{compactitemize}
\item \key{number} il tipo numerico\footnote{Solamente dalla versione 5.3 di Lua
vengono internamente distinti gli interi e i numeri in virgola mobile};
\item \key{table} il tipo tabella \( \to \) capitolo \ref{chFondTabella};
\item \key{boolean} il tipo booleano \( \to \) capitolo \ref{chFondOpLogic};
\item \key{string} il tipo stringa \( \to \) capitolo \ref{chFondStringhe};
\item \key{function} il tipo funzione \( \to \) capitolo \ref{chFondFunzioni}.
\item \key{nil} il tipo nullo \( \to \) sezione \ref{secFondTipoNil};
\end{compactitemize}

Il breve elenco suscita due osservazioni: tranne la tabella non esistono
tipi strutturati mentre le funzioni hanno il rango di tipo.

Questo fa capire molto bene il carattere di Lua: da un lato l'essenzialità ha
ridotto all'indispensabile i tipi predefiniti nel linguaggio, ma dall'altro ha
spinto all'inclusione di concetti intelligenti e potenti.


\subsection{Il tipo \key{nil}}
\label{secFondTipoNil}

Uno dei concetti più importanti che caratterizzano un linguaggio di
programmazione è la presenza o meno del tipo nullo. In Lua esiste e viene
chiamato \key{nil}\luak{nil}. Il tipo nullo ha un solo valore possibile,
anch'esso chiamato \key{nil}. Il nome è così sia l'unico valore possibile che il
tipo.

Leggere una variabile che non esiste non è un errore perché Lua restituisce
semplicemente \key{nil}, mentre assegnare il valore nullo a una variabile la
distrugge:
\begin{lines}
print(z)      --> stampa nil, la variabile 'z' non esiste
local z = 123 --> assegnazione di un tipo numerico
print(z)      --> stampa 123
z = nil       --> distruzione della variabile
\end{lines}


\section{Gli identificatori}

I nomi che possiamo dare a variabili e funzioni, sono stringhe di lettere
maiuscole o minuscole, numeri e trattini bassi \key{'\_'}\luas{\_} purché non
comincino con una cifra e non corrispondano alle \emph{keyword}, le parole
riservate del linguaggio.

In Lua gli identificatori sono \emph{case sensitive}: le stesse lettere formano
identificatori distinti se esse sono diverse nel maiuscolo o minuscolo, come per
\key{vaR} e \key{Var}.

Si è quindi liberi di utilizzare nomi qualsiasi, tuttavia è conveniente aderire
a qualche \emph{convenzione} che aggiunga al nome un significato di categoria.
Per esempio, gli identificatori che iniziano con una lettera maiuscola si
possono riservare agli oggetti, come nel capitolo \ref{chFondOop}, oppure quelli
che iniziano con un solo trattino basso, l'underscore \key{'\_'}, pur essendo
identificatori come tutti gli altri, si possono riservare per dati privati come
campi di tabella o funzioni ausiliarie.

Infatti nel capitolo \ref{chFondIteratori} degli iteratori si fa uso
dell'identificatore \key{'\_'} per le variabili di ciclo non utilizzate.
Qualsiasi sia la convenzione adottata per i nomi, l'importante è rispettarne
sempre le regole per evitare errori o un listato meno chiaro.

Sono invece comunque da evitare i nomi che iniziano con un doppio trattino
basso, che potrebbero collidere con quelli dei metametodi (vedi il capitolo
\ref{chFondOop}) e quelli che pur iniziando con un singolo trattino basso hanno
poi lettere tutte in maiuscolo come per esempio
\key{\_VERSION}\luastd{\_VERSION}, perché potrebbero collidere con i nomi
predefiniti di Lua.


\section{Il Garbage Collector}

I dati non più utili come quelli di cui non esiste più un \emph{riferimento} a
essi durante l'esecuzione, per esempio perché la variabile è stata riassegnata a
\key{nil}, oppure quelli locali nel momento in cui escono di scopo, vengono
automaticamente eliminati dal \emph{garbage collector} di Lua. Questo componente
solleva l'utente dalla gestione diretta della memoria e sopratutto dai deleteri
errori di programmazione che si possono facilmente compiere nel farlo, al prezzo
di una piccola diminuzione delle prestazioni in fase di esecuzione.

Al termine del programma tutte le risorse in memoria vengono automaticamente
liberate.


\section{Esercizi}

\begin{Exercise}[label=fond-01]
Usando due sole istruzioni, scrivere il codice Lua che crea due variabili
\key{x} e \key{y} di valore 12.34 e assegni alle altre due variabili \key{sum} e
\key{prod} rispettivamente la somma e il prodotto delle prime due. Si stampino
in console i risultati.
\end{Exercise}

\begin{Exercise}[label=fond-02]
Scrivere il codice Lua che dimostri che modificare una variabile locale non
modifica il valore della variabile globale con lo stesso nome. Suggerimento:
utilizzare la coppia \key{do}/\key{end} per creare un blocco di codice con le
proprie variabili locali.
\end{Exercise}

% end of file


\input{chapter/II-02-tabella}



\chapter{Costrutti di base}

\section{Il ciclo \key{for} e il condizionale \key{if}}
\label{secFondCicloIf}

Cominciamo con il contare i numeri pari contenuti in una tabella che funziona
come un array, ricordandoci che gli indici partono da 1 e non da 0. Rileggete
il capitolo precedente come utile riferimento.

Creiamo la tabella con il costruttore in linea e iteriamo con un ciclo
\key{for}:
\sourcecode{file = [[code/e1-001.lua]], run = true}

Il corpo del ciclo \key{for}\luak{for} di Lua è il blocco compreso tra le parole
chiave obbligatorie \key{do}\luak{do} ed \key{end}. La variabile \key{i} interna
al ciclo assumerà i valori da 1 fino al numero di elementi della tabella,
ottenuto con l'operatore lunghezza \key{\#} valido anche per le stringhe.

Per ciascuna iterazione con il costrutto condizionale \key{if}\luak{if}
incrementiamo un contatore solo se l'elemento della tabella è pari. L'\key{if}
ha anch'esso bisogno di definire il blocco di codice e lo fa con le parole
chiavi obbligatorie \key{then}\luak{then} ed \key{end}\luak{end}, mentre
\key{else}\luak{else} o \key{elseif}\luak{elseif} sono rami di codice
facoltativi.

Il controllo di parità degli interi si basa sull'operatore modulo
\key{\%}\luas{\%}, resto della divisione intera. Infatti un numero pari è tale
se il resto della divisione per 2 è zero. 

L'operatore di \emph{uguaglianza} è il doppio carattere di uguale
\key{==}\luas{==} e quello di \emph{disuguaglianza} è la coppia dei segni tilde
e uguale \verb|~=|\luas{~=}. Naturalmente funzionano anche gli operatori di
confronto \key{>}\luas{>}, \key{>=}\luas{>=} e \key{<}\luas{<},
\key{<=}\luas{<=}.


\section{Operatore lunghezza}
\label{secFondLenOperator}

Ma come si comporta l'operatore di lunghezza \key{\#}\luas{\#} per le
tabelle array con indici non lineari? Per esempio, qual è il risultato del
seguente codice:
\begin{lines}
local t = {}
t[1] = 1
t[2] = 2
t[1000] = 3
print(#t)
\end{lines}
e in questo caso cosa verrà stampato?
\begin{lines}
local t = {}
t[1000] = 123
print(#t)
\end{lines}
e ancora in questo caso con il costruttore?
\begin{lines}
local t1 = {nil, nil,  nil, nil, nil, nil, nil, 8}
print(#t1)
local t2 = {nil, nil,  nil, nil, nil, nil, nil, 8, nil}
print(#t2)
\end{lines}

L'operatore \key{\#} restituisce la lunghezza di una tabella array come uno dei
sui \emph{bordi}. Un bordo è la posizione di un valore non \key{nil} seguito da
un valore \key{nil}, oppure zero se la posizione 1 è \key{nil}.

Questo comportamento riflette la particolare e sofisticata implementazione della
tabella di Lua. L'operatore lunghezza \key{\#} può restituire uno dei
qualsiasi indici interi corrispondenti a un bordo della tabella, in funzione di
come è stata creata o anche della presenza di chiavi non intere.

Se la tabella è una \emph{sequenza}, cioè se non ci sono buchi di valori tra le
posizioni intere partendo da 1 fino a \( n \), allora la tabella ha un solo
bordo che vale \( n \). Se la tabella è vuota il suo bordo vale zero.

L'operatore \key{\#} viene calcolato molto velocemente anche per tabelle array
grandi, ed è usato per espressioni idiomatiche come quella usatissima per
l'inserimento di un nuovo dato in coda a una sequenza:
\begin{lines}
#[run]
local t = {101, 102, 103, 104, 105} -- una sequenza
t[#t + 1] = 106
print(#t, t[6])
\end{lines}


\section{Il ciclo \key{while}}
\label{secCicloWhile}

Passiamo a scrivere il codice per inserire in una tabella i fattori primi di un
numero. Fatelo per esercizio e poi confrontate il codice seguente che utilizza
l'operatore modulo \key{\%}:
\sourcecode{file = [[code/e1-002.lua]], run = true}

Così abbiamo introdotto anche il ciclo \key{while}\luak{while}
perfettamente coerente con la sintassi dei costrutti visti fino a ora: il blocco
di codice ripetuto fino a che la condizione è vera, è obbligatoriamente definito
da due parole chiave, quella di inizio è \key{do} e quella di fine è \key{end}.

Le variabili definite come locali nei blocchi del ciclo \key{for}, nei rami del
condizionale \key{if} e nel ciclo \key{while}, non sono visibili all'esterno.


\section{Intermezzo}

In Lua non è obbligatorio inserire un carattere delimitatore sintattico ma è
facoltativo il segno \key{;}. I caratteri spazio, tabulazione e ritorno a capo
vengono considerati dalla grammatica come separatori, perciò si è liberi di
formattare il codice come si desidera inserendo per esempio più istruzioni sulla
stessa linea. Solitamente non si utilizzano i punti e virgola finali, ma se ci
sono due assegnazioni sulla stessa linea --- stile sconsigliabile perché poco
leggibile --- li si può separare almeno con un segno \key{;}. Come sempre una
forma stilistica chiara e semplice vi aiuterà a scrivere codice più
comprensibile anche a distanza di tempo.

Generalmente è buona norma definire le nuove variabili il più vicino possibile
al punto in cui verranno utilizzate per la prima volta, un beneficio per la
comprensione ma anche per la correttezza del codice perché può evitare di
confondere i nomi e magari di introdurre errori.


\section{Il ciclo \key{for} con il passo}

Provate a scrivere il codice Lua che verifica se un numero è \emph{palindromo},
ovvero che gode della proprietà che le cifre decimali sono simmetriche come per
esempio avviene per il numero 123321. Confrontate poi questa soluzione:
\sourcecode{file = [[code/e1-003.lua]], select = [[prima_sol]], run = true}

La soluzione utilizza una tabella per memorizzare le cifre in ordine inverso del
numero da verificare, che vengono poi utilizzate successivamente nel ciclo
\key{for}\luak{for} dall'ultima --- la cifra più significativa --- fino alla
prima per ricalcolare il valore. Se il numero iniziale è palindromo allora il
corrispondente numero a cifre invertite è uguale al numero di partenza.

Nel ciclo \key{for} il terzo parametro opzionale -1 imposta il passo per la
variabile \key{i} che quindi passa dal numero di cifre del numero da
controllare (6 nel nostro caso) a 1.

In effetti non è necessaria la tabella:
\sourcecode{file = [[code/e1-003.lua]], select = [[seconda_sol]], run=true}


\section{\key{if} a rami multipli}

Il prossimo problema è il seguente: determinare il numero di cifre di un
intero. Ancora una volta, confrontate il codice proposto solo dopo aver cercato
una vostra soluzione.
\sourcecode{file = [[code/e1-004.lua]]}

Questo esempio mostra in azione l'\key{if}\luak{if} a più rami che in Lua svolge
la funzione del costrutto \key{switch} presente in altri linguaggi, con una
nuova parola chiave: \key{elseif}.

L'esempio è interessante anche per come viene introdotta la variabile
\key{digits}, cioè senza inizializzarla per poi assegnarla nel ramo opportuno
dell'\key{if}. Infatti una variabile interna a un blocco non sopravvive oltre,
per questo motivo dichiararla all'interno dell'\key{if} non è sufficiente.

Come è necessario \emph{non} premettere \key{local}\luak{local} nelle
assegnazioni nei rami del condizionale: in questo caso verrebbe creata una nuova
variabile locale al blocco che \emph{oscurerebbe} quella esterna con lo stesso
nome. In altre parole, al termine del condizionale \key{digits} varrebbe ancora
\key{nil}, il valore che assume nel momento della dichiarazione.


\section{Esercizi}

\begin{Exercise}[label=cos-01]
Contare quanti interi sono divisibili sia per 2 che per 3 nell'intervallo \( [1,
10\,000]\). Suggerimento: utilizzare l'operatore modulo \key{\%}, resto della
divisione intera tra due operandi.
\end{Exercise}

\begin{Exercise}[label=cos-02]
Determinare i fattori del numero intero \(5\,461\,683\) modificando il codice
riportato alla sezione~\ref{secCicloWhile} per includerne la molteplicità.
\end{Exercise}

\begin{Exercise}[label=cos-03]
Calcolare il determinante della matrice corrispondente alla seguente tabella,
che contiene tre tabelle/array con tre numeri in sequenza.
\begin{lines}
local t = {
    { 0,  5, -1},
    { 2, -2,  0},
    {-1,  0,  1},
}
\end{lines}
\end{Exercise}

\begin{Exercise}[label=cos-04]
Data la tabella seguente stampare in console il conteggio dei numeri pari e dei
numeri dispari contenuti in essa. Verificare che la somma di questi due
conteggi sia uguale alla dimensione della tabella.
\begin{lines}
local t = {
    45, 23, 56, 88, 96, 11,
    80, 32, 22, 85, 50, 10,
    32, 75, 10, 66, 55, 30,
    10, 13, 23, 91, 54, 19,
    50, 17, 91, 44, 92, 66,
    71, 25, 19, 80, 17, 21,
    81, 60, 39, 15, 18, 28,
    23, 10, 18, 30, 50, 11,
    50, 88, 28, 66, 13, 54,
    91, 25, 23, 17, 88, 90,
    85, 99, 22, 91, 40, 80,
    56, 62, 81, 71, 33, 30,
    90, 22, 80, 58, 42, 10,
}
\end{lines}
\end{Exercise}

\begin{Exercise}[label=cos-05]
Data la tabella precedente, scrivere il codice per costruire una seconda tabella
uguale alla prima ma priva di duplicati e senza alterare l'ordine degli interi.
\end{Exercise}

\begin{Exercise}[label=cos-06]
Data la tabella precedente costruire una tabella le cui chiavi siano i numeri
contenuti in essa e i valori siano il corrispondente numero di volte che la
chiave stessa compare nella tabella di partenza. Stampare poi in console il
numero che si presenta il maggior numero di volte.
\end{Exercise}

% end of file




\chapter{Operatori logici}

In Lua un'espressione è vera se essa corrisponde al valore booleano
\key{true}\luak{true} oppure a un valore che non è \key{nil}\luak{nil}.

Gli operatori logici \key{and}\luak{and}, \key{or}\luak{or} e
\key{not}\luak{not} danno luogo ad alcune espressioni idiomatiche di Lua.
Cominciamo con \key{or}: è un operatore logico binario. Se il primo operando è
vero lo restituisce altrimenti restituisce il secondo. Per esempio nel seguente
codice \key{a} vale 123.
\begin{lines}
local a = 123 or "mai assegnato"
\end{lines}

L'operatore \key{and} --- anche questo binario come \key{or} --- restituisce il
primo operando se esso è falso altrimenti restituisce il secondo operando.

\section{Operatore ternario}
\label{secFondOperatoreTernario}

Con \key{and} e \key{or} combinati otteniamo l'operatore ternario del C++ in
Lua: Ecco l'espressione in un esempio: se \key{a} è vera il risultato è \key{b}
altrimenti \key{c}:
\begin{lines}
local val = (a and b) or c
\end{lines}

Poiché \key{and} ha priorità maggiore rispetto a \key{or} nell'espressione
precedente possiamo omettere le parentesi per un codice ancor più idiomatico:
\begin{lines}
local val = a and b or c -- a ? b : c del C++
\end{lines}

Il massimo tra due numeri è un'espressione condizionale:
\begin{lines}
local x, y = 45.69, 564.3
local max
if x > y then
    max = x
else
    max = y
end
\end{lines}
ma con gli operatori logici è tutto più Lua:
\begin{lines}
local x, y = 45.69, 564.3
local max = (x > y) and x or y
\end{lines}

L'operatore logico \key{not} restituisce \key{true} se l'operando è \key{nil}
oppure se è \key{false} e, viceversa, restituisce \key{false} se l'operando non
è \key{nil} oppure è \key{true}. Alcuni esempi:
\begin{lines}
print(not 5)       --> 'false'
print(not not 5)   --> 'true'
print(not true)    --> 'false'
print(not false)   --> 'true'
print(not nil)     --> 'true'
\end{lines}

L'operatore di negazione può essere usato per controllare se una variabile è
valida oppure no. Per esempio possiamo controllare se in una tabella esiste il
campo \key{prezzo}:
\begin{lines}
local t = {} -- una tabella vuota
if not t.prezzo then -- t.prezzo è nil
    print("assente")
else
    print("presente")
end

t.prezzo = 12.00
if not t.prezzo then
    print("assente")
else
    print("presente")
end
\end{lines}


\section{Esercizi}

\begin{Exercise}[label=oplogic-01]
Prevedere il risultato delle seguenti espressioni Lua:
\begin{lines}
local a = 1 or 2
local b = 1 and 2
local c = "text" or 45

local d = not 12 or "ok"
local e = not nil or "ok"
\end{lines}
\end{Exercise}

\begin{Exercise}[label=oplogic-02]
Nel seguente codice, se il valore del primo condizionale è \key{true} cosa
stamperà invece il secondo condizionale?
\begin{lines}
if "stringa" then print "it's not 'nil'" end
if "stringa" == true then
    print("it's 'true'")
else
    print("it's not 'true'")
end
\end{lines}
\end{Exercise}

\begin{Exercise}[label=oplogic-03]
Come distinguere se una variabile contiene il valore \key{false} o il valore
\key{nil}?  
\end{Exercise}

\begin{Exercise}[label=oplogic-04]
Usando gli operatori logici di Lua codificare l'espressione che restituisce la
stringa \verb|"più grande di 100"|, \verb|"uguale"| o
\verb|"più piccolo di 100"| a seconda del valore numerico fornito.
\end{Exercise}

% end file



\chapter{Il tipo stringa}
\label{chFondStringhe}

In Lua le stringhe rappresentano uno dei tipi di base del linguaggio.
Per rappresentare valori stringa letterali ci sono tre diversi delimitatori:
\begin{compactitemize}
\item doppi apici: carattere \verb|"|;
\item apice semplice: carattere \key{'};
\item doppie parentesi quadre: delimitatori \key{[[} e \key{]]} con o senza un
numero corrispondente di caratteri \key{=}, per esempio \key{[==[} e \key{]==]}.
\end{compactitemize}

In una stringa delimitata da doppi apici possiamo inserire liberamente apici
semplici e viceversa, e caratteri non stampabili come il ritorno a capo
(\cs{n}) e la tabulazione (\cs{t}), tramite il carattere di escape backslash
che quindi va inserito esso stesso come doppio backslash (\cs{\textbackslash}):
\sourcecode{file = [[code/e1-005.lua]], run = true}

In Lua non esiste il tipo carattere quindi gli Autori del linguaggio hanno
pensato di utilizzare i delimitatori normalmente destinati a rappresentarne la
forma letterale, per consentire all'utente di creare stringhe contenenti i
delimitatori stessi, senza utilizzare l'escaping.

Sono comunque ammessi i simboli \key{\textbackslash}\verb|"| e
\key{\textbackslash '} che rappresentano i caratteri corrispondenti, come si
vede nella variabile \key{s5} del codice precedente.

Il terzo tipo di delimitatore per le stringhe è una coppia si parentesi quadre e
ha la proprietà di ammettere il ritorno a capo. Si possono così introdurre nel
sorgente interi brani di testo nel quale i caratteri di escaping non saranno
interpretati.
\sourcecode{file = [[code/e1-006.lua]], run = true}

Se per caso nel testo fossero presenti i delimitatori di chiusura è possibile
inserire un numero qualsiasi di caratteri \key{=} tra le parentesi quadre,
purché il numero sia lo stesso per i delimitatori di apertura e chiusura,
esempio:
\sourcecode{ file = [[code/e1-007.lua]]}


\section{Commenti multiriga}

Questi delimitatori variabili con numero qualsiasi di segni \key{=} li troviamo
anche nei commenti multiriga di Lua. Abbiamo incontrato fino a ora i commenti di
riga che si introducono nel codice con un doppio trattino \key{--}.

I commenti multiriga sono comodi quando si vuol escludere dall'esecuzione un
intero blocco di righe: iniziano con i doppi trattini seguiti da un delimitatore
di stringa multiriga e terminano con la corrispondente chiusura:
\begin{lines}
-- questo è un commento di riga

--[[
questo è un commento
multiriga
]]

--[=[
e anche questo è un commento
multiriga
]=]
\end{lines}

Normalmente in Lua i commenti multiriga vengono chiusi premettendo i doppi
trattini anche al gruppo delimitatore di chiusura. Questo è solo un trucco per
riattivare rapidamente il codice eventualmente contenuto nel commento, basta uno
spazio per far trasformare il commento multiriga in uno semplice:
\begin{lines}
--[[ righe di codice non attive
local tab = {}
--]]

-- [[ notare lo spazio dopo i doppi trattini
-- questo codice invece viene eseguito

local tab = {}
--]] -- e questo diventa una normale riga di commento
\end{lines}


\section{Concatenazione stringhe e immutabilità}

In Lua l'operatore \key{..} concatena due stringhe, in questo modo:
\begin{lines}
#[run]
local s1 = "Hello" .. " " .. "world"
local s2 = s1 .. " OK"
s2 = s2 .. "."

print(s1 .. "!")
print(s2)
\end{lines}

Il concetto importante riguardo alle stringhe è se queste siano o no immutabili.
Se non lo sono la concatenazione di stringhe non comporta la creazione di una
nuova stringa ma la modifica in memoria.

In Lua, come in molti altri linguaggi, le stringhe sono invece immutabili.
Ciò significa che una volta create, le stringhe non possono essere modificate e
nel codice precedente, l'operazione di concatenare il carattere punto in coda
alla stringa \key{s2}, genera una nuova stringa che è assegnata alla stessa
variabile.

Per poche operazioni di concatenazione ciò non è un problema, ma in alcuni casi
invece si. Consideriamo il seguente codice apparentemente innocuo:
\begin{lines}
local s = ""

for i = 1, 100 do
    s = s .. "**"
end
print(#s) -- # funziona anche per le stringhe!
--             contandone i byte
\end{lines}

Ma cosa succede in dettaglio? Perché questo codice non è efficiente? A ogni
concatenazione viene creata una nuova stringa. La prima volta vengono copiati
due byte per dare la stringa \verb|"**"|. La seconda iterazione la memoria
copiata sarà di 4 byte, e alla terza di 6 byte, eccetera.

A ogni iterazione la memoria copiata cresce di due byte con il risultato che per
produrre una stringa di 200 asterischi (200 byte) avremo copiato in totale la
memoria equivalente a 10100 byte!

In Java e negli altri linguaggi con stringhe immutabili normalmente si corre ai
ripari mettendo a disposizione una struttura dati o una funzione che risolve il
problema, per esempio un tipo \key{StringBuffer}. In Lua la soluzione è una
funzione della libreria \key{table} che, anticipando rispetto alle nostre
chiaccherate è \fn{table.concat}\luastd{table.concat}:
\begin{lines}
local t = {}
for i = 1, 100 do
    t[#t + 1] = "**"
end

print(#table.concat(t))
\end{lines}

Nel caso specifico avremo dovuto usare la funzione
\fn{string.rep}\luastd{string.rep} anche se \fn{table.concat} è più generale.


\section{Esercizi}

\begin{Exercise}[label=string-01]
Come fare in Lua per creare una stringa letterale contenente sia il
carattere apice semplice che doppio?
\end{Exercise}

\begin{Exercise}[label=string-02]
Quale sarà il risultato dell'esecuzione del seguente codice?
\begin{lines}
local s = "'"..'"'.."ok"..[["']]
print(s)
\end{lines}
\end{Exercise}

\begin{Exercise}[label=string-03]
Creare la stringa \verb|"\/"|.
\end{Exercise}

\begin{Exercise}[label=string-04]
Scrivere un programma che a partire dalla stringa \verb|"*"| crei e stampi la
stringa di 64 asterischi senza utilizzare l'operatore di concatenazione o la
funzione \fn{string.rep}.
\end{Exercise}

\begin{Exercise}[label=string-05]
Scrivere un programma che a partire dalla stringa \verb|"*"| crei e stampi
la stringa di 64 asterischi usando l'operatore di concatenazione il minimo
indispensabile di volte.
\end{Exercise}

% end of file



\chapter{Funzioni}
\label{chFondFunzioni}

Le funzioni sono il principale mezzo di astrazione e organizzazione del codice.

Coerentemente con il resto del linguaggio la sintassi dichiarativa di una
funzione comprende due parole chiave che servono per delimitare il blocco di
codice: \key{function} ed \key{end}. Una funzione può restituire dati tramite la
parola chiave \key{return}.

Come primo esempio, consideriamo una funzione per calcolare l'ennesimo numero
della \href{http://it.wikipedia.org/wiki/Successione_di_Fibonacci}{serie di
Fibonacci}. Un elemento si ottiene sommando i due precedenti elementi avendo
posto uguale a 1 i primi due:
\sourcecode{file = [[code/e2-001.lua]], select = [[uno]]}

Con le regole dell'assegnazione multipla una funzione può accettare più
argomenti. Se gli argomenti passati sono in eccesso rispetto a quelli che essa
prevede, quelli in più verranno ignorati. Viceversa, se gli argomenti sono
inferiori a quelli previsti allora a quelli mancanti verrà assegnato il valore
\key{nil}\luak{nil}.

Le stesse regole valgono anche per i dati di uscita quando la funzione è usata
come espressione in un'istruzione di assegnazione. Dopo l'istruzione
\key{return}\luak{return} si può scrivere una lista delle espressioni separate
da virgole che saranno assegnate alle corrispondenti variabili.

Per esempio, potremo modificare la funzione precedente per restituire anche la
somma dei primi \( n \) numeri di Fibonacci oltre che l'ennesimo elemento della
serie stessa e considerare un valore di default se l'argomento è \key{nil}:
\sourcecode{file = [[code/e2-001.lua]], select = [[due]], run = true,}


\section{Funzioni: valori di prima classe, I}

In Lua le funzioni sono un tipo. Possono essere assegnate a una variabile,
passate come argomento a un'altra funzione e restituite da una funzione come
valore, una proprietà che non si trova spesso nei linguaggi di scripting e che
offre più flessibilità al codice.

A ben vedere in Lua tutte le funzioni sono memorizzate in variabili e di per se
non hanno un nome. Il modo di definizione è quindi la \emph{sintassi anonima}:
\begin{lines}
add = function (a, b)
    return a + b
end
print(add(45.4564, 161.486))
\end{lines}

Questa importante proprietà si riassume dicendo che in Lua le funzioni sono
valori di \emph{prima classe}: sono assegnate a variabili e non hanno un nome
esattamente come non lo hanno gli altri tipi come numeri e stringhe.

Per comodità il linguaggio ammette anche la sintassi classica di definizione:
\begin{lines}
function variable_name(args)
    -- function body
end
\end{lines}
L'interprete Lua la tradurrà automaticamente nel codice equivalente in sintassi
anonima, una caratteristica nascosta detta \emph{zucchero sintattico}:
\begin{lines}
variable_name = function (args)
    -- function body
end
\end{lines}


\section{Funzioni: valori di prima classe, II}

Un esempio di funzione con un argomento funzione è il seguente, dove si vuole
eseguire per un certo numero di volte consecutivamente la funzione argomento:
\sourcecode{
    file = [[code/e3-001.lua]],
    select = [[uno]],
}

Molto interessante. Nell'ultima riga di codice l'argomento è una funzione
definita in sintassi anonima (che verrà eseguita 12 volte).

Essendo le funzioni valori di prima classe è possibile riassegnare variabili per
cambiarne il significato, come per esempio con la funzione predefinita
\fn{print}:
\begin{lines}
#[run]
local orig_print = print
print = function (n)
    orig_print("Argomento funzione -> "..n)
end

print(12)
\end{lines}


\section{Tabelle e funzioni}

Se una tabella può contenere chiavi con qualsiasi valore allora può contenere
anche funzioni! Le sintassi previste sono queste, esplicitate con il codice
riportato di seguito:
\begin{compactitemize}
\item assegnare la variabile di funzione a una chiave di tabella;
\item assegnare direttamente la chiave di tabella con la definizione di funzione
in sintassi anonima;
\item usare il costruttore di tabelle per assegnare funzioni in sintassi
anonima.
\end{compactitemize}
\sourcecode{
    file = [[code/e3-001.lua]],
    select = [[due]],
}

Con questo meccanismo una tabella può svolgere il ruolo di \emph{modulo}
memorizzando funzioni utili in un gruppo. In effetti la libreria standard
di Lua si presenta all'utente proprio in questo modo.


\section{Numero di argomenti variabile}

Una funzione può ricevere un numero variabile di argomenti rappresentati da
\key{...}\luas{...} tre caratteri punto consecutivi\footnote{Questa
caratteristica è chiamata \emph{variadic function}.}. Nel corpo della funzione i
tre punti rappresenteranno la lista degli argomenti, dunque possiamo o costruire
con essi una tabella oppure effettuare un'assegnazione multipla.

Un esempio è una funzione che restituisce la somma di tutti gli argomenti
numerici:
\sourcecode{
    file = [[code/e3-001.lua]],
    select = [[tre]],
}

Nell'ultima riga di codice si può notare che anche la funzione predefinita
\fn{print} accetta un numero variabile di argomenti.

Il meccanismo è ancora più flessibile perché tra i primi argomenti vi possono
essere variabili "fisse". Per esempio il primo parametro potrebbe essere un
moltiplicatore:

\sourcecode{
    file = [[code/e3-001.lua]],
    select = [[quattro]],
}

La funzione predefinita \fn{select}\luastd{select} consente di accedere alla
lista degli argomenti in dettaglio. Infatti nel codice precedente, se tra gli
argomenti compare un valore \key{nil} avremo problemi ad accedere ai valori
successivi perché --- come sappiamo già --- l'operatore di lunghezza \key{\#}
determina la lunghezza della sequenza in base ai bordi cioè alle posizioni dei
valori non nulli seguite da valori \key{nil} (maggiori dettagli alla sezione \S
\ref{secFondLenOperator}).

Il selettore prevede un primo parametro fisso seguito da una lista variabile di
valori rappresentata dai tre punti \key{...}. Se questo parametro è un intero
allora verrà considerato come indice per restituire l'argomento corrispondente.
Se invece il parametro è la stringa \key{\#} allora la funzione restituisce il
numero totale di argomenti \emph{inclusi} i \key{nil}.

Il codice seguente preso pari pari dal \href{http://www.lua.org/pil/}{PIL} ne è
un esempio:
\begin{lines}
for i = 1, select("#", ...) do
    local arg = select(i, ...)
    -- loop body
end
\end{lines}


\section{Omettere le parentesi}

In Lua esiste la sintassi di chiamata a funzione semplificata che consiste nella
possibilità di ommettere le parentesi tonde \key{()}\luas{()}. È ammessa solo se
alla funzione si passa un unico argomento di tipo stringa o di tipo tabella:
\sourcecode{
    file = [[code/e7-funzioni.lua]],
    select = [[anchequesto]],
}


\section{Closure}
\label{secClosure}

Chiudiamo il capitolo parlando di uno strano termine forse meglio noto agli
sviluppatori nei linguaggi funzionali: la \emph{closure}.

Questa proprietà di Lua amplia il concetto di funzione rendendo possibile
l'accesso dall'interno di essa a dati presenti nel contesto esterno. Ciò è
possibile perché alla chiamata di una funzione viene creato uno spazio di
memoria del contesto esterno unico e indipendente.

\begin{quote}
\emph{%
Tutte le chiamate a una stessa funzione condivideranno una stessa closure.%
}
\end{quote}

Se questo è vero una funzione potrebbe incrementare un contatore creato al suo
interno, e anche qui prendo l'esempio di codice dal PIL:
\sourcecode{
    file = [[code/e7-funzioni.lua]],
    select = [[uno]],
}

Il codice definisce una funzione \fn{new\_counter} che restituisce una
funzione che ha accesso indipendente al contesto (la variabile \key{i}).

Tecnicamente la closure \emph{è} la funzione effettiva mentre invece la
funzione non è altro che il prototipo della closure.

Le closure consentono di implementare diverse tecniche utili in modo naturale e
concettualmente semplice. Una funzione di ordinamento potrebbe per esempio
accettare come parametro una funzione di confronto per stabilire l'ordine tra
due elementi tramite l'accesso a una seconda tabella esterna contenente
informazioni utili per l'ordinamento stesso.

Nel prossimo esempio mettiamo in pratica l'idea. Il codice utilizza la funzione
\fn{table.sort} della libreria di Lua che introdurremo nel prossimo capitolo,
per applicare l'algoritmo di ordinamento alla tabella primo argomento in
base al criterio stabilito con la funzione passata come secondo
argomento in sintassi anonima.
\sourcecode{
    file = [[code/e7-funzioni.lua]],
    select = [[due]],
    run = true,
}


\section{Esercizi}

\begin{Exercise}[label=fn-01]
Scrivere una funzione che sulla base della stringa in ingresso \verb|"+"|,
\verb|"-"|, \verb|"*"|, \verb|"/"| restituisca la funzione corrispondente per
due operandi.
\end{Exercise}

\begin{Exercise}[label=fn-02]
Scrivere la funzione che accetti due argomenti numerici e ne restituisca i
risultati delle quattro operazioni aritmetiche.
\end{Exercise}

\begin{Exercise}[label=fn-03]
Scrivere una funzione che restituisca il fattoriale di un numero memorizzandone
in una tabella di closure i risultati per evitare di ripetere il calcolo in
chiamate successive con pari argomento.
\end{Exercise}

\begin{Exercise}[label=fn-04]
Scrivere una funzione con un argomento opzionale rispetto al primo parametro
numerico che ne restituisca il seno interpretandolo in radianti se l'argomento
opzionale è \key{nil} oppure \verb|"rad"|, in gradi sessadecimali se
\verb|"deg"| o in gradi centesimali se \verb|"cen"|.
\end{Exercise}

\begin{Exercise}[label=fn-05]
Scrivere una funzione \fn{ordinates} che accetti come primo argomento una
funzione \( f: ℝ \to ℝ \) (prende un numero e restituisce un numero), come
secondo e terzo argomento i due valori dell'intervallo di calcolo e come quarto
argomento il numero di punti \( n \geq 2 \) in cui suddividere equamente
l'intervallo, e restituisca i valori in sequenza che la funzione \( f \) assume
nei punti d'ascissa così definiti, in una tabella/array con indice iniziale 1.

Verificare poi che le due tabelle calcolate da \fn{ordinates} con le funzioni di
libreria \key{math.sin} e \key{math.cos}, nello stesso intervallo \( 0 \), \(
\pi/2 \) con \( n = 101 \) contengano lo stesso valore nella posizione di indice
51.
\end{Exercise}


% end of file



\chapter{La libreria standard di Lua}
\label{iiChLibstd}

In Lua sono immediatamente disponibili un folto gruppo di funzioni che ne
formano la \emph{libreria standard}. Si tratta di una collezione di funzioni
utili a svolgere compiti ricorrenti su stringhe, file, tabelle, eccetera, e si
trovano precaricate in una serie di tabelle.

L'elenco completo ma in ordine sparso con il nome della tabella/modulo
contenitore e la descrizione applicativa è il seguente:
\begin{center}
\begin{tabular}{ll}
\key{math} & matematica;\\
\key{table} & utilità sulle tabelle;\\
\key{string} & ricerca, sostituzione e pattern matching;\\
\key{io} & input/output facility, operazioni sui file;\\
\key{bit32} & operazioni bitwise (solo in Lua 5.2);\\
\key{os} & date e chiamate di sistema;\\
\key{coroutine} & creazione e controllo delle coroutine;\\
\key{utf8} & utilità per la codifica Unicode UTF-8 (da Lua 5.3);\\
\key{package} & caricamento di librerie esterne;\\
\key{debug} & accesso alle variabili e performance assessment.\\
\end{tabular}
\end{center}
La pagina web a \href{www.lua.org/manual/5.3/contents.html}{questo link}
fornisce tutte le informazioni di dettaglio sulla libreria standard di Lua~5.3.


\section{Libreria matematica}
\label{iiSecMathLibrary}

Nella libreria memorizzata nella tabella \key{math} ci sono le funzioni
trigonometriche \fn{sin}\luastd{math.sin}, \fn{cos}\luastd{math.cos},
\fn{tan}\luastd{math.tan}, \fn{asin}\luastd{math.asin} eccetera --- che lavorano
in radianti --- le funzioni esponenziali \fn{exp}\luastd{math.exp},
\fn{log}\luastd{math.log}, \fn{log10}\luastd{math.log10}, quelle di
arrotondamento \fn{ceil}\luastd{math.ceil}, \fn{floor}\luastd{math.floor}, e
quelle per la generazione pseudocasuale di numeri come
\fn{random}\luastd{math.random}, e \fn{randomseed}\luastd{math.randomseed}.
Oltre a funzioni, la tabella include campi numerici come la costante \( \pi
\)\luak{math.pi}.

Un esempio introduttivo è questo dove nella funzione \fn{one} viene definita una
funzione locale:
\sourcecode{file = [[code/e8-libstd.lua]], select = [[one]]}

Accanto a funzioni matematiche vere e proprie troviamo la funzione
\fn{math.type}\luastd{{math.type}} che esamina l'argomento e restituisce
\verb|"integer"| o \verb|"float"| a seconda se esso è un intero o un numero in
virgola mobile, oppure \key{nil} se non si tratta di un numero. Questa funzione
è utile per eseguire il controllo di validità dei parametri in ingresso.

La funzione \fn{math.tointeger}\luastd{math.tointeger} è, come la precedente,
utile per verificare se l'argomento è un intero. Essa infatti tenta la
conversione e nel caso non sia possibile il risultato sarà \key{nil}. 

L'argomento può essere anche un numero o una stringa, e in questo è simile alla
funzione \fn{tonumber}\luastd{tonumber} ma solo per i valori interi:
\begin{lines}
#[run]
print(tonumber(123), math.tointeger(123))
print(tonumber(-89), math.tointeger(-89))
print(tonumber(40.5 * 2), math.tointeger(40.5 * 2))
print(tonumber(40.5 * 3), math.tointeger(40.5 * 3))
print(tonumber("89"), math.tointeger("89"))
print(tonumber("89.6"), math.tointeger("89.6"))
\end{lines}


\section{Libreria tabelle}
\label{iiSecTableLibrary}

Per le tabelle la libreria standard fornisce funzionalità attraverso la tabella
\key{table}. Tra le funzioni più importanti troviamo
\fn{table.concat}\luastd{table.concat} che concatena tutti i valori nelle
posizioni sequenza in una stringa. Per esempio:
\begin{lines}
-- table.concat(list [, sep [, i [, j]]])
assert(table.concat{1,2,3,4,5,6} == "123456")
\end{lines}

Ci sono anche utili argomenti opzionali: un secondo argomento stringa che sarà
usato come separatore dei vari elementi, un terzo argomento intero che è
l'indice dal quale gli elementi saranno raccolti e un quarto argomento come
indice finale.

Questa funzione è molto importante nelle applicazioni perché consente di
concatenare stringhe in un modo efficiente quando il loro numero è maggiore di
una decina\footnote{L'ordine di grandezza del numero delle stringhe da
concatenare per cui conviene l'uso di \fn{table.concat} dovrebbe essere una
decina ma non ho fatto benchmark in proposito.}.

Altre funzioni molto utili sono \fn{table.insert}\luastd{table.insert} con la
sua controparte \fn{table.remove}\luastd{table.remove}, e
\fn{table.sort}\luastd{table.sort} per riordinare gli elementi della sequenza di
una tabella secondo criteri definibili di cui un esempio si trova nella
sezione~\ref{secClosure}.


\section{Libreria stringhe}

La libreria per le stringhe è memorizzata nella tabella \key{string} ed è una
delle più utili per questo l'approfondiremo più in dettaglio rispetto alle
altre. Con essa si possono formattare campi e compiere operazioni di ricerca e
sostituzione. In effetti, in Lua non è infrequente elaborare grandi porzioni di
testo.


\subsection{Funzione \fn{string.format}}
\label{secFondStringFormat}

La funzione più semplice è quella di formattazione
\fn{string.format}\luastd{string.format}. Essa restituisce una stringa prodotta
con il formato definito dal primo argomento dei dati forniti dal secondo
argomento in poi.

Il formato è esso stesso specificato come una stringa contenente dei segnaposto
creati con il simbolo percentuale e uno specificatore di tipo. Per esempio
\verb|"%d"| indica il formato relativo a un numero intero, dove \key{d} sta per
digit mentre \verb|"%f"| indica il segnaposto per un numero decimale con \key{f}
che sta per float.

I campi formato derivano da quelli della funzione classica di libreria
\fn{printf} del C. Di seguito un esempio di codice:
\sourcecode{
    file = [[code/e8-libstd.lua]],
    select = [[fmt]],
    run = true,
}

Come avete potuto notare nel codice, è anche possibile fornire un ulteriore
specifica di dettaglio tra il carattere \key{\%} e lo specificatore di tipo, per
esempio per indicare il numero delle cifre decimali di arrotondamento.

Per elaborare il testo si utilizza di solito una libreria per le espressioni
regolari. Lua mette a disposizione alcune funzioni di sostituzione e
\emph{pattern matching} meno complete dell'implementazione dello standard
POSIX per le espressioni regolari ma molto spesso più semplici da utilizzare.

Esistono due strumenti di base, il primo è il \emph{pattern} e il secondo è la
\emph{capture}.


\subsection{Pattern I}
\label{secFondPattern}

Il pattern è una stringa che può contenere campi chiamati \emph{classi} simili
a quelli per la funzione di formato visti in precedenza, che stavolta però si
riferiscono al \emph{singolo carattere} e questa differenza è essenziale.

La funzione di base che accetta un pattern è
\fn{string.match}\luastd{string.match}. Essa restituisce la prima corrispondenza
trovata in una stringa data come primo argomento che corrispondente al pattern
dato come secondo argomento.

Per esempio, possiamo ricercare un gruppo di tre cifre intere all'interno di un
testo con il pattern \verb|"%d%d%d"|:
\sourcecode{
    file = [[code/e8-libstd.lua]],
    select = [[pattern_one]],
    run = true,
}

Le classi carattere possibili sono le seguenti:
\begin{compactdescription}
  \item[\key{.}] un carattere qualsiasi;
  \item[\key{\%a}] una lettera;
  \item[\key{\%c}] un carattere di controllo;
  \item[\key{\%d}] una cifra;
  \item[\key{\%l}] una lettera minuscola;
  \item[\key{\%u}] una lettera maiuscola;
  \item[\key{\%p}] un carattere di interpunzione;
  \item[\key{\%s}] un carattere spazio;
  \item[\key{\%w}] un carattere alfanumerico;
  \item[\key{\%x}] un carattere esadecimale;
  \item[\key{\%z}] il carattere rappresentato con il codice 0.
\end{compactdescription}

Le classi ammettono quattro modificatori per esprimere le ripetizioni dei
caratteri:
\begin{compactdescription}
  \item[\key{+}] indica 1 o più ripetizioni;
  \item[\key{*}] indica 0 o più ripetizioni;
  \item[\key{-}] come \key{*} ma nella sequenza più breve;
  \item[\key{?}] indica 0 o 1 ripetizione;
\end{compactdescription}

Esempio:
\sourcecode{
    file = [[code/e8-libstd.lua]],
    select = [[pattern_two]],
    run = true,
}

\subsection{Capture}
\label{secFondCapture}

Il pattern può essere arricchito non solo per trovare corrispondenze ma anche
per restituirne parti componenti. Questa funzionalità viene chiamata
\emph{capture} e consiste semplicemente nel racchiudere tra parentesi tonde le
classi.

Per esempio per estrarre l'anno di una data nel formato \key{dd/mm/yyyy}
possiamo usare il pattern con la capture seguente \verb|"%d%d/%d%d/(%d%d%d%d)"|:
\sourcecode{
    file = [[code/e8-libstd.lua]],
    select = [[capture_one]],
    run = true,
}

Più capture ci sono nel pattern e altrettanti argomenti multipli di uscita
saranno restituiti:
\sourcecode{file = [[code/e8-libstd.lua]],
   select = [[capture_two]],
   run = true,
}


\subsection{Pattern II}

Mentre il carattere percento unito a una lettera forma una delle classi
carattere elencate alla sezione~\ref{secFondPattern}, la stessa lettera scritta
in maiuscolo rappresenta la classe carattere negazione.

La classe \verb|"%D"| rappresenta un carattere che non sia una cifra decimale,
così come \verb|"%S"| corrisponde al carattere che non ha corrispondenza con
\verb|"%s"|. Per esempio, per catturare una prima sequenza di eventuali
caratteri spazio in una stringa si può usare il pattern di ricerca
\verb|"(%s*)%S?"|. 

Due ulteriori caratteri speciali \verb|^| per l'inizio riga e \key{\$} per la
fine riga, svolgono il ruolo di ancoraggio del pattern. Possono essere usati
entrambi per esempio per eliminare da una stringa i caratteri spazio iniziali o
finali:
\begin{lines}
local pattern = "^%s*(.-)%s*$"
print("|"..string.match("ok", pattern).."|" )      --> "|ok|"
print("|"..string.match(" ok", pattern).."|" )     --> "|ok|"
print("|"..string.match(" ok ", pattern).."|" )    --> "|ok|"
print("|"..string.match(" ok ok ", pattern).."|" ) --> "|ok ok|"
\end{lines}

Passiamo ora ai \emph{char-set}. Si tratta di classi definite dall'utente
racchiudendo tra parentesi quadre caratteri o classi di carattere già definiti.
Per esempio, per estrarre da una stringa una sequenza di \( 0 \) e \( 1 \)
possiamo definire una nuova classe carattere che corrisponde o alla prima o alla
seconda cifra con \verb|[01]|.

Per esempio, per trovare corrispondenze di \emph{identificatori}, cioè sequenze
di caratteri alfanumerici che comprendono il trattino basso e che non iniziano
con una cifra, potremo usare il pattern \verb|"^[_%a][_%w]*"|:
\begin{lines}
local pattern = "^[_%a][_%w]*"
print(string.match("1var", pattern))    --> nil
print(string.match("_", pattern))       --> "_"
print(string.match("var", pattern))     --> "var"
print(string.match("_var", pattern))    --> "_var"
print(string.match("1var", pattern))    --> nil
print(string.match("var1_=5", pattern)) --> "var1_"
print(string.match("0", pattern))       --> nil
\end{lines}

Ulteriore esempio è la ricerca di corrispondenze per numeri interi con segno:
\begin{lines}
local pattern = "[+-]?%d+"
print(string.match("56", pattern))   --> "56"
print(string.match("+56", pattern))  --> "+56"
print(string.match("-56", pattern))  --> "-56"
print(string.match("-+56", pattern)) --> "+56"
print(string.match("0000", pattern)) --> "0000"
\end{lines}

In un pattern, tra parentesi quadre è ammesso l'uso del carattere trattino per
definire intervalli. Il pattern \verb|"[012345]"| che corrisponde a una classe
carattere per le prime sei cifre decimali, può essere definito anche con
\verb|"[0-5]"|.


\subsection{La funzione \fn{string.gsub}}
\label{secFondGsub}

Abbiamo appena cominciato a scoprire le funzionalità dedicate al testo
disponibili nella libreria standard di Lua. Diamo solo un altro sguardo
presentando la funzione \fn{string.gsub}\luastd{string.gsub}. Il suo nome sta
per \emph{global substitution}, ovvero la sostituzione di tutte le parti di un
testo che corrispondono a uno schema.

Per individuare le parti da sostituire è naturale pensare di utilizzare un
pattern e che sia possibile utilizzare le capture nel testo di sostituzione, per
esempio:
\sourcecode{
    file = [[code/e8-libstd.lua]],
    select = [[gsub]],
    run = true,
}

Il primo argomento è la stringa da ricercare, il secondo è il pattern, il terzo
è il testo di sostituzione per ciscuna corrispondenza, ma può anche essere una
tabella dove le chiavi corrispondenti al pattern saranno sostituite con i
rispettivi valori, oppure una funzione che dal singolo frammento di testo
estratto elabora la stringa di sostituzione.

Una funzione quindi assai flessibile. Mi viene in mente questo esercizio:
moltiplicare per 12 tutti gli interi in una stringa, ed ecco il codice:
\sourcecode{
    file = [[code/e8-libstd.lua]],
    select = [[gsubfn]],
    run = true,
}

A questo punto degli esempi avrete certamente capito che \fn{gsub} restituisce
anche il numero delle sostituzioni effettuate.

Tutte queste funzioni restituiscono una stringa costruita ex-novo e non
modificano la stringa originale di ricerca. In Lua le stringhe sono dati
immutabili.

Infine, le funzioni della libreria \key{string} possono essere chiamate su
variabili stringa usando la \emph{colon notation} che introdurremo alla
sezione~\ref{secFondColonNotation}, grazie ai metametodi:
\begin{lines}
string.match(s, pattern) -- equivalente a:
s:match(pattern)
\end{lines}


\subsection{La funzione \fn{string.gmatch}}

Con la funzione \fn{string.gmatch}\luastd{string.gmatch} è possibile iterare
tutte le corrispondenze di un pattern in una stringa in un ciclo generic for.
Gli argomenti da fornire sono la stringa e il pattern che può contenere delle
capture. Vediamo un primo esempio:
\begin{lines}
#[run]
local s = "trova le vocali"
for vowel in string.gmatch(s, "[aeiou]") do
    print(vowel)
end
\end{lines}

Se invece il pattern contiene delle capture, ad ogni iterazione altrettante
variabili di ciclo assumeranno i valori stringa delle catture:
\begin{lines}
#[run]
local s = "da A=(1,9) a B=(-9,5) passando per C=(0,3)"
local pattern = "([A-Z])=%(([+-]?%d+),([+-]?%d+)%)"
for p, x, y in string.gmatch(s, pattern) do
    print(p, tonumber(x), tonumber(y))
end
\end{lines}


\subsection{Altre funzioni utili}

Nominiamo per prima la funzione \fn{string.find}\luastd{string.find}. Essa
restituisce la posizione della corrispondenza nella stringa di ricerca, nella
forma della coppia d'indici di inizio e di fine della sequenza. La ricerca può
cominciare da una posizione qualsiasi in virtù del terzo argomento opzionale, e
se il pattern contiene una capture la corrispondenza verrà restituita come
ulteriore terzo risultato:
\begin{lines}
#[run]
-- string.find(s, pattern [, init [, plain]])
local s = "Un intero 123 e un secondo 456!"
local i, j = string.find(s, "123") -- i = 11, j = 13
print(string.find(s, "(%d+)", j+1))
\end{lines}

Conviene citare altre due funzioni utili per esempio per poter usare il simbolo
di percentuale nel codice Lua interno a sorgenti \TeX{}. Sono
\fn{string.byte}\luastd{string.byte} e \fn{string.char}\luastd{string.char}: la
prima da una lista di interi positivi calcola la stringa codificata con essi e
la seconda calcola i codici dei caratteri della stringa argomento. Per esempio:
\begin{lines}
#[run]
-- string.byte(s [, i [, j]])
-- string.char(···)
local s = "LuaTeX"
local t = {string.byte(s, 1, -1)}
assert(table.concat(t, ", ") == "76, 117, 97, 84, 101, 88")
assert(string.char(76, 117, 97, 84, 101, 88) == s)
\end{lines}


\section{Esercizi}

\begin{Exercise}[label=libstd-01]
Qual è la differenza tra i campi di formato della funzione \fn{string.format} e
le classi dei pattern? Quali le somiglianze?
\end{Exercise}

\begin{Exercise}[label=libstd-02]
Stampare una data nel formato \key{dd/mm/yyyy} a partire dagli interi contenuti
nelle variabili \key{d}, \key{m} e \key{y}.
\end{Exercise}

\begin{Exercise}[label=libstd-03]
Cosa restituisce l'esecuzione della seguente funzione?
\sourcecode{
    file = [[code/e8-libstd.lua]],
    select = [[esercizio3]],
}
Quale pattern corrisponde a un numero decimale la cui parte intera può essere
omessa?
\end{Exercise}

\begin{Exercise}[label=libstd-04]
Come estrarre dal nome di un file l'estensione?
\end{Exercise}

\begin{Exercise}[label=libstd-05]
Come eliminare da un testo eventuali caratteri spazio iniziali e/o finali?
\end{Exercise}

\begin{Exercise}[label=libstd-06]
Il pattern \verb|"(%d+)/(%d+)/(%d+)"| è adatto per estrarre giorno, mese e
anno di una data presente in una stringa nel formato \verb|"dd/mm/yyyy"|?
\end{Exercise}

\begin{Exercise}[label=libstd-07]
Creare un esempio che utilizzi \fn{string.gsub} con una funzione in sintassi
anonima a due argomenti corrispondenti ad altrettante capture nel pattern di
ricerca.
\end{Exercise}


% end of file



\chapter{Iteratori}

Gli iteratori offrono un approccio semplice e unificato per scorrere uno alla
volta gli elementi di una collezione di dati. Vi dedicheremo un capitolo proprio
perché sono molto utili per scrivere codice efficiente ed elegante.

Il linguaggio Lua prevede il ciclo d'iterazione \emph{generic for} che
introduce la nuova parola chiave \key{in} secondo questa sintassi:
\lines
for <lista variabili> in iterator_function() do
-- codice
end
\endlines
\sourcecode{from_lines()}

Le tabelle di Lua sono oggetti che possono essere impiegati per rappresentare
degli array oppure dei dizionari. In entrambe i casi Lua mette a disposizione
due iteratori predefiniti rispettivamente tramite le funzioni \fn{ipairs} e
\fn{pairs}.

Queste funzioni restituiscono un iteratore conforme alle specifiche del generic
for. Mentre impareremo più tardi a scrivere iteratori personalizzati,
dedicheremo le prossime due sezioni a questi importanti iteratori predefiniti
per le tabelle.


\section{Funzione \fn{ipairs}}

La funzione \fn{ipairs} restituisce un iteratore che a ogni ciclo genera due
valori: l'indice dell'array e il valore corrispondente. L'iterazione comincia
dalla posizione 1 e termina quando il valore è \key{nil}:
\lines
-- una tabella array
local t = {45, 56, 89, 12, 0, 2, -98}

-- iterazione tabella come array
for i, v in ipairs(t) do
    print(i, v)
end
\endlines
\sourcecode{from_lines()}

Il ciclo con \fn{ipairs} è equivalente a questo codice:
\lines
-- una tabella array
local t = {45, 56, 89, 12, 0, 2, -98}
do
    local i, v = 1, t[1]
    while v do
        print(i, v)
        i = i + 1
        v = t[i]
    end
end
\endlines
\sourcecode{from_lines()}

Se non interessa il valore dell'indice è buona norma dare al nome di variabile
corridspondente un segno di underscore \key{\_} che in Lua è un identificatore
valido. Per esempio:
\lines
-- una tabella array
local t = {45, 56, 89, 12, 0, 2, -98}
local sum = 0
for _, elem in ipairs(t) do
    sum = sum + elem
end
print(sum)
\endlines
\sourcecode{from_lines()}

Se non vogliamo incorrere in errori è molto importante ricordarsi che con
\fn{ipairs} verranno restituiti i valori in ordine di posizione da 1 in poi e
fino a che non verrà trovato un valore \key{nil}. Se desiderassimo raggiungere
tutte le coppie chiave/valore dovremo far ricorso all'iteratore \fn{pairs}
che tratteremo nella prossima sezione.


\section{Funzione \fn{pairs}}
\label{secFondPairsIterator}

Questa funzione primitiva di Lua considera la tabella come un dizionario
pertanto l'iteratore restituirà in un ordine casuale tutte le coppie chiave
valore contenute nella tabella stessa.

Una tabella con indici a salti verrà iterata parzialmente da \fn{ipairs} ma
completamente da \fn{pairs} al prezzo di perdere l'ordinamento:
\sourcecode{
    from_file [[code/e9-iter.lua]]
    :select_lines [[uno]]
    :add_output{delim_run=  true}
}

Il comportamento di questi due iteratori potrebbe lasciare perplessi ma è
coerente con le caratteristiche di Lua.


\section{Generic \key{for}}

Come può essere implementato un iteratore in Lua? Per iterare è necessario
mantenere alcune informazioni essenziali chiamate \emph{stato} dell'iteratore.
Per esempio l'indice a cui siamo arrivati nell'iterazione di una tabella/array
e la tabella stessa.

Perchè non utilizzare la closure per memorizzare lo stato dell'iteratore?

Abbiamo incontrato le closure nella sezione \ref{secClosure}. Proviamo a
scrivere il codice per iterare una tabella:
\sourcecode{
    from_file [[code/e9-iter.lua]]
    :select_lines [[due]]
}

Funziona, molto semplicemente. Non è stato necessario introdurre nessun nuovo
elemento al linguaggio. L'iteratore è solamente una questione d'implementazione
che tra l'altro in questo caso ricrea l'iteratore \fn{ipairs} visto poco fa.

Infatti, la funzione \fn{iter\_fn} mantiene nella closure lo stato
dell'iteratore --- l'indice \key{i} e la tabella \key{t} --- e restituisce uno
dopo l'altro gli elementi della tabella. Il ciclo \key{while} infinito,
s'interrompe quando il valore è \key{nil}.

Tuttavia, data l'importanza degli iteratori, Lua introduce il nuovo costrutto
chiamato \emph{generic for} che si aspetta una funzione proprio come la
\fn{iter} del codice precedente. E in effetti funziona:
\sourcecode{
    from_file [[code/e9-iter.lua]]
    :select_lines [[tre]]
}

Riassumendo, la costruzione di un iteratore in Lua si basa sulla creazione di
una funzione che restituisce uno alla volta gli elementi dell'insieme nella
sequenza desiderata. Una volta costruito l'iteratore, questo potrà essere
impiegato in un ciclo generic for.

Se per esempio si volesse iterare la collezione dei numeri pari compresi
nell'intervallo da 1 a 10, avendo a disposizione l'apposito iteratore
\fn{evenNum} che definiremo in seguito, potrei scrivere semplicemente:
\lines
for n in evenNum(1,10) do
    print(n)
end
\endlines
\sourcecode{from_lines()}


\section{L'esempio dei numeri pari}

Per definire un iteratore sui numeri pari di un intervallo dobbiamo creare una
funzione che restituisce a sua volta una funzione in grado di generare la
sequenza. L'iterazione termina quando giunti all'ultimo elemento, la funzione
restituirà il valore nullo \key{nil}, cosa che succede in automatico senza dover
esplicitare un'istruzione di \key{return}.

Potremo fare così: dato il numero iniziale per prima cosa potremo calcolare il
primo numero pari dell'intervallo usando l'operatore modulo \key{\%} e poi
creare la funzione di iterazione in sintassi anonima che prima incrementa di 2
la variabile di ciclo --- ed ecco perché dovremo inizialmente sottrarle la
stessa quantità --- e, se questo è inferiore all'estremo superiore
dell'intervallo ritornare indice e numero pari della sequenza. Ecco il codice
completo:
\sourcecode{
    from_file [[code/e9-iter.lua]]
    :select_lines [[iter_even]]
    :add_output{delim_run = true}
}

In questo esempio, oltre ad approfondire il concetto di iterazione basata sulla
closure di Lua, possiamo notare che il generic for effettua correttamente anche
l'assegnazione a più variabili di ciclo con le regole viste nella
sezione~\ref{secFondAssegnamento}.

Naturalmente, l'implementazione data di \fn{evenNum} è solo una fra quelle
possibili, e non è detto che non debbano essere considerate situazioni
particolari come quella in cui si passa all'iteratore un solo numero o
addirittura nessun argomento.


\section{Stateless iterator}
\label{secFondStatelessIter}

Una seconda versione del generatore di numeri pari può essere un buon esempio
di un iteratore in Lua che non necessita di una closure, per un risultato ancora
più efficiente, implementando uno \emph{stateless iterator}.

Per capire come ciò sia possibile dobbiamo conoscere nel dettaglio come
funziona il generic for in Lua; dopo la parola chiave \key{in} esso si aspetta
altri due parametri oltre alla funzione da chiamare a ogni ciclo: una variabile
che rappresenta lo stato invariante e la variabile di controllo.
\lines
for <vars> in <iter_fn>, <state>, <ctl_var> do
    ...
end
\endlines
\sourcecode{from_lines()}

La funzione d'iterazione verrà chiamata a ogni ciclo con due argomenti: lo stato
invariante e la variabile di controllo e ci si aspetta che restituisca uno o più
dati di ciclo. Quando questi valori saranno \key{nil} il ciclo a termine.

Nel seguente codice la funzione \fn{evenNum} provvede a restituire i tre
parametri necessari: la funzione \fn{next\_even} come iteratore, lo stato
invariante, ovvero il numero a cui la sequenza dovrà fermarsi e la variabile di
controllo che è proprio il valore nella sequenza dei numeri pari.
\sourcecode{
    from_file [[code/e9-iter.lua]]
    :select_lines [[generic_for]]
}

Con gli iteratori abbiamo terminato l'esplorazione di base del linguaggio Lua.
Questi otto capitoli sono sufficienti per scrivere programmi utili perché
trattano tutti gli argomenti essenziali, il prossimo invece, tratterà del
paradigma della programmazione a oggetti in Lua.


\section{Esercizi}

\begin{Exercise}[label=iter-01]
Dopo aver definito una tabella con chiavi e valori stampare le singole coppie
tramite l'iteratore predefinito \fn{pairs}.
\end{Exercise}

\begin{Exercise}[label=iter-02]
Scrivere una funzione che accetta un array (una tabella con indici interi in
sequenza) di stringhe e utilizzando la funzione di libreria \fn{string.upper}
restituisca un nuovo array con il testo trasformato in maiuscolo. Per esempio da
\code{\{"abc", "def", "ghi"\}} a \code{\{"ABC", "DEF", "GHI"\}}).
\end{Exercise}

\begin{Exercise}[label=iter-03]
Scrivere la funzione/closure per l'iteratore che restituisce la sequenza dei
quadrati dei numeri naturali a partire da 1 fino a un valore dato.
\end{Exercise}

\begin{Exercise}[label=iter-04]
Scrivere la versione \emph{stateless} dell'iteratore dell'esercizio precedente.
\end{Exercise}

\begin{Exercise}[label=iter-05]
Scrivere la versione \emph{stateless} dell'iteratore \fn{ipairs}. È possibile
implementarlo in modo che la funzione d'iterazione restituisca per il ciclo
generic for solamente l'elemento della tabella e non anche l'indice?
\end{Exercise}


% end of file



\chapter{Programmazione a oggetti in Lua}
\label{iiChOop}

In sintesi, il paradigma della \emph{Object Oriented Programming} \textsc{oop},
si basa sulla creazione di entità indipendenti chiamate \emph{oggetti}. Ciascun
oggetto incorpora sia dati che funzioni, che prendono il nome di \emph{metodi}.

Ogni oggetto è un'istanza che fa parte di una stessa famiglia chiamata
\emph{classe}, una sorta di prototipo che rappresenta un ``tipo di dati''. Le
classi possono essere ricavate da altre classi con il meccanismo
dell'\emph{ereditarietà} per specializzarne il comportamento.

Per instanziare un oggetto di una classe si utilizza un metodo speciale chiamato
\emph{costruttore}, che valida gli eventuali dati in ingresso e instanzia in
memoria l'oggetto.

In questo capitolo ritroveremo tutti questi concetti del paradigma della
programmazione a oggetti dal punto di vista di Lua. Con essi la struttura del
problema non è più pensata in termini di funzioni, ma attraverso la
rappresentazione dei suoi elementi concettuali in classi e le loro relazioni di
ereditarietà.

Negli ultimi anni, la programmazione a oggetti è stata ripensata tant'è che nei
linguaggi di nuova generazione come Go e Rust non è inclusa nel modo classico.
Ciò non toglie che essa possa rendere più intuitiva la programmazione in Lua, in
special modo per chi sviluppa applicazioni per \LuaTeX.


\section{Il minimalismo di Lua}

Il linguaggio Lua non è progettato con gli stessi obiettivi di Java o del C++, i
due linguaggi più noti per la programmazione a oggetti, non possiede un
controllo preventivo del tipo, non prevede il concetto sintattico di classe, non
offre alcun meccanismo per dichiarare come privati campi e metodi, e lascia al
programmatore più di un modo per implementare la \textsc{oop}.

Tuttavia Lua basandosi sulle tabelle offre il pieno supporto ai principi del
paradigma a oggetti senza perdere le caratteristiche minimali del linguaggio.


\section{Una classe Rettangolo}
\label{secRectOop}

Costruiremo una classe per rappresentare un rettangolo. Si tratta di un ente
geometrico definito da due soli parametri \emph{larghezza} e \emph{altezza}, e
dotato di proprietà come l'area e il perimetro, che implementeremo come metodi.

Un primo tentativo potrebbe essere questo:
\sourcecode{
    file = [[code/e10-oop.lua]],
    select = [[primo]],
}

Ci accorgiamo presto che questo tentativo è difettoso in quanto non rispetta
l'indipendenza degli oggetti rispetto al loro nome. Infatti il prossimo test
fallisce:
\sourcecode{
    file = [[code/e10-oop.lua]],
    select = [[due]],
}

Il problema sta nel fatto che nel metodo \fn{area} compare il particolare
riferimento alla tabella \key{Rettangolo}. La soluzione non può che essere
l'introduzione del riferimento dell'oggetto come parametro esplicito nel metodo
stesso, ed è la stessa utilizzata anche dagli altri linguaggi di programmazione
che supportano gli oggetti. Vedremo tra poco come, allo stesso modo dei
linguaggi \textsc{oop}, anche in Lua si possa nascondere il riferimento
con la colon notation.

Secondo quest'idea dovremo riscrivere il metodo \fn{area} in questo modo (in
Lua il riferimento implicito all'oggetto deve chiamarsi \key{self} pertanto
abituiamoci fin dall'inizio a questa convenzione):
\sourcecode{
    file = [[code/e10-oop.lua]],
    select = [[tre]],
}

Fino a ora abbiamo costruito l'oggetto sfruttando le caratteristiche della
tabella e la particolarità che consente di assegnare una funzione a una
variabile. Questo punto è importante: le chiavi nella tabella dell'oggetto
possono contenere sia funzioni/metodi sia valori/campi perciò sovrascrivere una
chiave nell'intenzione di introdurre un nuovo campo/metodo porta a errori.


\section{Colon notation}

Da questo momento entra in scena l'operatore \key{:}\luas{:} --- che chiameremo
\emph{colon notation}. L'operatore \key{:} fa in modo che le seguenti due
espressioni siano perfettamente equivalenti anche se le rende differenti dal
punto di vista concettuale agli occhi del programmatore:
\sourcecode{
    file = [[code/e10-oop.lua]],
    select = [[colon_notation]],
}

Questo operatore è il primo nuovo elemento che Lua introduce per facilitare la
programmazione orientata agli oggetti. Se si accede a un metodo memorizzato in
una tabella con l'operatore due punti \key{:} anziché con l'operatore
\key{.}\luas{.} allora l'interprete aggiungerà implicitamente un primo parametro
chiamandolo \key{self}\luak{self} con il riferimento alla tabella stessa,
insomma puro zucchero sintattico.


\section{Metatabelle}

Il linguaggio Lua si fonda sull'essenzialità tanto che supporta la
programmazione a oggetti utilizzando quasi esclusivamente le proprie risorse di
base senza introdurre nessun nuovo costrutto. In particolare in Lua si utilizza
la tabella, l'unica struttura dati predefinita nel linguaggio, assieme a
particolari funzionalità dette \emph{metatabelle} e \emph{metametodi}.

Il salto definitivo nella programmazione \textsc{oop} consiste nel poter
costruire una \emph{classe} senza ogni volta assemblare campi e metodi,
introducendo un qualcosa che faccia da stampo per gli oggetti.

In Lua l'unico meccanismo disponibile per compiere questo ultimo importante
passo consiste nelle \emph{metatabelle}. Esse sono normali tabelle contenenti
funzioni dai nomi prestabiliti che vengono chiamati quando si verificano
particolari eventi come l'esecuzione di un'espressione di somma tra due tabelle
con l'operatore \key{+}. Ogni tabella può essere associata a una metatabella e
questo consente di creare degli insiemi di oggetti che condividono una stessa
aritmetica.

Metatabelle e metametodi quindi, possono rendere il codice intuitivo e compatto,
sono quindi una funzionalità indipendente dalla programmazione a oggetti.

I nomi delle funzioni di una metatabella vengono detti \emph{metametodi} e
iniziano tutti con un doppio trattino basso. Per esempio nel caso della somma
sarà richiesta la funzione \fn{\_\_add} nella metatabella associata al primo
addendo o se non esiste a quella del secondo addendo.

Per assegnare una metatabella si utilizza la funzione \fn{setmetatable}. Essa ha
due argomenti tabella, la prima è l'oggetto e la seconda è la tabella con i
metametodi.


\section{Il metametodo \fn{\_\_tostring}}

Il metametodo più semplice di tutti è \fn{\_\_tostring}. Esso viene invocato se
una tabella è data come argomento alla funzione \fn{print}\luastd{print} per
ottenere il valore stringa da stampare. Se non esiste una metatabella associata
con questo metametodo verrà stampato l'indirizzo di memoria della tabella:
\sourcecode{
    file = [[code/e10-oop.lua]],
    select = [[tostring]],
}


\section{Il metametodo \key{\_\_index}}
\label{secFondMetaIndex}

Il metametodo che interessa la programmazione a oggetti in Lua è
\key{\_\_index}\luak{\_\_index}. Esso interviene quando viene chiamato un campo
di una tabella che non esiste e che normalmente restituirebbe il valore
\key{nil}. Un esempio di codice chiarirà il meccanismo:
\sourcecode{
    file = [[code/e10-oop.lua]],
    select = [[index]],
}

Tornando all'oggetto \key{Rettangolo} riscriviamo il codice creando adesso una
tabella che assume il ruolo concettuale di una vera e propria classe:
\sourcecode{
    file = [[code/e10-oop.lua]],
    select = [[classe_rect]],
}

Queste poche righe di codice racchiudono il meccanismo della creazione di una
nuova classe in Lua: abbiamo infatti assegnato a una nuova tabella \key{r} la
metatabella con funzione di classe \key{Rettangolo}. Quando viene richiesta la
stampa del campo \key{b}, poiché tale campo non esiste nella tabella vuota
\key{r} verrà ricercato il metametodo \key{\_\_index} nella metatabella
associata che è appunto la tabella \key{Rettangolo}.

A questo punto il metametodo restituisce semplicemente la tabella
\key{Rettangolo} stessa e questo fa sì che tutti i campi e i metodi siano
ereditati da essa per essere accessibili da \key{r}. Il campo \key{b} e il
metodo \fn{area} del nuovo oggetto \key{r} sono in realtà quelli definiti nella
tabella \key{Rettangolo}.

Se volessimo creare invece un rettangolo assegnando direttamente la dimensione
dei lati dovremo semplicemente crearli in \key{r} con i nomi previsti dalla
classe: \key{b} e \key{h}. Il metodo \fn{area} sarà ancora caricato dalla
tabella \key{Rettangolo} ma i campi numerici con le nuove misure dei lati
saranno quelli interni dell'oggetto \key{r} e non quelli della metatabella
poiché semplicemente esistono in \key{r}.

Questa costruzione funziona ma può essere migliorata con l'introduzione del
costrutture come vedremo meglio in seguito. L'oggetto \key{Rettangolo} apparirà
sempre più concettualmente simile a una classe.


\section{Il costruttore}
\label{secFondCostruttore}

Proponendoci ancora la rappresentazione del concetto di rettangolo, completiamo
il quadro introducendo il costruttore della classe. Il lavoro che dovrà
svolgere questa speciale funzione sarà quello di inizializzare i campi
argomento in una delle tante modalità possibili e una volta effettuato il
controllo di validità degli argomenti.

Il codice completo della classe \key{Rettangolo} è il seguente:
\sourcecode{
    file = [[code/e10-oop.lua]],
    select = [[newrect]],
}

Il costruttore \fn{new} accetta una tabella come argomento, altrimenti ne crea
una vuota, controlla gli eventuali parametri geometrici, assegna la metatabella
e restituisce l'oggetto. Alla funzione arriva il riferimento implicito a
\key{Rettangolo} grazie alla colon notation, per cui \key{self} è un riferimento
alla stessa tabella del di quello della variabile \key{Rettangolo}.

Quando viene passata una tabella con uno o due campi sulle misure dei lati al
costruttore, l'oggetto disporrà delle misure come valori interni effettivi,
cioè dei parametri indipendenti che costituiscono il suo stato interno. Lo
sviluppatore può fare anche una diversa scelta, quella per esempio di
considerare la tabella argomento del costruttore come semplice struttura di
chiavi/valori da sottoporre al controllo di validità e poi includere in una
nuova tabella con modalità e nomi che riguardano solo l'implementazione interna
della classe.

Essendo il costruttore una normale funzione Lua, i suoi argomenti possono essere
più di uno e di diverso tipo, mentre la classe può averne anche più di uno con
nomi differenti.


\section{Questa volta un cerchio}

Per capire ancor meglio i dettagli e renderci conto di come funziona il
meccanismo automatico delle metatabelle, costruiamo una classe \key{Cerchio} che
annoveri fra i suoi metodi una funzione che modifichi il valore del raggio
aggiungendovi una misura:
\sourcecode{
    file = [[code/e10-oop.lua]],
    select = [[cerchio]],
}

Nella sezione del codice utente viene dapprima creato un cerchio senza fornire
alcun valore per il raggio. Ciò significa che quando stampiamo il valore del
raggio con la successiva istruzione otteniamo \( 0 \) che è il valore di default
del raggio dell'oggetto \key{Cerchio}, per effetto della chiamata a
\key{\_\_index}\luak{\_\_index} della metatabella.

Fino a questo momento la tabella dell'oggetto \key{o} non contiene alcun campo
\key{radius}. Cosa succede allora quando è chiamato il comando
\key{o:addToRadius(12.342)}?

Il metodo \fn{addToRadius} contiene una sola espressione. Come da regola viene
prima valutata la parte a destra ovvero \key{self.radius + v}. Il primo termine
assume il valore previsto in \key{Cerchio} --- quindi zero --- grazie al
metametodo, e successivamente il risultato della somma uguale all'argomento
\key{v} è memorizzato nel campo \key{o.radius} che viene creato effettivamente
solo in quel momento.


\section{Ereditarietà}

Il concetto di ereditarietà nella programmazione a oggetti consiste nella
possibilità di derivare una classe da un'altra per specializzarne il
comportamento.

L'operazione di derivazione incorpora automaticamente nella sottoclasse tutti i
campi e i metodi della classe base. Dopodiché si implementano o si modificano i
metodi della classe derivata creando una gerarchia di oggetti.

In Lua l'operazione di derivazione consiste molto semplicemente nel creare un
oggetto con il costruttore della classe base e modificarne o aggiungerne metodi
o campi.

Vediamo un esempio semplice dove si rappresenta il concetto generale di una
persona che svolge attività sportiva e da questo, il concetto di una persona
che svolge uno specifico sport:
\sourcecode{
    file = [[code/e10-oop.lua]],
    select = [[sportivo]],
}

Continua tutto a funzionare per via della ricerca effettuata dal metametodo
\key{\_\_index} che funziona a ritroso fino alla classe base.


\section{Esercizi}

\begin{Exercise}[label=oop-01]
Aggiungere alla classe \key{Rettangolo} riportata nel testo il metametodo
\fn{\_\_tostring} che stampi in console il rettangolo dalle dimensioni
corrispondenti ad altezza e larghezza usando i caratteri \key{+} per gli
spigoli e i caratteri \key{-} e \key{|} per disegnare i lati. Utilizzare le
funzioni di libreria \fn{string.rep} e \fn{string.format}.
\end{Exercise}

\begin{Exercise}[label=oop-02]
Creare una classe corrispondente al concetto di numero complesso e implementare
le quattro operazioni aritmetiche tramite metametodi (riferimento matematico
\href{http://it.wikipedia.org/wiki/Numero_complesso#Operazioni_con_i_numeri_complessi}{qui}).
Aggiungere anche il metodo \fn{\_\_tostring} per stampare il numero complesso e
poter controllare i risultati di operazioni di test.
\end{Exercise}

\begin{Exercise}[label=oop-03]
Ideare una classe base e una classe derivata dandone un'implementazione.
\end{Exercise}

% end of file


\part{Cose di \LuaTeX}
\label{partLuaTeX}



\chapter{Introduzione ai nodi}
\label{iiiChNodi}

I \emph{nodi} sono gli oggetti tipografici elementari che \TeX{} costruisce e
assembla in liste di più livelli come risultato della lettura e dell'esecuzione
del codice del file sorgente.

I nodi sono quindi i mattoni che intessuti uno sull'altro costruiscono la
rappresentazione di ciò che poi sarà effettivamente scritto sulla pagina e che
dipende solamente dal formato di uscita, per esempio PDF o DVI.

I nodi, le liste di nodi, e le liste di liste di nodi, sono il risultato
intermedio di un processo per fasi inizialmente alimentato dal testo sorgente,
byte per byte. Il compositore costruisce queste strutture complesse come dato di
uscita in memoria in attesa di inviarle alla fase successiva come dato in
ingresso. Può Lua interagire in modo diretto e nativo con le operazioni della
fase dei nodi?


\section{Lua e i nodi}

Il linguaggio Lua è stato progettato tra le altre cose per poter offrire
all'utente di un programma un linguaggio di scripting con l'accesso ai dati
interni. Lua può quindi \emph{estendere} un programma lasciando che l'utente
possa configurarne il comportamento o usufruire dei servizi di elaborazione in
modo del tutto generale.

\LuaTeX{} così come gli altri motori che includono l'interprete Lua, grazie a
questa fondamentale caratteristica di questo linguaggio, implementa la libreria
di basso livello \code{node} che rende possibile la costruzione delle strutture
tipografiche dei nodi in Lua e l'interazione con il relativo processo interno.

È così che con \code{node} è possibile costruire ogni tipo di struttura
tipografica pronta per essere posizionata sulla pagina, le stesse che il
compositore costruirebbe leggendo un sorgente \code{.tex}, saltando le prime
fasi di lettura dei token, espansione ed esecuzione, per interagire direttamente
con la fase finale detta visuale.


\section{Ciclo di vita dei nodi}
\label{iiiSecCicloVitaNodi}

Tecnicamente gli oggetti nodo in Lua sono tipi detti \emph{userdata}, con
implementazione in C e fuori dalla gestione automatica della memoria. Vanno
quindi usati con cautela facendo in modo che il ciclo di vita sia sempre e solo
questo:
\begin{compactenumerate}
\item \emph{creazione}, 
\item \emph{impostazione parametri},
\item \emph{invio a unica destinazione oppure distruzione}.
\end{compactenumerate}

Riutilizzare un oggetto nodo già finalizzato comporta il rischio elevato, se non
la certezza, del crash del programma. La destinazione deve essere unica
intendendo che un nodo può essere inviato una sola volta alla fase successiva,
per invio diretto con la funzione \fn{node.write}\luastd{node.write} oppure per
invio indiretto con la memorizzazione in un registro di tipo scatola.

Altri problemi derivano invece dal non eliminare esplicitamente nodi creati ma
poi non destinati. Per esempio, se si deve costruire un oggetto formato da nodi
tutti uguali si può creare il nodo prototipo per copiarlo più volte con la
funzione \fn{node.copy}\luastd{node.copy}, ma alla fine ci si deve ricordare di
destinarlo oppure eliminarlo dalla memoria con le apposite funzioni, per esempio
con \fn{node.free}\luastd{node.free}.


\section{Introduzione ai nodi principali}

Come si crea un nodo? Ciascun nodo si crea sempre con lo stesso semplice schema:
si chiama la funzione \fn{new} di \code{node}\luastd{node.new} con
l'identificatore del tipo e si configurano i parametri dell'oggetto restituito.

L'identificatore del tipo è una stringa oppure un numero e i numerosi parametri
di un oggetto nodo funzionano come campi di una tabella Lua perciò è lecito
usare la dot notation.

Per meglio comprendere i concetti essenziali e introdurre l'argomento con
gradualità, lascieremo i dettagli al manuale di \LuaTeX{} \cite{prg:luatex} 


\subsection{Nodi glifo}

Il nodo glifo è un signolo elemento di un font, un nodo tra i più usati che
compone parole, frasi, pagine. I suoi parametri obbligatori sono il numero del
font, che deve essere già caricato, e il codice del glifo nel font.

Questo codice Lua mette in pratica i due passaggi:
\begin{lines}
local n = node.new("glyph") -- creazione nodo
n.font = font.current()     -- impostazione parametri
n.char = 65
\end{lines}


\subsection{Nodi dimensione}

I nodi dimensione sono chiamati \key{glue} e rappresentano le lunghezze
elastiche grazie alla quali \TeX{} riesce ad essere così flessibile nel comporre
la pagina. Un esempio in cui interessa solamente la dimensione rigida è:
\begin{lines}
local g = node.new("glue") -- creazione nodo
g.width = 12 * 65536       -- impostazione parametri
\end{lines}

Le dimensioni per i nodi, in particolare per i parametri dei nodi, devono essere
in \emph{scaled point} in sigla \key{sp}. Quest'unità è la frazione intera di un
punto tipografico con fattore \( 2^{16} = 65536 \) e deriva
dall'implementazione in virgola fissa scelta da Donald~Knuth stesso, per le
dimensioni in \TeX.


\subsection{Nodi rule}

Con il nodo \emph{rule} si disegna un rettangolo pieno. Il suo uso abituale è
quello di disegnare una linea verticale o orizzontale come i filetti di una
tabella. Il codice Lua di creazione e impostazione segue il solito schema:
\begin{lines}
local r = node.new("rule") -- creazione nodo
r.width = tex.sp "0.24pt"  -- impostazione parametri
r.height = tex.sp "14pt"
r.depth = tex.sp "4pt"
\end{lines}

Il nodo rule è il diretto equivalente della macro \cs{rule}. L'altezza totale è
la somma del tratto discendente al di sotto dela linea base chiamato profondità,
e del tratto ascendente al di sopra di essa.


\subsection{Assemblaggio nodi in liste}

I nodi si assemblano in liste che possono essere strutturate su più livelli. Un
capoverso per esempio, è un unico nodo di tipo scatola verticale che contiene
una lista di scatole orizzontali ciascuna corrispondente alle singole righe
alternate con nodi di spaziatura verticale. Ogni scatola orizzontale a sua volta
è una lista di nodi glifo, dimensione elastica e altri tipi di nodi come quelli
di kerning.

Tutti i nodi hanno due specifici campi \key{.prev} e \key{.next}, che contengono
i riferimenti al nodo che precede e a quello che segue. Le liste possono essere
quindi percorse sia in avanti, dalla testa alla coda, che all'indietro.

Invece che gestire direttamente questi campi per il collegamento delle liste,
consiglio di utilizzare le funzioni come
\fn{node.insert\_after}\luastd{node.insert\_after}. I suoi argomenti sono tre
riferimenti: il nodo di testa della lista, il nodo corrente dopo il quale si
vuole inserire il nuovo nodo, e infine il nodo da inserire.

Il riferiementi restituiti sono due: il nodo di testa della lista e il nuovo
nodo inserito. Questo schema consente semplicemente di aggiungere in coda alla
lista nuovi elementi oppure creare una lista da un primo elemento.

Come esempio, realizziamo il disegno di barre affiancate che crescono sempre di
più in altezza. Usiamo il formato plain in \LuaTeX{} e la copia di nodi
prototipo:
\sourcecode{file = [[tol/n1.tex]]}

Alla funzione di uscita \fn{node.write} si passa solamente un nodo, che sia
semplice o che sia il nodo di testa di una lista, tutti i nodi collegati saranno
inviati all'output di pagina.

Il risultato del codice è: \directlua{
% geometria barre
local mm = tex.sp '1mm'
local function hbar(x)
    return (0.215*x*x + 1)*mm
end
% nodo rule prototipo
local bar = node.new('rule')
bar.width = 2*mm
% nodo glue prototipo
local g = node.new('glue')
g.width = 1*mm
% creazione lista
local head, last
for x = 0, 8 do
    local b = node.copy(bar)
    b.height = hbar(x)
    head, last = node.insert_after(head, last, b)
    head, last = node.insert_after(head, last, node.copy(g))
end
bar.height = hbar(9)
head, last = node.insert_after(head, last, bar)
% free memory
node.free(g)
% send list to output
node.write(head)
}


% Scatole orizz e verticale
\subsection{Nodi scatola}

Le liste di nodi possono essere impacchettate in scatole orizzontali o
verticali. Le prime posizionano la lista disponendo uno dopo l'altro i nodi su
una linea orizzontale che chiamiamo linea base, le seconde dispongono i nodi in
ordine uno sull'altro in una pila verticale.

Gli oggetti scatola sono i normalissimi oggetti di \TeX{} che si costruiscono
con le macro \cs{hbox} e \cs{vbox}, o le corrispondenti macro \LaTeX. In Lua
essi si ottengono con le funzioni \fn{node.hpack}\luastd{node.hpack} e
\fn{node.vpack}\luastd{node.vpack}, rispettivamente per scatole orizzontali e
verticali, al solito fornendo come argomento il nodo di testa di una lista.

Vediamo all'opera queste operazioni con l'esercitazione pratica della prossima
sezione. Teniamo in conto che il contenuto di un nodo scatola, è la lista il cui
nodo di testa è memorizzato nel campo \key{.head}.


\section{Esercitazione: testo in caselle}

In alcuni moduli si incontrano dati che devono essere scritti lettera per
lettera all'interno di una sequenza di caselle tutte di uguale dimensione. Per
riprodurre in \TeX{} una simile struttura dovremo alternare i singoli caratteri
con uno spazio che dipende dalle larghezze dei caratteri adiacenti.

La distanza \( G_\mathrm{gap} \) tra i centri dei glifi è constante ed è uguale
alla somma di metà della larghezza del carattere di sinistra, della metà della
larghezza del carattere di destra più lo spazio tra di loro è costante. Se
\(w_\mathrm{i} \) e \(w_\mathrm{i+1} \) sono le larghezze di due glifi
consecutivi allora la distanza \( g \) interna dai loro bordi, quella che
dovremo effettivamente inserire, soddisfa l'equazione:
\[
    G_\mathrm{gap} = \frac{w_\mathrm{i}}{2} + g + \frac{w_\mathrm{i+1}}{2}
\]

Assumendo già definite le funzioni Lua \fn{gliph} e \fn{glue}, la costruzione
della lista dei nodi glifo alternati a nodi distanza, è riportata nell'estratto
di condice seguente:
\begin{lines}
#[indexfile=tol/n2.tex]
-- filename: tol/n2.tex
local s = "TESTO"
local gap = tex.sp("20pt")
local w1 = gap
local head, last
for n in string.utfvalues(s) do
    local wn, gn = glyph(n)
    local g = glue(gap - (w1 + wn)/2)
    head, last = node.insert_after(head, last, g)
    head, last = node.insert_after(head, last, gn)
    w1 = wn
end
local hbox = node.hpack(head)
tex.box["sptxt"] = hbox
\end{lines}

La lista che fa capo al nodo iniziale contenuto nella variabile \key{head},
viene inserita in una scatola orizzontale con la funzione
\fn{node.hpack}\luastd{node.hpack}. La scatola è poi memorizzata nel registro
\TeX{} dello stesso tipo con il nome \key{'sptxt'} già creato precedentemente
nel sorgente.

A questo punto si può inserire nel sorgente ovunque serva il contenuto del
registro con le primitive \TeX{} \cs{box} o \cs{copy}, o con il comando \LaTeX{}
\cs{usebox}.


\subsection{Disegno delle caselle}

Cominciamo con il modificare il codice precedente per inserire le barre
verticali tra le caselle. Se ci si pensa un momento, questo significa che
dobbiamo spezzare in due il nodo che distanzia i glifi per costruire la sequenza
del tipo \key{'rule'}, \key{'glue'}, \key{'gliph'}, \key{'glue'}, \key{'rule'},
eccetera. In conseguenza non è più necessaria la variabile \key{w1} che
conteneva la larghezza del glifo precedente.

Se \( t \) è lo spessore della barra verticale, la regola della distanza
diventa:
\[
    G_\mathrm{gap} = \frac{t}{2} + g + w + g + \frac{t}{2}
\]

I termini che usiamo sono convenzionali perché la nostra lista di nodi non sa
nulla di caselle o distanze interne, è solo una catena di nodi fatta di tante
barre verticali uguali e di coppie di distanze dal glifo.

In caso di nodi uguali, è molto importante ricordarci delle regole base. Se
costruissimo un solo oggetto rule per inserirlo in lista ogni volta che arriva
il turno di una barra verticale, commetteremo un grave errore di violazione del
principio: avremmo infatti inviato un nodo a più destinazioni, cosa che produce
una lista impossibile da scorrere per circolarità dei riferimenti.

È molto importante scrivere codice Lua che si attenga alla regola del ciclo di
vita dei nodi. Non farlo da luogo a liste non corrette e violazioni di memoria

Per duplicare i nodi può essere utile la funzione
\fn{node.copy}\luastd{node.copy}. Questa tecnica è utilizzata per creare la
lista con le sole barre verticali:
\begin{lines}
#[indexfile=tol/n3.tex]
-- filename: tol/n3.tex
local s = "TESTO"
local gap = tex.sp("20pt")
local t = tex.sp("0.2pt")
local r = rule(t, tex.sp("12.5pt"), tex.sp("4pt"))
local head, last
for n in string.utfvalues(s) do
    head, last = node.insert_after(head, last, node.copy(r))
    local w, gn = glyph(n)
    local g = glue((gap - t - w)/2)
    head, last = node.insert_after(head, last, g)
    head, last = node.insert_after(head, last, gn)
    head, last = node.insert_after(head, last, node.copy(g))
end
head, last = node.insert_after(head, last, r)
local hbox = node.hpack(head)
tex.box["sptxt"] = hbox
\end{lines}

L'iteratore \fn{string.utfvalues}\luastd{string.utfvalues} è un'estensione della
libreria standard di Lua presente in \LuaTeX{} molto utile per rendere il nostro
codice compatibile con la codifica Unicode UTF-8, e gestire correttamente i
caratteri accentati eccetera.

La prossima operazione è assemblare nuovi nodi rule per le linee di chiusura
delle caselle, quella sopra \key{'r1'} e quella sotto \key{'r2'}.

La nostra lista diventa una scatola verticale con all'interno la lista delle tre
scatole orizzontali \key{hbox(r1)}, testo spaziato \key{hbox(spacedtext)}, e
\key{hbox(r2)}. Il compositore infatti sovrappone uno sull'altro gli elementi di
una scatola verticale così come affianca uno all'altro quelli di una scatola
orizzontale.

Aggiungiamo poi un ulteriore livello, inserendo la scatola verticale a sua volta
in una scatola orizzontale per fare in modo di regolare l'allineamento alla
linea base. Lo schema finale della struttura è questo:
\begin{Verbatim}[numbers=none]
result_hbox (
    head = vbox (
        hbox(r1) -- hbox(spacedtext) -- hbox(r2)
    )
)
\end{Verbatim}

Per calcolare la lunghezza dei nodi rule \key{r1} e \key{r2}, dovremo
conteggiare il numero dei glifi. Introduciamo allora la funzione \fn{spacedtext}
che restituisce il numero dei glifi e il nodo della scatola orizzontale che
contiene il testo spaziato.

La traduzione in codice Lua dello schema è diretto:
\begin{lines}
#[indexfile=tol/n4.tex]
-- filename: tol/n4.tex
local gap = tex.sp("20pt") -- distanza assiale glifi
local t = tex.sp("0.2pt")  -- spessore barra verticale
local h = tex.sp("12.5pt") -- altezza barra
local d = tex.sp("4pt")    -- profondità barra
-- nodo hbox(spacedtext)
local n, hbox = spacedtext(s, gap, t, h, d)
-- nodo hbox(r1)
local r1 = rule(n*gap, t)
r1 = node.hpack(r1)
-- nodo hbox(r2)
local r2 = node.copy(r1)
-- lista hh = hbox(r1) -- hbox(specedtext) -- hbox(r2)
local hh, ll = node.insert_after(nil, nil, r1)
hh, ll = node.insert_after(hh, ll, hbox)
hh, ll = node.insert_after(hh, ll, r2)
-- nodo vbox
local vbox = node.vpack(hh)
local result_hbox = node.hpack(vbox)
result_hbox.head.shift = d
tex.box["sptxt"] = result_hbox
\end{lines}

% visualizzare il risultato del codice

\directlua{
spacedtext = {}
local s = spacedtext
function s.glyph(code)
    local n = node.new('glyph')
    n.font = font.current()
    n.char = code
    return n.width, n
end
function s.glue(x)
    local g = node.new 'glue'
    g.width = x
    return g
end
function s.rule(w, h, d)
    local r = node.new('rule')
    r.width = w
    r.height = h
    r.depth = d
    return r
end
function s:spacedtext(s, gap, t, h, d)
    local r = self.rule(t, h, d)
    local head, last
    local c = 0
    for n in string.utfvalues(s) do
        head, last = node.insert_after(head, last, node.copy(r))
        local w, gn = self.glyph(n)
        local g = self.glue((gap - t - w)/2)
        head, last = node.insert_after(head, last, g)
        head, last = node.insert_after(head, last, gn)
        head, last = node.insert_after(head, last, node.copy(g))
        c = c + 1
    end
    head, last = node.insert_after(head, last, r)
    local hbox = node.hpack(head)
    return c, hbox
end
}

\newbox\sptxt
\newcommand{\spacedtext}[2]{\directlua{
    local lib = spacedtext
    local gap = assert(tex.sp([[#1]]))
    local s = [==[#2]==]
    local t, h = tex.sp('0.2pt'), tex.sp('12.5pt')
    local d = tex.sp('4pt')
    local n, hbox = lib:spacedtext(s, gap, t, h, d)
    local r1 = lib.rule(n*gap, t)
    r1 = node.hpack(r1)
    local r2 = node.copy(r1)
    local hh, ll = node.insert_after(nil, nil, r1)
    hh, ll = node.insert_after(hh, ll, hbox)
    hh, ll = node.insert_after(hh, ll, r2)
    local vbox = node.vpack(hh)
    local res_hbox = node.hpack(vbox)
    res_hbox.head.shift = d
    tex.box['sptxt'] = res_hbox
}\box\sptxt}

% fine codice

Eccone una dimostrazione: \spacedtext{18pt}{TESTO}.

% end of file


\part{Applicazioni Lua in \LuaTeX}
\label{partApp}


% esercitazioni semplici con richiami ai fondamenti

\chapter{La calcolatrice}
\label{ivChCalcolatrice}

Siamo giunti al primo capitolo di taglio applicativo. Creeremo una
\emph{calcolatrice} sottoforma di una macro \cs{expr} che accetti un'espressione
numerica e ne stampi il risultato. È davvero utile non dover più calcolare a
parte il risultato e ricopiarlo nel sorgente del documento \LaTeX{}.

Tentiamo qualcosa di molto semplice con Lua, assegnare l'espressione a una
variabile per poi stamparla nel documento:

\CLRmarginpar{Lua in \TeX\\
\gotosec{} \ref{iSecLuaInLuaLaTeX}}[true]% true option means +16pt of vspace
%
\CLRmarginpar{Variabili locali\\
\gotosec{} \ref{iSecGlobaleLocale}, \gotosec{} \ref{iiSecLocaleGlobale}}
%
\CLRmarginpar{\fn{tex.print}\\
\gotosec{} \ref{iSecPassaggioDati}}
%
\begin{lines}
#[tex]
#[indexfile=app-tut/E0-001-expr.tex]
% !TeX program = LuaLaTeX
% filename: app-tut/E0-001-expr.tex
\documentclass{article}
\newcommand\expr[1]{\directlua{
    local result = #1
    tex.print(tostring(result))
}}
\begin{document}
Finalmente una calcolatrice:
\( 1.24 (7.45 + 11.21) = \expr{1.24*(7.45 + 11.21)}\)
\end{document}
\end{lines}

\newcommand\expr[1]{\directlua{
    local result = #1
    tex.print(tostring(result))
}}
compilando con \LuaLaTeX{} il risultato è:
\begin{tcolorbox}
Finalmente una calcolatrice: \( 1.24 (7.45 + 11.21) = \expr{1.24*(7.45 + 11.21)}\)
\end{tcolorbox}

Un buon inizio. Nel sorgente all'interno della macro \cs{directlua} il primo
argomento è stato sostituito con l'espressione che viene poi valutata da Lua.
Nessun pacchetto aggiuntivo caricato, qualsiasi espressione numerica è lecita, e
questo solo e soltanto usando Lua incluso in \LuaTeX.

Funziona anche con le stringhe, a patto di delimitarne il valore, e con le
espressioni booleane. Proviamo:
\begin{tcolorbox}
\begin{lines}
#[tex]
#[tcolorbox]
\( 56.9 > 78.42 \) è \texttt{\expr{ 56.9 > 78.42 }}
\end{lines}
\tcblower
\( 56.9 > 78.42 \) è \texttt{\expr{ 56.9 > 78.42 }}
\end{tcolorbox}


\section{Espressioni booleane personalizzate}

E se si volessero sostituire le rappresentazioni testuali dei valori vero e
falso? Ecco la modifica:

\CLRmarginpar{%
Ciclo \key{if}\\
\gotosec{} \ref{iiSecCicloIf}}[true]
%
\CLRmarginpar{%
Tipo \key{boolean}\\
\gotosec{} \ref{iiSecManciataTipi}}
%
\CLRmarginpar{%
Operatore ternario\\
\gotosec{} \ref{iiSecOperatoreTernario}}
%
\begin{lines}
#[tex]
\newcommand\expr[1]{\directlua{
    local result = #1
    if type(result) == "boolean" then
        result = result and "vero" or "falso"
    end
    tex.print(tostring(result))
}}
\end{lines}

\renewcommand\expr[1]{\directlua{
    local result = #1
    if type(result) == 'boolean' then
        result = result and 'vero' or 'falso'
    end
    tex.print(tostring(result))
}}

Un semplice test ci conforterà sulla correttezza del codice e si funziona:
\begin{tcolorbox}[sidebyside]
\begin{lines}
#[tcolorbox]
\expr{100 == 100 and 7 > 3}
\expr{-10 < -100}
\end{lines}
\tcblower
\expr{100 == 100 and 7 > 3}\\
\expr{-10 < -100}
\end{tcolorbox}


\section{Arrotondamento numerico}

Vorrei poter regolare l'arrotondamento del risultato numerico della calcolatrice
ricorrendo a un argomento opzionale separato dall'espressione da una virgola:

\begin{center}
\CLRmarginpar{%
\fn{string.format}\\
\gotosec{} \ref{iiSecStringFormat}}[true]%
\begin{lines}
#[tex]
\newcommand\expr[1]{\directlua{
    local result, dec = #1
    if type(result) == "boolean" then
        result = result and "vero" or "falso"
    elseif type(result) == "number" and dec then
        local perc = string.char(37)
        local fmt1 = perc..perc.."0."..perc.."df"
        local fmt2 = string.format(fmt1, dec)
        result = string.format(fmt2, result)
    end
    tex.print(tostring(result))
}}
\end{lines}
\end{center}

\renewcommand\expr[1]{\directlua{
    local result, dec = #1
    if type(result) == 'boolean' then
        result = result and 'vero' or 'falso'
    elseif type(result) == 'number' and dec then
        local perc = string.char(37)
        local fmt1 = perc..perc..'0.'..perc..'df'
        local fmt2 = string.format(fmt1, dec)
        result = string.format(fmt2, result)
    end
    tex.print(tostring(result))
}}

Stavolta il codice perde un po' di chiarezza perché non è possibile usare
direttamente il carattere percento \code{\%} che verrebbe interpretato come
inizio di un commento nel costruire le stringhe di formato. Ovviamente questo
non succederebbe se il codice fosse in un file separato o se fosse racchiuso in
un ambiente \pack{luacode} dell'omonimo pacchetto \LaTeX{} \cite{pkg:luacode}.

Mettiamo alla prova la modifica alla macro \cs{expr}:
\begin{tcolorbox}
\begin{lines}
#[tex]
#[tcolorbox]
\(\sqrt{2} + \sqrt{3} \approx \expr{ 2^0.5 + 3^0.5, 2}\)
\end{lines}
\tcblower
\(\sqrt{2} + \sqrt{3} \approx \expr{ 2^0.5 + 3^0.5, 2}\)
\end{tcolorbox}


\section{Funzioni matematiche}

Potremo trovare una sintassi un po' più chiara per indicare il numero di cifre a
cui arrontondare il risultato, tuttavia c'è di mezzo un problema più urgente:
poter usare funzioni matematiche come seno e coseno. Se scrivessimo
\verb=\expr{sin(1)^2 + cos(1)^2}= non otterremo il valore unitario ma un errore.
Dovremo infatti usare la scomoda notazione
\code{math.}\meta{funzione}/\meta{costante} come in
\verb=\expr{math.cos(math.pi)}= invece della più naturale \verb=\expr{cos(pi)}=.
Ma ci vuole poco a riassegnare le funzioni a nomi locali per far si che
l'identità trigonometrica precedente sia un'espressione valida:
\begin{lines}
#[tex]
\newcommand\expr[1]{\directlua{
    local cos = math.cos
    local sin = math.sin
    local result, dec = #1
    if type(result) == "boolean" then
        result = result and "vero" or "falso"
    elseif type(result) == "number" and dec then
        local perc = string.char(37)
        local fmt1 = perc..perc.."0."..perc.."df"
        local fmt2 = string.format(fmt1, dec)
        result = string.format(fmt2, result)
    end
    tex.print(tostring(result))
}}
\end{lines}

\renewcommand\expr[1]{\directlua{
    local cos = math.cos
    local sin = math.sin
    local result, dec = #1
    if type(result) == 'boolean' then
        result = result and 'vero' or 'falso'
    elseif type(result) == 'number' and dec then
        local perc = string.char(37)
        local fmt1 = perc..perc..'0.'..perc..'df'
        local fmt2 = string.format(fmt1, dec)
        result = string.format(fmt2, result)
    end
    tex.print(tostring(result))
}}

Una prova della calcolatrice potenziata con le funzioni trigonometriche ci dirà
se tutto funziona ancora bene:
\begin{tcolorbox}
\begin{lines}
#[tex]
#[tcolorbox]
\(\sin^2(1/2) + \cos^2(1/2) =\expr{sin(0.5)^2 + cos(0.5)^2, 8}\).
\end{lines}
\tcblower
\(\sin^2(1/2) + \cos^2(1/2) = \expr{sin(0.5)^2 + cos(0.5)^2, 8}\).
\end{tcolorbox}
e per un'espressione booleana:
\begin{tcolorbox}
\begin{lines}
#[tex]
#[tcolorbox]
A \(1/3\) l'identità è \emph{\expr{sin(1/3)^2 + cos(1/3)^2 == 1}}
\end{lines}
\tcblower
A \( 1/3 \) l'identità è \emph{\expr{sin(1/3)^2 + cos(1/3)^2 == 1}}
\end{tcolorbox}

Finora ogni nuova funzionalità aggiunta alla calcolatrice non ha presentato
difficoltà. Possiamo inserire o meno il risultato in ambiente matematico,
arrotondarlo al numero di decimali desiderato e usare funzioni matematiche
dell'efficiente libreria in virgola mobile di Lua, un linguaggio che si sta
dimostrando semplice da usare e molto efficace.


\section{Costanti numeriche}

Continuamo con un nuovo passo: aggiungere costanti numeriche definite
dall'utente, una sorta di \emph{funzione di memoria} della calcolatrice. Per
inserire variabili letterali in un'espressione abbiamo bisogno che il loro
valore numerico sia inizializzato ma non possiamo ricorrere alla stessa tecnica
con cui abbiamo risolto l'inserimento delle funzioni trigonometriche.

Non è possibile infatti codificare variabili locali senza conoscerne il nome,
perché è un dato fornito dall'utente. Servirebbe una sorta di metaprogrammazione
come con le macro dei linguaggi compilati. Leggendo più a fondo la
documentazione di Lua, si scopre che è possibile intervenire sull'ambiente delle
variabili globali \code{\_ENV}\footnote{Per ulteriori informazioni
sull'\emph{environment} di Lua rimando al PIL dove si trova un intero capitolo
ad esso dedicato.} di un \emph{chunk}, e anzi, a ben vedere il problema di
rendere visibili simboli di costanti è lo stesso che quello di rendere
disponibili nell'espressione le funzioni matematiche con nomi abbreviati.

Facciamo un tentativo ripartendo con il codice da zero:
\CLRmarginpar{%
Tipo \code{table}\\
\gotosec{} \ref{iiChTabella}}[true]
%
\CLRmarginpar{%
Iteratore \fn{pairs}\\
\gotosec{} \ref{iiSecPairsIterator}}
%
\begin{lines}
#[tex]
\directlua{
calclib = {}
for name, object in pairs(math) do
    calclib[name] = object
end
}
\newcommand\expr[1]{\directlua{
do
    local _ENV = calclib
    ans = #1
end
tex.print(tostring(calclib.ans))
}}
\end{lines}

Stiamo sfruttando una tecnica piuttosto interessante: all'interno di un blocco
viene riassegnata localmente la variabile \code{\_ENV} a \code{calclib}, una
tabella in cui vi abbiamo riversato tutte le funzioni e le costanti matematiche
della libreria \code{math} di Lua. Alla riga 10, tutti quei nomi saranno
visibili come variabili globali proprio quando l'espressione deve essere
valutata.

Non solo, come effetto collaterale, il risultato dell'ultimo calcolo sarà
disponibile nella successiva espressione memorizzato nella variabile \code{ans}
come succede con altri tool matematici! Proviamolo:
%
\directlua{
calclib = {}
for name, object in pairs(math) do
    calclib[name] = object
end
}
\renewcommand\expr[1]{\directlua{
do
    local _ENV = calclib
    ans = #1
end
tex.print(tostring(calclib.ans))
}}
%
\begin{tcolorbox}[sidebyside]
\begin{lines}
#[tex]
#[tcolorbox]
\expr{pi/4}\\
\expr{cos(ans)}\\
\expr{acos(ans)}
\end{lines}
\tcblower
\expr{pi/4}\\
\expr{cos(ans)}\\
\expr{acos(ans)}
\end{tcolorbox}

Molto bene. Non ci resta che aggiungere con la stessa tecnica la memorizzazione
delle costanti attivando l'argomento opzionale della macro \cs{expr}. Tra le
parentesi quadre potremo fornire all'espressione una lista di costanti nel
formato chiave/valore con separatore il carattere virgola.

Memorizzeremo le costanti indicate dall'utente solamente se il loro nome non è
già stato utilizzato o se non è un un nome di funzione. Inoltre, specificando
una stringa solo come chiave tra le opzioni, potremo implementare la
memorizzazione del risultato dell'espressione stessa, così che sia poi
riutilizzabile:
\begin{lines}
#[tex]
\directlua{
calclib = {}
for name, object in pairs(math) do
    calclib[name] = object
end
}
\newcommand\expr[2][]{\directlua{
do
    local error, pairs, assert, type = error, pairs, assert, type
    local _ENV = calclib
    local opt = {#1}
    local mem = opt[1]; opt[1] = nil
    for c, val in pairs(opt) do
        if _ENV[c] then
            error("Duplicated key '"..c.."' in constant name")
        else
            _ENV[c] = val
        end
    end
    ans = #2
    if mem then
        assert(type(mem) == "string")
        if _ENV[mem] then
            error("Duplicated key '"..mem.."' in memory index")
        else
            _ENV[mem] = ans
        end
    end
end
tex.print(tostring(calclib.ans))
}}
\end{lines}

\renewcommand\expr[2][]{\directlua{
do
    local error, pairs, assert, type = error, pairs, assert, type
    local _ENV = calclib
    local opt = {#1}
    local mem = opt[1]; opt[1] = nil
    for c, val in pairs(opt) do
        if _ENV[c] then
            error([[Duplicated key ']]..c..[[' in constant name]])
        else
            _ENV[c] = val
        end
    end
    ans = #2
    if mem then
        assert(type(mem) == [[string]])
        if _ENV[mem] then
            error([[Duplicated key ']]..mem..[[' in memory index]])
        else
            _ENV[mem] = ans
        end
    end
end
tex.print(tostring(calclib.ans))
}}

Eccone un esempio:
\begin{tcolorbox}
\begin{lines}
#[tex]
#[tcolorbox]
\( b = \expr[b=10, h=20]{b} \), % oppure \expr["b", h=20]{10}
\( h = \expr{h} \),
\( M = \expr[m = 1000]{m}\), % oppure \expr["m"]{1000}
\( \sigma = M/W_\mathrm{x} = \expr[w=(b*h^2)/6]{m/w}\).
\end{lines}
\tcblower
\( b = \expr[b = 10]{b} \),
\( h = \expr[h = 20]{h} \),
\( M = \expr[m = 1000]{m}\),
\( \sigma = M/W_\mathrm{x} = \expr[w=(b*h^2)/6]{m/w}\).
\end{tcolorbox}

Poiché anche i valori assegnati alle costanti sono valutati da Lua \emph{dopo}
la modifica dell'environment, anche per le costanti nelle opzioni della macro
\cs{expr} è possibile assegnare espressioni usando tutte le funzioni matematiche
e tutte le costanti precedentemente definite. Da questo punto in poi, possiamo
presentare il valore di \( W_\mathrm{x} \) scrivendo nel sorgente
\verb=\expr{w}= che da \expr{w}.

\section{Conclusioni}

Le principali funzionalità della calcolatrice sono state implementate in Lua,
certo non tutte. Per esempio potremo far eseguire calcoli coinvolgendo anche i
registri contatori o dimensionali di \TeX{}, per esempio per determinare delle
coordinate in un disegno, oppure considerare che constanti dai nomi speciali
come \code{M1}, \code{M2} eccetera si comportino come registri di memoria della
calcolatrice e quindi che possano essere sovrascritti e possano funzionare da
accumulatori.

% end of file


\input{chapter/IV-02-tab}

\input{chapter/IV-03-registro}

\input{chapter/IV-04-tartaglia}

% \input{chapter/IV-05-cerchio}

% \input{chapter/IV-06-dati}

\appendix

% 


Come tutti i linguaggi di scripting non occorre compilare il codice Lua. Basta
scrivere il programma in un file di testo --- preferibilmente assegnandogli
l'estensione \key{.lua} --- e lanciare il comando d'esecuzione in un terminale
\tcmd{\$ lua nomefile}.

Per scrivere il codice vi consiglio di utilizzare l'editor
\href{http://www.scintilla.org/SciTE.html}{SciTE} perché è già predisposto per
eseguire programmi Lua ed è programmabile in Lua. Tra l'altro l'output del
programma viene visualizzato nella \emph{output tab}. Il tasto funzione F5
esegue il programma visualizzato nella scheda attiva in quel momento e il tasto
F8 attiva/disattiva la finestra di output, che è in pratica un terminale
incorporato nell'editor.

Prepariamoci a scrivere il primo programma in Lua, ovviamente il fatidico
\href{http://en.wikipedia.org/wiki/%22Hello,_world!%22_program"}{Hello world!},
salvando il seguente codice in un file chiamato \key{primo.lua} ed eseguiamolo
direttamente in SciTE premendo il tasto funzione F5 o al terminale con il
comando \tcmd{\$ lua primo.lua}:
\lines
print("Hello world!")
\endlines
\sourcecode{from_lines():add_output()}

In ambiente Linux o Mac OS~X, come per tutti gli altri linguaggi di scripting,
possiamo inserire come prima riga del file a cui sono stati assegnati i permessi
di esecuzione, la
\href{http://en.wikipedia.org/wiki/Shebang_%28Unix%29}{Sha-Bang}. In questo modo
basterà semplicemente scriverne il nome nel terminale, e in questo caso è
conveniente omettere l'estensione. Ecco il programma con la Sha-Bang:
\lines
#!/usr/bin/env lua
print("Hello world!")
\endlines
\sourcecode{from_lines()}

Lua è in grado di passare al programma gli argomenti che l'utente digita nel
terminale separandoli da spazi, tramite la tabella array chiamata \key{arg} che
conterrà all'indice 0 il nome dello script, e ai successivi indici i vari
argomenti utente come tipo stringa. Questa proprietà assieme alla tecnica della
sha-bang, dà all'utente la possibilità di scrivere programmi in Lua come se
fossero comandi da terminale, almeno per gli ambienti di classe *nix.

\chapter{Preparazione di un interprete Lua}

Poiché l'interprete Lua non interferisce con \LuaTeX{} e occupa una quantità di
memoria davvero piccola, è consigliabile installarlo per poter usare il modo
interattivo REPL e per sperimentare più comodamente.

Per seguire queste chiaccherate andrà benissimo una versione di Lua uguale o
superiore alla 5.1. Tuttavia è preferibile installare la versione che
corrisponde a quella inclusa in \LuaTeX{}, che viene riportata anche nel suo
manuale.

L'interprete Lua è scritto in Ansi~C e per questo è disponibile praticamente
per tutte le piattaforme, eventualmente compilandone i sorgenti.

Per verificare l'installazione è possibile dare il seguente comando da
terminale per stampare la versione dell'interprete:
\begin{Verbatim}
$ lua -v
Lua 5.3.3  Copyright (C) 1994-2016 Lua.org, PUC-Rio
\end{Verbatim}



\subsection{Linux}

Per i sistemi Linux fate ricorso al package manager della vostra distribuzione,
per esempio per le distribuzioni Debian e derivate come Ubuntu, è sufficiente
dare il comando:
\begin{Verbatim}
$ sudo apt-get install lua
\end{Verbatim}



\subsection{Windows}

Per Windows fate riferimento alla pagina
\href{http://luabinaries.sourceforge.net/index.html}{LuaBinaries}. In caso di
problemi potete scaricare un file autoinstallante dal progetto
\href{https://github.com/rjpcomputing/luaforwindows}{Lua for Windows} fermo
però alla versione 5.1.5 di Lua ma completo di alcune librerie utili.



\subsection{Mac OSX}

Per Mac OSX andate alla pagine \href{http://rudix.org/packages/lua.html}{Rudix}
e individuate il package adatto per la versione del vostro Mac.



\section{Come eseguire il codice}

Operativamente, ci sono molti modi per eseguire codice Lua. La modalità usuale
è installarne l'interprete e questo capitolo riporta le procedure per i tre
principali sistemi operativi.

Se si dispone di una recente distribuzione del sistema \TeX{}, per esempio
TeX~Live, uno \emph{script} in Lua può anche essere eseguito da \LuaTeX{}, per
esempio con il comando da terminale:
\begin{Verbatim}
$ luatex --luaonly nomefile
\end{Verbatim}

Nella parte~\ref{partLuaTeX} vedremo come sia possibile inserire il codice Lua
all'interno di un sorgente \texttt{.tex}. Per esempio, poiché sto componendo
questa guida in \LuaLaTeX{} --- che altro non è che \LuaTeX{} con il formato
\LaTeX{} precaricato --- scrivendo il codice:
\begin{Verbatim}
\directlua{tex.print(lua.version)}
\end{Verbatim}
posso dirvi che la versione di Lua inclusa è \directlua{tex.print(lua.version)}.


\subsection{Modalità interattiva}

Aprite un terminale, o console in Windows, e digitate semplicemente il comando
\texttt{lua}. Entreremo nella modalità interattiva dove potremo
digitare istruzioni una alla volta. Per uscire tornando al prompt, digitate la
funzione \texttt{os.exit()} seguita da un bel tasto invio.

In modalità interattiva possiamo comodamente controllare quale tipo ha un
valore, utilizzando la funzione \texttt{type()}. Essa restituisce una stringa
che corrisponde al nome del tipo di un'espressione, stringa che possiamo a sua
volta stampare con la funzione \texttt{print()} in console. Digitiamo
\footnote{Il doppio trattino inserisce un commento di riga.}:
\lines
a = 123
print(type(123))    --> stampa "number"
print(type(a))      --> stampa "number"
print(type("123"))  --> stampa "string"
\endlines
\sourcecode{from_lines()}


% end of file



\backmatter



\chapter{Note finali}

I miei ringraziamenti vanno a Claudio Beccari per aver scritto la classe
\class{guidatematica.cls} con cui è stata composta la guida e per avermi dato
risposte come sempre precise e complete alle mie domande.

Ringrazio Gianluca Pignalberi che mi ha proposto un uso avanzato del pacchetto
\pack{siunitx} nella composizione della tabella proposta nel secondo tutorial.

Ci sono sempre buone occasioni per imparare, speriamo che i maestri non manchino
mai.

% end of file


% bibliografia
\printbibliography[heading=bibintoc]

% indexes
\printindex

\end{document}
