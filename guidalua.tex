% !TeX program = LuaLaTeX--shell-escape
%
% "Guida Tematica alla programmazione Lua in LuaTeX"
% Copyright (C) 2020-2021 Roberto Giacomelli,
% Gruppo utilizzatori Italiani di TeX e LaTeX
% All rights reserved
%
% https://www.guitex.org
% 
% License information see LICENSE text file

\documentclass[b5paper,11pt,openany]{guidatematica}
\ProvidesFile{guidalua.tex}[2021/04/08 v0.4.5 Guida al linguaggio Lua per LuaTeX]
\GetFileInfo{guidalua.tex}
\setmonofont[Scale=0.82]{Fira Mono}
\usepackage{sourcecode}
\usepackage{hologo}
\usepackage[backend=biber,style=alphabetic]{biblatex}
\addbibresource{guidalua.bib}

\fvset{
    fontsize=\small,
    numbers=left,
    xleftmargin=7.5mm,
    numbersep=5.75mm
}

\lstset{% general setup
    numbers=left,
    numberstyle=\tiny,
    xleftmargin=7.5mm,
    numbersep=5.75mm
}

\usepackage[
    pdftitle={Guida al linguaggio Lua per LuaTeX},
    pdfauthor={Roberto Giacomelli},
    pdfsubject={Guida tematica del GuIT},
    pdfkeywords={Lua, LuaTeX, guida, programming}
]{hyperref}

\usepackage{exercise}
\renewcounter{Exercise}[chapter]
\renewcommand{\ExerciseHeader}{\noindent\textsc{esercizio \ExerciseHeaderNB}}

\usepackage{tcolorbox}
\tcbuselibrary{skins}
\tcbset{
    sharp corners=all,
    colback=verdeguit!12!white,
    colframe=verdeguit!70!white,
    bicolor,
    colbacklower=white,
    boxrule=0.4pt,
    leftrule=3.2pt,
    fontupper=\small,
    fontlower=\small
}

\setmarginnotes{6pt}{\dimexpr\foremargin-12pt\relax}{5pt}
\tcbuselibrary{documentation}
\tcbset{
    doc marginnote={
        colframe=verdeguit,
        colback=verdeguit!5!white,
        if odd page or oneside={flushleft upper}{flushright upper}
    }
}

\usepackage{siunitx}
\sisetup{output-decimal-marker=\virgoladecimale}

\frenchspacing

\newcommand\tcmd[1]{\normalfont\texttt{#1}}
\newcommand\key[1]{\normalfont\texttt{#1}}
\newcommand\code[1]{\normalfont\texttt{#1}}
\newcommand\fn[1]{\normalfont\texttt{#1()}}
\newcommand\gotosec{\textcolor{verdeguit}{\tiny\S}}

\newlength{\margindown}
\setlength{\margindown}{2.5mm}

\makeatletter
\newcommand\guidalualicensebox{
\noindent
\begin{tikzpicture}
\begin{scope}[scale=0.18]
\guidalualogocmd
\end{scope}
\node[anchor=base] at (40mm,1.25mm) {\@@title};
\node[anchor=base] at (40mm,-3.5mm) {Copyright \textcopyright{} \the\year, \@@author};
\end{tikzpicture}\\[\baselineskip]
Questa documentazione è soggetta alla licenza LPPL
\href{http://www.latex-project.org/lppl.txt}{\LaTeX{} Project Public
License}, versione 1.3 o successive ed è curata dall'autore.}
\makeatother

\directlua{
    local d = [[\filedate]]
    token.set_macro('guidaluadate', d:gsub('/', '-'))
}

\licenza{%
\guidalualicensebox

\medskip Guide's \pack{biblatex} entry:\\
\begingroup
\ttfamily\footnotesize\noindent
\begin{tabular}{@{}l@{}l@{}l@{}}
\toprule
\multicolumn{3}{l}{\hspace{-6pt}@manual\{guit:guidalua,}\\
\hspace{8pt}
 & title        & =\{Guida al linguaggio Lua per Lua\cs{TeX}\},\\
 & author       & =\{Giacomelli, Roberto\},\\
 & date         & =\{\guidaluadate\},\\
 & version      & =\{\fileversion\},\\
 & pagetotal    & =\{\thelastpage\},\\
 & langid       & =\{italian\},\\
 & url          & =\{https://www.guitex.org/home/images/doc/GuideGuIT/guidalua.pdf\},\\
 & urldate      & =\{2021-04-01\},\\
 & organization & =\{GuIT, Gruppo Utilizzatori Italiani di \cs{TeX}\},\\
 & series       & =\{Guide Tematiche del GuIT\},\\
\}\\
\bottomrule
\end{tabular}
\endgroup
}

\newcommand\guidalualogocmd{
\draw[fill,verdeguit] (0,0) circle (2.2);
\draw[fill,verdeguit] (2.31, 2.31) circle (0.72);
\draw[fill, white] (0.8,0.8) circle (0.72);
}

\begin{document}
\newtcolorbox[blend into=listings]{tcbfloat}[2][]{float=htb,title={#2},#1}
\hypersetup{
    colorlinks=true,
    linkcolor=verdeguit,
    citecolor=verdeguit,
    urlcolor=verdeguit!80!red
}
\author{Roberto Giacomelli}
\date{\filedate{} --- \fileversion}
\title{Guida al linguaggio Lua per \LuaTeX}
\subtitle{
\begin{tikzpicture}[scale=0.82]
\node at (-2.31-0.72, 0) {};
\guidalualogocmd
\end{tikzpicture}
\vspace*{8.5ex}
}
\maketitle

\input{section/0-01-intro}

\newpage

\tableofcontents

\mainmatter*

\part{Tutorial}
\label{partTutorial}

\input{section/I-01-calc}

% secondo tutorial

\section{Tabella dei pesi}

Dopo la calcolatrice si presenta un altro problema compositivo: una tabella che
riporta per vari diametri, area e peso della barra d'acciaio di lunghezza
unitaria. I diametri variano da 6 a 32 millimetri con passo 2.

L'idea è definire una sorta di iteratore a due componenti. Per esempio, se
volessimo una tabella con due colonne, la prima con gli interi da 1 a 10 e la
seconda con i rispettivi quadrati, dovremo definire solo la funzione di calcolo
e il numero delle righe totali, perché un secondo componente si occuperà di
applicarla le volte necessarie.

Il primo componente variabile della funzione generatrice, può essere qualsiasi
purché sia definita per due argomenti: il primo il contatore di riga e il
secondo l'array di riga. Nel caso d'esempio si dovrà memorizzare nell'array il
contatore in posizione 1 e il quadrato in posizione 2:
\begin{Verbatim}
local function row_func(counter, row)
    row[1] = counter
    row[2] = counter^2
end
\end{Verbatim}

L'idea iniziale è quindi realizzata se attribuiamo alla funzione che calcola la
generica riga il concreto ruolo di \emph{regola di definizione} dell'intera
tabella. A ben vedere potremo fare a meno del secondo parametro \code{row} se
restituissimo direttamente un nuovo array di riga, tuttavia in questo modo il
codice risulta più efficiente.

\directlua{
Row = {}
Row.__index = Row
function Row:new(fn_next, start, stop, step)
    if not stop then
        start, stop = 1, start
    end
    local o = {
        fn_next = fn_next,
        start = start,
        stop = stop,
        step = step or 1
    }
    setmetatable(o, self)
    return o
end

function Row:next()
    local var = self.var
    if not var then
        var = self.start
    else
        var = var + self.step
    end
    if var <= self.stop then
        self.var = var
        local fn = self.fn_next
        fn(var, self)
        return true
    end
end
}

Il secondo componente costante, la classe di libreria \key{Row}, ha il compito
di applicare la regola ad ogni riga della tabella, qualsiasi essa sia. Prima di
passare alla sua implementazione esaminiamo la costruzione della tabella
d'esempio in \LuaLaTeX{}. Il metodo \fn{new} accetta proprio una funzione come
primo argomento e il valore totale di righe come secondo argomento.

In Lua le funzioni sono valori come tutti gli altri. L'esempio minimo
compilabile è il seguente:
\begin{tcolorbox}[sidebyside,righthand width=21mm]
\begin{Verbatim}[numbers=none,xleftmargin=0pt]
% !TeX program = LuaLaTeX
% filename: app-start/E0-003-tab.tex
\documentclass{article}
% preambolo non riportato
\begin{document}
\begin{tabular}{rr}\directlua{
local row = Row:new(
    function (c, r) r[1]=c; r[2]=c^2 end, 10
)
local par = string.char(92)..string.char(92)
while row:next() do
    tex.print(row[1].."&"..row[2]..par)
end
}\end{tabular}
\end{document}
\end{Verbatim}
\tcblower
\begin{tabular}{rr}
\directlua{
local row = Row:new(function (c, r) r[1]=c; r[2] = c^2 end, 10)
local par = string.char(92)..string.char(92)
while row:next() do
    tex.print(row[1]..[[&]]..row[2]..par)
end
}
\end{tabular}
\end{tcolorbox}

All'interno dell'ambiente \amb{tabular} c'è solo codice Lua: costruito l'oggetto
\key{row} un ciclo \key{while} esegue l'iterazione con il metodo \fn{next}. Come
si può verificare dall'implementazione con il paradigma a oggetti che segue, è
\fn{next} a chiamare a ogni passo la funzione di generazione di riga:
%
\tcbdocmarginnote[enlarge top initially by=\margindown]{%
Metametodo \key{\_\_index}\\
\gotosec{} \ref{secFondMetaIndex}}
%
\tcbdocmarginnote[enlarge top initially by=\dimexpr\margindown+12mm\relax]{%
Costruttore\\
\gotosec{} \ref{secFondCostruttore}}
%
\begin{Verbatim}
Row = {}
Row.__index = Row
-- costruttore
function Row:new(fn_next, start, stop, step)
    if not stop then
        start, stop = 1, start
    end
    local o = {
        fn_next = fn_next,
        start = start,
        stop = stop,
        step = step or 1
    }
    setmetatable(o, self)
    return o
end
-- iteratore
function Row:next()
    local var = self.var
    if not var then
        var = self.start
    else
        var = var + self.step
    end
    if var <= self.stop then
        self.var = var
        local fn = self.fn_next
        fn(var, self)
        return true
    end
end
\end{Verbatim}

Siamo in fase iniziale e questo giustifica l'assenza di controlli sui dati di
input e la conseguente gestione degli errori, parte essenziale di ogni
programma. Ci si potrà preoccupare in seguito in una fase di consolidamento di
errori e altri dettagli.

Per esempio, stiamo trascurando le conseguenze possibili del fatto che il codice
Lua è all'interno del sorgente \TeX{} e che per questo ci potrebbero essere
problemi dovuti all'espansione dell'argomento della primitiva \cs{directlua}. Un
esempio? Riceveremmo un errore con il blocco della compilazione se usassimo i
delimitatori delle stringhe doppi apici se nel preambolo si caricasse il
pacchetto \pack{polyglossia} con l'opzione \opz{babelshorthands} per la lingua
italiana, che rende attivo proprio il doppio apice. Per fortuna ci sono altri
modi più sicuri di delimitare i valori letterali delle stringhe in Lua in questi
casi, proprie del linguaggio.

Ogni tipo di dati potrà essere rappresentato in forma tabellare, anche dati non
calcolati come nomi di file con relativa dimensione in byte, dati come i
seguenti, istanziati dal costruttore di tabelle di Lua che elaborando una
tabella che a sua volta ne contiene altre quattro:
\begin{Verbatim}
local data = {
    {"files.txt",  4710},
    {"lib.lua"  ,   330},
    {"parse.lua", 50995},
    {"path.txt" ,  2150},
}
\end{Verbatim}

Come dovremo definire il componente variabile, ovvero la funzione
\fn{row\_func}, per costruire la tabella a due colonne nome file, e dimensione?
\begin{tcolorbox}[sidebyside,righthand width=32mm]
\begin{Verbatim}[numbers=none,xleftmargin=0pt]
% !TeX program = LuaLaTeX
% filename: app-start/E0-004-tab.tex
\documentclass{article}
% preambolo non riportato
\begin{document}
\begin{tabular}{lr}\directlua{
local data = {
    {"files.txt"    ,  4710},
    {"lib.lua"      ,   330},
    {"parse.lua"    , 50995},
    {"path.txt"     ,  2150},
}
local function row_func(counter, row)
    row[1] = data[counter][1]
    row[2] = data[counter][2]
end
local row = Row:new(row_func, 4)
local p = string.char(92); p = p..p
while row:next() do
    tex.print(row[1].."&"..row[2]..p)
end
}\end{tabular}
\end{document}
\end{Verbatim}
\tcblower
\begin{tabular}{lr}
\directlua{
local data = {
  {'files.txt'    ,  4710},
  {'lib.lua'      ,   330},
  {'parse.lua'    , 50995},
  {'path.txt'     ,  2150},
}
local function row_func(c, row)
    row[1] = data[c][1]
    row[2] = data[c][2]
end
local row = Row:new(row_func, 4)
local p = string.char(92); p = p..p
while row:next() do
tex.print(row[1]..'&'..row[2]..p)
end
}
\end{tabular}
\end{tcolorbox}

In questo esempio incontriamo una ridondanza perché dobbiamo specificare il
numero di righe nel costruttore \fn{new} quando questo dato è invece
derivabile dai dati. 

In Lua è normale chiamare una funzione passando per lo stesso argomento una
variabile di un tipo di dato oppure un'altra variabile di un tipo diverso. Se
passassimo al metodo \fn{new} direttamente la tabella dati anziché la funzione
generatrice, si potrebbe riconoscerne il tipo e costruire di conseguenza sia la
funzione \fn{row\_func} che il numero totale di righe al posto dell'utente. In
effetti nel codice del file \file{E0-004-tab.tex} si trova un'implementazione
che fa proprio questo.

Tuttavia potremo optare per implementare altri costruttori della classe, per
esempio con il nome di \fn{from\_data} o \fn{from\_file}, specifici per un tipo
di sorgente dati, e indubbiamente è la soluzione da preferire. Potremo anche
sperimentare il punto di vista per colonne anziché quello per righe mantenendo
la generalità della classe.

Proseguiamo adesso migliorando il modo in cui generare il codice di riga
nell'ambiente \key{tabular}. Stiamo infatti usando la concatenazione di
stringhe, un modo non molto efficiente e nemmeno comodo. Potrebbe essere più
conveniente specificare una sorta di template con segnaposto come la stringa
seguente per un'ipotetica tabella a due colonne:
\begin{Verbatim}[numbers=none]
template = [[\textbf{<1>} & <2>\\]]
\end{Verbatim}

Il numero tra parentesi acute, i segni di minore e maggiore, indica l'indice di
riga così come definito nelle funzioni \fn{row\_func}.

Per far questo, basta aggiungere alla classe \code{Row} un metodo d'iterazione
che a ogni passo ritorni la stringa risultato, e che segua le specifiche perché
possa essere usato in un ciclo \key{for}:
%
\tcbdocmarginnote[enlarge top initially by=\margindown]{%
\fn{string.gsub}\\
\gotosec{} \ref{secFondGsub}}
%
\tcbdocmarginnote[enlarge top initially by=\dimexpr\margindown+9mm\relax]{%
Pattern\\
\gotosec{} \ref{secFondPattern}}
%
\tcbdocmarginnote[enlarge top initially by=\dimexpr\margindown+18mm\relax]{%
Capture\\
\gotosec{} \ref{secFondCapture}}
%
\begin{Verbatim}
function Row:iter_template(tmpl)
    local iter_fn = function(row, i)
        if not i then
            i = row.start
        else
            i = i + row.step
        end
        if i <= self.stop then
            self.fn_next(i, self)
            local s = tmpl:gsub("<(%d+)>", function (s)
                local n = tonumber(s)
                return row[n]
            end)
            return i, s
        end
    end
    return iter_fn, self, nil
end
\end{Verbatim}

Al di la di considerazioni di efficienza legate all'uso della funzione di
libreria \fn{gsub}, l'iteratore in effetti funziona come dimostra il seguente
codice per \LuaLaTeX{} estratto dal file \file{app-start/E0-005-tab.tex}
parte della guida, dove abbiamo inserito la macro \cs{noexpand} per bloccare
l'espansione delle control sequence\footnote{Certo mi ostino ancora a non
utilizzare il pacchetto \pack{luacode}.}:
\begin{Verbatim}
\begin{tabular}{lr}
\directlua{
local tmpl = [[\noexpand\textbf{<1>} & <2>\noexpand\\]]
for _, s in row:iter_template(tmpl) do
   tex.print(s)
end
}
\end{tabular}
\end{Verbatim}

Torniamo alla nostra tabella dei pesi. La funzione generatrice e il template di
riga saranno le seguenti:
\begin{Verbatim}
local function row_func(diam, row)
    row[1] = diam
    local area = math.pi * (diam/20)^2
    local fmt = string.char(37)..'0.3f'
    row[2] = fmt:format(area)
    row[3] = fmt:format(0.785*area)
end
row = Row:new(row_func, 6, 32, 2)
tmpl = [[\noexpand\textbf{<1>} & <2> & <3>\noexpand\\]]
\end{Verbatim}

\directlua{
function Row:iter_template(tmpl)
    local iter_fn = function(row, i)
        if not i then
            i = row.start
            row.counter = 0
        else
            i = i + row.step
        end
        if i <= self.stop then
            row.counter = row.counter + 1
            self.fn_next(i, self)
            local perc = string.char(37)
            local s = tmpl:gsub('<('..perc..'d+)>', function (s)
                local n = tonumber(s)
                return assert(row[n])
            end)
            return i, s
        end
    end
    return iter_fn, self, nil
end

local function row_func(diam, row)
    row[1] = diam
    local area = math.pi * (diam/20)^2
    local fmt = string.char(37)..'0.3f'
    row[2] = fmt:format(area)
    row[3] = fmt:format(0.785*area)
end
row = Row:new(row_func, 6, 32, 2)
tmpl = [[\noexpand\textbf{<1>} & <2> & <3>\noexpand\\]]
}
e il risultato è:
\begin{center}
\begin{tabular}{lrr}
\directlua{
for _, s in row:iter_template(tmpl) do
   tex.print(s)
end
}
\end{tabular}
\end{center}

Miglioriamo ora il codice della funzione generatrice aggiungendo il metodo
\fn{insert} alla classe \code{Row}, a tre argomenti: il numero di colonna
\key{col}, il valore da inserire nella cella \key{val} e infine il valore
opzionale di arrotondamento numerico \key{prec}. Eccone una sua implementazione
molto semplice:
\begin{Verbatim}
function Row:insert(col, val, prec)
    if prec then
        local p = string.char(37)
        local fmt = string.format(p..p.."0."..p.."df", prec)
        val = string.format(fmt, val)
    end
    self[col] = val
    return self
end
\end{Verbatim}

Il nuovo metodo restituisce l'oggetto stesso così che possiamo concatenare più
inserimenti di cella. Ecco come la funzione di generazione può semplificarsi:
\begin{Verbatim}
local function row_func(diam, row)
    local area = math.pi * (diam/20)^2
    row:insert(1, diam)
       :insert(2, area, 3)
       :insert(3, 0.785*area, 3)
end
\end{Verbatim}

Sono scomparse le acrobazie per il formato numerico a favore della compattezza.
Un ulteriore miglioramento ci consente di evitare di dover controllare
l'espansione quando inseriamo il testo del template di riga grazie al comando
\cs{detokenize}.

Introduciamo in proposito una nuova macro \cs{printrow} che ha come argomento il
template che rappresenta il modello della genrica riga della tabella:
\begin{Verbatim}
\newcommand{\printrow}[1]{\directlua{
local tmpl = [=[\detokenize{#1}]=]
for _, s in row:iter_template(tmpl) do
   tex.print(s)
end
}}
\end{Verbatim}

Per non introdurre un secondo argomento, nell'istanziare l'oggetto della classe
\code{Row} dovremo solo ricordarci di chiamare la variabile come \key{row}, lo
stesso nome usato nella definizione di \cs{printrow}. Mettiamo subito al lavoro
la nuova macro:
\begin{Verbatim}
\begin{tabular}{lrr}
\printrow{\textbf{<1>} & <2> & <3>\\}
\end{tabular}
\end{Verbatim}

Molto semplice: si definisce prima la funzione generatrice e con essa si
costruisce l'oggetto \key{Row}, poi si scrive il codice \LaTeX{} dando alla
macro \cs{printrow} il template con i segnaposto.

Molto importante è far corrispondere i numeri di cella dei segnaposto del
template con i valori che la funzione di riga inserisce nella varie posizioni.

L'ultimo passo è migliorare l'aspetto della tabella. Con il pacchetto
\pack{booktabs} aggiungiamo un'intestazione e un filetto ogni tre righe per
facilitare la lettura dei dati. Dobbiamo così modificare la funzione di riga per
derminare se il numero di riga è multiplo di tre --- senza usare l'operatore
modulo \key{\%} di Lua perché non ci troviamo in un file esterno:
\begin{Verbatim}
\directlua{
local function fn(diam, row)
    local area = math.pi * (diam/20)^2
    local peso = 0.785*area
    local c = row.counter
    local midrule = ""
    if c - 3*math.floor(c/3) == 0
    and not (diam == row.stop) then
        midrule = string.char(92).."midrule"
    end
    row:insert(1, diam)
       :insert(2, area, 3)
       :insert(3, peso, 6)
       :insert(0, midrule)
end
}
\end{Verbatim}

Introduciamo anche il pacchetto \pack{siunitx} \cite{pkg:siunitx} utilissimo per
comporre numeri, unità di misura e tabelle, con questo ambiente \amb{tabular}
ridisegnato:
\begin{Verbatim}
\begin{tabular}{
    c
    S[table-format=4.3]
    S[table-format=1.3]
    S[table-format=1.6]
}
\toprule
\diameter & {Sviluppo} & {Sezione} & {Peso}\\
\small\si{mm} & {\small\si{cm^2/m}} & {\small\si{cm^2}} &
    {\small\si{daN/m}}\\
\midrule
\printrow{\(\mathbf{<1>}\) & <4> & <2> & <3>\\<0>}
\bottomrule
\end{tabular}
\end{Verbatim}

Il progetto completato si trova nel file \file{app-star/E0-006-tab.tex}, dove ho
aggiunto alla tabella la colonna con il calcolo della superficie laterale delle
barre.


\subsection{Conclusioni}

La nostra classe \code{Row} ci permette di costruire tabelle iterative in Lua in
modo del tutto generale, compiendo calcoli numerici e ogni sorta di possibili
elaborazioni. Molti altri affinamenti sono possibili come il caricamento di dati
esterni oppure l'uso di pipeline di operatori all'interno dei segnaposto dei
template. Anche questo tutorial si chiude perciò con una lista di nuove idee da
implementare con Lua.

La tabella sottostante mostra il risultato finale.

\directlua{
function Row:insert(col, val, prec)
    if prec then
        local p = string.char(37)
        local fmt = string.format(p..p..'0.'..p..'df', prec)
        val = string.format(fmt, val)
    end
    self[col] = val
    return self
end

local function fn(diam, row)
    local area = math.pi * (diam/20)^2
    local peso = 0.785*area
    local sup_lat = 10 * math.pi * diam
    local c = row.counter
    local midrule = ''
    if c - 3*math.floor(c/3) == 0 then
        midrule = string.char(92)..'midrule'
    end
    row:insert(1, diam)
       :insert(2, area, 3)
       :insert(3, peso, 6)
       :insert(4, sup_lat, 3)
       :insert(0, midrule)
end
row = Row:new(fn, 6, 32, 2)
}

\newcommand{\printrow}[1]{\directlua{
local tmpl = [=[\detokenize{#1}]=]
for _, s in row:iter_template(tmpl) do
   tex.print(s)
end
}}

\begin{center}
\begin{tabular}{
    c
    S[table-format=4.3]
    S[table-format=1.3]
    S[table-format=1.6]
}
\toprule
\diameter & {Sviluppo} & {Sezione} & {Peso}\\
\small\si{mm} & {\small\si{cm^2/m}} & {\small\si{cm^2}} & {\small\si{daN/m}}\\
\midrule
\printrow{\(\mathbf{<1>}\) & <4> & <2> & <3>\\<0>}
\bottomrule
\end{tabular}
\end{center}

% end of file




\chapter{Sul sistema \TeX{} e Lua}
\label{iichExplain}

Questo capitolo fornisce informazioni sulla differenza tra motore di
composizione e formato, e sull'esecuzione di codice Lua all'interno di un
sorgente \TeX.


\section{Motori di composizione e formati}

Un sorgente \TeX{} contiene testo e macro. Il testo formerà i capoversi, i
titoli e il resto del documento, mentre le macro ne stabiliranno l'aspetto e la
struttura. I \emph{motori di composizione} costruiscono il documento disponendo
sia delle macro dette \emph{primitive} implementate direttamente in essi, sia
delle macro utente definite per svolgere compiti specifici e ricorrenti.

Queste nuove macro il cui codice è generalmente scritto da veri esperti,
possono essere aggregate in una sorta di libreria di alto livello che prende
il nome di \emph{formato}, come \LaTeX{} o Con\TeX t, perché per esse sono
state stabilite nuove e coerenti regole di sintassi.

Per esempio, nel formato \LaTeX{} è presente il concetto di \emph{ambiente} con
la coppia di macro di delimitazione \cs{begin\{\}} e \cs{end\{\}}. I motori di
composizione hanno l'abilità nella fase iniziale della compilazione di caricare
il formato in forma precompilata.

Se si avvia un qualsiasi motore di composizione della famiglia \TeX{} verrà
caricato di default il formato più semplice chiamato \emph{plain}. Se si vuole
invece utilizzare un diverso formato, per esempio il più diffuso \LaTeX{},
occorre passarne il nome al compilatore con l'opzione \texttt{-{}-fmt} nel
comando al terminale.

Tuttavia, data l'importanza per gli utenti dei formati di alto livello, sono
stati predisposti appositi comandi scorciatoia. Per esempio il programma
\prog{pdflatex} rimanda all'effettivo motore di composizione \prog{pdftex} con
l'istruzione di caricare il formato \LaTeX{}, e così anche per \prog{lualatex}
con \prog{luatex}.

I due comandi seguenti sono del tutto equivalenti, nel primo si invoca
direttamente il motore di composizione mentre nel secondo lo si fa
indirettamente:
\begin{Verbatim}[numbers=none]
$ pdftex --fmt=pdflatex mydoc
$ pdflatex mydoc
\end{Verbatim}

Se utilizzassimo il formato \code{latex} con il compositore \prog{pdftex} avremo
in uscita non un PDF ma un DVI, il formato nativo di \TeX{}. In questo caso,
nelle moderne distribuzioni il nome del programma scorciatoia è proprio
\prog{latex} perché l'originale \prog{tex} è stato sostituito da \prog{pdftex},
quindi questi due comandi di compilazione sono equivalenti:
\begin{Verbatim}[numbers=none]
$ pdftex --fmt=latex mydoc
$ latex mydoc
\end{Verbatim}

Quando il nome del formato corrisponde al nome del programma scorciatoia, non li
si deve confondere: sono due oggetti molto diversi. I file dei formati sono
usualmente generati o rigenerati in automatico dalle utility della distribuzione
nel momento dell'installazione o quando l'utente esegue un aggiornamento del
sistema che contiene modifiche al formato, perciò è molto raro che ci si trovi
nella situazione di doverne generare uno.

Riassumendo, i motori di composizione sono programmi tipografici mentre i
formati sono insiemi coerenti di macro basate sulle primitive del compositore. I
nomi dei programmi disponibili nel sistema \TeX{} possono confondere se non si
conosce questa importante distinzione: alcuni di essi sono comandi scorciatoia
per identificare sia il motore sia il formato e non un vero e proprio motore di
composizione.


\subsection{Compositori Lua-powered}

\LuaTeX{} è un programma che elabora un file di testo contenente codice \TeX{}
per comporne il corrispondente file PDF, il formato di uscita di default, quindi
è un motore di composizione. Nella famiglia \TeX{} ci sono almeno altri due
compositori dotati dell'interprete Lua, LuaHB\TeX{} e Luajit\TeX{}.

Tutti e tre questi compositori possono eseguire il formato \LaTeX. Come detto in
apertura di sezione, esiste il programma \prog{lualatex} scorciatoia a un
compositore che carica il formato \LaTeX.

Dalla TeX Live 2020 questo compositore è \prog{luahbtex}. Per rendercene conto
basta scrivere in un terminale il nome del programma, premere invio e leggere
l'output poi premere CTRL+C per chiuderne l'esecuzione di prova:
\begin{Verbatim}[numbers=none]
> lualatex
This is LuaHBTeX, Version 1.12.0 (TeX Live 2020/W32TeX)
 restricted system commands enabled.
**
\end{Verbatim}

E in definitiva per compilare un sorgente \LaTeX{} con \LuaLaTeX{}, con la TeX
Live 2020 i due comandi equivalenti sono:
\begin{Verbatim}[numbers=none]
$ luahbtex --fmt=lualatex mydoc
$ lualatex mydoc
\end{Verbatim}

Per ulteriore informazione, \prog{luahbtex} è il motore di composizione
\prog{luatex} in cui è stato sostituito il componente per il calcolo della forma
dei font con il modulo HarfBuzz \cite{lib:harfbuzz}, mentre \prog{luajittex} è
un'altra variante di \prog{luatex} in cui l'interprete Lua è stato sostituito
con LuaJIT \cite{prg:luajit} un'implementazione indipendente più veloce
dell'interprete ufficiale che sfrutta le tecniche di compilazione denominate
\emph{Just In Time}.

In generale un sorgente \TeX{} che contiene codice Lua viene correttamente
compilato da qualsiasi dei tre compositori grazie al mantenimento della
compatibilità.


\subsection{\LaTeX{} con \code{pdflatex} o \code{lualatex}}

I sorgenti \LaTeX{} possono essere compilati sia con il tradizionale motore
\code{pdftex} che con il compositore \code{luahbtex} con minime modifiche.
Vediamo quali sono.

Nel preambolo di un sorgente scritto per \code{pdflatex} si caricano tre
pacchetti fondamentali, come in questo frammento di codice:
\begin{Verbatim}
% !TeX program = pdfLaTeX
\documentclass{article}
\usepackage[T1]{fontenc}
\usepackage[utf8]{inputenc}
\usepackage[italian]{babel}
\end{Verbatim}

Oggi la stragrande maggioranza degli utenti, e se non siete tra loro fatelo
subito, codificano i file sorgenti con la codifica UTF-8. Il testo in Unicode
infatti elimina ogni incoveniente quando si condividono i file o li si apre in
un sistema operativo diversi\footnote{Per informazioni dettagliate sulle
codifiche rimando alla guida tematica \GuIT{}
\href{http://www.guitex.org/home/images/doc/GuideGuIT/introcodifiche.pdf}{%
Introduzione alle codifiche in entrata e uscita} di Claudio Beccari e Tommaso
Gordini \cite{gt:codifiche}.}.

Per \LuaTeX{} è obbligatorio la codifica UTF-8 perciò la prima modifica al
sorgente è semplicemente eliminare il caricamento del pacchetto \pack{inputenc}.
Se il vostro file non è codificato Unicode, e oggi l'eventualità dovrebbe essere
rara, si può utilizzare un tool come \code{iconv} come spiegato nella guida
tematica menzionata in nota.

Per i font invece è necessario eliminare il pacchetto \pack{fontenc} ed
eventualmente specificare quelli da usare nel documento con il pacchetto
\pack{fontspec} e le sue macro \cs{setmainfont}, \cs{setmonofont} eccetera. I
font matematici vanno invece gestiti con il pacchetto \pack{unicode-math} con la
macro \cs{setmathfont}.

Da quel che abbiamo detto il preambolo cambia così:
\begin{Verbatim}
% !TeX program = LuaLaTeX
\documentclass{article}
\usepackage{fontspec}
\usepackage[italian]{babel}
\end{Verbatim}

L'immediato vantaggio di passare da pdf\LaTeX{} a \LuaLaTeX{} è proprio nel
poter comporre il documento con i font Open Type. Inoltre non si usa un
programma che il cui sviluppo è cessato da tempo. Per la mia esperienza la
compatibilità con i numerosi pacchetti \LaTeX{} è ottima mentre è possibile
utilizzare i pacchetti scritti in Lua ormai numerosi.

Ulteriori informazioni dove viene incluso anche il compositore \XeLaTeX{} si
possono trovare nel documento \texttt{lualatex-doc}, al solito disponibile a
video con l'utility \code{texdoc}.

\subsection{Codice Lua in \LuaTeX}

Per illustrare l'esecuzione di codice Lua all'interno di un sorgente \LuaTeX,
consideriamo la stampa di un semplice testo nell'output di console con il
seguente sorgente completo, dove il codice Lua va inserito come argomento della
primitiva \cs{directlua}:
\begin{Verbatim}
% !TeX program = LuaTeX
\directlua{
    print("Hello World!")
}
\bye
\end{Verbatim}
il testo uscirà tra gli altri messaggi di output senza che sia prodotto un file
PDF. Ciò significa che \cs{directlua} è una macro espandibile con risultato
vuoto.

La prima linea di commento è una \emph{riga magica}, comodissima nel dare
istruzione allo shell editor sul programma da usare per compilare il documento
ma ignorata durante la compilazione stessa\footnote{La sintassi delle righe
magiche dipende dall'editor, in questa guida essa è scritta secondo le regole
di TeX Works.}. Qui le useremo se pertinenti per aiutare il lettore a
stabilire il contesto di esecuzione del codice.

Se il sorgente è memorizzato nel file \texttt{primo.tex}, possiamo verificare
quanto previsto in un terminale lanciando il comando:
\begin{Verbatim}[numbers=none]
$ luatex primo
\end{Verbatim}
e per il sistema operativo Windows e la distribuzione TeX Live 2020, l'output
in console è:
\begin{Verbatim}
This is LuaTeX, Version 1.12.0 (TeX Live 2020/W32TeX) 
    restricted system commands enabled.
(./primo.texHello World!
)
warning  (pdf backend): no pages of output.
Transcript written on primo.log.
\end{Verbatim}


\subsection{Codice Lua in Lua\LaTeX}
\label{secLuaInLuaLaTeX}

Con Lua\LaTeX{} si ottiene lo stesso risultato ma con il sorgente scritto nella
sintassi \LaTeX, ovvero:
\begin{Verbatim}
% !TeX program = LuaLaTeX
\documentclass{article}
\directlua{
    print("Hello World!")
}
\begin{document}
\end{document}
\end{Verbatim}
e questa volta il comando di compilazione è:
\begin{Verbatim}[numbers=none]
$ lualatex primo
\end{Verbatim}

Avremo potuto inserire la macro all'interno dell'ambiente \amb{document} anziché
nel preambolo. Quando \TeX{} incontra \cs{directlua} ne \emph{espande}
l'argomento e passa a Lua il controllo che esegue immediatamente il codice per
poi restituirlo di nuovo a \TeX{} al termine dell'esecuzione.


\section{Passaggio di dati}
\label{secPassaggioDati}

La comunicazione dati bidirezionale tra \TeX{} e Lua può avvenire con la
tecnologia dei nodi, come vedremo, certamente quella più avanzata e complessa
ma non è l'unica: i dati possono arrivare a Lua tramite l'espansione, mentre
nella direzione opposta è possibile scrivere del testo nella lista di input
del compositore con le funzioni della famiglia \fn{tex.print}.

Interrompete la lettura della guida per provare a scrivere la funzione Lua
\fn{fact} che calcola il fattoriale di un intero. La useremo per dimostrare come
avviene questo scambio di dati. L'idea è definire un nuovo comando che usando la funzione stampi il fattoriale del numero argomento:
\begin{Verbatim}
\newcommand{\fattoriale}[1]{\directlua{
    local n = #1
    tex.print(tostring(fact(n)))
}}
\end{Verbatim}

Quando \TeX{} incontra una macro utente con un segnaposto\footnote{Fate
riferimento ha una buona guida per \LaTeX{} per saperne di più sulla definizione
di macro utente.} lo sostituisce con i dati corrispondenti token singolo o
gruppo di token tra parentesi graffe. Per questo quando nel sorgente scriviamo
\cs{fattoriale}\code{\{5\}}, dopo la sostituzione il codice Lua effettivamente
eseguito sarà:
\begin{Verbatim}
local n = 5
tex.print(tostring(fact(n)))
\end{Verbatim}

La funzione \fn{tex.print} inserisce l'argomento stringa nell'input d'ingresso
come se fossero stati letti dal sorgente. Quando a termine l'esecuzione del
blocco di codice Lua della macro \cs{fattoriale} i successivi caratteri che si
troverà a elaborare il compositore saranno le cifre 120 e l'espansione della
macro sarà stata vuota.

Per documentarsi nel dettaglio esiste un'ottima risorsa su Overleaf, la popolare
piattaforma di compilazione e gestione progetti \LaTeX{} online,
\href{https://cs.overleaf.com/learn/latex/Articles/An_Introduction_to_LuaTeX_(Part_2):_Understanding_%5Cdirectlua}{questo indirizzo web}.

Il listato del sorgente completo dell'esercizio proposto da confrontare con la vostra personale soluzione è:
\VerbatimInput{app-start/E1-001-fattoriale.tex}


\section{Globale o locale}
\label{secGlobaleLocale}

Prendendo spunto ancora dall'esempio precedente, soffermiamoci sul comportamento
nei diversi blocchi \cs{directlua} delle definizioni locali e globali: come
descritto dalle specifiche di Lua, tutto quello che è locale a un blocco non è
più disponibile al di fuori di esso. Nel contesto di \LuaTeX{} il codice
contenuto in una macro \cs{directlua} forma un blocco.

Per questo motivo se separassimo il codice che calcola il fattoriale dalla
definizione della macro utente \cs{fattoriale}, per non ricevere un errore di
chiamata di un valore \code{nil} dal secondo blocco dovremo definire la funzione
come globale come in:
\VerbatimInput{app-start/E1-002-fattoriale.tex}

Le definizioni globali tuttavia possono determinare errori dovuti alla
collisione dei nomi definiti in altre parti del sorgente essendo l'ambiente
appunto globale e quindi unico. Spesso, come per gli esempi della guida, non è
un problema ma è buona norma definire una tabella di \emph{namespace} dove
memorizzare le funzioni limitando la collisione dei nomi solamente al nome del
riferimento alla tabella.

Questa buona prassi diviene \emph{obbligatoria} quando stiamo scrivendo codice
applicativo in contesti di utilizzo reale.

Come ulteriore esempio, nel listato\footnote{Questo listato è interessante
perché è migliorabile sia per eliminare le ripetizioni di codice a cui è
costretto l'utente sia nell'efficienza di esecuzione.} che segue è mostrato come
si possa formattare il numero del fattoriale con il pacchetto
\pack{siunitx}: \VerbatimInput{app-start/E1-003-fattoriale.tex}


\section{Espansione di macro}

La comunicazione di un dato tra \TeX{} e Lua può avvenire anche per espansione
di macro. Come esempio minimo consideriamo il seguente sorgente \LuaTeX{} che
stampa l'ora di inizio della compilazione avvalendosi del contatore \cs{time}
che indica i minuti trascorsi dall'inizio del giorno:
\VerbatimInput{app-start/E1-004-time.tex}

Al termine dell'espansione l'istruzione di assegnazione della variabile numerica
\key{time} sarà l'effettiva e corretta sintassi Lua. Per il formato \LuaLaTeX{}
lo stesso file potrebbe essere:
\VerbatimInput{app-start/E1-005-time.tex}

Nel capitolo~\ref{iichRegistro} vedremo che anche con la libreria standard di
Lua è possibile ottenere la data e l'ora corrente.


\section{Caratteri speciali}

Alcuni simboli hanno un diverso significato per \TeX{} e per Lua, per esempio il
carattere cancelletto, la stessa backslash o il simbolo di percento. \LuaTeX{}
non si occupa direttamente di modificare il significato dei simboli che si
sovrappongono.

Le descrizioni dei conseguenti errori possono essere criptici ma ci sono almeno
quattro diverse soluzioni:
\begin{compactitemize}
\item scrivere il codice Lua in file esterni,
\item utilizzare codice \TeX{} per gestire l'espansione dei simboli o i codici
di categoria,
\item usare i codici ASCII per creare stringhe che contengono i simboli, con la
funzione \fn{string.char},
\item utilizzare il pacchetto \pack{luacode}.
\end{compactitemize}

Come preferenza personale non utilizzo l'ottimo pacchetto \pack{luacode},
anche per evitare una dipendenza in più. Tuttavia non appena il codice Lua
cresce in numero di linee o diventa un componente di un progetto reale, quasi
sempre si devono utilizzare file esterni per una più agevole gestione del
progetto.

Rimando alla documentazione del pacchetto \pack{luacode} per i dettagli sulla
collisione dei simboli. Vedrete che esso offre due comandi e due ambienti per
poter far inserire codice Lua in un sorgente \LaTeX{} con varie impostazioni di
caratteri.

Nel codice Lua si possono usare i commenti in stile \TeX{} con il simbolo del
percento perché il processo d'espansione li elimina \emph{prima} di passare il
codice all'interprete, ma non si possono usare i commenti in stile Lua con il
doppio trattino a meno che non siano all'ultima linea. Infatti, per la stessa
espansione tutto il codice Lua nella macro \cs{directlua} è inviato
all'interprete come un unica linea di codice perciò il commento dopo un doppio
trattino si estenderebbe non solo a fine riga ma a tutto il codice che segue.

Per fortuna la grammatica Lua a differenza di Python, consente la libera
scrittura e identazione del codice altrimenti questo meccanismo non potrebbe
funzionare.


\section{Le librerie disponibili in \LuaTeX}

Agli usuali moduli della libreria standard di Lua, sono stati aggiunti in
\LuaTeX{}, così come negli altri motori di composizione estesi, speciali
funzionalità dedicate al controllo dello stato interno e alla creazione di
elementi tipografici.





\section{La primitiva \key{directlua}}

Abbiamo ora tutte le informazioni per svolgere alcuni esercizi e applicazioni
fin dai prossimi capitoli. Non rimane che ricordare che il principale modo di
eseguire codice Lua in \LuaTeX{} è assegnarlo come argomento alla macro
\cs{directlua}. Quello che avviene è stabilito da queste regole:
\begin{compactenumerate}
\item l'argomento di \cs{directlua} viene espanso ed eseguito come blocco, può
quindi contenere macro o argomenti macro con un testo di sostituzione;

\item le variabili locali hanno validità solo all'interno del gruppo/blocco
mentre quelle globali saranno valide anche in quelli di successive
\cs{directlua};

\item l'espansione di \cs{directlua} è vuota;
\end{compactenumerate}


% end of file


\part{Fondamenti del linguaggio Lua}
\label{partFoundation}



\chapter{Come eseguire gli esercizi}
\label{iiChEseguireEsercizi}

È certamente fondamentale eseguire noi stessi esempi ed esercizi di
programmazione allo scopo di acquisire la padronanza di Lua. Nella guida ne
trovate alcuni alla fine di ciascun capitolo della parte seconda. 

Questa sezione vi introduce brevemente al programma \prog{texlua} che già
trovate compreso in ogni recente distribuzione \TeX{}. Si tratta dell'interprete
Lua controparte di \prog{luatex}.

Rispetto all'interprete \prog{lua} standard \prog{texlua} non ha la modalità
interattiva REPL\footnote{Read–eval–print loop.} con cui si digita una linea di
codice alla volta in un prompt interattivo, modalità molto utile per fare prove
velocemente.

Il codice dunque, andrà memorizzato in un file con estensione \texttt{.lua}.
Come esempio elementare, digitiamo questa unica riga di codice in un file
di testo chiamato \file{primo.lua}:
\begin{lines}
print("Hello World!")
\end{lines}
apriamo una finestra di terminale\footnote{Maggiori dettagli per diversi sistemi
operativi sulla linea di comando possono essere trovati nella guida tematica
dedicata \emph{Guida alla console} scaricabile dal sito \GuIT.} e lanciamo il
comando:
\begin{Verbatim}[numbers=none,xleftmargin=0pt]
$ texlua primo.lua
\end{Verbatim}

Ora che sappiamo come eseguire il codice Lua, concentriamoci con i prossimi
capitoli sulle basi del linguaggio. Torneremo nella terza parte della guida su
ulteriori modalità di esecuzione anche per il codice Lua interno a sorgenti
\TeX.





\chapter{Assegnazione e tipi predefiniti}
\label{chFondAssignment}


\section{Lua, proprio un bel nome}

Lua è un linguaggio semplice ma non banale. Il suo ambito di applicazione è
quello dei linguaggi di scripting: text processing, manutenzione del sistema,
elaborazioni su file dati, eccetera e lo si può anche trovare come linguaggio
embedded di programmi complessi come i videogiochi o altri applicativi
che danno la possibilità di essere programmati con esso dall'utente.

Lua è stato ideato da un gruppo di programmatori esperti
dell'\href{http://www.puc-rio.br/index.html}{Università Cattolica di Rio de
Janeiro} in Brasile. ``Lua'' si pronuncia LOO-ah e significa ``Luna'' in
portoghese!


\section{L'assegnamento}
\label{secFondAssegnamento}

Ci occupiamo ora di uno degli elementi di base dei linguaggi informatici:
l'istruzione di \emph{assegnamento}. Con questa operazione viene introdotto un
\emph{simbolo} nel programma associandolo a un valore che apparterrà a uno
dei possibili \emph{tipi} di dato.

La sintassi di Lua non sorprende: a sinistra compare il nome della variabile e
a destra l'espressione che fornirà il valore da assegnare al simbolo. Il
carattere di `\texttt{=}' funge da separatore:
\begin{lines}
a = 123
\end{lines}

Durante l'esecuzione di questo codice, Lua determina dinamicamente il tipo del
valore letterale `123' --- un numero --- creandolo in memoria col nome di
`\texttt{a}'.

L'istruzione di assegnamento omette il tipo di dato non essendone prevista una
dichiarazione esplicita. In altre parole, i dati hanno un tipo, ma ciò entra in
gioco solamente a tempo di esecuzione.

Altro concetto importante di Lua è che le variabili sono tutte globali a meno
che non si dichiari il contrario.


\subsection{Locale o globale?}
\label{secFondLocaleGlobale}

Una proprietà dell'assegnamento è che se non diversamente specificato Lua
istanzia i simboli nell'ambiente globale del codice in esecuzione. Se si
desidera creare una variabile locale rispetto al blocco di codice in cui è
definita, occorre premettere alla definizione la parola chiave \key{local}.

Le variabili locali evitano alcuni errori di programmazione e in Lua rendono il
codice più veloce. Le useremo \emph{sempre} quando un simbolo appartiene in modo
semantico a un blocco, per esempio al corpo di una funzione\footnote{Da notare
che in sessione interattiva, ovvero nel modo REPL dell'interprete Lua, ogni riga
è un blocco quindi le variabili locali non sopravvivono alla riga successiva.
Perciò in questa modalità si usano solo variabili globali.}.

Se si crea una variabile locale con lo stesso nome di una variabile globale
quest'ultima viene \emph{oscurata} e il suo valore sarà protetto da modifiche
fino a che il blocco in cui è definita la variabile locale non termina.


\subsection{Assegnazioni multiple}

In Lua possono essere assegnate più variabili alla volta nella stessa
istruzione. Questo significa che l'assegnamento è in realtà più complesso di
quello presentato fino a ora perché è possibile scrivere una lista di variabili
separate da virgole che assumeranno i valori corrispondenti della lista di
espressioni, sempre separate da virgole che compare dopo il segno di uguale:
\begin{lines}
local a, b = 0.45 + 0.23, "text"
\end{lines}

Quando il numero delle variabili non corrispondono a quello delle espressioni,
Lua assegnerà automaticamente valori \texttt{nil} o ignorerà le espressioni in
eccesso. Per esempio:
\begin{lines}
local a, b, c = 0.45, "text"    -- 'c' vale nil
print(a, b, c)
local x, y = "op", "qw", "lo"   -- "lo" è un dato ignorato
print(x, y)
\end{lines}

Nell'assegnazione Lua prima valuta le espressioni a destra e solo
successivamente crea le rispettive variabili secondo l'ordine della lista.
Perciò per scambiare il valore di due variabili, operazione chiamata
\emph{switch}, è possibile scrivere semplicemente:
\begin{lines}
x, y = y, x
\end{lines}

Un ulteriore esempio di assegnazione multipla è il seguente, a dimostrazione
che le espressioni della lista a destra vengono prima valutate e solo dopo
assegnate alle corrispondenti variabili nella lista di sinistra:
\begin{lines}
#[run]
local pi = 3.14159
local r = 10.8 -- raggio del cerchio
-- grandezze cerchio
local diam, circ, area = 2*r, 2*pi*r, pi*r^2
-- stampa grandezze
print("Diametro:", diam)
print("Circonferenza:", circ)
print("Area:", area)
\end{lines}

Le assegnazioni multiple sono interessanti ma sembra non siano così importanti,
possiamo infatti ricorrere ad assegnazioni singole. Diverranno invece molto
utili con le funzioni e con gli iteratori di cui ci occuperemo in seguito.


\section{Una manciata di tipi}
\label{secFondManciataTipi}

In Lua esistono una manciata di tipi. Essenzialmente, omettendone due, sono solo
questi:
\begin{compactitemize}
\item \key{number} il tipo numerico\footnote{Solamente dalla versione 5.3 di Lua
vengono internamente distinti gli interi e i numeri in virgola mobile};
\item \key{string} il tipo stringa;
\item \key{boolean} il tipo booleano;
\item \key{table} il tipo tabella;
\item \key{nil} il tipo nullo;
\item \key{function} il tipo funzione.
\end{compactitemize}

Il breve elenco suscita due osservazioni: tranne la tabella non esistono
tipi strutturati mentre le funzioni hanno il rango di tipo.

Questo fa capire molto bene il carattere di Lua: da un lato l'essenzialità ha
ridotto all'indispensabile i tipi predefiniti nel linguaggio, ma dall'altro ha
spinto all'inclusione di concetti intelligenti e potenti.


\subsection{Il tipo \key{nil}}

Uno dei concetti più importanti che caratterizzano un linguaggio di
programmazione è la presenza o meno del tipo nullo. In Lua esiste e viene
chiamato \key{nil}. Il tipo nullo ha un solo valore possibile, anch'esso
chiamato \key{nil}. Il nome è così sia l'unico valore possibile che il tipo.

Leggere una variabile non istanziata non è un errore perché Lua restituisce
semplicemente \key{nil}, mentre assegnare il valore nullo a una variabile la
distrugge:
\begin{lines}
print(z)      --> stampa nil, la variabile 'z' non esiste
local z = 123 --> assegnamento di un tipo numerico
print(z)      --> stampa 123
z = nil       --> distruzione della variabile
\end{lines}


\subsection{Il garbage collector}

I dati non più utili come quelli di cui non esiste più un \emph{riferimento} a
essi durante l'esecuzione, per esempio perché la variabile è stata riassegnata a
\key{nil}, oppure quelli locali nel momento in cui escono di scopo, vengono
automaticamente eliminati dal \emph{garbage collector} di Lua. Questo componente
solleva l'utente dalla gestione diretta della memoria e sopratutto dai deleteri
errori di programmazione che si possono facilmente compiere nel farlo, al prezzo
di una piccola diminuzione delle prestazioni in fase di esecuzione.

Al termine del programma tutte le risorse in memoria vengono automaticamente
liberate.

\section{Esercizi}

\begin{Exercise}[label=fond-01]
Scrivere il codice Lua che instanzi due variabili \key{x} e \key{y} al valore
12.34. Si assegni alle altre due variabili \key{sum} e \key{prod} la somma e il
loro prodotto delle prime. Si stampi in console i risultati.
\end{Exercise}

\begin{Exercise}[label=fond-02]
Scrivere il codice Lua che dimostri che modificare una variabile locale non
modifica il valore della variabile globale con lo stesso nome. Suggerimento:
utilizzare la coppia \key{do}/\key{end} per creare un blocco di codice con le
proprie variabili locali.
\end{Exercise}

% end of file


\input{section/II-02-tabella}



\chapter{Costrutti di base}

\section{Il ciclo \key{for} e il condizionale \key{if}}
\label{secFondCicloIf}

Cominciamo con il contare i numeri pari contenuti in una tabella che funziona
come un array, ricordandoci che gli indici partono da 1 e non da 0. Rileggete
il capitolo precedente come utile riferimento.

Creiamo la tabella con il costruttore in linea e iteriamo con un ciclo
\key{for}:
\sourcecode{file = [[code/e1-001.lua]], run = true}

Il corpo del ciclo \key{for}\luak{for} di Lua è il blocco compreso tra le parole
chiave obbligatorie \key{do}\luak{do} ed \key{end}. La variabile \key{i} interna
al ciclo assumerà i valori da 1 fino al numero di elementi della tabella,
ottenuto con l'operatore lunghezza \key{\#} valido anche per le stringhe.

Per ciascuna iterazione con il costrutto condizionale \key{if}\luak{if}
incrementiamo un contatore solo se l'elemento della tabella è pari. L'\key{if}
ha anch'esso bisogno di definire il blocco di codice e lo fa con le parole
chiavi obbligatorie \key{then}\luak{then} ed \key{end}\luak{end}, mentre
\key{else}\luak{else} o \key{elseif}\luak{elseif} sono rami di codice
facoltativi.

Il controllo di parità degli interi si basa sull'operatore modulo
\key{\%}\luas{\%}, resto della divisione intera. Infatti un numero pari è tale
se il resto della divisione per 2 è zero. 

L'operatore di \emph{uguaglianza} è il doppio carattere di uguale
\key{==}\luas{==} e quello di \emph{disuguaglianza} è la coppia dei segni tilde
e uguale \verb|~=|\luas{~=}. Naturalmente funzionano anche gli operatori di
confronto \key{>}\luas{>}, \key{>=}\luas{>=} e \key{<}\luas{<},
\key{<=}\luas{<=}.


\section{Operatore lunghezza}
\label{secFondLenOperator}

Ma come si comporta l'operatore di lunghezza \key{\#}\luas{\#} per le
tabelle array con indici non lineari? Per esempio, qual è il risultato del
seguente codice:
\begin{lines}
local t = {}
t[1] = 1
t[2] = 2
t[1000] = 3
print(#t)
\end{lines}
e in questo caso cosa verrà stampato?
\begin{lines}
local t = {}
t[1000] = 123
print(#t)
\end{lines}
e ancora in questo caso con il costruttore?
\begin{lines}
local t1 = {nil, nil,  nil, nil, nil, nil, nil, 8}
print(#t1)
local t2 = {nil, nil,  nil, nil, nil, nil, nil, 8, nil}
print(#t2)
\end{lines}

L'operatore \key{\#} restituisce la lunghezza di una tabella array come uno dei
sui \emph{bordi}. Un bordo è la posizione di un valore non \key{nil} seguito da
un valore \key{nil}, oppure zero se la posizione 1 è \key{nil}.

Questo comportamento riflette la particolare e sofisticata implementazione della
tabella di Lua. L'operatore lunghezza \key{\#} può restituire uno dei
qualsiasi indici interi corrispondenti a un bordo della tabella, in funzione di
come è stata creata o anche della presenza di chiavi non intere.

Se la tabella è una \emph{sequenza}, cioè se non ci sono buchi di valori tra le
posizioni intere partendo da 1 fino a \( n \), allora la tabella ha un solo
bordo che vale \( n \). Se la tabella è vuota il suo bordo vale zero.

L'operatore \key{\#} viene calcolato molto velocemente anche per tabelle array
grandi, ed è usato per espressioni idiomatiche come quella usatissima per
l'inserimento di un nuovo dato in coda a una sequenza:
\begin{lines}
#[run]
local t = {101, 102, 103, 104, 105} -- una sequenza
t[#t + 1] = 106
print(#t, t[6])
\end{lines}


\section{Il ciclo \key{while}}
\label{secCicloWhile}

Passiamo a scrivere il codice per inserire in una tabella i fattori primi di un
numero. Fatelo per esercizio e poi confrontate il codice seguente che utilizza
l'operatore modulo \key{\%}:
\sourcecode{file = [[code/e1-002.lua]], run = true}

Così abbiamo introdotto anche il ciclo \key{while}\luak{while}
perfettamente coerente con la sintassi dei costrutti visti fino a ora: il blocco
di codice ripetuto fino a che la condizione è vera, è obbligatoriamente definito
da due parole chiave, quella di inizio è \key{do} e quella di fine è \key{end}.

Le variabili definite come locali nei blocchi del ciclo \key{for}, nei rami del
condizionale \key{if} e nel ciclo \key{while}, non sono visibili all'esterno.


\section{Intermezzo}

In Lua non è obbligatorio inserire un carattere delimitatore sintattico ma è
facoltativo il segno \key{;}. I caratteri spazio, tabulazione e ritorno a capo
vengono considerati dalla grammatica come separatori, perciò si è liberi di
formattare il codice come si desidera inserendo per esempio più istruzioni sulla
stessa linea. Solitamente non si utilizzano i punti e virgola finali, ma se ci
sono due assegnazioni sulla stessa linea --- stile sconsigliabile perché poco
leggibile --- li si può separare almeno con un segno \key{;}. Come sempre una
forma stilistica chiara e semplice vi aiuterà a scrivere codice più
comprensibile anche a distanza di tempo.

Generalmente è buona norma definire le nuove variabili il più vicino possibile
al punto in cui verranno utilizzate per la prima volta, un beneficio per la
comprensione ma anche per la correttezza del codice perché può evitare di
confondere i nomi e magari di introdurre errori.


\section{Il ciclo \key{for} con il passo}

Provate a scrivere il codice Lua che verifica se un numero è \emph{palindromo},
ovvero che gode della proprietà che le cifre decimali sono simmetriche come per
esempio avviene per il numero 123321. Confrontate poi questa soluzione:
\sourcecode{file = [[code/e1-003.lua]], select = [[prima_sol]], run = true}

La soluzione utilizza una tabella per memorizzare le cifre in ordine inverso del
numero da verificare, che vengono poi utilizzate successivamente nel ciclo
\key{for}\luak{for} dall'ultima --- la cifra più significativa --- fino alla
prima per ricalcolare il valore. Se il numero iniziale è palindromo allora il
corrispondente numero a cifre invertite è uguale al numero di partenza.

Nel ciclo \key{for} il terzo parametro opzionale -1 imposta il passo per la
variabile \key{i} che quindi passa dal numero di cifre del numero da
controllare (6 nel nostro caso) a 1.

In effetti non è necessaria la tabella:
\sourcecode{file = [[code/e1-003.lua]], select = [[seconda_sol]], run=true}


\section{\key{if} a rami multipli}

Il prossimo problema è il seguente: determinare il numero di cifre di un
intero. Ancora una volta, confrontate il codice proposto solo dopo aver cercato
una vostra soluzione.
\sourcecode{file = [[code/e1-004.lua]]}

Questo esempio mostra in azione l'\key{if}\luak{if} a più rami che in Lua svolge
la funzione del costrutto \key{switch} presente in altri linguaggi, con una
nuova parola chiave: \key{elseif}.

L'esempio è interessante anche per come viene introdotta la variabile
\key{digits}, cioè senza inizializzarla per poi assegnarla nel ramo opportuno
dell'\key{if}. Infatti una variabile interna a un blocco non sopravvive oltre,
per questo motivo dichiararla all'interno dell'\key{if} non è sufficiente.

Come è necessario \emph{non} premettere \key{local}\luak{local} nelle
assegnazioni nei rami del condizionale: in questo caso verrebbe creata una nuova
variabile locale al blocco che \emph{oscurerebbe} quella esterna con lo stesso
nome. In altre parole, al termine del condizionale \key{digits} varrebbe ancora
\key{nil}, il valore che assume nel momento della dichiarazione.


\section{Esercizi}

\begin{Exercise}[label=cos-01]
Contare quanti interi sono divisibili sia per 2 che per 3 nell'intervallo \( [1,
10\,000]\). Suggerimento: utilizzare l'operatore modulo \key{\%}, resto della
divisione intera tra due operandi.
\end{Exercise}

\begin{Exercise}[label=cos-02]
Determinare i fattori del numero intero \(5\,461\,683\) modificando il codice
riportato alla sezione~\ref{secCicloWhile} per includerne la molteplicità.
\end{Exercise}

\begin{Exercise}[label=cos-03]
Calcolare il determinante della matrice corrispondente alla seguente tabella,
che contiene tre tabelle/array con tre numeri in sequenza.
\begin{lines}
local t = {
    { 0,  5, -1},
    { 2, -2,  0},
    {-1,  0,  1},
}
\end{lines}
\end{Exercise}

\begin{Exercise}[label=cos-04]
Data la tabella seguente stampare in console il conteggio dei numeri pari e dei
numeri dispari contenuti in essa. Verificare che la somma di questi due
conteggi sia uguale alla dimensione della tabella.
\begin{lines}
local t = {
    45, 23, 56, 88, 96, 11,
    80, 32, 22, 85, 50, 10,
    32, 75, 10, 66, 55, 30,
    10, 13, 23, 91, 54, 19,
    50, 17, 91, 44, 92, 66,
    71, 25, 19, 80, 17, 21,
    81, 60, 39, 15, 18, 28,
    23, 10, 18, 30, 50, 11,
    50, 88, 28, 66, 13, 54,
    91, 25, 23, 17, 88, 90,
    85, 99, 22, 91, 40, 80,
    56, 62, 81, 71, 33, 30,
    90, 22, 80, 58, 42, 10,
}
\end{lines}
\end{Exercise}

\begin{Exercise}[label=cos-05]
Data la tabella precedente, scrivere il codice per costruire una seconda tabella
uguale alla prima ma priva di duplicati e senza alterare l'ordine degli interi.
\end{Exercise}

\begin{Exercise}[label=cos-06]
Data la tabella precedente costruire una tabella le cui chiavi siano i numeri
contenuti in essa e i valori siano il corrispondente numero di volte che la
chiave stessa compare nella tabella di partenza. Stampare poi in console il
numero che si presenta il maggior numero di volte.
\end{Exercise}

% end of file




\chapter{Operatori logici}

In Lua un'espressione è vera se essa corrisponde al valore booleano
\key{true}\luak{true} oppure a un valore che non è \key{nil}\luak{nil}.

Gli operatori logici \key{and}\luak{and}, \key{or}\luak{or} e
\key{not}\luak{not} danno luogo ad alcune espressioni idiomatiche di Lua.
Cominciamo con \key{or}: è un operatore logico binario. Se il primo operando è
vero lo restituisce altrimenti restituisce il secondo. Per esempio nel seguente
codice \key{a} vale 123.
\begin{lines}
local a = 123 or "mai assegnato"
\end{lines}

L'operatore \key{and} --- anche questo binario come \key{or} --- restituisce il
primo operando se esso è falso altrimenti restituisce il secondo operando.

\section{Operatore ternario}
\label{secFondOperatoreTernario}

Con \key{and} e \key{or} combinati otteniamo l'operatore ternario del C++ in
Lua: Ecco l'espressione in un esempio: se \key{a} è vera il risultato è \key{b}
altrimenti \key{c}:
\begin{lines}
local val = (a and b) or c
\end{lines}

Poiché \key{and} ha priorità maggiore rispetto a \key{or} nell'espressione
precedente possiamo omettere le parentesi per un codice ancor più idiomatico:
\begin{lines}
local val = a and b or c -- a ? b : c del C++
\end{lines}

Il massimo tra due numeri è un'espressione condizionale:
\begin{lines}
local x, y = 45.69, 564.3
local max
if x > y then
    max = x
else
    max = y
end
\end{lines}
ma con gli operatori logici è tutto più Lua:
\begin{lines}
local x, y = 45.69, 564.3
local max = (x > y) and x or y
\end{lines}

L'operatore logico \key{not} restituisce \key{true} se l'operando è \key{nil}
oppure se è \key{false} e, viceversa, restituisce \key{false} se l'operando non
è \key{nil} oppure è \key{true}. Alcuni esempi:
\begin{lines}
print(not 5)       --> 'false'
print(not not 5)   --> 'true'
print(not true)    --> 'false'
print(not false)   --> 'true'
print(not nil)     --> 'true'
\end{lines}

L'operatore di negazione può essere usato per controllare se una variabile è
valida oppure no. Per esempio possiamo controllare se in una tabella esiste il
campo \key{prezzo}:
\begin{lines}
local t = {} -- una tabella vuota
if not t.prezzo then -- t.prezzo è nil
    print("assente")
else
    print("presente")
end

t.prezzo = 12.00
if not t.prezzo then
    print("assente")
else
    print("presente")
end
\end{lines}


\section{Esercizi}

\begin{Exercise}[label=oplogic-01]
Prevedere il risultato delle seguenti espressioni Lua:
\begin{lines}
local a = 1 or 2
local b = 1 and 2
local c = "text" or 45

local d = not 12 or "ok"
local e = not nil or "ok"
\end{lines}
\end{Exercise}

\begin{Exercise}[label=oplogic-02]
Nel seguente codice, se il valore del primo condizionale è \key{true} cosa
stamperà invece il secondo condizionale?
\begin{lines}
if "stringa" then print "it's not 'nil'" end
if "stringa" == true then
    print("it's 'true'")
else
    print("it's not 'true'")
end
\end{lines}
\end{Exercise}

\begin{Exercise}[label=oplogic-03]
Come distinguere se una variabile contiene il valore \key{false} o il valore
\key{nil}?  
\end{Exercise}

\begin{Exercise}[label=oplogic-04]
Usando gli operatori logici di Lua codificare l'espressione che restituisce la
stringa \verb|"più grande di 100"|, \verb|"uguale"| o
\verb|"più piccolo di 100"| a seconda del valore numerico fornito.
\end{Exercise}

% end file



\chapter{Il tipo stringa}
\label{chFondStringhe}

In Lua le stringhe rappresentano uno dei tipi di base del linguaggio.
Per rappresentare valori stringa letterali ci sono tre diversi delimitatori:
\begin{compactitemize}
\item doppi apici: carattere \verb|"|;
\item apice semplice: carattere \key{'};
\item doppie parentesi quadre: delimitatori \key{[[} e \key{]]} con o senza un
numero corrispondente di caratteri \key{=}, per esempio \key{[==[} e \key{]==]}.
\end{compactitemize}

In una stringa delimitata da doppi apici possiamo inserire liberamente apici
semplici e viceversa, e caratteri non stampabili come il ritorno a capo
(\cs{n}) e la tabulazione (\cs{t}), tramite il carattere di escape backslash
che quindi va inserito esso stesso come doppio backslash (\cs{\textbackslash}):
\sourcecode{file = [[code/e1-005.lua]], run = true}

In Lua non esiste il tipo carattere quindi gli Autori del linguaggio hanno
pensato di utilizzare i delimitatori normalmente destinati a rappresentarne la
forma letterale, per consentire all'utente di creare stringhe contenenti i
delimitatori stessi, senza utilizzare l'escaping.

Sono comunque ammessi i simboli \key{\textbackslash}\verb|"| e
\key{\textbackslash '} che rappresentano i caratteri corrispondenti, come si
vede nella variabile \key{s5} del codice precedente.

Il terzo tipo di delimitatore per le stringhe è una coppia si parentesi quadre e
ha la proprietà di ammettere il ritorno a capo. Si possono così introdurre nel
sorgente interi brani di testo nel quale i caratteri di escaping non saranno
interpretati.
\sourcecode{file = [[code/e1-006.lua]], run = true}

Se per caso nel testo fossero presenti i delimitatori di chiusura è possibile
inserire un numero qualsiasi di caratteri \key{=} tra le parentesi quadre,
purché il numero sia lo stesso per i delimitatori di apertura e chiusura,
esempio:
\sourcecode{ file = [[code/e1-007.lua]]}


\section{Commenti multiriga}

Questi delimitatori variabili con numero qualsiasi di segni \key{=} li troviamo
anche nei commenti multiriga di Lua. Abbiamo incontrato fino a ora i commenti di
riga che si introducono nel codice con un doppio trattino \key{--}.

I commenti multiriga sono comodi quando si vuol escludere dall'esecuzione un
intero blocco di righe: iniziano con i doppi trattini seguiti da un delimitatore
di stringa multiriga e terminano con la corrispondente chiusura:
\begin{lines}
-- questo è un commento di riga

--[[
questo è un commento
multiriga
]]

--[=[
e anche questo è un commento
multiriga
]=]
\end{lines}

Normalmente in Lua i commenti multiriga vengono chiusi premettendo i doppi
trattini anche al gruppo delimitatore di chiusura. Questo è solo un trucco per
riattivare rapidamente il codice eventualmente contenuto nel commento, basta uno
spazio per far trasformare il commento multiriga in uno semplice:
\begin{lines}
--[[ righe di codice non attive
local tab = {}
--]]

-- [[ notare lo spazio dopo i doppi trattini
-- questo codice invece viene eseguito

local tab = {}
--]] -- e questo diventa una normale riga di commento
\end{lines}


\section{Concatenazione stringhe e immutabilità}

In Lua l'operatore \key{..} concatena due stringhe, in questo modo:
\begin{lines}
#[run]
local s1 = "Hello" .. " " .. "world"
local s2 = s1 .. " OK"
s2 = s2 .. "."

print(s1 .. "!")
print(s2)
\end{lines}

Il concetto importante riguardo alle stringhe è se queste siano o no immutabili.
Se non lo sono la concatenazione di stringhe non comporta la creazione di una
nuova stringa ma la modifica in memoria.

In Lua, come in molti altri linguaggi, le stringhe sono invece immutabili.
Ciò significa che una volta create, le stringhe non possono essere modificate e
nel codice precedente, l'operazione di concatenare il carattere punto in coda
alla stringa \key{s2}, genera una nuova stringa che è assegnata alla stessa
variabile.

Per poche operazioni di concatenazione ciò non è un problema, ma in alcuni casi
invece si. Consideriamo il seguente codice apparentemente innocuo:
\begin{lines}
local s = ""

for i = 1, 100 do
    s = s .. "**"
end
print(#s) -- # funziona anche per le stringhe!
--             contandone i byte
\end{lines}

Ma cosa succede in dettaglio? Perché questo codice non è efficiente? A ogni
concatenazione viene creata una nuova stringa. La prima volta vengono copiati
due byte per dare la stringa \verb|"**"|. La seconda iterazione la memoria
copiata sarà di 4 byte, e alla terza di 6 byte, eccetera.

A ogni iterazione la memoria copiata cresce di due byte con il risultato che per
produrre una stringa di 200 asterischi (200 byte) avremo copiato in totale la
memoria equivalente a 10100 byte!

In Java e negli altri linguaggi con stringhe immutabili normalmente si corre ai
ripari mettendo a disposizione una struttura dati o una funzione che risolve il
problema, per esempio un tipo \key{StringBuffer}. In Lua la soluzione è una
funzione della libreria \key{table} che, anticipando rispetto alle nostre
chiaccherate è \fn{table.concat}\luastd{table.concat}:
\begin{lines}
local t = {}
for i = 1, 100 do
    t[#t + 1] = "**"
end

print(#table.concat(t))
\end{lines}

Nel caso specifico avremo dovuto usare la funzione
\fn{string.rep}\luastd{string.rep} anche se \fn{table.concat} è più generale.


\section{Esercizi}

\begin{Exercise}[label=string-01]
Come fare in Lua per creare una stringa letterale contenente sia il
carattere apice semplice che doppio?
\end{Exercise}

\begin{Exercise}[label=string-02]
Quale sarà il risultato dell'esecuzione del seguente codice?
\begin{lines}
local s = "'"..'"'.."ok"..[["']]
print(s)
\end{lines}
\end{Exercise}

\begin{Exercise}[label=string-03]
Creare la stringa \verb|"\/"|.
\end{Exercise}

\begin{Exercise}[label=string-04]
Scrivere un programma che a partire dalla stringa \verb|"*"| crei e stampi la
stringa di 64 asterischi senza utilizzare l'operatore di concatenazione o la
funzione \fn{string.rep}.
\end{Exercise}

\begin{Exercise}[label=string-05]
Scrivere un programma che a partire dalla stringa \verb|"*"| crei e stampi
la stringa di 64 asterischi usando l'operatore di concatenazione il minimo
indispensabile di volte.
\end{Exercise}

% end of file



\chapter{Funzioni}
\label{chFondFunzioni}

Le funzioni sono il principale mezzo di astrazione e organizzazione del codice.

Coerentemente con il resto del linguaggio la sintassi dichiarativa di una
funzione comprende due parole chiave che servono per delimitare il blocco di
codice: \key{function} ed \key{end}. Una funzione può restituire dati tramite la
parola chiave \key{return}.

Come primo esempio, consideriamo una funzione per calcolare l'ennesimo numero
della \href{http://it.wikipedia.org/wiki/Successione_di_Fibonacci}{serie di
Fibonacci}. Un elemento si ottiene sommando i due precedenti elementi avendo
posto uguale a 1 i primi due:
\sourcecode{file = [[code/e2-001.lua]], select = [[uno]]}

Con le regole dell'assegnazione multipla una funzione può accettare più
argomenti. Se gli argomenti passati sono in eccesso rispetto a quelli che essa
prevede, quelli in più verranno ignorati. Viceversa, se gli argomenti sono
inferiori a quelli previsti allora a quelli mancanti verrà assegnato il valore
\key{nil}\luak{nil}.

Le stesse regole valgono anche per i dati di uscita quando la funzione è usata
come espressione in un'istruzione di assegnazione. Dopo l'istruzione
\key{return}\luak{return} si può scrivere una lista delle espressioni separate
da virgole che saranno assegnate alle corrispondenti variabili.

Per esempio, potremo modificare la funzione precedente per restituire anche la
somma dei primi \( n \) numeri di Fibonacci oltre che l'ennesimo elemento della
serie stessa e considerare un valore di default se l'argomento è \key{nil}:
\sourcecode{file = [[code/e2-001.lua]], select = [[due]], run = true,}


\section{Funzioni: valori di prima classe, I}

In Lua le funzioni sono un tipo. Possono essere assegnate a una variabile,
passate come argomento a un'altra funzione e restituite da una funzione come
valore, una proprietà che non si trova spesso nei linguaggi di scripting e che
offre più flessibilità al codice.

A ben vedere in Lua tutte le funzioni sono memorizzate in variabili e di per se
non hanno un nome. Il modo di definizione è quindi la \emph{sintassi anonima}:
\begin{lines}
add = function (a, b)
    return a + b
end
print(add(45.4564, 161.486))
\end{lines}

Questa importante proprietà si riassume dicendo che in Lua le funzioni sono
valori di \emph{prima classe}: sono assegnate a variabili e non hanno un nome
esattamente come non lo hanno gli altri tipi come numeri e stringhe.

Per comodità il linguaggio ammette anche la sintassi classica di definizione:
\begin{lines}
function variable_name(args)
    -- function body
end
\end{lines}
L'interprete Lua la tradurrà automaticamente nel codice equivalente in sintassi
anonima, una caratteristica nascosta detta \emph{zucchero sintattico}:
\begin{lines}
variable_name = function (args)
    -- function body
end
\end{lines}


\section{Funzioni: valori di prima classe, II}

Un esempio di funzione con un argomento funzione è il seguente, dove si vuole
eseguire per un certo numero di volte consecutivamente la funzione argomento:
\sourcecode{
    file = [[code/e3-001.lua]],
    select = [[uno]],
}

Molto interessante. Nell'ultima riga di codice l'argomento è una funzione
definita in sintassi anonima (che verrà eseguita 12 volte).

Essendo le funzioni valori di prima classe è possibile riassegnare variabili per
cambiarne il significato, come per esempio con la funzione predefinita
\fn{print}:
\begin{lines}
#[run]
local orig_print = print
print = function (n)
    orig_print("Argomento funzione -> "..n)
end

print(12)
\end{lines}


\section{Tabelle e funzioni}

Se una tabella può contenere chiavi con qualsiasi valore allora può contenere
anche funzioni! Le sintassi previste sono queste, esplicitate con il codice
riportato di seguito:
\begin{compactitemize}
\item assegnare la variabile di funzione a una chiave di tabella;
\item assegnare direttamente la chiave di tabella con la definizione di funzione
in sintassi anonima;
\item usare il costruttore di tabelle per assegnare funzioni in sintassi
anonima.
\end{compactitemize}
\sourcecode{
    file = [[code/e3-001.lua]],
    select = [[due]],
}

Con questo meccanismo una tabella può svolgere il ruolo di \emph{modulo}
memorizzando funzioni utili in un gruppo. In effetti la libreria standard
di Lua si presenta all'utente proprio in questo modo.


\section{Numero di argomenti variabile}

Una funzione può ricevere un numero variabile di argomenti rappresentati da
\key{...}\luas{...} tre caratteri punto consecutivi\footnote{Questa
caratteristica è chiamata \emph{variadic function}.}. Nel corpo della funzione i
tre punti rappresenteranno la lista degli argomenti, dunque possiamo o costruire
con essi una tabella oppure effettuare un'assegnazione multipla.

Un esempio è una funzione che restituisce la somma di tutti gli argomenti
numerici:
\sourcecode{
    file = [[code/e3-001.lua]],
    select = [[tre]],
}

Nell'ultima riga di codice si può notare che anche la funzione predefinita
\fn{print} accetta un numero variabile di argomenti.

Il meccanismo è ancora più flessibile perché tra i primi argomenti vi possono
essere variabili "fisse". Per esempio il primo parametro potrebbe essere un
moltiplicatore:

\sourcecode{
    file = [[code/e3-001.lua]],
    select = [[quattro]],
}

La funzione predefinita \fn{select}\luastd{select} consente di accedere alla
lista degli argomenti in dettaglio. Infatti nel codice precedente, se tra gli
argomenti compare un valore \key{nil} avremo problemi ad accedere ai valori
successivi perché --- come sappiamo già --- l'operatore di lunghezza \key{\#}
determina la lunghezza della sequenza in base ai bordi cioè alle posizioni dei
valori non nulli seguite da valori \key{nil} (maggiori dettagli alla sezione \S
\ref{secFondLenOperator}).

Il selettore prevede un primo parametro fisso seguito da una lista variabile di
valori rappresentata dai tre punti \key{...}. Se questo parametro è un intero
allora verrà considerato come indice per restituire l'argomento corrispondente.
Se invece il parametro è la stringa \key{\#} allora la funzione restituisce il
numero totale di argomenti \emph{inclusi} i \key{nil}.

Il codice seguente preso pari pari dal \href{http://www.lua.org/pil/}{PIL} ne è
un esempio:
\begin{lines}
for i = 1, select("#", ...) do
    local arg = select(i, ...)
    -- loop body
end
\end{lines}


\section{Omettere le parentesi}

In Lua esiste la sintassi di chiamata a funzione semplificata che consiste nella
possibilità di ommettere le parentesi tonde \key{()}\luas{()}. È ammessa solo se
alla funzione si passa un unico argomento di tipo stringa o di tipo tabella:
\sourcecode{
    file = [[code/e7-funzioni.lua]],
    select = [[anchequesto]],
}


\section{Closure}
\label{secClosure}

Chiudiamo il capitolo parlando di uno strano termine forse meglio noto agli
sviluppatori nei linguaggi funzionali: la \emph{closure}.

Questa proprietà di Lua amplia il concetto di funzione rendendo possibile
l'accesso dall'interno di essa a dati presenti nel contesto esterno. Ciò è
possibile perché alla chiamata di una funzione viene creato uno spazio di
memoria del contesto esterno unico e indipendente.

\begin{quote}
\emph{%
Tutte le chiamate a una stessa funzione condivideranno una stessa closure.%
}
\end{quote}

Se questo è vero una funzione potrebbe incrementare un contatore creato al suo
interno, e anche qui prendo l'esempio di codice dal PIL:
\sourcecode{
    file = [[code/e7-funzioni.lua]],
    select = [[uno]],
}

Il codice definisce una funzione \fn{new\_counter} che restituisce una
funzione che ha accesso indipendente al contesto (la variabile \key{i}).

Tecnicamente la closure \emph{è} la funzione effettiva mentre invece la
funzione non è altro che il prototipo della closure.

Le closure consentono di implementare diverse tecniche utili in modo naturale e
concettualmente semplice. Una funzione di ordinamento potrebbe per esempio
accettare come parametro una funzione di confronto per stabilire l'ordine tra
due elementi tramite l'accesso a una seconda tabella esterna contenente
informazioni utili per l'ordinamento stesso.

Nel prossimo esempio mettiamo in pratica l'idea. Il codice utilizza la funzione
\fn{table.sort} della libreria di Lua che introdurremo nel prossimo capitolo,
per applicare l'algoritmo di ordinamento alla tabella primo argomento in
base al criterio stabilito con la funzione passata come secondo
argomento in sintassi anonima.
\sourcecode{
    file = [[code/e7-funzioni.lua]],
    select = [[due]],
    run = true,
}


\section{Esercizi}

\begin{Exercise}[label=fn-01]
Scrivere una funzione che sulla base della stringa in ingresso \verb|"+"|,
\verb|"-"|, \verb|"*"|, \verb|"/"| restituisca la funzione corrispondente per
due operandi.
\end{Exercise}

\begin{Exercise}[label=fn-02]
Scrivere la funzione che accetti due argomenti numerici e ne restituisca i
risultati delle quattro operazioni aritmetiche.
\end{Exercise}

\begin{Exercise}[label=fn-03]
Scrivere una funzione che restituisca il fattoriale di un numero memorizzandone
in una tabella di closure i risultati per evitare di ripetere il calcolo in
chiamate successive con pari argomento.
\end{Exercise}

\begin{Exercise}[label=fn-04]
Scrivere una funzione con un argomento opzionale rispetto al primo parametro
numerico che ne restituisca il seno interpretandolo in radianti se l'argomento
opzionale è \key{nil} oppure \verb|"rad"|, in gradi sessadecimali se
\verb|"deg"| o in gradi centesimali se \verb|"cen"|.
\end{Exercise}

\begin{Exercise}[label=fn-05]
Scrivere una funzione \fn{ordinates} che accetti come primo argomento una
funzione \( f: ℝ \to ℝ \) (prende un numero e restituisce un numero), come
secondo e terzo argomento i due valori dell'intervallo di calcolo e come quarto
argomento il numero di punti \( n \geq 2 \) in cui suddividere equamente
l'intervallo, e restituisca i valori in sequenza che la funzione \( f \) assume
nei punti d'ascissa così definiti, in una tabella/array con indice iniziale 1.

Verificare poi che le due tabelle calcolate da \fn{ordinates} con le funzioni di
libreria \key{math.sin} e \key{math.cos}, nello stesso intervallo \( 0 \), \(
\pi/2 \) con \( n = 101 \) contengano lo stesso valore nella posizione di indice
51.
\end{Exercise}


% end of file



\chapter{La libreria standard di Lua}
\label{iChLibstd}

In Lua sono immediatamente disponibili un folto gruppo di funzioni che ne
formano la \emph{libreria standard}. Si tratta di una collezione di funzioni
utili a svolgere compiti ricorrenti su stringhe, file, tabelle, eccetera, e si
trovano precaricate in una serie di tabelle.

L'elenco completo ma in ordine sparso con il nome della tabella/modulo
contenitore e la descrizione applicativa è il seguente:
\begin{center}
\begin{tabular}{ll}
\key{math} & matematica;\\
\key{table} & utilità sulle tabelle;\\
\key{string} & ricerca, sostituzione e pattern matching;\\
\key{io} & input/output facility, operazioni sui file;\\
\key{bit32} & operazioni bitwise (solo in Lua 5.2);\\
\key{os} & date e chiamate di sistema;\\
\key{coroutine} & creazione e controllo delle coroutine;\\
\key{utf8} & utilità codifica Unicode UTF-8 (da Lua 5.3);\\
\key{package} & caricamento di librerie esterne;\\
\key{debug} & accesso alle variabili e performance assessment.\\
\end{tabular}
\end{center}
La pagina web a \href{www.lua.org/manual/5.3/contents.html}{questo link}
fornisce tutte le informazioni di dettaglio sulla libreria standard di Lua~5.3.


\section{Libreria matematica}

Nella libreria memorizzata nella tabella \key{math} ci sono le funzioni
trigonometriche \fn{sin}, \fn{cos}, \fn{tan}, \fn{asin} eccetera --- che come
di consueto lavorano in radianti --- le funzioni esponenziali \fn{exp},
\fn{log}, \fn{log10}, quelle di arrotondamento \fn{ceil}, \fn{floor}, e quelle
per la generazione pseudocasuale di numeri come \fn{random}, e \fn{randomseed}.
Oltre a funzioni, la tabella include campi numerici come la costante \( \pi \).

Un esempio introduttivo è questo dove nella funzione \fn{one} viene definita una
funzione locale:
\sourcecode{file = [[code/e8-libstd.lua]], select = [[one]]}


\section{Libreria stringhe}

La libreria per le stringhe è memorizzata nella tabella \key{string} ed è una
delle più utili. Con essa si possono formattare campi e compiere operazioni di
ricerca e sostituzione. In effetti, in Lua non è infrequente elaborare grandi
porzioni di testo.


\subsection{Funzione \fn{string.format}}
\label{secFondStringFormat}

La funzione più semplice è quella di formattazione \fn{string.format}. Essa
restituisce una stringa prodotta con il formato definito dal primo argomento
dei dati forniti dal secondo argomento in poi.

Il formato è esso stesso specificato come una stringa contenente dei segnaposto
creati con il simbolo percentuale e uno specificatore di tipo. Per esempio
\verb|"%d"| indica il formato relativo a un numero intero, dove \key{d} sta per
digit mentre \verb|"%f"| indica il segnaposto per un numero decimale con \key{f}
che sta per float.

I campi formato derivano da quelli della funzione classica di libreria
\fn{printf} del C. Di seguito un esempio di codice:
\sourcecode{
    file = [[code/e8-libstd.lua]],
    select = [[fmt]],
    run = true,
}

Come avete potuto notare nel codice, è anche possibile fornire un ulteriore
specifica di dettaglio tra il carattere \key{\%} e lo specificatore di tipo, per
esempio per indicare il numero delle cifre decimali di arrotondamento.

Per elaborare il testo si utilizza di solito una libreria per le espressioni
regolari. Lua mette a disposizione alcune funzioni di sostituzione e
\emph{pattern matching} meno complete dell'implementazione dello standard
POSIX per le espressioni regolari ma molto spesso più semplici da utilizzare.

Esistono due strumenti di base, il primo è il \emph{pattern} e il secondo è la
\emph{capture}.


\subsection{Pattern}
\label{secFondPattern}

Il pattern è una stringa che può contenere campi chiamati \emph{classi} simili
a quelli per la funzione di formato visti in precedenza, che stavolta però si
riferiscono al singolo carattere, e questa differenza è essenziale.

La funzione di base che accetta pattern è \fn{string.match} che restituisce
la prima corrispondenza trovata in una stringa primo argomento corrispondente
al pattern dato come secondo argomento.

Per esempio, possiamo ricercare in un numero di tre cifre intere all'interno di
un testo con il pattern \verb|"%d%d%d"|:
\sourcecode{
    file = [[code/e8-libstd.lua]],
    select = [[pattern_one]],
    run = true,
}

Le classi carattere possibili sono le seguenti:
\begin{compactdescription}
  \item[\key{.}] un carattere qualsiasi;
  \item[\key{\%a}] una lettera;
  \item[\key{\%c}] un carattere di controllo;
  \item[\key{\%d}] una cifra;
  \item[\key{\%l}] una lettera minuscola;
  \item[\key{\%u}] una lettera maiuscola;
  \item[\key{\%p}] un carattere di interpunzione;
  \item[\key{\%s}] un carattere spazio;
  \item[\key{\%w}] un carattere alfanumerico;
  \item[\key{\%x}] un carattere esadecimale;
  \item[\key{\%z}] il carattere rappresentato con il codice 0.
\end{compactdescription}

Le classi ammettono quattro modificatori per esprimere le ripetizioni dei
caratteri:
\begin{compactdescription}
  \item[\key{+}] indica 1 o più ripetizioni;
  \item[\key{*}] indica 0 o più ripetizioni;
  \item[\key{-}] come \key{*} ma nella sequenza più breve;
  \item[\key{?}] indica 0 o 1 occorrenza;
\end{compactdescription}

Esempio:
\sourcecode{
    file = [[code/e8-libstd.lua]],
    select = [[pattern_two]],
    run = true,
}

\section{Capture}
\label{secFondCapture}

Il pattern può essere arricchito non solo per trovare corrispondenze ma anche
per restituirne parti componenti. Questa funzionalità viene chiamata
\emph{capture} e consiste semplicemente nel racchiudere tra parentesi tonde le
classi.

Per esempio per estrarre l'anno di una data nel formato \key{dd/mm/yyyy}
possiamo usare il pattern con la capture seguente \verb|"%d%d/%d%d/(%d%d%d%d)"|:
\sourcecode{
    file = [[code/e8-libstd.lua]],
    select = [[capture_one]],
    run = true,
}

Più capture ci sono nel pattern e altrettanti argomenti multipli di uscita
saranno restituiti:
\sourcecode{file = [[code/e8-libstd.lua]],
   select = [[capture_two]],
   run = true,
}


\subsection{La funzione \fn{string.gsub}}
\label{secFondGsub}

Abbiamo appena cominciato a scoprire le funzionalità dedicate al testo
disponibili nella libreria standard di Lua precaricata a runtime.

Diamo solo un altro sguardo alla libreria presentando la funzione
\fn{string.gsub}. Il suo nome sta per \emph{global substitution}, ovvero
la sostituzione di tutte le occorrenze in un testo.

Intanto per individuare le occorrenze è naturale pensare di utilizzare un
pattern e che sia possibile utilizzare le capture nel testo di sostituzione,
per esempio:
\sourcecode{
    file = [[code/e8-libstd.lua]],
    select = [[gsub]],
    run = true,
}

Il primo argomento è la stringa da ricercare, il secondo è il pattern e il
terzo è il testo di sostituzione dell'occorrenza, ma può anche essere una
tabella dove le chiavi corrispondenti al pattern saranno sostituite con
i rispettivi valori, oppure anche una funzione che riceverà le catture e
calcolerà il testo da sostituire.

Una funzione quindi assai flessibile. Mi viene in mente questo esercizio:
moltiplicare per 12 tutti gli interi in una stringa, ed ecco il codice:
\sourcecode{
    file = [[code/e8-libstd.lua]],
    select = [[gsubfn]],
    run = true,
}

A questo punto degli esempi avrete certamente capito che \fn{gsub} restituisce
anche il numero delle sostituzioni effettuate.

Tutte queste funzioni restituiscono una stringa costruita ex-novo e non
modificano la stringa originale di ricerca. In Lua le stringhe sono dati
immutabili.



\section{Esercizi}

\begin{Exercise}[label=libstd-01]
Qual è la differenza tra i campi di formato della funzione \fn{string.format} e
le classi dei pattern? Quali le somiglianze?
\end{Exercise}

\begin{Exercise}[label=libstd-02]
Stampare una data nel formato \key{dd/mm/yyyy} a partire dagli interi contenuti
nelle variabili \key{d}, \key{m} e \key{y}.
\end{Exercise}

\begin{Exercise}[label=libstd-03]
Cosa restituisce l'esecuzione della seguente funzione?
\sourcecode{
    file = [[code/e8-libstd.lua]],
    select = [[esercizio3]],
}
Quale pattern corrisponde a un numero decimale la cui parte intera può essere
omessa?
\end{Exercise}

\begin{Exercise}[label=libstd-04]
Come estrarre dal nome di un file l'estensione?
\end{Exercise}

\begin{Exercise}[label=libstd-05]
Come eliminare da un testo eventuali caratteri spazio iniziali e/o finali?
\end{Exercise}

\begin{Exercise}[label=libstd-06]
Il pattern \verb|"(%d+)/(%d+)/(%d+)"| è adatto per catturare giorno, mese e
anno di una data presente in una stringa nel formato \key{dd/mm/yyyy}?
\end{Exercise}

\begin{Exercise}[label=libstd-07]
Creare un esempio che utilizzi \fn{string.gsub} con una funzione in sintassi
anonima a due argomenti corrispondenti a due capture nel pattern di ricerca.
\end{Exercise}


% end of file



\chapter{Iteratori}

Gli iteratori offrono un approccio semplice e unificato per scorrere uno alla
volta gli elementi di una collezione di dati. Vi dedicheremo un capitolo proprio
perché sono molto utili per scrivere codice efficiente ed elegante.

Il linguaggio Lua prevede il ciclo d'iterazione \emph{generic for} che
introduce la nuova parola chiave \key{in} secondo questa sintassi:
\lines
for <lista variabili> in iterator_function() do
-- codice
end
\endlines
\sourcecode{from_lines()}

Le tabelle di Lua sono oggetti che possono essere impiegati per rappresentare
degli array oppure dei dizionari. In entrambe i casi Lua mette a disposizione
due iteratori predefiniti rispettivamente tramite le funzioni \fn{ipairs} e
\fn{pairs}.

Queste funzioni restituiscono un iteratore conforme alle specifiche del generic
for. Mentre impareremo più tardi a scrivere iteratori personalizzati,
dedicheremo le prossime due sezioni a questi importanti iteratori predefiniti
per le tabelle.


\section{Funzione \fn{ipairs}}

La funzione \fn{ipairs} restituisce un iteratore che a ogni ciclo genera due
valori: l'indice dell'array e il valore corrispondente. L'iterazione comincia
dalla posizione 1 e termina quando il valore è \key{nil}:
\lines
-- una tabella array
local t = {45, 56, 89, 12, 0, 2, -98}

-- iterazione tabella come array
for i, v in ipairs(t) do
    print(i, v)
end
\endlines
\sourcecode{from_lines()}

Il ciclo con \fn{ipairs} è equivalente a questo codice:
\lines
-- una tabella array
local t = {45, 56, 89, 12, 0, 2, -98}
do
    local i, v = 1, t[1]
    while v do
        print(i, v)
        i = i + 1
        v = t[i]
    end
end
\endlines
\sourcecode{from_lines()}

Se non interessa il valore dell'indice è buona norma dare al nome di variabile
corridspondente un segno di underscore \key{\_} che in Lua è un identificatore
valido. Per esempio:
\lines
-- una tabella array
local t = {45, 56, 89, 12, 0, 2, -98}
local sum = 0
for _, elem in ipairs(t) do
    sum = sum + elem
end
print(sum)
\endlines
\sourcecode{from_lines()}

Se non vogliamo incorrere in errori è molto importante ricordarsi che con
\fn{ipairs} verranno restituiti i valori in ordine di posizione da 1 in poi e
fino a che non verrà trovato un valore \key{nil}. Se desiderassimo raggiungere
tutte le coppie chiave/valore dovremo far ricorso all'iteratore \fn{pairs}
che tratteremo nella prossima sezione.


\section{Funzione \fn{pairs}}
\label{secFondPairsIterator}

Questa funzione primitiva di Lua considera la tabella come un dizionario
pertanto l'iteratore restituirà in un ordine casuale tutte le coppie chiave
valore contenute nella tabella stessa.

Una tabella con indici a salti verrà iterata parzialmente da \fn{ipairs} ma
completamente da \fn{pairs} al prezzo di perdere l'ordinamento:
\sourcecode{
    from_file [[code/e9-iter.lua]]
    :select_lines [[uno]]
    :add_output{delim_run=  true}
}

Il comportamento di questi due iteratori potrebbe lasciare perplessi ma è
coerente con le caratteristiche di Lua.


\section{Generic \key{for}}

Come può essere implementato un iteratore in Lua? Per iterare è necessario
mantenere alcune informazioni essenziali chiamate \emph{stato} dell'iteratore.
Per esempio l'indice a cui siamo arrivati nell'iterazione di una tabella/array
e la tabella stessa.

Perchè non utilizzare la closure per memorizzare lo stato dell'iteratore?

Abbiamo incontrato le closure nella sezione \ref{secClosure}. Proviamo a
scrivere il codice per iterare una tabella:
\sourcecode{
    from_file [[code/e9-iter.lua]]
    :select_lines [[due]]
}

Funziona, molto semplicemente. Non è stato necessario introdurre nessun nuovo
elemento al linguaggio. L'iteratore è solamente una questione d'implementazione
che tra l'altro in questo caso ricrea l'iteratore \fn{ipairs} visto poco fa.

Infatti, la funzione \fn{iter\_fn} mantiene nella closure lo stato
dell'iteratore --- l'indice \key{i} e la tabella \key{t} --- e restituisce uno
dopo l'altro gli elementi della tabella. Il ciclo \key{while} infinito,
s'interrompe quando il valore è \key{nil}.

Tuttavia, data l'importanza degli iteratori, Lua introduce il nuovo costrutto
chiamato \emph{generic for} che si aspetta una funzione proprio come la
\fn{iter} del codice precedente. E in effetti funziona:
\sourcecode{
    from_file [[code/e9-iter.lua]]
    :select_lines [[tre]]
}

Riassumendo, la costruzione di un iteratore in Lua si basa sulla creazione di
una funzione che restituisce uno alla volta gli elementi dell'insieme nella
sequenza desiderata. Una volta costruito l'iteratore, questo potrà essere
impiegato in un ciclo generic for.

Se per esempio si volesse iterare la collezione dei numeri pari compresi
nell'intervallo da 1 a 10, avendo a disposizione l'apposito iteratore
\fn{evenNum} che definiremo in seguito, potrei scrivere semplicemente:
\lines
for n in evenNum(1,10) do
    print(n)
end
\endlines
\sourcecode{from_lines()}


\section{L'esempio dei numeri pari}

Per definire un iteratore sui numeri pari di un intervallo dobbiamo creare una
funzione che restituisce a sua volta una funzione in grado di generare la
sequenza. L'iterazione termina quando giunti all'ultimo elemento, la funzione
restituirà il valore nullo \key{nil}, cosa che succede in automatico senza dover
esplicitare un'istruzione di \key{return}.

Potremo fare così: dato il numero iniziale per prima cosa potremo calcolare il
primo numero pari dell'intervallo usando l'operatore modulo \key{\%} e poi
creare la funzione di iterazione in sintassi anonima che prima incrementa di 2
la variabile di ciclo --- ed ecco perché dovremo inizialmente sottrarle la
stessa quantità --- e, se questo è inferiore all'estremo superiore
dell'intervallo ritornare indice e numero pari della sequenza. Ecco il codice
completo:
\sourcecode{
    from_file [[code/e9-iter.lua]]
    :select_lines [[iter_even]]
    :add_output{delim_run = true}
}

In questo esempio, oltre ad approfondire il concetto di iterazione basata sulla
closure di Lua, possiamo notare che il generic for effettua correttamente anche
l'assegnazione a più variabili di ciclo con le regole viste nella
sezione~\ref{secFondAssegnamento}.

Naturalmente, l'implementazione data di \fn{evenNum} è solo una fra quelle
possibili, e non è detto che non debbano essere considerate situazioni
particolari come quella in cui si passa all'iteratore un solo numero o
addirittura nessun argomento.


\section{Stateless iterator}
\label{secFondStatelessIter}

Una seconda versione del generatore di numeri pari può essere un buon esempio
di un iteratore in Lua che non necessita di una closure, per un risultato ancora
più efficiente, implementando uno \emph{stateless iterator}.

Per capire come ciò sia possibile dobbiamo conoscere nel dettaglio come
funziona il generic for in Lua; dopo la parola chiave \key{in} esso si aspetta
altri due parametri oltre alla funzione da chiamare a ogni ciclo: una variabile
che rappresenta lo stato invariante e la variabile di controllo.
\lines
for <vars> in <iter_fn>, <state>, <ctl_var> do
    ...
end
\endlines
\sourcecode{from_lines()}

La funzione d'iterazione verrà chiamata a ogni ciclo con due argomenti: lo stato
invariante e la variabile di controllo e ci si aspetta che restituisca uno o più
dati di ciclo. Quando questi valori saranno \key{nil} il ciclo a termine.

Nel seguente codice la funzione \fn{evenNum} provvede a restituire i tre
parametri necessari: la funzione \fn{next\_even} come iteratore, lo stato
invariante, ovvero il numero a cui la sequenza dovrà fermarsi e la variabile di
controllo che è proprio il valore nella sequenza dei numeri pari.
\sourcecode{
    from_file [[code/e9-iter.lua]]
    :select_lines [[generic_for]]
}

Con gli iteratori abbiamo terminato l'esplorazione di base del linguaggio Lua.
Questi otto capitoli sono sufficienti per scrivere programmi utili perché
trattano tutti gli argomenti essenziali, il prossimo invece, tratterà del
paradigma della programmazione a oggetti in Lua.


\section{Esercizi}

\begin{Exercise}[label=iter-01]
Dopo aver definito una tabella con chiavi e valori stampare le singole coppie
tramite l'iteratore predefinito \fn{pairs}.
\end{Exercise}

\begin{Exercise}[label=iter-02]
Scrivere una funzione che accetta un array (una tabella con indici interi in
sequenza) di stringhe e utilizzando la funzione di libreria \fn{string.upper}
restituisca un nuovo array con il testo trasformato in maiuscolo. Per esempio da
\code{\{"abc", "def", "ghi"\}} a \code{\{"ABC", "DEF", "GHI"\}}).
\end{Exercise}

\begin{Exercise}[label=iter-03]
Scrivere la funzione/closure per l'iteratore che restituisce la sequenza dei
quadrati dei numeri naturali a partire da 1 fino a un valore dato.
\end{Exercise}

\begin{Exercise}[label=iter-04]
Scrivere la versione \emph{stateless} dell'iteratore dell'esercizio precedente.
\end{Exercise}

\begin{Exercise}[label=iter-05]
Scrivere la versione \emph{stateless} dell'iteratore \fn{ipairs}. È possibile
implementarlo in modo che la funzione d'iterazione restituisca per il ciclo
generic for solamente l'elemento della tabella e non anche l'indice?
\end{Exercise}


% end of file



\chapter{Programmazione a oggetti in Lua}
\label{iiChOop}

In sintesi, il paradigma della \emph{Object Oriented Programming} \textsc{oop},
si basa sulla creazione di entità indipendenti chiamate \emph{oggetti}. Ciascun
oggetto incorpora sia dati che funzioni, che prendono il nome di \emph{metodi}.

Ogni oggetto è un'istanza che fa parte di una stessa famiglia chiamata
\emph{classe}, una sorta di prototipo che rappresenta un ``tipo di dati''. Le
classi possono essere ricavate da altre classi con il meccanismo
dell'\emph{ereditarietà} per specializzarne il comportamento.

Per instanziare un oggetto di una classe si utilizza un metodo speciale chiamato
\emph{costruttore}, che valida gli eventuali dati in ingresso e instanzia in
memoria l'oggetto.

In questo capitolo ritroveremo tutti questi concetti del paradigma della
programmazione a oggetti dal punto di vista di Lua. Con essi la struttura del
problema non è più pensata in termini di funzioni, ma attraverso la
rappresentazione dei suoi elementi concettuali in classi e le loro relazioni di
ereditarietà.

Negli ultimi anni, la programmazione a oggetti è stata ripensata tant'è che nei
linguaggi di nuova generazione come Go e Rust non è inclusa nel modo classico.
Ciò non toglie che essa possa rendere più intuitiva la programmazione in Lua, in
special modo per chi sviluppa applicazioni per \LuaTeX.


\section{Il minimalismo di Lua}

Il linguaggio Lua non è progettato con gli stessi obiettivi di Java o del C++, i
due linguaggi più noti per la programmazione a oggetti, non possiede un
controllo preventivo del tipo, non prevede il concetto sintattico di classe, non
offre alcun meccanismo per dichiarare come privati campi e metodi, e lascia al
programmatore più di un modo per implementare la \textsc{oop}.

Tuttavia Lua basandosi sulle tabelle offre il pieno supporto ai principi del
paradigma a oggetti senza perdere le caratteristiche minimali del linguaggio.


\section{Una classe Rettangolo}
\label{secRectOop}

Costruiremo una classe per rappresentare un rettangolo. Si tratta di un ente
geometrico definito da due soli parametri \emph{larghezza} e \emph{altezza}, e
dotato di proprietà come l'area e il perimetro, che implementeremo come metodi.

Un primo tentativo potrebbe essere questo:
\sourcecode{
    file = [[code/e10-oop.lua]],
    select = [[primo]],
}

Ci accorgiamo presto che questo tentativo è difettoso in quanto non rispetta
l'indipendenza degli oggetti rispetto al loro nome. Infatti il prossimo test
fallisce:
\sourcecode{
    file = [[code/e10-oop.lua]],
    select = [[due]],
}

Il problema sta nel fatto che nel metodo \fn{area} compare il particolare
riferimento alla tabella \key{Rettangolo}. La soluzione non può che essere
l'introduzione del riferimento dell'oggetto come parametro esplicito nel metodo
stesso, ed è la stessa utilizzata anche dagli altri linguaggi di programmazione
che supportano gli oggetti. Vedremo tra poco come, allo stesso modo dei
linguaggi \textsc{oop}, anche in Lua si possa nascondere il riferimento
con la colon notation.

Secondo quest'idea dovremo riscrivere il metodo \fn{area} in questo modo (in
Lua il riferimento implicito all'oggetto deve chiamarsi \key{self} pertanto
abituiamoci fin dall'inizio a questa convenzione):
\sourcecode{
    file = [[code/e10-oop.lua]],
    select = [[tre]],
}

Fino a ora abbiamo costruito l'oggetto sfruttando le caratteristiche della
tabella e la particolarità che consente di assegnare una funzione a una
variabile. Questo punto è importante: le chiavi nella tabella dell'oggetto
possono contenere sia funzioni/metodi sia valori/campi perciò sovrascrivere una
chiave nell'intenzione di introdurre un nuovo campo/metodo porta a errori.


\section{Colon notation}

Da questo momento entra in scena l'operatore \key{:}\luas{:} --- che chiameremo
\emph{colon notation}. L'operatore \key{:} fa in modo che le seguenti due
espressioni siano perfettamente equivalenti anche se le rende differenti dal
punto di vista concettuale agli occhi del programmatore:
\sourcecode{
    file = [[code/e10-oop.lua]],
    select = [[colon_notation]],
}

Questo operatore è il primo nuovo elemento che Lua introduce per facilitare la
programmazione orientata agli oggetti. Se si accede a un metodo memorizzato in
una tabella con l'operatore due punti \key{:} anziché con l'operatore
\key{.}\luas{.} allora l'interprete aggiungerà implicitamente un primo parametro
chiamandolo \key{self}\luak{self} con il riferimento alla tabella stessa,
insomma puro zucchero sintattico.


\section{Metatabelle}

Il linguaggio Lua si fonda sull'essenzialità tanto che supporta la
programmazione a oggetti utilizzando quasi esclusivamente le proprie risorse di
base senza introdurre nessun nuovo costrutto. In particolare in Lua si utilizza
la tabella, l'unica struttura dati predefinita nel linguaggio, assieme a
particolari funzionalità dette \emph{metatabelle} e \emph{metametodi}.

Il salto definitivo nella programmazione \textsc{oop} consiste nel poter
costruire una \emph{classe} senza ogni volta assemblare campi e metodi,
introducendo un qualcosa che faccia da stampo per gli oggetti.

In Lua l'unico meccanismo disponibile per compiere questo ultimo importante
passo consiste nelle \emph{metatabelle}. Esse sono normali tabelle contenenti
funzioni dai nomi prestabiliti che vengono chiamati quando si verificano
particolari eventi come l'esecuzione di un'espressione di somma tra due tabelle
con l'operatore \key{+}. Ogni tabella può essere associata a una metatabella e
questo consente di creare degli insiemi di oggetti che condividono una stessa
aritmetica.

Metatabelle e metametodi quindi, possono rendere il codice intuitivo e compatto,
sono quindi una funzionalità indipendente dalla programmazione a oggetti.

I nomi delle funzioni di una metatabella vengono detti \emph{metametodi} e
iniziano tutti con un doppio trattino basso. Per esempio nel caso della somma
sarà richiesta la funzione \fn{\_\_add} nella metatabella associata al primo
addendo o se non esiste a quella del secondo addendo.

Per assegnare una metatabella si utilizza la funzione \fn{setmetatable}. Essa ha
due argomenti tabella, la prima è l'oggetto e la seconda è la tabella con i
metametodi.


\section{Il metametodo \fn{\_\_tostring}}

Il metametodo più semplice di tutti è \fn{\_\_tostring}. Esso viene invocato se
una tabella è data come argomento alla funzione \fn{print}\luastd{print} per
ottenere il valore stringa da stampare. Se non esiste una metatabella associata
con questo metametodo verrà stampato l'indirizzo di memoria della tabella:
\sourcecode{
    file = [[code/e10-oop.lua]],
    select = [[tostring]],
}


\section{Il metametodo \key{\_\_index}}
\label{secFondMetaIndex}

Il metametodo che interessa la programmazione a oggetti in Lua è
\key{\_\_index}\luak{\_\_index}. Esso interviene quando viene chiamato un campo
di una tabella che non esiste e che normalmente restituirebbe il valore
\key{nil}. Un esempio di codice chiarirà il meccanismo:
\sourcecode{
    file = [[code/e10-oop.lua]],
    select = [[index]],
}

Tornando all'oggetto \key{Rettangolo} riscriviamo il codice creando adesso una
tabella che assume il ruolo concettuale di una vera e propria classe:
\sourcecode{
    file = [[code/e10-oop.lua]],
    select = [[classe_rect]],
}

Queste poche righe di codice racchiudono il meccanismo della creazione di una
nuova classe in Lua: abbiamo infatti assegnato a una nuova tabella \key{r} la
metatabella con funzione di classe \key{Rettangolo}. Quando viene richiesta la
stampa del campo \key{b}, poiché tale campo non esiste nella tabella vuota
\key{r} verrà ricercato il metametodo \key{\_\_index} nella metatabella
associata che è appunto la tabella \key{Rettangolo}.

A questo punto il metametodo restituisce semplicemente la tabella
\key{Rettangolo} stessa e questo fa sì che tutti i campi e i metodi siano
ereditati da essa per essere accessibili da \key{r}. Il campo \key{b} e il
metodo \fn{area} del nuovo oggetto \key{r} sono in realtà quelli definiti nella
tabella \key{Rettangolo}.

Se volessimo creare invece un rettangolo assegnando direttamente la dimensione
dei lati dovremo semplicemente crearli in \key{r} con i nomi previsti dalla
classe: \key{b} e \key{h}. Il metodo \fn{area} sarà ancora caricato dalla
tabella \key{Rettangolo} ma i campi numerici con le nuove misure dei lati
saranno quelli interni dell'oggetto \key{r} e non quelli della metatabella
poiché semplicemente esistono in \key{r}.

Questa costruzione funziona ma può essere migliorata con l'introduzione del
costrutture come vedremo meglio in seguito. L'oggetto \key{Rettangolo} apparirà
sempre più concettualmente simile a una classe.


\section{Il costruttore}
\label{secFondCostruttore}

Proponendoci ancora la rappresentazione del concetto di rettangolo, completiamo
il quadro introducendo il costruttore della classe. Il lavoro che dovrà
svolgere questa speciale funzione sarà quello di inizializzare i campi
argomento in una delle tante modalità possibili e una volta effettuato il
controllo di validità degli argomenti.

Il codice completo della classe \key{Rettangolo} è il seguente:
\sourcecode{
    file = [[code/e10-oop.lua]],
    select = [[newrect]],
}

Il costruttore \fn{new} accetta una tabella come argomento, altrimenti ne crea
una vuota, controlla gli eventuali parametri geometrici, assegna la metatabella
e restituisce l'oggetto. Alla funzione arriva il riferimento implicito a
\key{Rettangolo} grazie alla colon notation, per cui \key{self} è un riferimento
alla stessa tabella del di quello della variabile \key{Rettangolo}.

Quando viene passata una tabella con uno o due campi sulle misure dei lati al
costruttore, l'oggetto disporrà delle misure come valori interni effettivi,
cioè dei parametri indipendenti che costituiscono il suo stato interno. Lo
sviluppatore può fare anche una diversa scelta, quella per esempio di
considerare la tabella argomento del costruttore come semplice struttura di
chiavi/valori da sottoporre al controllo di validità e poi includere in una
nuova tabella con modalità e nomi che riguardano solo l'implementazione interna
della classe.

Essendo il costruttore una normale funzione Lua, i suoi argomenti possono essere
più di uno e di diverso tipo, mentre la classe può averne anche più di uno con
nomi differenti.


\section{Questa volta un cerchio}

Per capire ancor meglio i dettagli e renderci conto di come funziona il
meccanismo automatico delle metatabelle, costruiamo una classe \key{Cerchio} che
annoveri fra i suoi metodi una funzione che modifichi il valore del raggio
aggiungendovi una misura:
\sourcecode{
    file = [[code/e10-oop.lua]],
    select = [[cerchio]],
}

Nella sezione del codice utente viene dapprima creato un cerchio senza fornire
alcun valore per il raggio. Ciò significa che quando stampiamo il valore del
raggio con la successiva istruzione otteniamo \( 0 \) che è il valore di default
del raggio dell'oggetto \key{Cerchio}, per effetto della chiamata a
\key{\_\_index}\luak{\_\_index} della metatabella.

Fino a questo momento la tabella dell'oggetto \key{o} non contiene alcun campo
\key{radius}. Cosa succede allora quando è chiamato il comando
\key{o:addToRadius(12.342)}?

Il metodo \fn{addToRadius} contiene una sola espressione. Come da regola viene
prima valutata la parte a destra ovvero \key{self.radius + v}. Il primo termine
assume il valore previsto in \key{Cerchio} --- quindi zero --- grazie al
metametodo, e successivamente il risultato della somma uguale all'argomento
\key{v} è memorizzato nel campo \key{o.radius} che viene creato effettivamente
solo in quel momento.


\section{Ereditarietà}

Il concetto di ereditarietà nella programmazione a oggetti consiste nella
possibilità di derivare una classe da un'altra per specializzarne il
comportamento.

L'operazione di derivazione incorpora automaticamente nella sottoclasse tutti i
campi e i metodi della classe base. Dopodiché si implementano o si modificano i
metodi della classe derivata creando una gerarchia di oggetti.

In Lua l'operazione di derivazione consiste molto semplicemente nel creare un
oggetto con il costruttore della classe base e modificarne o aggiungerne metodi
o campi.

Vediamo un esempio semplice dove si rappresenta il concetto generale di una
persona che svolge attività sportiva e da questo, il concetto di una persona
che svolge uno specifico sport:
\sourcecode{
    file = [[code/e10-oop.lua]],
    select = [[sportivo]],
}

Continua tutto a funzionare per via della ricerca effettuata dal metametodo
\key{\_\_index} che funziona a ritroso fino alla classe base.


\section{Esercizi}

\begin{Exercise}[label=oop-01]
Aggiungere alla classe \key{Rettangolo} riportata nel testo il metametodo
\fn{\_\_tostring} che stampi in console il rettangolo dalle dimensioni
corrispondenti ad altezza e larghezza usando i caratteri \key{+} per gli
spigoli e i caratteri \key{-} e \key{|} per disegnare i lati. Utilizzare le
funzioni di libreria \fn{string.rep} e \fn{string.format}.
\end{Exercise}

\begin{Exercise}[label=oop-02]
Creare una classe corrispondente al concetto di numero complesso e implementare
le quattro operazioni aritmetiche tramite metametodi (riferimento matematico
\href{http://it.wikipedia.org/wiki/Numero_complesso#Operazioni_con_i_numeri_complessi}{qui}).
Aggiungere anche il metodo \fn{\_\_tostring} per stampare il numero complesso e
poter controllare i risultati di operazioni di test.
\end{Exercise}

\begin{Exercise}[label=oop-03]
Ideare una classe base e una classe derivata dandone un'implementazione.
\end{Exercise}

% end of file


\part{Applicazioni Lua in \LuaTeX}
\label{partApp}



\chapter{Un registro delle compilazioni}
\label{iiichRegistro}

Siamo giunti al primo capitolo di taglio applicativo. Ci occuperemo di creare
in Lua un sistema di registrazione delle compilazioni che si concretizza in un
file posizionato nella directory principale del progetto di documento.

Non è possibile reperire dati generali sulla compilazione eseguendo codice
durante la compilazione stessa. Dovremo farlo con un tool esterno. Solo in
questo modo è possibile ottenere il tempo di compilazione totale o accertarsi
del nome del file sorgente. Tuttavia la registrazione in fase di compilazione è
utile per imparare con Lua a lavorare con i file e sperimentare un processo di
sviluppo iterativo molto più rapido che con \TeX.

Chiameremo questo registro \code{history.log} verso cui invieremo una linea di
testo informativa per ciscuna compilazione che il compositore completi
correttamente.


\section{Scrivere sul registro}

Iniziamo a implementare la gestione automatica del registro dalla funzionalità
chiave: la scrittura su file. Ci rivolgeremo alla libreria standard di Lua,
modulo \code{io}\footnote{La completa documentazione della libreria \code{io} si
trova alla pagina \url{https://www.lua.org/manual/5.3/manual.html\#6.8}.}:
\begin{lines}
-- write a line of text in a file
local function append(filename, line)
    local f = io.open(filename, "a+")
    f:write(line.."\n")
    f:close()
end
\end{lines}

La funzione \fn{io.open} restituisce un riferimento a un oggetto che rappresenta
un canale di input/output del sistema operativo verso un file i cui metodi vanno
chiamati in colon notation (vedi alla sezione~\ref{secRectOop}). \fn{open}
accetta il nome e la modalità di apertura del file. Lo specificatore \verb|"a+"|
indica di aprire il file in \emph{append} senza distruggerne il contenuto
preesistente. In caso il file non esista ne verrà creato uno nuovo perciò in
ogni caso otterremo il file aperto in scrittura.

La funzione \fn{append} può essere provata in un sorgente plain \LuaTeX{}
compilabile come in questo:
\sourcecode{file = [[app-registro/01.tex]]}


\section{Dati di compilazione}

Costruite le fondamenta stabiliamo quali dati inserire nel registro: l'utente 
mano a mano che il lavoro sul documento procede eseguirà senza dubbio delle
compilazioni intermedie perciò registreremo le seguenti informazioni di base:
\begin{compactenumerate}
\item il nome del file sorgente,
\item la dimensione del file sorgente,
\item data e ora della compilazione,
\item il tempo di esecuzione della composizione,
\item il nome del motore di composizione.
\end{compactenumerate}

Per formare la linea di registro, al posto della concatenazione di stringhe
useremo quella di una tabella di stringhe, più efficiente. Riempiremo la tabella
un dato alla volta in un blocco \cs{directlua} posizionato immediatamente prima
del termine del sorgente. Come prima istruzione invece, inseriremo il blocco di
codice Lua che definirà alcuni parametri come il nome del file del registro e il
separatore di campo testuale, e la funzione \fn{append}:
\sourcecode{file = [[app-registro/02.tex]]}


\subsection{La variabile \code{jobname}}

Il nome del file sorgente non è detto che sia contenuto nella variabile
\code{jobname} che possiamo trovare anche nella macro omonima oppure nel campo
omonimo della tabella \code{tex}.

Il campo o la macro conterrà il nome del sorgente soltanto se l'utente non ha
valorizzato l'opzione \code{--jobname} nel comando di compilazione con un altro
nome o se non è stato modificato il campo \code{tex.jobname}.

Infatti, se provassimo a compilare il listato \code{02.tex} con il seguente
comando
\begin{Verbatim}[numbers=none,xleftmargin=0pt]
$ luatex --jobname=abc 02
\end{Verbatim}
otterremo un errore che blocca la compilazione. Nel secondo \cs{directlua} la
variabile locale \key{jobname} conterrebbe infatti la stringa \code{abc.tex}
che è un file che non esiste. Di conseguenza la funzione \fn{lfs.attributes}
restituisce \key{nil} che ovviamente non può essere indicizzato con la chiave
\key{size}.

La tabella \code{lfs} contiene la libreria Lua File System disponibile per i
sistemi operativi più diffusi, con alcune funzionalità sui file che non troviamo
di base in Lua perché l'interprete ha regole stringenti di portabilità. La
troviamo in \LuaTeX{} già compilata staticamente. La documentazione si trova
all'indirizzo web
\url{https://keplerproject.github.io/luafilesystem/manual.html}.

Se la funzione \fn{lfs.attributes} non restituisce la tabella con i dati del
file significa dunque che nel comando di compilazione è stata attivata l'opzione
\code{--jobname} per assegnare un diverso nome al file PDF di uscita.
Modifichiamo dunque il codice eseguito in chiusura in questo senso:
\begin{lines}
local tline = {}
local jobname = tex.jobname
local attr = lfs.attributes(jobname..".tex")
if attr then
    tline[1] = jobname
    tline[2] = attr.size
else
    tline[1] = "unknow"
    tline[2] = "unknow"    
end
tline[3] = os.date()
tline[4] = os.clock() - register.start
tline[5] = status.luatex_engine
register:append(tline)
\end{lines}

Anche l'estensione \code{.tex} è un problema perché se fosse diversa ancora una
volta il codice non funzionerebbe, per esempio se il file sorgente avesse
estensione \code{.TEX}.


\subsection{Gli altri dati}

Come sappiamo dal capitolo~\ref{iiChLibstd} la tabella \key{os} appartiene alla
libreria standard di Lua. La documentazione delle funzioni \fn{os.date} e
\fn{os.clock} si trova quindi nel reference ufficiale del linguaggio alla pagina
\url{https://www.lua.org/manual/5.3/manual.html#6.9}.

Infine, maggiori informazioni sulla tabella \key{status} da cui leggiamo il nome
del motore di composizione, possono essere reperite nel manuale di \LuaTeX{}.

Quanto alla misura del tempo di compilazione che registriamo, più precisamente
si tratta del tempo che trascorre tra i due momenti in cui eseguiamo la funzione
\fn{os.clock} perciò non comprende il tempo iniziale di avvio e caricamento del
formato, né i tempi finali di chiusura come la composizione del materiale non
ancora emesso sulla pagina.


\section{Creare il modulo e il pacchetto}

Per consentire l'utilizzo del registro da qualsiasi sorgente occorre spostare il
codice Lua in un file esterno di \emph{modulo} e scrivere un secondo file di
\emph{pacchetto} per fornire un'interfaccia utente conforme al formato, per
esempio per \LaTeX{}.

Questa separazione non è solo utile al processo di sviluppo, facilitando
l'accesso alle funzionalità da parte degli utenti indipendentemente dal formato.

Allo stato attuale dello sviluppo del codice, il modulo Lua è il seguente:
\sourcecode{file = [[app-registro/mod-history.lua]]}

Tutte le funzionalità sono incluse in un unica tabella Lua chiamata
\code{register}. Dall'esterno, caricando il modulo con la funzione \fn{require}
potremo assegnarne il riferimento a una variabile locale o globale a seconda
delle necessità.

L'uso della colon notation per chiamare le funzioni non fa parte di
un'implementazione a oggetti ma è solo un semplice modo per poter disporre di un
riferimento alla tabella contenitore con la chiave \key{self}. Ciò rende pulito
il codice ed elimina la necessità di creare una closure con un piccolo
incremento delle prestazioni per includere il riferimento alla tabella stessa.

Il sorgente in plain \LuaTeX{} si semplifica in questo:
\sourcecode{file = [[app-registro/03.tex]]}

Per compilarlo correttamente il file Lua del modulo chiamato
\code{mod-history.lua} deve essere nel percorso di ricerca. La soluzione più
semplice è copiarlo in locale nella directory del sorgente.

A questo punto è facile creare un pacchetto per \LaTeX{}. Non ci sarà alcuna
macro utente ma un aggiunta al codice di chiusura del documento con la macro
\cs{AtEndDocument}. Ecco il codice:
\sourcecode{file = [[app-registro/historylog.sty]]}

Basta caricare il pacchetto e tutte le compilazioni del sorgente verranno
registrate come per questo file sorgente compilabile con Lua\LaTeX{}.
Compilatelo più volte e controllate poi il file locale \file{history.log}:
\sourcecode{file = [[app-registro/doc.tex]]}


\section{Sviluppo}

Abbiamo costruito passo passo un modulo Lua per inserire in un registro alcune
informazioni sulle compilazioni dei sorgenti. Le funzionalità sono minime ma si
tratta di un buon punto di partenza per dare risposta a nuove domande: quale
struttura dovremo dare al modulo in modo che sia semplice aggiungere nuove
funzionalità, come permettere all'utente di modificare le informazioni
registrabili.

A poco a poco prende forma e si precisa la struttura di un \emph{framework}.

%Per esempio, se si vuole che la data sia trascritta nel formato ISO
%\code{yyyy-mm-dd} oppure che la dimensione del file riporti unità di misura
%che si adattano a seconda della magnitudo del valore.

% end of file




% Costruzione con i nodi del triangolo di Tartaglia.

\chapter{Tartaglia}
\label{iichTartaglia}

In questo capitolo utilizzeremo le funzionalità di creazione dei nodi per
realizzare il triangolo di Tartaglia. I nodi sono oggetti tipografici pronti da
posizionare sulla pagina che vengono prodotti da \TeX{} in uno degli ultimi
momenti del processo. \LuaTeX{} è in grado di creare i nodi anche per altra via,
tramite la creazione di oggetti in Lua.

I nodi sono di vario tipo, per esempio dimensioni elastiche o glifi, e vengono
assemblati in liste anche molto complesse perché strutturate in scatole
verticali od orizzontali. Infine, senza entrare troppo nel dettaglio tecnico, i
nodi non rientrano nella gestione automatica della memoria operata dal Garbage
Collector di Lua, mentre occorre programmare correttamente i riferimenti
al nodo seguente e precedente se si vuole creale una lista.

Per questi motivi conviene manipolare le liste dei nodi attraverso le funzioni
che si trovano nella tabella \code{node}, anziché operare su di essi
direttamente.

Con i nodi ogni dettaglio deve essere costruito. Si opera come un tipografo
che lavora con strumenti elementari, assemblando un pezzo alla volta.


\section{Costruzione del triangolo di Tartaglia}

Il Triangolo di Tartaglia%
\footnote{\url{https://it.wikipedia.org/wiki/Triangolo_di_Tartaglia}.} fino
all'ottavo livello è qui rappresentato:
\begin{center}
\includegraphics{image/tart-triangolo.pdf}
\end{center}

Ogni nuovo livello è costruito sul precedente sommando i due interi che
sovrastano un dato elemento in modo che il primo e l'ultimo numero siano sempre
1. La proprietà più nota del triangolo è che il livello \( n \) è formato dai
coefficienti binomiali \( (a + b)^n \).

Procediamo con il codice. Chiudete la guida e cercate una vostra implementazione
in \LuaTeX{} stampando i numeri dei livelli fino all'ottavo su una stessa linea
separandoli con uno spazio. Confrontate poi la soluzione fornita nel prossimo
listato.

Al solito stiamo procedendo per gradi. Otteniamo prima il codice che produce i
numeri del triangolo e poi il codice che costruisce la lista dei nodi da
inserire in una scatola che lo compone sulla pagina.

Ecco la mia versione, che utilizza una sola tabella che cresce livello dopo
livello:
\sourcecode{file = [[app-tartaglia/01.tex]]}


\section{Nodi}

I numeri del triangolo vanno posizionati in punti ben precisi. Otterremo
la disposizione geometrica regolando le distanze tra i gruppi di cifre in modo
che il centro del testo che rappresenta un numero sia sull'asse verticale
opportuno per il livello, e disponendo una scatola sull'altra per l'insieme dei
livelli. I passi che svolgeremo in plain \LuaTeX{} sono i seguenti:
\begin{compactenumerate}
\item comporre una cifra,
\item comporre un numero in una lista,
\item comporre più numeri in linea congiungendo le liste con un nodo spazio,
\item assemblare una scatola sull'altra.
\end{compactenumerate}


\subsection{Un numero}

Per costruire un nodo di tipo glifo, un singolo elemento nella collezione
di un font, si utilizza la funzione \fn{node.new}. Poi è obbligatorio
valorizzare almeno il codice del carattere per il campo \code{char} e il numero
del font per il campo \code{font}.

Ottenuto l'oggetto nodo possiamo comporlo sulla pagina con la funzione
\fn{node.write}. Per esempio se volessimo stampare un 8:
\sourcecode{file = [[app-tartaglia/02.tex]]}

La macro \cs{leavevmode} è importante perché è vietato inserire un oggetto glifo
in modo verticale, ed è questa la modalità in cui si trova \TeX{} all'inizio.


\subsection{Dal numero alla lista}

Dal numero, con l'operatore modulo a 10 è possibile ricavare in un ciclo le
cifre componenti la rappresentazione decimale a partire da quella meno
significativa. Con la singola cifra si crea il glifo e lo si concatena in una
lista tramite la funzione \fn{node.insert\_before} che funziona anche per
aggiungere un elemento in testa.

Curando il caso particolare dello zero, questo è quello che fa la funzione
\fn{digit} nel seguente sorgente compilabile:
\sourcecode{file = [[app-tartaglia/03.tex]]}

Il metodo \fn{node.write} accetta un nodo e non una lista. Ma se il nodo
argomento ha un riferimento a un nodo nel campo \code{next}, verrà composta
tutta la catena. Questi riferimenti sono stati inseriti per noi da
\fn{node.insert\_before}.

Dunque la lista costruita è la sequenza di glifi delle cifre del numero 12090.
Dobbiamo ricordarci che il testo composto che ne risulta è tipografia minimale,
perché la lista non è stata modificata per inserire legature, kerning, o punti
di cesura a fin di riga. Con i nodi siamo noi gli artigiani digitali.


\subsection{Numeri e spazi}

Un nodo \emph{glue} distanzia il nodo precedente da quello successivo, in
orizzontale. Può essere elastico in estensione o riduzione, oppure rigido.
Per il nostro scopo dovremo calcolare la distanza rigida tra i nodi in modo
tale che i centri dei due numeri successivi a un livello del triangolo distino
sempre lo stesso valore.

Occorre quindi misurare la larghezza del numero composto. La cosa più semplice è
inserire la lista dei nodi glifo in una scatola orizzontale per poi misurarla
con la funzione \fn{node.dimensions} che ne restituisce larghezza, altezza sulla
lina base e profondità dalla linea base.

La dimensione tra due scatole dovrà essere la differenza tra la distanza assiale
con la semisomma delle larghezze delle scatole adiacenti.

Il passo successivo è quindi aggiungere la funzione \fn{pack\_level} per
costruire la scatola orizzontale contentente la lista di un livello intero del
triangolo di Tartaglia a partire dalla tabella di interi.

Una scatola orizzontale di una lista si costruisce passando il nodo capolista
alla funzione \fn{node.hpack}. nel codice ho modificato la funzione \fn{digits}
affinché restituisca due paramentri: il nodo della scatola orizzontale che
contiene la lista dei nodi glifi e la larghezza della scatola stessa.

Il sorgente compilabile diventa questo:
\sourcecode{file = [[app-tartaglia/04.tex]]}

Se esageriamo con la grandezza dei numeri allora si sovrapporrano. Questo
succede se la lunghezza elastica è negativa poiché la distanza assiale di 24pt
(vedi la variabile locale \key{a}) è troppo piccola. A questo livello del
codice, l'utente deve controllare che non ci siano sovrapposizioni specie
all'ultima riga del triangolo dove si trovano i numeri più grandi.


\subsection{Sovrapposizione scatole}

Il passo finale è quello di sovrapporre le scatole orizzontali a formare il
triangolo. Basterà impacchettare le scatole in una scatola verticale con la
funzione \fn{node.vpack} dopo aver costruito la lista di scatole e spazi
verticali.

Dobbiamo prima modificare la funzione \fn{pack\_level} perché restituisca una
scatola orizzontale per il materiale di un intero livello. Fino a ora la lista
poteva anche essere una sequenza di scatole e nodi glue perché la immettevamo in
modo orizzontale. Adesso invece immettiamo le righe del triangolo in ambiente
verticale impacchettando la lista delle righe separate con un nodo di lunghezza
con la funzione \fn{node.vpack}.

Queste sono le nuove funzioni: \fn{next\_level} calcola la riga del triangolo
rispetto a quella precedente, e \fn{tartaglia} genera il triangolo fino al
livello specificato in una scatola verticale:
\begin{lines}
-- filename: app-tartaglia/05.tex
local function next_level(t, row)
    t[row+1] = 1
    for e = row, 2, -1 do
        t[e] = t[e] + t[e-1]
    end
end
local function tartaglia(level)
    assert(type(level)=="number")
    local il = tex.sp "8.5pt"
    local head, last
    local t = {}
    for l = 0, level do
        next_level(t, l)
        local hbox = pack_level(t)
        if head then
            local g = node.new("glue")
            g.width = il
            head, last = node.insert_after(head, last, g)
            head, last = node.insert_after(head, last, hbox)
        else
            head, last = node.insert_after(head, last, hbox)
        end
    end
    local vbox = node.vpack(head)
    return vbox
end
\end{lines}

Il risultato è questo:
\begin{center}
\includegraphics{image/tart-left}
\end{center}


\subsection{Opzione allineamento}

Per allineare al centro o a destra le linee possiamo introdurre dei nodi
lunghezza nella scatola orizzontale della singola riga. Conviene inserire questi
nodi con la funzione \fn{pack\_level} perché se lo facessimo all'esterno dovremo
poi reimpacchettare la lista in un'ulteriore scatola orizzontale per poterle poi
sovrapporre in ambiente verticale.

A questo scopo aggiungeremo il parametro \code{align}. Per dimostrare quanto si
riveli utile la dinamicità del linguaggio Lua, considereremo tre diversi
possibili gruppi di valori di tipo diverso per il parametro:
\begin{compactitemize}
\item \code{align} vale \key{nil}, per esempio perché nella chiamata di funzione
principale il valore non è stato assegnato: l'allineamento assume il valore di
default di triangolo centrato;
\item \code{align} è una stringa, allora potrà valere \code{"left"},
\code{"center"} o \code{"right"};
\item \code{align} è un numero come frazione di spazio che deve rimanere a
sinistra del primo elemento in alto. Quindi 0 è la stessa cosa dell'allineamento
\code{"left"}, \( 1/2 \) di \code{"center"} e 1 di \code{"right"}. Sono
possibili valori negativi o maggiori di 1.
\end{compactitemize}

Il trucco per implementare facilmente l'aggiunta delle lunghezze di allineamento
davanti e in coda alla lista degli elementi di un livello del triangolo è quello
di conoscere quanto vale lo spazio \( w \) da distribuire opportunamente.

Al livello \( r \), se \( a \) è la distanza assiale tra numeri adiacenti e \(
r_\mathrm{tot} \) è il numero totale dei livelli, allora \( w \) vale:
\[
w = k_\mathrm{left} + k_\mathrm{right} = a\left(r - r_\mathrm{tot}\right)
\]

L'esattezza matematica dell'espressione è dovuta al fatto che il numero in testa
e in coda per ogni riga del triangolo è sempre 1.

Il listato completo della funzione \fn{pack\_level} è il seguente:
\begin{lines}
-- filename: app-tartaglia/06.tex
local function pack_level(t, diff_level, k_left, k_right)
    local a = tex.sp "24pt"
    local w1
    local head, last
    if diff_level == 0 then
        k_left, k_right = nil, nil
    end
    if k_left then
        head = node.new("glue")
        head.width = a*diff_level*k_left
        last = head
    end
    for _, n in ipairs(t) do
        local hbox, w2 = pack_digits(n)
        if w1 then
            local g = node.new("glue")
            g.width = a - (w1+w2)/2
            w1 = w2
            head, last = node.insert_after(head, last, g)
            head, last = node.insert_after(head, last, hbox)
        else
            w1 = w2
            head, last = node.insert_after(head, last, hbox)
        end
    end
    if k_right then
        local g = node.new("glue")
        g.width = a*diff_level*k_right
        head, last = node.insert_after(head, last, g)
    end
    return node.hpack(head)
end
\end{lines}

La funzione tiene conto delle situazioni in cui non è necessario inserire il
distanziamento di allineamento su un lato, cioè quando la lunghezza vale zero
oppure quando la linea da impacchettare è l'ultima, riga che non ha mai
necessità di essere traslata.

Tuttavia, non viene fatto affidamento sulla \emph{direzione di composizione} per
allineare a destra o a sinistra e viene inserito sempre lo spazio. Se
l'allineamento fosse a sinistra le scatole sarebbero allineate a sinistra dal
compositore che dispone gli oggetti in modo orizzontale da sinistra a destra. Ma
se la direzione fosse impostata al contrario l'effetto sarebbe l'opposto.

Per questo nel codice viene inserita la lunghezza a destra nonostante
l'allineamento a sinistra. I parametri \( k_\mathrm{left} \) e \(
k_\mathrm{right} \) sono definiti dalla funzione principale \fn{tartaglia} a
seconda del parametro \code{align}. Il listato è il seguente:
\begin{lines}
-- filename: app-tartaglia/06.tex
local function tartaglia(level, align)
    assert(type(level)=="number")
    local k_left, k_right; if align then
        if type(align) == "string" then
            if align == "center" then
                k_left, k_right = 0.5, 0.5
            elseif align == "right" then
                k_right = 1
            elseif align == "left" then
                k_left = 1
            end
        elseif type(align) == "number" then
            if align == 0 then
                k_right = 1
            elseif align == 1 then
                k_left = 1
            else
                k_left, k_right = align, 1 - align
            end
        else
            error("Unexpected alignment value")
        end
    else
        k_left, k_right = 0.5, 0.5
    end
    local il = tex.sp "8.5pt"
    local head, last
    local t = {}
    for l = 0, level do
        next_level(t, l)
        local hbox = pack_level(t, level - l, k_left, k_right)
        if head then
            local g = node.new("glue")
            g.width = il
            head, last = node.insert_after(head, last, g)
            head, last = node.insert_after(head, last, hbox)
        else
            head, last = node.insert_after(head, last, hbox)
        end
    end
    local vbox = node.vpack(head)
    return vbox
end
\end{lines}

Domanda: se avessimo avuto numeri diversi da 1 come primo e ultimo elemento, se
ritenuto necessario, quali modifiche occorrebbe considerare nel codice?


\subsection{Verifica grafica degli allineamenti con TikZ}

Per controllare visivamente gli allineamenti verticali nel triangolo di
Tartaglia è possibile sovrapporre linee verticali sottili di passo \( a \) al
disegno. Realizzare questo disegno è in realtà molto semplice poiché una volto
costruito il nodo del contenitore, la scatola può essere assegnata direttamente
a uno dei registri tramite indicizzazione della tabella \code{tex.box}:
\begin{lines}
#[tex]
% !TeX program = LuaTeX
\newbox\tartbox % nuovo registro
\directlua{
    ... definizioni come prima
    tex.box.tartbox = tartaglia(8)
}
\box\tartbox
\bye
\end{lines}

La macro \cs{box} è una primitiva di \TeX{}. Quello che fa è comporre sulla
pagina il contenuto del box indicato dalla control sequence che lo segue e poi
svuotarlo.

A questo punto è facile separare la costruzione del triangolo dal suo impiego, e
un esempio è proprio far espandere la scatola in una macro \cs{node} del
pacchetto grafico TikZ:
\begin{lines}
#[tex]
% !TeX program = LuaTeX
% filename: app-tartaglia/07.tex
\input tikz.tex
\newbox\tartbox
% ...
\directlua{
    ... definizioni come prima
    tex.box.tartbox = tartaglia(8)
}
\tikzpicture
\foreach \x in {-96,-72,...,96} {
\draw[blue] (\x pt,68pt) -- (\x pt,-68pt);
}
\foreach \x in {-84,-60,...,84} {
\draw[red] (\x pt,68pt) -- (\x pt,-68pt);
}
\node at (0, 0) {\box\tartbox};
\endtikzpicture
\bye
\end{lines}

Le rette rosse e quelle blue distano 24pt una dall'altra. Sono posizionate a
partire dall'ascissa zero poiché TikZ inserirà la scatola usando il suo punto
centrale nell'origine del sistema di riferimento.

Le linee rosse corrispondono alle posizioni dei numeri sui livelli dispari,
e quelle blu a quelle dei livelli pari. Il risultato è:
\begin{center}
\includegraphics{image/tart-tikz.pdf}
\end{center}


\subsection{Regolazione automatica della distanza}

Sappiamo che la distanza \( a \) tra i centri di due numeri consecutivi su una
linea del triangolo è fissa. Se non fosse abbastanza grande i due numeri si
sovrapporrebbero e a quel punto l'utente dovrebbe reimpostarne il valore nel
sorgente e ricompilare.

Possiamo invece rendere l'operazione automatica e in diversi modi. Per esempio,
potremo intendere che il triangolo venga costruito in base a una distanza minima
tra un numero e l'altro, oppure impostando la distanza \( a \) come fissa per
incrementarla in caso di sovrapposizioni.

Una prima soluzione è costruire la scatola con il triangolo solo alla fine,
salvare cioè le scatole orizzontali dei numeri in una tabella e nel frattempo
calcolarne il valore minimo di \( a \); costruire poi la scatola contenitore del
triangolo distanziando opportunamente i nodi.

Una seconda strada è quella di impacchettare il triangolo come fatto fino a ora
negli esempi, per poi eventualmente scorrere la lista dei nodi per incrementare
la distanza tra i centri. Questa seconda strategia è quella che seguirò per
mostrare come una lista di nodi già costruita possa essere utilmente
modificata.


% \subsection{Visitare la lista dei nodi}

% missing material

\subsection{Modificare la distanza}

Useremo la funzione \fn{pack\_level} per creare la lista di un livello del
triangolo di Tartaglia già scritta in precedenza, e una nuova funzione
\fn{add\_distance} per modificare la distanza tra i centri.

Alcune informazioni utili tratte dal manuale di \LuaTeX{} che è bene richiamare:
la lista contenuta in un nodo scatola orizzontale o verticale inizia con il nodo
contenuto nel campo \code{head}. Nella lista il nodo successivo può essere
ricavato leggendo il campo \code{next} del nodo attuale. Il campo numerico
\code{id} indica il tipo di nodo, per esempio 12 individua un nodo \code{glue} e
0 un nodo \code{hbox}.

Aggiungere una distanza fissa è molto semplice. Sappiamo che il nodo \code{hbox}
è l'alternanza tra scatole orizzontali e nodi \code{glue}, perciò se \code{hbox}
è il mnome di variabile che contiene la scatola, il riferimento
\code{hbox.head.next} punta al primo nodo distanza. In un ciclo \key{while}
scorrere i soli nodi \code{glue} significa saltare un nodo e quindi preparare il
prossimo riferimento con il campo \code{glue.next.next}:
\begin{lines}
-- filename app-tartaglia/08.tex
local function add_distance(hbox, d)
    assert(hbox and hbox.id == 0)
    local glue = hbox.head.next
    while glue do
        assert(glue.id == 12)
        glue.width = glue.width + d
        glue = glue.next.next
    end
end
\end{lines}

Nel triangolo dobbiamo tener in conto tuttavia degli eventuali nodi di
spaziatura iniziale e finale per l'allineamento orizzontale del triangolo. In
questi spazi l'incremento della distanza è proporzionale alla differenza tra il
numero dei livelli totali con il numero di quello corrente. Dovremo adattare la
funzione \fn{add\_distance} per modificare le lunghezze non solo dei nodi
intermedi tra un numero e l'altro, ma anche per gli eventuali spazi di
allineamento citati, in questo modo:
\begin{lines}
-- filename app-tartaglia/09.tex
local
function add_distance(hbox, d, k_left, k_right, level, totlevel)
    assert(hbox and hbox.id == 0)
    local glue = hbox.head
    local tdist = d*(totlevel - level)
    if k_left then
        assert(glue.id == 12)
        glue.width = glue.width + tdist*k_left
        glue = glue.next
    end
    glue = glue.next
    for _= 1, level do
        assert(glue.id == 12)
        glue.width = glue.width + d
        glue = glue.next.next
    end
    if k_right then
        assert(glue.id == 12)
        glue.width = glue.width +  tdist*k_right
    end
end
\end{lines}

Non si utilizza il ciclo \code{while} ma un ciclo \code{for} che itera tante
volte quanti sono gli spazi nel livello, ovvero il numero del livello a
cominciare da 1 perché il livello 0 non ha spazi intermedi. In questo modo è più
semplice per il codice gestire il puntatore al nodo invece che passare da un
nodo al successivo.

La versione finale contenuta nel file \code{app-tartaglia/09.tex}, conta 160
linee di codice Lua in grado di generare il triangolo di Tartaglia con il numero
di livelli richiesti, diverse opzioni di allineamento orizzontale, e con la
capacità di mantenere la distanza assiale tra i numeri di 24pt più l'eventuale
distanza perché ci siano almeno 3pt tra due numeri consecutivi.

Domanda: se e come cambiereste il codice per considerare la simmetria dei numeri
sulla riga di uno stesso livello del triangolo di Tartaglia?


\section{Riepilogo}

La tecnologia dei nodi consente di comporre oggetti tipografici di complessità
arbitraria. Seguendo vari passi di sviluppo, in questo capitolo abbiamo
costruito con essa il triangolo di Tartaglia, un esempio applicativo
interessante proprio per poter implementare nuove funzionalità come quella di
rendere variabile l'allineamento o lasciare che sia il codice ad aggiustare la
distanza tra i numeri se necessario.

Rimane da esplorare la gestione in Lua delle opzioni e dei parametri perché
l'utente possa modificare l'aspetto del triangolo. Considero questo tema
separato da quello dei nodi di \LuaTeX{}. Per questo motivo l'esercitazione
può dirsi conclusa.


% end of file


% \input{section/III-03-cerchio-pdfliteral}

% \input{section/III-04-dati-esterni}

\appendix

% 


Come tutti i linguaggi di scripting non occorre compilare il codice Lua. Basta
scrivere il programma in un file di testo --- preferibilmente assegnandogli
l'estensione \key{.lua} --- e lanciare il comando d'esecuzione in un terminale
\tcmd{\$ lua nomefile}.

Per scrivere il codice vi consiglio di utilizzare l'editor
\href{http://www.scintilla.org/SciTE.html}{SciTE} perché è già predisposto per
eseguire programmi Lua ed è programmabile in Lua. Tra l'altro l'output del
programma viene visualizzato nella \emph{output tab}. Il tasto funzione F5
esegue il programma visualizzato nella scheda attiva in quel momento e il tasto
F8 attiva/disattiva la finestra di output, che è in pratica un terminale
incorporato nell'editor.

Prepariamoci a scrivere il primo programma in Lua, ovviamente il fatidico
\href{http://en.wikipedia.org/wiki/%22Hello,_world!%22_program"}{Hello world!},
salvando il seguente codice in un file chiamato \key{primo.lua} ed eseguiamolo
direttamente in SciTE premendo il tasto funzione F5 o al terminale con il
comando \tcmd{\$ lua primo.lua}:
\lines
print("Hello world!")
\endlines
\sourcecode{from_lines():add_output()}

In ambiente Linux o Mac OS~X, come per tutti gli altri linguaggi di scripting,
possiamo inserire come prima riga del file a cui sono stati assegnati i permessi
di esecuzione, la
\href{http://en.wikipedia.org/wiki/Shebang_%28Unix%29}{Sha-Bang}. In questo modo
basterà semplicemente scriverne il nome nel terminale, e in questo caso è
conveniente omettere l'estensione. Ecco il programma con la Sha-Bang:
\lines
#!/usr/bin/env lua
print("Hello world!")
\endlines
\sourcecode{from_lines()}

Lua è in grado di passare al programma gli argomenti che l'utente digita nel
terminale separandoli da spazi, tramite la tabella array chiamata \key{arg} che
conterrà all'indice 0 il nome dello script, e ai successivi indici i vari
argomenti utente come tipo stringa. Questa proprietà assieme alla tecnica della
sha-bang, dà all'utente la possibilità di scrivere programmi in Lua come se
fossero comandi da terminale, almeno per gli ambienti di classe *nix.

\chapter{Preparazione di un interprete Lua}

Poiché l'interprete Lua non interferisce con \LuaTeX{} e occupa una quantità di
memoria davvero piccola, è consigliabile installarlo per poter usare il modo
interattivo REPL e per sperimentare più comodamente.

Per seguire queste chiaccherate andrà benissimo una versione di Lua uguale o
superiore alla 5.1. Tuttavia è preferibile installare la versione che
corrisponde a quella inclusa in \LuaTeX{}, che viene riportata anche nel suo
manuale.

L'interprete Lua è scritto in Ansi~C e per questo è disponibile praticamente
per tutte le piattaforme, eventualmente compilandone i sorgenti.

Per verificare l'installazione è possibile dare il seguente comando da
terminale per stampare la versione dell'interprete:
\begin{Verbatim}
$ lua -v
Lua 5.3.3  Copyright (C) 1994-2016 Lua.org, PUC-Rio
\end{Verbatim}



\subsection{Linux}

Per i sistemi Linux fate ricorso al package manager della vostra distribuzione,
per esempio per le distribuzioni Debian e derivate come Ubuntu, è sufficiente
dare il comando:
\begin{Verbatim}
$ sudo apt-get install lua
\end{Verbatim}



\subsection{Windows}

Per Windows fate riferimento alla pagina
\href{http://luabinaries.sourceforge.net/index.html}{LuaBinaries}. In caso di
problemi potete scaricare un file autoinstallante dal progetto
\href{https://github.com/rjpcomputing/luaforwindows}{Lua for Windows} fermo
però alla versione 5.1.5 di Lua ma completo di alcune librerie utili.



\subsection{Mac OSX}

Per Mac OSX andate alla pagine \href{http://rudix.org/packages/lua.html}{Rudix}
e individuate il package adatto per la versione del vostro Mac.



\section{Come eseguire il codice}

Operativamente, ci sono molti modi per eseguire codice Lua. La modalità usuale
è installarne l'interprete e questo capitolo riporta le procedure per i tre
principali sistemi operativi.

Se si dispone di una recente distribuzione del sistema \TeX{}, per esempio
TeX~Live, uno \emph{script} in Lua può anche essere eseguito da \LuaTeX{}, per
esempio con il comando da terminale:
\begin{Verbatim}
$ luatex --luaonly nomefile
\end{Verbatim}

Nella parte~\ref{partLuaTeX} vedremo come sia possibile inserire il codice Lua
all'interno di un sorgente \texttt{.tex}. Per esempio, poiché sto componendo
questa guida in \LuaLaTeX{} --- che altro non è che \LuaTeX{} con il formato
\LaTeX{} precaricato --- scrivendo il codice:
\begin{Verbatim}
\directlua{tex.print(lua.version)}
\end{Verbatim}
posso dirvi che la versione di Lua inclusa è \directlua{tex.print(lua.version)}.


\subsection{Modalità interattiva}

Aprite un terminale, o console in Windows, e digitate semplicemente il comando
\texttt{lua}. Entreremo nella modalità interattiva dove potremo
digitare istruzioni una alla volta. Per uscire tornando al prompt, digitate la
funzione \texttt{os.exit()} seguita da un bel tasto invio.

In modalità interattiva possiamo comodamente controllare quale tipo ha un
valore, utilizzando la funzione \texttt{type()}. Essa restituisce una stringa
che corrisponde al nome del tipo di un'espressione, stringa che possiamo a sua
volta stampare con la funzione \texttt{print()} in console. Digitiamo
\footnote{Il doppio trattino inserisce un commento di riga.}:
\lines
a = 123
print(type(123))    --> stampa "number"
print(type(a))      --> stampa "number"
print(type("123"))  --> stampa "string"
\endlines
\sourcecode{from_lines()}


% end of file



\backmatter



\chapter{Note finali}

I miei ringraziamenti vanno a Claudio Beccari per aver scritto la classe
\class{guidatematica.cls} con cui è stata composta la guida e per avermi dato
risposte come sempre precise e complete alle mie domande.

Ringrazio Gianluca Pignalberi che mi ha proposto un uso avanzato del pacchetto
\pack{siunitx} nella composizione della tabella proposta nel secondo tutorial.

Ci sono sempre buone occasioni per imparare, speriamo che i maestri non manchino
mai.

% end of file


% bibliografia
\input{section/Z-90-biblio.tex}
\end{document}

