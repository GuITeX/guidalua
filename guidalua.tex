% !TeX program = LuaLaTeX--shell-escape

\documentclass[b5paper,11pt,openany]{guidatematica}
\ProvidesFile{guidalua.tex}[2021/03/28 v0.4.1 Guida al linguaggio Lua per LuaTeX]
\GetFileInfo{guidalua.tex}
\setmonofont[Scale=0.82]{Fira Mono}
\usepackage{sourcecode}

\fvset{
    fontsize=\small,
    numbers=left,
    xleftmargin=6mm,
    numbersep=2.25mm
}

\usepackage[
    colorlinks=true,
    linkcolor=blue,
    citecolor = blue,
    urlcolor = blue,
    pdftitle={Guida al linguaggio Lua per LuaTeX},
    pdfauthor={Roberto Giacomelli},
    pdfsubject={Guida tematica del GuIT},
    pdfkeywords={Lua, LuaTeX, guida, programming}
]{hyperref}
\usepackage{exercise}
\renewcounter{Exercise}[chapter]
\renewcommand{\ExerciseHeader}{\noindent\textsc{esercizio \ExerciseHeaderNB}}

\usepackage{tcolorbox}
\tcbuselibrary{skins}
\tcbset{
    sharp corners=all,
    colback=verdeguit!12!white,
    colframe=verdeguit!70!white,
    bicolor,
    colbacklower=white,
    boxrule=0.4pt,
    leftrule=3.2pt,
    fonttitle=\bfseries,
    fontupper=\small,
    fontlower=\small
}
\tcbuselibrary{documentation}
\tcbset{doc marginnote={
    colframe=verdeguit,
    colback=verdeguit!5!white,
    width=23mm,
    enlarge left by=-6mm,
    enlarge right by=-3mm
    }
}

\usepackage{siunitx}
\sisetup{output-decimal-marker=\virgoladecimale}

\frenchspacing

\newcommand\tcmd[1]{\normalfont\texttt{#1}}
\newcommand\key[1]{\normalfont\texttt{#1}}
\newcommand\code[1]{\normalfont\texttt{#1}}
\newcommand\fn[1]{\normalfont\texttt{#1()}}

\newlength{\margindown}
\setlength{\margindown}{12.72mm}

\begin{document}
\author{Roberto Giacomelli}
\date{\filedate{} --- \fileversion}
\title{Guida al linguaggio Lua per \LuaTeX}
\subtitle{
\begin{tikzpicture}[scale=0.82]
\node at (-2.31-0.72, 0) {};
\draw[fill,verdeguit] (0,0) circle (2.2);
\draw[fill,verdeguit] (2.31, 2.31) circle (0.72);
\draw[fill, white] (0.8,0.8) circle (0.72);
\end{tikzpicture}
\vspace*{8.5ex}
}
\maketitle

\tableofcontents
\newpage


% presentazione della guida e informazioni di base per eseguire gli esempi

\chapter{Presentazione}

\section{Motivazione}

Questa guida tematica è dedicata alla programmazione in Lua all'interno dei
motori di composizione del sistema \TeX.

Con Lua è possibile compiere sia elaborazioni generiche come l'interrogazione di
basi di dati che elaborazioni tipografiche interagendo con il compositore
interno. Con Lua rispetto a \TeX, è più semplice ed efficiente effettuare
calcoli numerici o avvalersi di avanzate librerie esterne.

La nuova generazione di compositori amplia così notevolmente gli scenari
applicativi. Se da un lato è auspicabile che queste potenzialità diventino
disponibili per gli utenti finali per mezzo di moduli e pacchetti, dall'altro è
utile fornire dettagli ed esempi per implementare proprie soluzioni o per poter
scrivere nuovi moduli condividendone lo sviluppo con la community.

Sono certamente molte le cose da conoscere: un nuovo linguaggio molto diverso da
\TeX{}, numerosi dettagli sul funzionamento interno dei compositori Lua-powered,
nuovi problemi di organizzazione del codice, di bilanciamento tra Lua e \TeX,
eccetera. Per questo, ho pensato di contribuire con questa guida cercando di
presentare il quadro della crescente complessità del sistema.


\section{Piano della guida}

La guida è divisa in tre parti: la prima offre una panoramica rapida per
iniziare subito con Lua seguendo i passi di un ipotetico utente alle prese con
la risoluzione di problemi compositivi raggruppati in \emph{tutorial} (parte
\ref{partTutorial}), la seconda tratta delle basi del linguaggio Lua (parte
\ref{partFoundation}) e la terza tratta di esempi applicativi con l'uso delle
librerie interne di composizione (parte \ref{partApp}).

Tra gli argomenti ci sono:
\begin{compactitemize}
\item basi del linguaggio Lua \( \to \) dal capitolo \ref{chAssignment},
\item differenza tra motore e formato di composizione \( \to \) capitolo
\ref{iichExplain},
\item tecniche di programmazione e di rappresentazione dei dati,
\item interazione tra Lua e lo stato interno del motore di composizione.
\end{compactitemize}


\section{Origine della guida}

Per illustrare i concetti del linguaggio ho preso spunto da un breve corso su
Lua che scrissi qualche tempo fa per il blog
\href{http://parliamodi-ubuntu.blogspot.it}{Lubit Linux} di Luigi Iannoccaro
che mi propose di realizzare un progetto di divulgazione su Lua. Luigi ha
acconsentito all'utilizzo di quegli appunti per produrre questa guida tematica.


\section{Contribuire e collaborare}

Spero che i lettori vorranno contribuire al testo inviando proprie soluzioni o
nuovi contributi piccoli o grandi. Lo si può fare attraverso lo strumento che
preferite, scrivendomi un messaggio di posta elettronica all'indirizzo
\href{mailto:giaconet.mailbox@gmail.com}{\texttt{giaconet.mailbox@gmail.com}},
oppure utilizzando il
\href{https://github.com/GuITeX/guidalua}{\texttt{repository git}} dei sorgenti
della guida, eseguendo un Pull Request o aprendo una discussione premendo il
pulsante Issues.


\section{Altre risorse}

La risorsa principale per imparare Lua, a cui si rimanda per tutti gli
approfondimenti, è certamente il PIL acronimo del titolo del libro
\emph{Programming In Lua} di Roberto Ierusalimschy, principale Autore
di Lua. Questo testo non solo è completo e autorevole ma è anche ben scritto e
composto\footnote{Tra l'altro il libro ufficiale su Lua viene composto in
\LaTeX{} e commercializzato per contribuire allo sviluppo del linguaggio
stesso.}.

Quanto a \LuaTeX{} il riferimento è il suo manuale che, come quasi tutta la
documentazione nel sistema \TeX{}, può essere visualizzato a video con il
comando da terminale:
\begin{Verbatim}[numbers=none]
$ texdoc luatex
\end{Verbatim}


\section{Note di lettura}

Nei listati compilabili riportati nelle pagine della guida compare alla prima
linea la \emph{riga magica}, un commento utile per dare istruzioni all'editor
sul compilatore da usare, ma che qui informerà il lettore aiutandolo a stabilire
il contesto del codice.

Se presente nel progetto, alla seconda riga dei listati si troverà invece il
nome del file che il lettore potrà scaricare ed eseguire per i propri
esperimenti.

Nella parte \ref{partFoundation} ho cercato di non dare per scontati i concetti
fondamentali della programmazione. Ovviamente il lettore già preparato procederà
più velocemente nel prendere dimestichezza con Lua. Ho invece escluso dalla
guida \TeX{}, per esempio non spiegando come si definisce una macro utente o
come si lavora con il formato \LaTeX3. Rimando senza indugio alla copiosa
documentazione disponibile a cominciare da quella scaricabile dal sito \GuIT.

Diamo quindi inizio a questa nuova avventura lunare.

% end of file


\mainmatter*

\part{Tutorial}
\label{partTutorial}


% esercitazioni semplici con richiami ai fondamenti


\chapter{Let's start with Lua}
\label{chGo}

Per dare l'idea di Lua e di come Lua possa ridefinire gli strumenti di
composizione del sistema \TeX, inizieremo la guida con dei \emph{tutorial}, una
sorta di racconto dei progressi compiuti da un ipotetico utente \LuaTeX{}
indaffarato nel risolvere alcuni problemi con i suoi documenti: fare calcoli con
una calcolatrice o comporre una tabella ripetitiva.

Nel margine di pagina il lettore troverà i riferimenti alle sezioni di
approfondimento della guida tematica su vari argomenti della programmazione Lua
che incontreremo nei tutorial.


\section{La calcolatrice}

Una calcolatrice, una macro \cs{expr} che accetti un'espressione numerica e
ne stampi il risultato. Sarebbe davvero utile non dover più calcolare a
parte il risultato e riportarlo nel sorgente del documento \LaTeX{} con un copia
incolla o peggio a mano.

\tcbdocmarginnote{Lua in \TeX\\\(\to\) \S \ref{secLuaInLuaLaTeX}}
\tcbdocmarginnote[enlarge top initially by=9mm]{%
Variabili locali\\\(\to\) \S \ref{secLuaInLuaLaTeX}}
\tcbdocmarginnote[enlarge top initially by=18mm]{%
\code{tex.print}\\\(\to\) \S \ref{secLuaInLuaLaTeX}}
Tentiamo qualcosa di molto semplice con Lua: passare l'espressione a una
variabile e stamparla nel documento:
\begin{Verbatim}
% !TeX program = LuaLaTeX
% filename: app-start/E0-001-expr.tex
\documentclass{article}
\newcommand\expr[1]{\directlua{
    local result = #1
    tex.print(tostring(result))
}}
\begin{document}
Finalmente una calcolatrice:
\( 1.24 (7.45 + 11.21) = \expr{1.24*(7.45 + 11.21)}\)
\end{document}
\end{Verbatim}

\newcommand\expr[1]{\directlua{
    local result = #1
    tex.print(tostring(result))
}}
compilando con \LuaLaTeX{} il risultato è:
\begin{tcolorbox}
Finalmente una calcolatrice: \( 1.24 (7.45 + 11.21) = \expr{1.24*(7.45 + 11.21)}\)
\end{tcolorbox}

Un buon inizio. Nel sorgente all'interno della macro \cs{directlua} il primo
argomento è stato sostituito con l'espressione che viene poi valutata da Lua.
Nessun pacchetto aggiuntivo caricato, qualsiasi espressione numerica è lecita, e
questo solo e soltanto usando Lua incluso in \LuaTeX.

Funziona anche con i valori booleani \key{true} e \key{false} e con le
stringhe a patto di racchiuderle tra apici. Proviamo:
\begin{tcolorbox}
\verb=\( 56.9 > 78.42 \) è \texttt{\expr{ 56.9 > 78.42 }}=
\tcblower
\( 56.9 > 78.42 \) è \texttt{\expr{ 56.9 > 78.42 }}
\end{tcolorbox}

E se si volessero sostituire le rappresentazioni dei valori vero e falso? Ecco
la modifica:
\begin{Verbatim}
\newcommand\expr[1]{\directlua{
    local result = #1
    if type(result) == "boolean" then
        if result then
            result = "vero"
        else
            result = "falso"
        end
    end
    tex.print(tostring(result))
}}
\end{Verbatim}

\renewcommand\expr[1]{\directlua{
    local result = #1
    if type(result) == 'boolean' then
        if result then
            result = 'vero'
        else
            result = 'falso'
        end
    end
    tex.print(tostring(result))
}}

Un semplice test ci conforterà sulla correttezza del codice:
\begin{tcolorbox}[sidebyside]
\verb|\expr{100 == 100 and 7 > 3}|
\tcblower
\expr{100 == 100 and 7 > 3}
\end{tcolorbox}

Si, funziona. A questo punto vorrei poter regolare l'arrotondamento del
risultato numerico della calcolatrice ricorrendo a un argomento
opzionale separato dall'espressione con una virgola:
\begin{Verbatim}
\newcommand\expr[1]{\directlua{
    local result, dec = #1
    if type(result) == "boolean" then
        if result then result = "vero" else result = "falso" end
    elseif type(result) == "number" and dec then
        local perc = string.char(37)
        local fmt1 = perc..perc.."0."..perc.."df"
        local fmt2 = string.format(fmt1, dec)
        result = string.format(fmt2, result)
    end
    tex.print(tostring(result))
}}
\end{Verbatim}

\renewcommand\expr[1]{\directlua{
    local result, dec = #1
    if type(result) == 'boolean' then
        if result then result = 'vero' else result = 'falso' end
    elseif type(result) == 'number' and dec then
        local perc = string.char(37)
        local fmt1 = perc..perc..'0.'..perc..'df'
        local fmt2 = string.format(fmt1, dec)
        result = string.format(fmt2, result)
    end
    tex.print(tostring(result))
}}

Il codice stavolta perde un po' di chiarezza perché non è possibile usare
direttamente il carattere percento \code{\%} che verrebbe interpretato come
inizio di un commento, nel costruire la stringhe di formato. Ovviamente questo
non succederebbe se il codice fosse in un file separato o se fosse racchiuso in
un ambiente \key{luacode} dell'omonimo pacchetto \LaTeX.

Mettiamo alla prova la nuova versione:
\begin{tcolorbox}
\verb|\(\sqrt{2} + \sqrt{3} = \expr{ 2^0.5 + 3^0.5, 3}\)|
\tcblower
\(\sqrt{2} + \sqrt{3} = \expr{ 2^0.5 + 3^0.5, 3}\)
\end{tcolorbox}

Potremo trovare una sintassi un po' più chiara, tuttavia occupiamoci di un
problema più urgente: non è possibile usare nell'espressione funzioni
matematiche come seno e coseno. Se scrivessimo \verb=\expr{sin(1)^2 + cos(1)^2}=
non otterremo il valore unitario ma un errore.

In Lua quelle funzioni matematiche sono disponibili nella libreria \code{math},
così ci vuole poco a riassegnarle a nomi locali per far si che l'identità
trigonometrica precedente sia un'espressione valida:
\begin{Verbatim}
\newcommand\expr[1]{\directlua{
    local cos = math.cos
    local sin = math.sin
    local result, dec = #1
    if type(result) == "boolean" then
        if result then result = "vero" else result = "falso" end
    elseif type(result) == "number" and dec then
        local perc = string.char(37)
        local fmt1 = perc..perc.."0."..perc.."df"
        local fmt2 = string.format(fmt1, dec)
        result = string.format(fmt2, result)
    end
    tex.print(tostring(result))
}}
\end{Verbatim}

\renewcommand\expr[1]{\directlua{
    local cos = math.cos
    local sin = math.sin
    local result, dec = #1
    if type(result) == 'boolean' then
        if result then result = 'vero' else result = 'falso' end
    elseif type(result) == 'number' and dec then
        local perc = string.char(37)
        local fmt1 = perc..perc..'0.'..perc..'df'
        local fmt2 = string.format(fmt1, dec)
        result = string.format(fmt2, result)
    end
    tex.print(tostring(result))
}}

Una prova della calcolatrice potenziata con le funzioni matematiche ci dirà se
tutto funziona ancora bene:
\begin{tcolorbox}
\begin{Verbatim}
\(\sin^2(1/2) + \cos^2(1/2) = \expr{sin(0.5)^2 + cos(0.5)^2, 8}\).

A \( 1/3 \) l'identità è pari a \( 1 \)?
\emph{\expr{sin(1/3)^2 + cos(1/3)^2 == 1}}
\end{Verbatim}
\tcblower
\(\sin^2(1/2) + \cos^2(1/2) = \expr{sin(0.5)^2 + cos(0.5)^2, 8}\).

A \( 1/3 \) l'identità è pari a \( 1 \)?
\emph{\expr{sin(1/3)^2 + cos(1/3)^2 == 1}}
\end{tcolorbox}

Finora ogni nuova funzionalità aggiunta alla calcolatrice non ha presentato
difficoltà. Possiamo inserire o meno il risultato in ambiente matematico,
arrotondarlo al numero di decimali desiderato e usare funzioni matematiche.
Lua si sta dimostrando semplice da usare e molto efficace.

Continuamo con un nuovo passo: aggiungere costanti numeriche definite
dall'utente, una sorta di memoria della calcolatrice. Per inserire variabili
letterali in un'espressione abbiamo bisogno che il loro valore numerico sia
inizializzato ma non possiamo ricorrere alla stessa tecnica con cui abbiamo
risolto l'inserimento delle funzioni trigonometriche.

Infatti, non è possibile codificare variabili locali senza conoscerne il nome,
perché dato fornito dall'utente. Serve una sorta di metaprogrammazione
come con le macro del linguaggi compilati. Leggendo più a fondo la
documentazione di Lua, si scopre che è possibile intervenire sull'ambiente delle
variabili globali \code{\_ENV} di un \emph{chunk}, anzi, a ben vedere il
problema di rendere visibili simboli di costanti è lo stesso di quello di
rendere disponibili nell'espressione le funzioni matematiche con nomi
abbreviati. Facciamo quindi un tentativo ripartendo con il codice iniziale:
\begin{Verbatim}
\directlua{
calclib = {}
for name, object in pairs(math) do
    calclib[name] = object
end
}
\newcommand\expr[1]{\directlua{
do
    local _ENV = calclib
    ans = #1
end
tex.print(tostring(calclib.ans))
}}
\end{Verbatim}

Un codice che sfrutta una tecnica piuttosto interessante: all'interno di un
blocco viene riassegnata localmente la variabile \code{\_ENV} a \code{calclib},
una tabella in cui vi abbiamo copiato tutte le funzioni e le costanti
matematiche della libreria \code{math} di Lua. Alla riga seguente, tutti quei
nomi saranno visibili come variabili globali nel momento in cui Lua valuta
l'espressione.

Non solo, come effetto collaterale, il risultato dell'ultimo calcolo sarà
disponibile nella successiva espressione nella variabile \code{ans} come succede
con altri tool matematici! Proviamolo:

\directlua{
calclib = {}
for name, object in pairs(math) do
    calclib[name] = object
end
}
\renewcommand\expr[1]{\directlua{
do
    local _ENV = calclib
    ans = #1
end
tex.print(tostring(calclib.ans))
}}

\begin{tcolorbox}[sidebyside]
\begin{Verbatim}
\expr{pi/4}\\
\expr{cos(ans)}\\
\expr{acos(ans)}
\end{Verbatim}
\tcblower
\expr{pi/4}\\
\expr{cos(ans)}\\
\expr{acos(ans)}
\end{tcolorbox}

Molto bene. Non ci resta che aggiungere la memorizzazione delle costanti.
Useremo l'argomento opzionale della macro, le informazioni tra parentesi quadre,
per fornire all'espressione una serie di costanti nel formato chiave/valore,
ciascuna delle quali separata da una virgola. Memorizzeremo le costanti indicate
dall'utente solamente se il loro nome non è già utilizzato, per esempio da un
nome di funzione. Inoltre, specificando una stringa isolata tra le opzioni,
potremo implementare la memorizzazione del risultato così che sia riutilizzabile
nelle successive espressioni. Proviamo con:
\begin{Verbatim}
\directlua{
calclib = {}
for name, object in pairs(math) do
    calclib[name] = object
end
}
\newcommand\expr[2][]{\directlua{
do
    local error, pairs, assert, type = error, pairs, assert, type
    local _ENV = calclib
    local opt = {#1}
    local mem = opt[1]; opt[1] = nil
    for c, val in pairs(opt) do
        if _ENV[c] then
            error("Duplicated key '"..c.."' for constant name")
        else
            _ENV[c] = val
        end
    end
    ans = #2
    if mem then
        assert(type(mem) == "string")
        if _ENV[mem] then
            error("Duplicated key '"..mem.."' for memory index")
        else
            _ENV[mem] = ans
        end
    end
end
tex.print(tostring(calclib.ans))
}}
\end{Verbatim}

\renewcommand\expr[2][]{\directlua{
do
    local error, pairs, assert, type = error, pairs, assert, type
    local _ENV = calclib
    local opt = {#1}
    local mem = opt[1]; opt[1] = nil
    for c, val in pairs(opt) do
        if _ENV[c] then
            error([[Duplicated key ']]..c..[[' for constant name]])
        else
            _ENV[c] = val
        end
    end
    ans = #2
    if mem then
        assert(type(mem) == [[string]])
        if _ENV[mem] then
            error([[Duplicated key ']]..mem..[[' for memory index]])
        else
            _ENV[mem] = ans
        end
    end
end
tex.print(tostring(calclib.ans))
}}

Poiché anche i valori assegnati alle costanti sono valutati da Lua dopo la
modifica dell'environment, anche per le costanti nelle opzioni della macro è
possibile assegnare espressioni usando tutte le funzioni matematiche e tutte le
costanti precedentemente definite. Eccone un esempio:
\begin{tcolorbox}[sidebyside, righthand width=30mm]
\begin{Verbatim}
\( b = \expr[b = 10]{b} \),\\
\( h = \expr[h = 20]{h} \),\\
\( M = \expr[m = 1000]{m}\),\\
\( \sigma = M/W_\mathrm{x} =
    \expr[w=(b*h^2)/6]{m/w}\).
\end{Verbatim}
\tcblower
\( b = \expr[b = 10]{b} \),\\
\( h = \expr[h = 20]{h} \),\\
\( M = \expr[m = 1000]{m}\),\\
\( \sigma = M/W_\mathrm{x} = \expr[w=(b*h^2)/6]{m/w}\).
\end{tcolorbox}

Da questo punto in poi, possiamo presentare il valore di \( W_\mathrm{x} \)
scrivendo nel sorgente \verb=\expr{w}= che da \expr{w}.

Tutte le principali funzionalità della calcolatrice sono state implementate in
Lua e possiamo considerare terminato il tutorial. Certo non tutte. Per esempio
potremo far eseguire calcoli coinvolgendo anche registri contatori o registri
dimensioni di \TeX{}, oppure considerare che constanti dai nomi speciali come
\code{M1}, \code{M2} eccetera si comportino come registri di memoria della
calcolatrice e quindi che possano essere sovrascritti o possano funzionare da
accumulatori.





% end of file




\section{Tabella dei pesi}

Dopo la calcolatrice si presenta un'altro problema compositivo: una tabella che
riporta per vari diametri, area e peso della barra d'acciaio di lunghezza di un
metro. I diametri variano da 6 a 32 millimetri con passo di 2 per i soli numeri
pari. 

L'idea è definire una sorta di iteratore che generi i valori delle righe della
tabella una dopo l'altra. Per esempio, se volessimo una tabella con due colonne,
la prima con gli interi da 1 a 10 e la seconda con i rispettivi quadrati,
dovremo poter scrivere solamente la funzione di calcolo e il valore finale del
contatore.

La funzione potrà essere qualsiasi purché sia definita per accettare due
argomenti: il primo è il contatore e il secondo è l'array di riga. Nel nostro
caso essa dovrà memorizzare il contatore in posizione 1 e il quadrato in
posizione 2 dell'array:
\begin{Verbatim}
local function regola(counter, r)
    r[1] = counter
    r[2] = counter^2
end
local row = Row:new(regola, 10)
\end{Verbatim}

Prima di passare al codice dell'oggetto \code{Row} verifichiamo la costruzione
della tabella in \LuaLaTeX{} notando che il metodo \fn{new} accetta proprio una
funzione come primo argomento e il valore finale del contatore come secondo
argomento: in Lua le funzioni sono valori come tutti gli altri.

L'idea originale è quindi realizzata se attribuiamo alla funzione che calcola la
generica riga della tabella il ruolo di \emph{regola} di definizione dell'intera
tabella. A ben vedere potremo fare a meno del secondo parametro per restituire
direttamente l'array di riga, tuttavia in questo modo il codice risulta più
efficiente.

\directlua{
Row = {}
Row.__index = Row
function Row:new(fn_next, start, stop, step)
    if not stop then
        start, stop = 1, start
    end
    local o = {
        fn_next = fn_next,
        start = start,
        stop = stop,
        step = step or 1
    }
    setmetatable(o, self)
    return o
end

function Row:next()
    local var = self.var
    if not var then
        var = self.start
    else
        var = var + self.step
    end
    if var <= self.stop then
        self.var = var
        local fn = self.fn_next
        fn(var, self)
        return true
    end
end
}

Il corpo del documento di prova che trovate per intero nel file indicato al
solito nella seconda riga, è il seguente:
\begin{tcolorbox}[sidebyside,righthand width=21mm]
\begin{Verbatim}
% !TeX program = LuaLaTeX
% filename: app-start/E0-003-tab.tex
\documentclass{article}
% preambolo non riportato
\begin{document}
\begin{tabular}{rr}
\directlua{
local row = Row:new(
    function (c, r) r[1] = c; r[2] = c^2 end, 10
)
local bs = string.char(92)
while row:next() do
    tex.print(row[1].."&"..row[2]..bs..bs)
end
}
\end{tabular}
\end{document}
\end{Verbatim}
\tcblower
\begin{tabular}{rr}
\directlua{
local row = Row:new(function (c, r) r[1]=c; r[2] = c^2 end, 10)
local bs = string.char(92)
while row:next() do
tex.print(row[1]..[[&]]..row[2]..bs..bs)
end
}
\end{tabular}
\end{tcolorbox}

L'implementazione della classe \code{Row} consiste in poche di linee di codice:
\begin{Verbatim}
Row = {}; Row.__index = Row
function Row:new(fn_next, start, stop, step)
    if not stop then
        start, stop = 1, start
    end
    local o = {
        fn_next = fn_next,
        start = start,
        stop = stop,
        step = step or 1
    }
    setmetatable(o, self)
    return o
end
function Row:next()
    local var = self.var
    if not var then
        var = self.start
    else
        var = var + self.step
    end
    if var <= self.stop then
        self.var = var
        local fn = self.fn_next
        fn(var, self)
        return true
    end
end
\end{Verbatim}

Possiamo considerarla come un generatore di righe. Tramite la regola
rappresentata dalla funzione e tramite i tre parametri di iterazione, ogni tipo
di dati potrà essere rappresentato in una forma tabellare. Per esempio, potremo
elencare alcuni file con la relativa dimensione in byte:
\begin{tcolorbox}[sidebyside,righthand width=30mm]
\begin{Verbatim}
% !TeX program = LuaLaTeX
% filename: app-start/E0-004-tab.tex
\documentclass{article}
% preambolo non riportato
\begin{document}
\begin{tabular}{lr}
\directlua{
local data = {
  {"files.txt"    ,  4710},
  {"lib.lua"      ,   330},
  {"liteparse.txt",  6451},
  {"parse.lua"    , 50995},
  {"path.txt"     ,  2150},
}
local function regola(counter, row)
    row[1] = data[counter][1]
    row[2] = data[counter][2]
end
local row = Row:new(regola, 5)
local bs = string.char(92)
while row:next() do
tex.print(row[1].."&"..row[2]..bs..bs)
end
}
\end{tabular}
\end{document}
\end{Verbatim}
\tcblower
\begin{tabular}{lr}
\directlua{
local data = {
  {'files.txt'    ,  4710},
  {'lib.lua'      ,   330},
  {'liteparse.txt',  6451},
  {'parse.lua'    , 50995},
  {'path.txt'     ,  2150},
}
local function regola(c, row)
    row[1] = data[c][1]
    row[2] = data[c][2]
end
local row = Row:new(regola, 5)
local bs = string.char(92)
while row:next() do
tex.print(row[1]..'&'..row[2]..bs..bs)
end
}
\end{tabular}
\end{tcolorbox}

A questo punto potremo sperimentare il punto di vista per colonne anziché quello
per righe mantenendone la stessa generalità, ma prefisco proseguire migliorando
il modo in cui generare il codice dell'ambiente \key{tabular}.

Invece di una concatenazione di stringhe, potrebbe essere più conveniente
specificare una sorta di template con segnaposto come la stringa:
\begin{Verbatim}
template = [[\textbf{<1>} & <2>\\]]
\end{Verbatim}
dove il numero tra parentesi acute, i segni di minore e maggiore, indica
l'indice di riga così come definito nella funzione \fn{regola}.

Per far questo, aggiungerei alla classe \code{Row} un metodo d'iterazione che a
ogni passo dia la stringa risultato, da usare in un ciclo \key{for}. La relativa
implementazione potrebbe essere la seguente:
\begin{Verbatim}
function Row:iter_template(tmpl)
    local iter_fn = function(row, i)
        if not i then
            i = row.start
        else
            i = i + row.step
        end
        if i <= self.stop then
            self.fn_next(i, self)
            local s = tmpl:gsub("<(%d+)>", function (s)
                local n = tonumber(s)
                return row[n]
            end)
            return i, s
        end
    end
    return iter_fn, self, nil
end
\end{Verbatim}

Al di la di considerazioni di efficienza legate all'uso della funzione di
libreria \fn{gsub}, l'iteratore in effetti funziona come dimostra il seguente
codice per \LuaLaTeX{} estratto dal file \file{app-start/E0-005-tab.tex}
allegato alla guida, dove abbiamo inserito la macro \cs{noexpand} per bloccare
l'espansione delle control sequence\footnote{Certo mi ostino ancora a non
utilizzare il pacchetto \pack{luacode}.}:
\begin{Verbatim}
\begin{tabular}{lr}
\directlua{
local tmpl = [[\noexpand\textbf{<1>} & <2>\noexpand\\]]
for _, s in row:iter_template(tmpl) do
   tex.print(s)
end
}
\end{tabular}
\end{Verbatim}

Torniamo alla nostra tabella dei pesi. La funzione generatrice e il template di
riga saranno le seguenti:
\begin{Verbatim}
local function fn(diam, row)
    row[1] = diam
    local area = math.pi * (diam/20)^2
    local fmt = string.char(37)..'0.3f'
    row[2] = fmt:format(area)
    row[3] = fmt:format(0.785*area)
end
row = Row:new(fn, 6, 32, 2)
tmpl = [[\noexpand\textbf{<1>} & <2> & <3>\noexpand\\]]
\end{Verbatim}

\directlua{
function Row:iter_template(tmpl)
    local iter_fn = function(row, i)
        if not i then
            i = row.start
            row.counter = 0
        else
            i = i + row.step
        end
        if i <= self.stop then
            row.counter = row.counter + 1
            self.fn_next(i, self)
            local perc = string.char(37)
            local s = tmpl:gsub('<('..perc..'d+)>', function (s)
                local n = tonumber(s)
                return assert(row[n])
            end)
            return i, s
        end
    end
    return iter_fn, self, nil
end

local function fn(diam, row)
    row[1] = diam
    local area = math.pi * (diam/20)^2
    local fmt = string.char(37)..'0.3f'
    row[2] = fmt:format(area)
    row[3] = fmt:format(0.785*area)
end
row = Row:new(fn, 6, 32, 2)
tmpl = [[\noexpand\textbf{<1>} & <2> & <3>\noexpand\\]]
}
e il risultato è:
\begin{center}
\begin{tabular}{lrr}
\directlua{
for _, s in row:iter_template(tmpl) do
   tex.print(s)
end
}
\end{tabular}
\end{center}

Miglioriamo ora il codice della funzione generatrice aggiungendo il metodo
\code{insert} alla class \code{Row}. Il nuovo metodo avrà tre argomenti:
il numero di colonna \key{col}, il valore da inserire nella cella  \key{val} e
infine il valore opzionale di arrotondamento numerico \key{prec}. Eccone una sua
implementazione molto semplice:
\begin{Verbatim}
function Row:insert(col, val, prec)
    if prec then
        local p = string.char(37)
        local fmt = string.format(p..p.."0."..p.."df", prec)
        val = string.format(fmt, val)
    end
    self[col] = val
    return self
end
\end{Verbatim}

In questa fase è lecito che nel codice non sia contemplata la gestione degli
errori. Da notare che il nuovo metodo restituisce l'oggetto stesso così che
possiamo concatenare più inserimenti di cella.

Ecco come la funzione di generazione può semplicarsi:
\begin{Verbatim}
local function fn(diam, row)
    local area = math.pi * (diam/20)^2
    row:insert(1, diam)
       :insert(2, area, 3)
       :insert(3, 0.785*area, 3)
end
\end{Verbatim}

Molto bene. Prima di passare a curare l'aspetto della tabella, un ulteriore
miglioramento ci consente di evitare di dover controllare l'espansione quando
inseriamo il testo del template di riga, grazie al comando \cs{detokenize}.

Introduciamo una nuova macro \cs{makerow} che ha come argomento il template in
grado di generare tutte le righe della tabella:
\begin{Verbatim}
\newcommand{\makerow}[1]{\directlua{
local tmpl = [=[\detokenize{#1}]=]
for _, s in row:iter_template(tmpl) do
   tex.print(s)
end
}}
\end{Verbatim}

Per non introdurre un secondo argomento, nell'istanziare l'oggetto della classe
\code{Row} dovremo solo ricordarci di chiamare la variabile come \key{row}.

Mettiamo subito al lavoro la nuova macro:
\begin{Verbatim}
\begin{tabular}{lrr}
\makerow{\textbf{<1>} & <2> & <3>\\}
\end{tabular}
\end{Verbatim}

Molto semplice: si definisce prima la funzione generatrice e con essa si
costruisce l'oggetto \key{Row}, poi si scrive il codice \LaTeX{} passando alla
macro \cs{makerow} il template con i segnaposto.

Molto importante è far corrispondere i numeri di cella nei segnaposti del
template con i valori che la funzione di riga inserisce nella varie posizioni.

L'ultimo passo è migliorare l'aspetto della tabella. Con il pacchetto
\pack{booktabs} aggiungiamo un'intestazione e un filetto ogni tre righe per
facilitare la lettura dei dati. Dobbiamo così modificare la funzione di riga per
derminare se il numero di riga è multiplo di tre --- senza usare l'operatore
modulo \key{\%} di Lua perché non ci troviamo in un file esterno:
\begin{Verbatim}
\directlua{
local function fn(diam, row)
    local area = math.pi * (diam/20)^2
    local peso = 0.785*area
    local c = row.counter
    local midrule = ""
    if c - 3*math.floor(c/3) == 0 then
        midrule = string.char(92).."midrule"
    end
    row:insert(1, diam)
       :insert(2, area, 3)
       :insert(3, peso, 6)
       :insert(0, midrule)
end
}
\end{Verbatim}

Introduciamo anche il pacchetto \pack{siunitx} utilissimo per comporre numeri,
unità di misura e tabelle, con questo ambiente \amb{tabular} ridisegnato:
\begin{Verbatim}
\begin{tabular}{crrr}
\toprule
\diameter & Sezione & Peso\\
\small\si{mm} & \small\si{cm^2} & \small\si{daN/m}\\
\midrule
\makerow{\textbf{<1>} & \num{<2>} & \num{<3>}\\<0>}
\bottomrule
\end{tabular}
\end{Verbatim}

Il progetto completo si trova nel file \file{app-star/E0-006-tab.tex}, dove ho
aggiunto alla tabella la colonna con il calcolo della superficie laterale delle
barre. Ecco il risultato:

\directlua{
function Row:insert(col, val, prec)
    if prec then
        local p = string.char(37)
        local fmt = string.format(p..p..'0.'..p..'df', prec)
        val = string.format(fmt, val)
    end
    self[col] = val
    return self
end

local function fn(diam, row)
    local area = math.pi * (diam/20)^2
    local peso = 0.785*area
    local sup_lat = 10 * math.pi * diam
    local c = row.counter
    local midrule = ''
    if c - 3*math.floor(c/3) == 0 then
        midrule = string.char(92)..'midrule'
    end
    row:insert(1, diam)
       :insert(2, area, 3)
       :insert(3, peso, 6)
       :insert(4, sup_lat, 3)
       :insert(0, midrule)
end
row = Row:new(fn, 6, 32, 2)
}

\newcommand{\makerow}[1]{\directlua{
local tmpl = [=[\detokenize{#1}]=]
for _, s in row:iter_template(tmpl) do
   tex.print(s)
end
}}

\begin{center}
\begin{tabular}{crrr}
\toprule
\diameter     & Sviluppo & Sezione & Peso\\
\small\si{mm} & \small\si{cm^2/m} & \small\si{cm^2} & \small\si{daN/m}\\
\midrule
\makerow{\(\mathbf{<1>}\) & \num{<4>} & \num{<2>} & \num{<3>}\\<0>}
\bottomrule
\end{tabular}
\end{center}

La nostra classe \code{Row} ci permette di costruire tabelle iterative in Lua
in modo del tutto generale, compiendo calcoli numerici e ogni sorta di possibili
elaborazioni. Molti altri affinamenti sono possibili: il caricamento di dati
esterni, la possibilità di utilizzare pipeline di comandi all'interno dei
segnaposto del template per esempio, ma è tempo di passare al prossimo
argomento.

% end of file


\input{section/1-intro/03-howto}

\part{Fondamenti del linguaggio Lua}
\label{partFoundation}



\chapter{Come eseguire gli esercizi}

È certamente fondamentale eseguire noi stessi esempi ed esercizi di
programmazione allo scopo di acquisire la padronanza di Lua. Nella guida ne
trovate alcuni alla fine di ciascun capitolo della parte seconda. 

Questa sezione vi introduce brevemente al programma \prog{texlua} che già
trovate compreso in ogni recente distribuzione \TeX{}. Si tratta dell'interprete
Lua controparte di \texttt{luatex}.

Rispetto all'interprete \prog{lua} standard \prog{texlua} non ha la modalità
interativa REPL\footnote{Read–eval–print loop.} con cui si digita una linea di
codice alla volta in un prompt interattivo, modalità utile per fare provare il
funzionamento di funzioni.

Il codice dunque, andrà memorizzato in un file con estensione \texttt{.lua}.
Come esempio elementare, digitiamo questa unica riga di codice in un file
di testo chiamato \texttt{primo.lua}:
\begin{Verbatim}
print("Hello World!")
\end{Verbatim}
apriamo una finestra di terminale\footnote{Maggiori dettagli per diversi sistemi
operativi sulla linea di comando possono essere trovati nella guida tematica
dedicata scaricabile dal sito GuIT.} e lanciamo il comando:
\begin{Verbatim}
$ texlua primo.lua
\end{Verbatim}

Ora che sappiamo come eseguire codice Lua, concentriamoci con i prossimi
capitoli sulle basi del linguaggio. Torneremo nella seconda parte della guida su
ulteriori modalità di esecuzione anche per il codice Lua interno a sorgenti \TeX.




\input{section/2-basic/01-fondamenti}



\chapter{La tabella}
\label{iChTabella}

In questo capitolo parleremo della \emph{tabella}, l'unico tipo strutturato
predefinito di Lua. Diamone subito la definizione: la tabella è un
\emph{dizionario} cioè l'insieme non ordinato di coppie chiavi/valore e, allo
stesso tempo, anche un \emph{array} cioè una sequenza ordinata di valori.

In Lua ne è previsto quindi un uso molteplice: se le chiavi sono numeri interi
la tabella sarà un array, se le chiavi sono di altro tipo, per esempio stringhe,
avremo un dizionario.

Le chiavi possono essere di tutti i tipi previsti da Lua tranne che \key{nil},
mentre i valori possono appartenere a qualsiasi tipo. Nulla vieta che in una
stessa tabella coesistano chiavi di tipo diverso.

Dal punto di vista sintattico, una tabella di Lua è un oggetto racchiuso tra
parentesi graffe e, la più semplice quella vuota, si crea così:
\lines
local t = {} -- una tabella vuota
\endlines
\sourcecode{from_lines()}

Per assegnare e ottenere il valore associato a una chiave si utilizzano le
parentesi quadre, l'operatore di indicizzazione, ecco un esempio:
\lines
local t = {}
t["key"] = "val"
print(t["key"]) --> stampa "val"
\endlines
\sourcecode{from_lines()}

Stando alla definizione che abbiamo dato, una tabella può avere chiavi anche di
tipo differente, e infatti è proprio così e ciò vale anche per i valori. In
questo esempio una tabella ha chiavi di tipo numerico e di tipo stringa con
valori a sua volta di tipo diverso:
\lines
local t = {}
t["key"] = 123
t[123] = "key"
print(t["key"]) --> stampa il tipo numerico 123
print(t[123])   --> stampa il tipo stringa "key"
\endlines
\sourcecode{from_lines()}


\section{La tabella è un oggetto}

Cosa significa che la tabella di Lua è un oggetto? Vuol dire che la tabella è
un dato in memoria gestito con un riferimento. In conseguenza, se si copia una
variabile a una tabella in una seconda variabile questa farà riferimento ancora
alla stessa tabella e non una sua copia:
\lines
local t = {}
t[1], t[2] = 10, 20
-- copia la tabella o il riferimento?
local other = t
t[1] = t[1] + t[2] -- modifichiamo t
-- l'altra variabile riflette la modifica?
assert(t[1] == other[1])
\endlines
\sourcecode{from_lines()}

Con la funzione \fn{assert} si può esprimere l'equivalenza logica tra due
espressioni. Essa ritorna l'argomento se questo è \key{true} oppure se non è
\key{nil}, altrimenti termina l'esecuzione del programma riportando l'errore
descritto eventualmente da un secondo argomento testuale.

Il fatto che la tabella è un oggetto è la premessa fondamentale per la
programmazione a oggetti in Lua e per scrivere codice più compatto nelle
elaborazioni su tabelle dalla struttura molto complessa.

In effetti possiamo annidare in una tabella ulteriori tabelle assegnandole come
valore a corrispondenti chiavi, con complessità arbitraria. In altri termini
una tabella può rappresentare una struttura ad albero senza limiti teorici.
Poiché essa è gestita attraverso un riferimento nell'albero vi saranno solamente
i corrispondenti riferimenti mentre il dato effettivo sarà presente in qualche
altra parte della memoria.


\section{Il costruttore e la dot notation}

Dunque la tabella è un tipo di dato molto flessibile, è un oggetto, ed è
sufficientemente efficiente. Può essere usata in moltissime diverse situazioni
ed è ancora più utile grazie all'efficacia del suo \emph{costruttore}.

Ispirato al formato di dati bibliografici di BibTeX, uno dei programmi storici
del sistema \TeX{} usato per la gestione delle bibliografie nei documenti
\LaTeX, il costruttore di Lua può creare tabelle da una sequenza di
chiavi/valori inserite tra parentesi graffe:
\lines
local t = { a = 123, b = 456, c = "valore"}
\endlines
\sourcecode{from_lines()}

La chiave appare come il nome di una variabile ma in realtà nel costruttore
essa viene interpretata come una chiave di tipo stringa. Così l'esempio
precedente è equivalente al seguente codice:
\lines
-- codice equivalente
local t = {}
t["a"] = 123
t["b"] = 456
t["c"] = "valore"
\endlines
\sourcecode{from_lines()}

La notazione del costruttore non ammette l'utilizzo diretto di chiavi
numeriche. Se occorrono è necessario utilizzare le parentesi quadre per
racchiudere il numero che fa da indice:
\lines
-- chiavi numeriche nel costruttore?
local t_error = { 20 = 123 }
local t_ok = { [20] = 123 }
\endlines
\sourcecode{from_lines()}

Invece, se nel costruttore omettiamo le chiavi, otteniamo una tabella array con
indici interi impliciti in sequenza a partire da 1, contrariamente alla maggior
parte dei linguaggi dove l'indice comincia da 0. Ecco un esempio:
\lines
local t = { 30, 8, 500 }
print(t[1] + t[2] + t[3]) --> stampa 538
\endlines
\sourcecode{from_lines()}

Non è tutto. L'efficacia sintattica del costruttore è completata dalla
\emph{dot notation}, valida solamente per le chiavi di tipo stringa: il campo di
una chiave di tipo stringa si indicizza scrivendone la chiave dopo il nome del
riferimento della tabella, separato dal carattere \key{.}:
\lines
local t = { chiave = "123" }
assert(t.chiave == t["chiave"])
\endlines
\sourcecode{from_lines()}

Prestate attenzione perché all'inizio si può male interpretare il risultato del
costruttore della tabella se unito alla dot notation:
\lines
local chiave = "ok"
local t = { ok = "123"} -- t.ok == t[chiave]

-- attenzione!
local k = "ok"
print( t.k ) --> stampa nil: "k" non è definita in t
print( t[k]) --> stampa "123"
-- t[k] == t["ok"] == t.ok
-- t.k diverso da t[k] !!!
\endlines
\sourcecode{from_lines()}

Non confondete il nome di variabile con il nome del campo in dot notation!

Riassumendo, indicizzare una tabella con una variabile restituisce il valore
associato alla chiave uguale al valore della variabile, mentre indicizzare in
dot notation con il nome uguale a quello della variabile restituisce il valore
associato alla chiave corrispondente alla stringa del nome.


\section{Esercizi}

\begin{Exercise}[label={tab-01}]
Scrivere il codice Lua che memorizzi in una tabella i primi 10 numeri primi
usandone il costruttore.
\end{Exercise}

\begin{Exercise}[label={tab-02}]
Utilizzando la dot notation è possibile utilizzare caratteri spazio nel nome
della chiave delle tabelle?
\end{Exercise}

\begin{Exercise}[label={tab-03}]
Scrivere il codice Lua che stampi il valore associato alle chiavi \key{paese} e
\key{codice}, e il numero medio di comuni per regione, per la seguente tabella.
Stampare inoltre il numero di abitanti della capitale.
\lines
local t = {
    paese = "Italia",
    lingua = "italiano",
    codice = "IT",
    regioni = 20,
    province = 110,
    comuni = 8047,
    capitale = {"Roma", "RM", abitanti = 2753000},
}
\endlines
\sourcecode{from_lines()}
\end{Exercise}


% end of file


\input{section/2-basic/03-costrutti-base}



\chapter{Operatori logici}

In Lua un'espressione è vera se essa corrisponde al valore booleano \key{true}
oppure a un valore che non è \key{nil}.

Gli operatori logici \key{and}, \key{or} e \key{not} danno luogo ad alcune
espressioni idiomatiche di Lua. Cominciamo con \key{or}: è un operatore logico
binario. Se il primo operando è vero lo restituisce altrimenti restituisce il
secondo. Per esempio nel seguente codice \key{a} vale 123.
\lines
local a = 123 or "mai assegnato"
\endlines
\sourcecode{from_lines()}

L'operatore \key{and} --- anche questo binario come \key{or} --- restituisce il
primo operando se esso è falso altrimenti restituisce il secondo operando.

Con \key{and} e \key{or} combinati otteniamo l'operatore ternario del C++ in
Lua: Ecco l'espressione in un esempio: se \key{a} è vera il risultato è \key{b}
altrimenti \key{c}:
\lines
local val = (a and b) or c
\endlines
\sourcecode{from_lines()}

Poiché \key{and} ha priorità maggiore rispetto a \key{or} nell'espressione
precedente possiamo omettere le parentesi per un codice ancor più idiomatico:
\lines
local val = a and b or c -- a ? b : c del C++
\endlines
\sourcecode{from_lines()}

Il massimo tra due numeri è un'espressione condizionale:
\lines
local x, y = 45.69, 564.3
local max
if x > y then
    max = x
else
    max = y
end
\endlines
\sourcecode{from_lines()}
ma con gli operatori logici è tutto più Lua:
\lines
local x, y = 45.69, 564.3
local max = (x > y) and x or y
\endlines
\sourcecode{from_lines()}

L'operatore logico \key{not} restituisce \key{true} se l'operando è \key{nil}
oppure se è \key{false} e, viceversa, restituisce \key{false} se l'operando non
è \key{nil} oppure è \key{true}. Alcuni esempi:
\lines
print(not 5)       --> 'false'
print(not not 5)   --> 'true'
print(not true)    --> 'false'
print(not false)   --> 'true'
print(not nil)     --> 'true'
\endlines
\sourcecode{from_lines()}

L'operatore di negazione può essere usato per controllare se una variabile è
valida oppure no. Per esempio possiamo controllare se in una tabella esiste il
campo \key{prezzo}:
\lines
local t = {} -- una tabella vuota
if not t.prezzo then -- t.prezzo è nil
    print("assente")
else
    print("presente")
end

t.prezzo = 12.00
if not t.prezzo then
    print("assente")
else
    print("presente")
end
\endlines
\sourcecode{from_lines()}


\section{Esercizi}

\begin{Exercise}[label=oplogic-01]
Prevedere il risultato delle seguenti espressioni Lua:
\lines
local a = 1 or 2
local b = 1 and 2
local c = "text" or 45

local d = not 12 or "ok"
local e = not nil or "ok"
\endlines
\sourcecode{from_lines()}
\end{Exercise}

\begin{Exercise}[label=oplogic-02]
Nel seguente codice, se il valore del primo condizionale è \key{true} cosa
stamperà invece il secondo condizionale?
\lines
if "stringa" then print "it's not 'nil'" end
if "stringa" == true then
    print("it's 'true'")
else
    print("it's not 'true'")
end
\endlines
\sourcecode{from_lines()}
\end{Exercise}

\begin{Exercise}[label=oplogic-03]
Come distinguere se una variabile contiene il valore \key{false} o il valore
\key{nil}?  
\end{Exercise}

\begin{Exercise}[label=oplogic-04]
Usando gli operatori logici di Lua codificare l'espressione che restituisce
la stringa \key{"più grande di 100"}, \key{"uguale"} o \key{"più piccolo di
100"} a seconda del valore numerico fornito.
\end{Exercise}



% end file



\input{section/2-basic/05-stringhe}


\chapter{Funzioni}

Le funzioni sono il principale mezzo di astrazione e lo strumento base per
strutturare il codice.

Coerentemente con il resto del linguaggio la sintassi di una funzione comprende
due parole chiave che servono per delimitare il blocco di codice di
definizione: \key{function} ed \key{end}. Una funzione può restituire dati
tramite la parola chiave \key{return}.

Come primo esempio, vi presento una funzione per calcolare l'ennesimo numero
della \href{http://it.wikipedia.org/wiki/Successione_di_Fibonacci}{serie di
Fibonacci}. Un elemento si ottiene sommando i due precedenti elementi avendo
posto uguale a 1 i primi due:
\sourcecode{
    from_file [[code/e2-001.lua]]
    :select_lines [[uno]]
}

Con le regole dell'assegnazione multipla una funzione può accettare più
argomenti. Se gli argomenti passati sono in eccesso rispetto a quelli che essa
prevede, quelli in più verranno ignorati. Se viceversa, gli argomenti sono
inferiori a quelli previsti allora a quelli mancanti verrà assegnato il valore
\key{nil}.

Ma questo vale anche per i dati di ritorno quando la funzione è usata come
espressione in un'istruzione di assegnamento. Basta inserire dopo l'istruzione
\key{return} la lista delle espressioni separate da virgole che saranno valutate
e assegnate alle corrispondenti variabili.

Per esempio, potremo modificare la funzione precedente per restituire la somma
dei primi \( n \) numeri di Fibonacci oltre che l'ennesimo elemento della serie
stessa e considerare un valore di default se l'argomento è \key{nil}:
\sourcecode{
    from_file [[code/e2-001.lua]]
    :select_lines [[due]]
    :add_output{delim_run = true}
}


\section{Funzioni: valori di prima classe, I}

In Lua le funzioni sono un tipo. Possono essere assegnate a una variabile e
possono essere passate come argomento a un'altra funzione, una proprietà che
non si trova spesso nei linguaggi di scripting e che offre una nuova
flessibilità al codice.

In realtà in Lua tutte le funzioni sono memorizzate in variabili. Per assegnare
direttamente una funzione a una variabile esiste la sintassi anonima:
\lines
add = function (a, b)
    return a + b
end
print(add(45.4564, 161.486))
\endlines
\sourcecode{from_lines()}

Essendo le funzioni valori di prima classe ne consegue che in Lua le funzioni
sono oggetti senza nome esattamente come lo sono gli altri tipi come i numeri e
le stringhe. Inoltre, la sintassi classica di definizione:
\lines
function variable_name (args)
    -- function body
end
\endlines
\sourcecode{from_lines()}

\noindent è solo \emph{zucchero sintattico} perché l'interprete Lua la tradurrà
automaticamente nel codice equivalente in sintassi anonima:
\lines
variable_name = function (args)
    -- function body
end
\endlines
\sourcecode{from_lines()}


\section{Funzioni: valori di prima classe, II}

Un esempio di funzione con un argomento funzione è il seguente, dove viene
eseguito un numero di volte dato, la stessa funzione priva di argomenti:
\sourcecode{
    from_file [[code/e3-001.lua]]
    :select_lines [[uno]]
}

Molto interessante. Nell'ultima riga di codice l'argomento è una funzione
definita in sintassi anonima che verrà eseguita 12 volte.

Per prendere confidenza con il concetto di \emph{funzioni come valori di prima
classe}, cambiamo il significato della funzione \fn{print}:
\lines
local orig_print = print
print = function (n)
    orig_print("Argomento funzione -> "..n)
end

print(12)
\endlines
\sourcecode{from_lines()}


\section{Tabelle e funzioni}

Se una tabella può contenere chiavi con qualsiasi valore allora può contenere
anche funzioni! Le sintassi previste sono queste, esplicitate con il codice
riportato di seguito:
\begin{compactitemize}
\item assegnare la variabile di funzione a una chiave di tabella;
\item assegnare direttamente la chiave di tabella con la definizione di funzione
in sintassi anonima;
\item usare il costruttore di tabelle per assegnare funzioni in sintassi
anonima.
\end{compactitemize}
\sourcecode{
    from_file [[code/e3-001.lua]]
    :select_lines [[due]]
}

Con questo meccanismo una tabella può svolgere il ruolo di \emph{modulo}
memorizzando funzioni utili a un certo scopo. In effetti la libreria standard
di Lua si presenta all'utente proprio in questo modo.



\section{Numero di argomenti variabile}

Una funzione può ricevere un numero variabile di argomenti rappresentati da tre
dot consecutivi \key{...}. Nel corpo della funzione i tre punti
rappresenteranno la lista degli argomenti, dunque possiamo o costruire con essi
una tabella oppure effettuare un'assegnazione multipla.

Un esempio è una funzione che restituisce la somma di tutti gli argomenti
numerici:
\sourcecode{
    from_file [[code/e3-001.lua]]
    :select_lines [[tre]]
}

Per inciso, anche la funzione base \fn{print} accetta un numero variabile di
argomenti. Il meccanismo è ancora più flessibile perché tra i primi argomenti
vi possono essere variabili ``fisse''. Per esempio il primo parametro potrebbe
essere un moltiplicatore:
\sourcecode{
    from_file [[code/e3-001.lua]]
    :select_lines [[quattro]]
}

Un'altra funzione predefinita \fn{select} consente di accedere alla lista degli
argomenti in dettaglio. Infatti nel codice precedente, se tra gli argomenti
compare un valore \key{nil} avremo problemi ad accedere ai valori successivi
perché --- come sappiamo già --- l'operatore di lunghezza \key{\#} considera il
\key{nil} come valore sentinella di fine array/tabella.

Il selettore prevede un primo parametro fisso seguito da una lista variabile di
valori rappresentata dai tre punti \key{...}. Se questo parametro è un intero
allora verrà considerato come indice per restituire l'argomento corrispondente.
Se invece il parametro è la stringa \key\# allora la funzione restituisce il
numero totale di argomenti, inclusi i \key{nil}.

Il codice seguente preso pari pari dal \href{http://www.lua.org/pil/}{PIL} ne è
un'applicazione:
\lines
for i = 1, select("#", ...) do
    local arg = select(i, ...)
    -- loop body
end
\endlines
\sourcecode{from_lines()}


\section{Omettere le parentesi se}

In Lua esiste la sintassi di chiamata a funzione semplificata che consiste nella
possibilità di ommettere le parentesi tonde \key{()}, ammessa solo se:
\begin{compactitemize}
\item alla funzione si passa un unico argomento di tipo stringa;
\item alla funzione si passa un unico argomento di tipo tabella.
\end{compactitemize}

Per esempio:
\sourcecode{
    from_file[[code/e7-funzioni.lua]]
    :select_lines [[anchequesto]]
}


\section{Closure}
\label{secClosure}

Chiudiamo il capitolo parlando con uno strano termine forse meglio noto agli
sviluppatori dei linguaggi funzionali: la \emph{closure}.

Questa proprietà di Lua amplia il concetto di funzione rendendo possibile
l'accesso dall'interno di essa ai dati presenti nel contesto esterno. Ciò è
possibile perché alla chiamata di una funzione viene creato uno spazio di
memoria del contesto esterno unico e indipendente.

\begin{quote}
\emph{%
Tutte le chiamate a una stessa funzione condivideranno una stessa closure.%
}
\end{quote}

Se questo è vero una funzione potrebbe incrementare un contatore creato al suo
interno, e anche qui prendo l'esempio di codice dal PIL:
\sourcecode{
    from_file [[code/e7-funzioni.lua]]
    :select_lines [[uno]]
}

Il codice definisce una funzione \fn{new\_counter} che restituisce una
funzione che ha accesso indipendente al contesto (la variabile \key{i}).

Tecnicamente la closure \emph{è} la funzione effettiva mentre invece la
funzione non è altro che il prototipo della closure.

Le closure consentono di implementare diverse tecniche utili in modo naturale e
concettualmente semplice. Una funzione di ordinamento potrebbe per esempio
accettare come parametro una funzione di confronto per stabilire l'ordine tra
due elementi tramite l'accesso a una seconda tabella esterna contenente
informazioni utili per l'ordinamento stesso.

Nel prossimo esempio mettiamo in pratica l'idea. Il codice utilizza una
funzione della libreria di Lua, che introdurremo nel prossimo capitolo, in
particolare \fn{table.sort}, per applicare l'algoritmo di ordinamento alla
tabella passata come argomento in base al criterio di ordine stabilito con la
funzione passata come secondo argomento in sintassi anonima.
\sourcecode{
    from_file [[code/e7-funzioni.lua]]
    :select_lines [[due]]
    :add_output{delim_run=true}
}


\section{Esercizi}

\begin{Exercise}[label=fn-01]
Scrivere una funzione che sulla base della stringa in ingresso \key{"+"},
\key{"-"}, \key{"*"}, \key{"/"} restituisca la funzione corrispondente per due
operandi.
\end{Exercise}

\begin{Exercise}[label=fn-02]
Scrivere la funzione che accetti due argomenti numerici e ne restituisca i
risultati delle quattro operazioni aritmetiche.
\end{Exercise}

\begin{Exercise}[label=fn-03]
Scrivere una funzione che restituisca il fattoriale di un numero memorizzandone
in una tabella di closure i risultati per evitare di ripetere il calcolo in
chiamate successive con pari argomento.
\end{Exercise}

\begin{Exercise}[label=fn-04]
Scrivere una funzione con un argomento opzionale rispetto al primo parametro
numerico che ne restituisca il seno interpretandolo in radianti se l'argomento
opzionale è \key{nil} oppure \key{rad}, in gradi sessadecimali se \key{deg} o
in gradi centesimali se \key{grd}.
\end{Exercise}

\begin{Exercise}[label=fn-05]
Scrivere una funzione che accetti come primo argomento una funzione \( f:
ℝ \to ℝ \) (prende un numero e restituisce un numero), come
secondo e terzo argomento i due valori dell'intervallo di calcolo e come quarto
argomento il numero di punti in cui suddividere l'intervallo. La funzione dovrà
stampare i valori che la funzione argomento assume nei punti d'ascissa così
definiti.
\end{Exercise}


% end of file


\input{section/2-basic/07-libreria-standard}


\chapter{Iteratori}

Gli iteratori offrono un approccio semplice e unificato per scorrere uno alla
volta gli elementi di una collezione di dati. Vi dedicheremo un capitolo proprio
perché sono molto utili per scrivere codice efficiente ed elegante.

Il linguaggio Lua prevede il ciclo d'iterazione \emph{generic for} che
introduce la nuova parola chiave \key{in} secondo questa sintassi:
\lines
for <lista variabili> in iterator_function() do
-- codice
end
\endlines
\sourcecode{from_lines()}

Le tabelle di Lua sono oggetti che possono essere impiegati per rappresentare
degli array oppure dei dizionari. In entrambe i casi Lua mette a disposizione
due iteratori predefiniti rispettivamente tramite le funzioni \fn{ipairs} e
\fn{pairs}.

Queste funzioni restituiscono un iteratore conforme alle specifiche del generic
for. Mentre impareremo più tardi a scrivere iteratori personalizzati,
dedicheremo le prossime due sezioni a questi importanti iteratori predefiniti
per le tabelle.


\section{Funzione ipairs()}

La funzione \fn{ipairs} restituisce un iteratore che a ogni ciclo genera due
valori: l'indice dell'array e il valore corrispondente. L'iterazione comincia
dalla posizione di indice 1 e termina quando il valore corrente è \key{nil}:
\lines
-- una tabella array
local t = {45, 56, 89, 12, 0, 2, -98}

-- iterazione tabella come array
for i, v in ipairs(t) do
    print(i, v)
end
\endlines
\sourcecode{from_lines()}

Il ciclo con \fn{ipairs} è equivalente a questo codice:
\lines
-- una tabella array
local t = {45, 56, 89, 12, 0, 2, -98}
do
    local i, v = 1, t[1]
    while v do
        print(i, v)
        i = i + 1
        v = t[i]
    end
end
\endlines
\sourcecode{from_lines()}

Se non interessa il valore dell'indice possiamo convenzionalmente utilizzare
per esso il nome di variabile corrispondente a un segno di underscore che in
Lua è un identificatore valido:
\lines
-- una tabella array
local t = {45, 56, 89, 12, 0, 2, -98}

local sum = 0
for _, elem in ipairs(t) do
    sum = sum + elem
end
print(sum)
\endlines
\sourcecode{from_lines()}

Se non vogliamo incorrere in errori è molto importante ricordarsi che con
\fn{ipairs} verranno restituiti i valori in ordine di posizione da 1 in poi e
fino a che non verrà trovato un valore \key{nil}. Se desiderassimo raggiungere
tutte le coppie chiave/valore dovremo far ricorso all'iteratore \fn{pairs}
che tratteremo nella prossima sezione.


\section{Funzione pairs()}

Questa funzione primitiva di Lua considera la tabella come un dizionario
pertanto l'iteratore restituirà in un ordine casuale tutte le coppie chiave
valore contenute nella tabella stessa.

Una tabella con indici a salti verrà iterata parzialmente da \fn{ipairs} ma
completamente da \fn{pairs}:
\sourcecode{
    from_file [[code/e9-iter.lua]]
    :select_lines [[uno]]
    :add_output{delim_run=  true}
}

Il comportamento di questi due iteratori potrebbe lasciare perplessi ma è
coerente con le caratteristiche della tabella di Lua.


\section{Generic for}

Come può essere implementato un iteratore in Lua? Per iterare è necessario
mantenere alcune informazioni essenziali chiamate \emph{stato} dell'iteratore.
Per esempio l'indice a cui siamo arrivati nell'iterazione di una tabella/array
e la tabella stessa.

Perchè non utilizzare la closure per memorizzare lo stato dell'iteratore?

Abbiamo incontrato le closure nella sezione \ref{secClosure}. Proviamo a
scrivere il codice per iterare una tabella:
\sourcecode{
    from_file [[code/e9-iter.lua]]
    :select_lines [[due]]
}

Funziona, molto semplicemente. Non è stato necessario introdurre nessun nuovo
elemento al linguaggio. L'iteratore è solamente una questione d'implementazione
che tra l'altro in questo caso ricrea l'iteratore \fn{ipairs} visto poco fa.

Infatti, la funzione \fn{iter\_test} mantiene nella closure lo stato
dell'iteratore --- l'indice \key{i} e la tabella \key{t} --- e restituisce uno
dopo l'altro gli elementi della tabella. Il ciclo \key{while} infinito,
s'interrompe quando il valore è \key{nil}.

Tuttavia, data l'importanza degli iteratori, Lua introduce il nuovo costrutto
chiamato \emph{generic for} che si aspetta una funzione proprio come la
\fn{iter} del codice precedente. E in effetti funziona:
\sourcecode{
    from_file [[code/e9-iter.lua]]
    :select_lines [[tre]]
}

Riassumendo, la costruzione di un iteratore in Lua si basa sulla creazione di
una funzione che restituisce uno alla volta gli elementi dell'insieme nella
sequenza desiderata. Una volta costruito l'iteratore, questo potrà essere
impiegato in un ciclo generic for.

Se per esempio si volesse iterare la collezione dei numeri pari compresi
nell'intervallo da 1 a 10, avendo a disposizione l'apposito iteratore
\fn{evenNum} che definiremo in seguito, potrei scrivere semplicemente:
\lines
for n in evenNum(1,10) do
    print(n)
end
\endlines
\sourcecode{from_lines()}


\section{L'esempio dei numeri pari}

Per definire questo iteratore dobbiamo creare una funzione che restituisce a
sua volta una funzione in grado di generare la sequenza dei numeri pari.
L'iterazione termina quando giunti all'ultimo elemento, la funzione restituirà
il valore nullo ovvero \key{nil}, cosa che succede in automatico senza dover
esplicitare un'istruzione di \key{return} grazie al funzionamento del generic
for.

Potremo fare così: dato il numero iniziale per prima cosa potremo calcolare il
numero pari successivo usando la funzione della libreria standard di Lua
\fn{math.ceil} che fornisce il numero arrotondato al primo intero superiore
dell'argomento.

Poi potremo creare la funzione di iterazione in sintassi anonima che prima
incrementa di 2 il numero pari precedente --- ed ecco perché dovremo
inizialmente sottrarre la stessa quantità all'indice --- e, se questo è
inferiore all'estremo superiore dell'intervallo ritornerà l'indice e il numero
pari della sequenza. Ecco il codice completo:
\sourcecode{
    from_file [[code/e9-iter.lua]]
    :select_lines [[iter_even]]
    :add_output{delim_run = true}
}

In questo esempio, oltre ad approfondire il concetto di iterazione basata sulla
closure di Lua, possiamo notare che il generic for effettua correttamente anche
l'assegnazione a più variabili di ciclo con le regole viste nel
capitolo~\ref{chAssignment}.

Naturalmente, l'implementazione data di \fn{evenNum} è solo una delle possibili
soluzioni, e non è detto che non debbano essere considerate situazioni
particolari come quella in cui si passa all'iteratore un solo numero o
addirittura nessun argomento.


\section{Stateless iterator}

Una seconda versione del generatore di numeri pari può essere un buon esempio
di un iteratore in Lua che non necessita di una closure, per un risultato ancora
più efficiente.

Per capire come ciò sia possibile dobbiamo conoscere nel dettaglio come
funziona il generic for in Lua; dopo la parola chiave \key{in} esso si aspetta
altri due parametri oltre alla funzione da chiamare a ogni ciclo: una variabile
che rappresenta lo stato invariante e la variabile di controllo.

Nel seguente codice la funzione \fn{evenNum} provvede a restituire i tre
parametri necessari: la funzione \fn{nextEven} come iteratore, lo stato
invariante, che per noi è il numero a cui la sequenza dovrà fermarsi e la
variabile di controllo che è proprio il valore nella sequenza dei numeri pari,
e con ciò abbiamo realizzato un stateless iterator in Lua, ovvero un iteratore
che non ha necessità di closure.

La funzione \fn{nextEven} verrà chiamata a ogni ciclo con, nell'ordine, lo
stato invariante e la variabile di controllo, pertanto fate attenzione, dovete
mettere in questo stesso ordine gli argomenti nella definizione:
\sourcecode{
    from_file [[code/e9-iter.lua]]
    :select_lines [[generic_for]]
}

Con gli iteratori abbiamo terminato l'esplorazione di base del linguaggio Lua.
Questi primi nove capitoli sono sufficienti per scrivere programmi utili perché
trattano tutti gli argomenti essenziali. Il prossimo capitolo tratterà del
paradigma della programmazione a oggetti in Lua.


\section{Esercizi}

\begin{Exercise}[label=iter-01]
Dopo aver definito una tabella con chiavi e valori stampare le singole coppie
tramite l'iteratore predefinito \fn{pairs}.
\end{Exercise}

\begin{Exercise}[label=iter-02]
Scrivere una funzione che accetta un array (una tabella con indici sequenziali
interi) di stringhe e utilizzando la funzione di libreria \fn{string.upper}
restituisca un nuovo array con il testo trasformato in maiuscolo (per esempio da
\key{\{"abc", "def", "ghi"\}} a \key{\{"ABC", "DEF", "GHI"\}}).
\end{Exercise}

\begin{Exercise}[label=iter-03]
Scrivere la funzione/closure per l'iteratore che restituisce la sequenza dei
quadrati dei numeri naturali a partire da 1 fino a un valore dato.
\end{Exercise}

\begin{Exercise}[label=iter-04]
Scrivere la versione \emph{stateless} dell'iteratore dell'esercizio precedente.
\end{Exercise}

\begin{Exercise}[label=iter-05]
Scrivere la versione \emph{stateless} dell'iteratore \fn{ipairs}. È possibile
implementarlo in modo che la funzione d'iterazione restituisca per il ciclo
generic for solamente l'elemento della tabella e non anche l'indice?
\end{Exercise}


% end of file


\input{section/2-basic/09-oop}

\part{Applicazioni Lua in \LuaTeX}
\label{partApp}

\input{section/3-app/01-registro}

\input{section/3-app/02-tartaglia-nodi}

%

\chapter{Disegno del cerchio}
\label{iichCerchio}

In questo capitolo esploreremo un'altro tipo di nodo: il \emph{pdfliteral}.
Con esso possono essere inserite nel file PDF figure vettoriali definite
attraverso istruzioni grafiche. Per questi nodi il compositore non fa altro che
inserirli direttamente e senza alcun controllo nel file di uscita perché siano
interpretati unicamente dal programma di visualizzazione e stampa PDF.

Dovremo quindi prestare attenzione alla correttezza delle istruzioni che
definiamo seguendo la sintassi prevista dal formato PDF e contenuta nel PDF
reference distribuito da Adobe. Un errore potrebbe invalidare l'intero
documento.

Se si è utenti del sistema \TeX{} ci sarà certamente capitato di dover
realizzare diagrammi o figure, perciò abbiamo già utilizzato le istruzioni
pdfliteral ma non direttamente. I pacchetti grafici come \pack{picture2e} o
\pack{TikZ} infatti, offrono un'interfaccia di alto livello più compatta e
sicura alle primitive grafiche PDF.

Per conoscere questo tipo di tecnologia, implementeremo in Lua il disegno di
cerchi.


\section{Le curve di Bezier}


\section{Cerchio a otto curve}


\section{SVG}





% end of file


%

\chapter{Dati esterni}
\label{iichDati}

Il tema molto vasto di questo capitolo è la tecnologia che ci permette di
includere nei documenti dati esterni. Può capitare di voler comporre un buon
numero di documenti che condividono un insieme di dati, oppure di comporre un
documento che riporti tabelle con dati sperimentali residenti in file esterni.

Nel primo caso, vorremmo evitare di digitare sempre gli stessi dati in ciascun
documento con il rischio di sbagliare, mentre nel secondo cercheremo una
procedura che carichi in automatico i dati all'interno del documento
presentandoli con aspetto professionale.

Oppure ancora, potremo trovarci in ambito aziendale dove i dati che dovremo
includere nei documenti sono responsabilità di unità operative diverse dalla
nostra. Cercheremo un modo per condividere le informazioni in modo efficiente e
sicuro.

Gli scenari sono quindi molto vari e di conseguenza le tecnologie utili possono
essere le più diverse, per esempio nel fornire o meno la lettura sincrona dei
dati da un database. Comunque sia presenterò degli esempi concreti solamente per
contesti di gestione dati in cui è il motore di composizione a compiere le
elaborazioni affinché i dati siano disponibili all'interno del sorgente. A
questo ambito daremo il nome di \emph{progetti documentali}.

Per esempio, non è un progetto di documento quello in cui il programma elabora i
dati e crea il relativo file di testo che rappresenta il sorgente \TeX{}.
Infatti, così facendo non solo sarebbe possibile compilare il sorgente con i
compositori tradizionali come PDF\LaTeX{} o Xe\LaTeX{}, e quindi non saremo in
tema, ma non lavoreremo scrivendo il sorgente in modo diretto, ovvero nel
contesto ideale per comporre documenti complessi che contengono variazioni uno
rispetto all'altro sia nel contenuto sia nella struttura.


\section{Lua table}

Il modo più semplice per includere dati esterni è quello che non richiede la
scrittura di alcun modulo per caricare i dati. In altre parole, i dati devono
essere scritti secondo un formato nativo del linguaggio, nel nostro caso Lua.

Questo formato nativo per Lua non può che essere il costruttore di tabelle, che
abbiamo incontrato al capitolo~\ref{iChTabella}. Applicheremo questa fantastica
tecnica per realizzare modi diversi di presentare gli stessi dati.








%\subsection{Lua type}


%\subsection{SQLite3 e i quadri economici}


%\subsection{Matematica simbolica}

% end of file


\appendix

%\input{section/8-appendix/01-install}

\backmatter



\chapter{Note finali}

I miei ringraziamenti vanno a Claudio Beccari per aver scritto la classe
\class{guidatemaica.cls} con cui è stata composta la guida.

% end of file


% \input{section/9-end/02-colophone}

% bibliografia
\end{document}

