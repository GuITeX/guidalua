% !TeX program = lualatex--shell-escape

\documentclass[b5paper,11pt,openany]{guidatematica}
\ProvidesFile{guidalua.tex}[2020/06/15 v0.3.1 Guida al linguaggio Lua per LuaTeX]
\GetFileInfo{guidalua.tex}
\setmonofont[Scale=0.82]{Fira Mono}
\usepackage{sourcecode}
\fvset{fontsize=\small}
\usepackage[
    colorlinks=true,
    linkcolor=blue,
    citecolor = blue,
    urlcolor = blue,
    pdftitle={Guida al linguaggio Lua per LuaTeX},
    pdfauthor={Roberto Giacomelli},
    pdfsubject={Guida tematica del GuIT},
    pdfkeywords={Lua, LuaTeX, guida, programming}
]{hyperref}
\usepackage{exercise}
\renewcounter{Exercise}[chapter]
\renewcommand{\ExerciseHeader}{\noindent\textsc{esercizio \ExerciseHeaderNB}}

\frenchspacing

\newcommand\tcmd[1]{\normalfont\texttt{#1}}
\newcommand\key[1]{\normalfont\texttt{#1}}
\newcommand\code[1]{\normalfont\texttt{#1}}
\newcommand\fn[1]{\normalfont\texttt{#1()}}

\begin{document}
\author{Roberto Giacomelli}
\date{\filedate{} --- \fileversion}
\title{Guida al linguaggio Lua per \LuaTeX}
\subtitle{
\begin{tikzpicture}[scale=0.82]
\node at (-2.31-0.72, 0) {};
\draw[fill,verdeguit] (0,0) circle (2.2);
\draw[fill,verdeguit] (2.31, 2.31) circle (0.72);
\draw[fill, white] (0.8,0.8) circle (0.72);
\end{tikzpicture}
\vspace*{8.5ex}
}
\maketitle

\tableofcontents
\newpage


% presentazione della guida e informazioni di base per eseguire gli esempi

\chapter{Introduzione}

\section{Presentazione della guida}

Questa guida tematica è dedicata alla programmazione in Lua all'interno dei
motori di composizione del sistema \TeX. Quel che occorre conoscere è l'ambito
in cui ci troviamo per poter avvalerci di funzionalità ben più semplici da
codificare che non con gli strumenti tradizionali messi a disposizione dal solo
\TeX{} o dalle macro del formato.

Elaborare dati, effettuare calcoli numerici, eseguire compiti in pratica
impossibili con il compositore tradizionale come connettersi a database
relazionali o avvalersi di avanzati sistemi esterni, sono funzionalità
possibili con la nuova generazione di compositori, in scenari applicativi
notevolmente ampliati.

Se da un lato è auspicabile che queste potenzialità diventino disponibili per
gli utenti finali per mezzo di moduli e pacchetti tali da minimizzare la
necessità di programmare, dall'altro è utile fornire dettagli ed esempi per
implementare proprie soluzioni e collaborare condividendo idee di sviluppo.

Sono certamente molte le cose da conoscere: un nuovo linguaggio molto diverso da
\TeX{}, numerosi dettagli sul funzionamento interno dei compositori Lua-powered,
nuovi problemi di organizzazione del codice. Per questo, ho pensato di
contribuire con questa guida tentando di illustrare la cresciuta complessità del
sistema.

Tra gli argomenti della guida ci sono:
\begin{compactitemize}
\item basi del linguaggio Lua, dal capitolo \ref{chAssignment},
\item la differenza tra motore e formato di composizione, capitolo \ref{iichExplain},
\item tecniche di programmazione e di rappresentazione dei dati,
\item l'interazione con lo stato interno del motore di composizione.
\end{compactitemize}

A mio parere se si vuole puntare sullo sviluppo razionale di progetti
documentali non bisogna mai dimenticare \TeX{}, anzi occorre conoscerlo più a
fondo per riuscire ad equilibrarne le componenti in azione per la costruzione di
documenti di qualità.

Auguro perciò ai lettori Happy LuaTeXing!


\section{Lua, proprio un bel nome}

Lua è un linguaggio semplice ma non banale. Il suo ambito di applicazione è
quello dei linguaggi di scripting: text processing, manutenzione del sistema,
elaborazioni su file dati, eccetera e lo si può anche trovare come linguaggio
embedded di programmi complessi come i videogiochi o altri applicativi
che danno la possibilità di essere programmati con esso dall'utente.

Lua è stato ideato da un gruppo di programmatori esperti
dell'\href{http://www.puc-rio.br/index.html}{Università Cattolica di Rio de
Janeiro} in Brasile. ``Lua'' si pronuncia LOO-ah e significa ``Luna'' in
portoghese!


\section{Piano della guida}

La guida è divisa in due parti: la prima tratta delle basi del linguaggio Lua e
la seconda le applica in esempi concreti con l'uso delle librerie interne del
compositore.

La risorsa principale per imparare Lua a cui si rimanda per tutti gli
approfondimenti è certamente il PIL acronimo del titolo del libro
\emph{Programming In Lua} di Roberto Ierusalimschy, principale Autore
di Lua. Questo testo non solo è completo e autorevole ma è anche ben scritto e
composto\footnote{Tra l'altro il libro ufficiale su Lua viene composto in
\LaTeX{} e commercializzato per contribuire allo sviluppo del linguaggio
stesso.}.

Quanto a \LuaTeX{} il riferimento è il suo manuale che, come quasi tutta la
documentazione nel sistema \TeX{}, può essere visualizzato a video con il
comando da terminale:
\begin{Verbatim}
$ texdoc luatex
\end{Verbatim}


\section{Contribuire alla guida}

Spero che i lettori vorranno contribuire al testo inviando le propri soluzioni o
nuovi contributi anche piccoli. Lo si può fare attraverso lo strumento che
preferite, scrivendomi un messaggio email, oppure utilizzando il repository
\texttt{git} dei sorgenti.


\section{Origine della guida}

Per illustrare i concetti del linguaggio ho preso spunto da un breve corso su
Lua che scrissi qualche tempo fa per il blog
\href{http://parliamodi-ubuntu.blogspot.it}{Lubit Linux} di Luigi Iannoccaro
che mi propose di realizzare un progetto di divulgazione su Lua. Luigi ha
acconsentito all'utilizzo di quegli appunti per produrre questa guida tematica.


\section{Come eseguire gli esercizi}

Per iniziare è certamente molto utile eseguire noi stessi esempi ed esercizi di
programmazione. Nella guida ne trovate alcuni alla fine di ciascun capitolo
della parte prima. 

Questa sezione vi introduce brevemente al programma \prog{texlua} che già
trovate compreso in ogni recente distribuzione \TeX{}. Si tratta dell'interprete
Lua controparte di \texttt{luatex} nell'esecuzione del codice Lua come
suggerisce il nome.

Rispetto all'interprete \prog{lua} standard esso non comprende la modalità
interativa detta REPL\footnote{Read–eval–print loop.} con cui si digita una
linea di codice alla volta senza dover creare un file per fare semplici prove. 

Il codice andrà memorizzato in un file con estensione \texttt{.lua}, in questo
modo: in un file \texttt{primo.lua} digitiamo questa unica riga di codice:
\begin{Verbatim}
print("Hello World!")
\end{Verbatim}
apriamo una finestra di terminale\footnote{Maggiori dettagli per diversi sistemi
operativi sulla linea di comando possono essere trovati nella guida tematica
dedicata scaricabile dal sito GuIT.} e lanciamo il comando:
\begin{Verbatim}
$ texlua primo.lua
\end{Verbatim}

Ora che sappiamo come eseguire codice Lua, concentriamoci con i prossimi
capitoli sulle basi del linguaggio. Torneremo nella seconda parte della guida su
ulteriori modalità di esecuzione e a conoscere importanti dettagli
sull'esecuzione di Lua all'interno dei motori di composizione.


% end of file


\mainmatter*

\part{Fondamenti}

\input{section/1-basic/02-fondamenti}

\input{section/1-basic/03-tabella}

\input{section/1-basic/04-costrutti-base}

\input{section/1-basic/05-oplogic}

\input{section/1-basic/06-stringhe}

\input{section/1-basic/07-funzioni}

\input{section/1-basic/08-libreria-standard}

\input{section/1-basic/09-iteratori}

\input{section/1-basic/10-oop}

\part{Applicazioni Lua in \LuaTeX}
\label{partLuaTeX}



\chapter{Sul sistema \TeX{} e Lua}
\label{iichExplain}

Questo capitolo ancora introduttivo fornisce informazioni sulla differenza tra
motore di composizione e formato, e sulla procedura per l'esecuzione di codice
Lua all'interno di un sorgente \TeX.


\section{Motori di composizione e formati}

Un sorgente \TeX{} contiene testo e macro. Il testo formerà i capoversi, i
titoli eccetera del documento, mentre le macro ne stabiliscono aspetto e
struttura. I \emph{motori di composizione} dispongono di particolari macro dette
\emph{primitive} implementate direttamente in essi, e la possibilità di definire
nuove macro per svolgere più facilmente compiti ripetitivi.

Queste nuove macro il cui codice è generalmente scritto da esperti offrono
funzionalità di più alto livello molto utili per l'utente. I motori di
composizione caricano sempre nella fase iniziale della compilazione, un insieme
di macro di alto livello chiamate \emph{formato} perché per esse sono state
stabilite anche nuove regole di sintassi.

Se si avvia un qualsiasi motore di composizione della famiglia \TeX{} verrà
caricato il formato più semplice chiamato \emph{plain}. Se si vuole invece
utilizzare un diverso formato, per esempio il più diffuso \LaTeX{}, occorre
specificarne il nome con l'opzione \texttt{-{}-fmt} quando si scrive il comando
di compilazione al terminale.

Tuttavia data l'importanza per gli utenti dei formati ad alto livello, sono
stati predisposti comandi scorciatoia. Per esempio il programma
\prog{pdflatex}, rimanda all'effettivo motore di composizione, il tradizionale
\prog{pdftex}, con l'istruzione di caricare il formato \LaTeX.

Riassumendo, i motori di composizione sono programmi tipografici mentre i
formati sono insiemi coerenti di macro basate sulle primitive di sistema. I nomi
dei programmi disponibili nel sistema \TeX{} possono quindi confondere se non si
conosce questa importante distinzione: alcuni di essi sono comandi scorciatoia
per identificare sia il motore sia il formato e non un motore di composizione a
se stante.


\subsection{Compositori Lua-powered}

\LuaTeX{} è un programma che elabora un file di testo contenente codice \TeX{}
per comporne il corrispondente file PDF, quindi è un motore di composizione.
Nella famiglia \TeX{} ci sono almeno altri due compositori dotati
dell'interprete Lua, LuaHB\TeX{} e Luajit\TeX{}.

Tutti e tre questi compositori possono eseguire il formato \LaTeX. Come detto in
apertura di sezione, esiste il programma \prog{lualatex} scorciatoia a un
compositore che carica il formato \LaTeX.

Nella TeX Live 2020 questo compositore è \prog{luahbtex}. Per rendercene conto
basta scrivere in un terminale il nome del programma, leggere l'output e premere
CTRL + C per chiuderne l'esecuzione di prova:
\begin{Verbatim}
> lualatex
This is LuaHBTeX, Version 1.12.0 (TeX Live 2020/W32TeX)
 restricted system commands enabled.
**
\end{Verbatim}

Per ulteriore informazione, \prog{luahbtex} è il motore di composizione
\prog{luatex} in cui è stato sostituito il componente per il calcolo della forma
dei font con il modulo HarfBuzz, mentre \prog{luajittex} è un'altra variante di
\prog{luatex} in cui l'interprete Lua è stato sostituito con LuaJIT
un'implementazione indipendente che sfrutta le tecniche di compilazione
denominate \emph{Just In Time}.

In generale un sorgente \TeX{} che contiene codice Lua viene correttamente
compilato da qualsiasi dei tre compositori grazie al mantenimento della
compatibilità.


\subsection{Lua in \LuaTeX}

Per illustrare l'esecuzione di codice Lua all'interno di un sorgente \LuaTeX,
consideriamo la stampa di un semplice testo nell'output di console con il
seguente sorgente completo, dove il codice Lua va inserito come argomento della
primitiva \cs{directlua}:
\begin{Verbatim}
% !TeX program = LuaTeX
\directlua{
    print("Hello World!")
}
\bye
\end{Verbatim}
il testo uscirà tra gli altri messaggi di output senza che sia prodotto un file
PDF. Ciò significa che \cs{directlua} è una macro espandibile con risultato
vuoto.

La prima riga di commento è una \emph{riga magica}, comodissima nel dare
istruzione allo shell editor su quale motore di composizione e formato
utilizzare per compilare il documento. In questo caso essa è scritta nella
sintassi prevista da TeX Works. Per la comprensione del codice le righe magiche
sono inutili ma aiutano il lettore a stabilire il contesto di esecuzione, perciò
le troverete nei listati della guida se pertinenti.

Se il sorgente è memorizzato nel file \texttt{primo.tex}, possiamo verificare
quanto previsto in un terminale lanciando il comando:
\begin{Verbatim}
$ luatex primo
\end{Verbatim}
per il sistema operativo Windows e la distribuzione TeX Live 2020, l'output
della console è:
\begin{Verbatim}
This is LuaTeX, Version 1.12.0 (TeX Live 2020/W32TeX) 
    restricted system commands enabled.
(./primo.texHello World!
)
warning  (pdf backend): no pages of output.
Transcript written on primo.log.
\end{Verbatim}


\subsection{Lua in Lua\LaTeX}

Con Lua\LaTeX{} si ottiene lo stesso risultato ma con il sorgente scritto nella
sintassi \LaTeX, ovvero:
\begin{Verbatim}
% !TeX program = LuaLaTeX
\documentclass{article}
\directlua{
    print("Hello World!")
}
\begin{document}
\end{document}
\end{Verbatim}
e questa volta il comando di compilazione è:
\begin{Verbatim}
$ lualatex primo
\end{Verbatim}

Avremo potuto inserire la macro all'interno dell'ambiente \amb{document} anziché
nel preambolo. Quando \TeX{} incontra \cs{directlua} ne \emph{espande}
l'argomento e passa a Lua il controllo che esegue immediatamente il
codice restituendo di nuovo il controllo dell'esecuzione a \TeX{} al termine.


\section{Passaggio di dati}

Da \TeX{} verso Lua i dati possono arrivare tramite espansione. Nella direzione
opposta tramite la scrittura di token con le funzioni della famiglia
\fn{tex.print}.



\section{Sovrapposizione di significati}

Alcuni simboli hanno un diverso significato per \TeX{} e per Lua, dando luogo a
errori o a comportamenti imprevisti.



% end of file


\input{section/2-app/02-registro}

\input{section/2-app/03-tartaglia-nodi}

%\input{section/2-app/04-cerchio-pdfliteral}

%\input{section/2-app/05-dati-esterni}

\appendix

%\input{section/__-install.tex}

\backmatter



\chapter{Note finali}

I miei ringraziamenti vanno a Claudio Beccari per aver scritto la classe
\clsname{guidatemaica.cls} con cui è stata composta la guida.

% end of file


% bibliografia
\end{document}

